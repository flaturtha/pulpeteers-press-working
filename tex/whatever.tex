% !TeX TS-program = LuaLaTeX
% !TeX encoding = UTF-8
\documentclass{novel}
%%% METADATA (FILE DATA):
\SetTitle{The Spider Lily}
\SetAuthor{Bruno Fischer}
\SetPDFX{X-1a:2001}
\SetTrimSize{4.25in}{6.875in}
\SetMediaSize{4.5in}{7.12in}
\SetMargins{0.5in}{0.5in}{0.5in}{0.7in}
\SetParentFont{Libertinus Serif}
\SetFontSize{9.5pt}
\SetHeadFootStyle{5}
\SetHeadJump{1.5}
\SetFootJump{1.5}
\SetLooseHead{50}
\SetEmblems{}{} % Default blanks.
\SetHeadFont[\parentfontfeatures,Letters=SmallCaps,Scale=0.92]{\parentfontname}
\SetPageNumberStyle{\thepage}
\SetVersoHeadText{\theAuthor}
\SetRectoHeadText{\theTitle}
%%% CHAPTERS:
\SetChapterStartStyle{footer} % Equivalent to empty, when style has no footer.
\SetChapterStartHeight{10}
\SetChapterFont[Numbers=Lining,Scale=1.6]{\parentfontname}
\SetSubchFont[Numbers=Lining,Scale=1.2]{\parentfontname}
\SetScenebreakIndent{false}
%%% BEGIN DOCUMENT:
\begin{document}
\frontmatter
\thispagestyle{empty}
% Half-Title Page.
\begin{parascale}[2]
\vspace*{3\nbs}
\centering\charscale[0.75]{The Spider Lily }\par
\centering\charscale[0.75]{The Spider Lily LINE 2}\par
\centering{The Spider Lily LINE 3}\par
\end{parascale}
\clearpage
\thispagestyle{empty}
\null % Necessary for blank page.
% Alternatively, List of Books.
\clearpage
\thispagestyle{empty}
% Title Page.
\begin{parascale}[4]
\centering\charscale[0.75]{The Spider Lily }\par
\centering\charscale[0.75]{The Spider Lily LINE 2}\par
\centering{The Spider Lily LINE 3}\par
\end{parascale}
\vspace*{2\nbs}

\begin{parascale}[1]
\centering\textit{SERIES}\par
\vspace*{3\nbs}
\charscale[2]{Bruno Fischer}\par
\end{parascale}
\vfill
\begin{parascale}[1]
% \centering\InlineImage[0, 3em]{/home/darkstar/dox/working-files/LaTeX/atticus.jpg}

A Tales of Murder Press, LLC\par
\textit{Noir} Novel\par
\end{parascale}
\clearpage
\thispagestyle{empty}
% Copyright Page.
\null\vfill
\allsmcp{First edition} Mammoth Magazine, Jan. 1948 — Vol. 2, No. 1 by Tales of Murder Press, LLC\par
\null\null
\allsmcp{ISBN}\par
\null\null
\vfill
\begin{adjustwidth}{3em}{3em}
\textit{This novel is in the public domain.} Certain \mbox{elements} in this edition are Copyright © 2024 Tales of Murder Press, LLC
\end{adjustwidth}
\clearpage
\thispagestyle{empty}
\clearpage % because ToC must start recto
\thispagestyle{empty}
\begin{toc}[0.5]{0em}
{\centering\charscale[1.25]{Contents}\par}
\null

\tocitem*[1]{Returning Home}{1}
\tocitem*[2]{Yesterday and Today}{2}
\tocitem*[3]{Blood on the Lily}{3}
\tocitem*[4]{The Jail}{4}
\tocitem*[5]{For the People}{5}
\tocitem*[6]{For Me}{6}
\tocitem*[7]{The Verdict}{7}
\tocitem*[8]{Sypathetic People}{8}
\tocitem*[9]{Afternoon of the Mathmetician}{9}
\tocitem*[10]{Negotiation for a Name}{10}
\tocitem*[11]{Poker}{11}
\tocitem*[12]{Emil Schneider}{12}
\tocitem*[13]{Flight}{13}
\tocitem*[14]{Shadows and Spectors}{14}
\tocitem*[15]{The Gamblers}{15}
\tocitem*[16]{Don Yard}{16}
\tocitem*[17]{Pursuit of an Equation}{17}
\tocitem*[18]{The Pay-Off}{18}
\tocitem*[19]{The Return}{19}
\tocitem*[20]{Q.E.D.}{20}


\end{toc}
\clearpage

\mainmatter
\cleartorecto
\thispagestyle{empty}


    \begin{ChapterStart}
    \vspace{3\nbs}
    \ChapterSubtitle[l]{Chapter ch1}
    \ChapterTitle[l]{ch1}
    \end{ChapterStart}

    \FirstLine{\noindent ### Chapter 1}

    
## Returning Home

She wasn’t there. I paused on the lowest step and looked past the two women running toward me. Lily wasn’t anywhere on the narrow station platform.

“Getting off, Lieutenant?” the conductor said.

I stepped down to the platform. My sister Ursula flung her arms about me and held me to her cushiony bosom and said my name over and over as if to taste the sound of it.

Miriam watched me with a tremulous smile. Her black hair was gathered back in a tight bun, showing her high forehead to the hairline. The summer sun had given an Oriental burnish to her normally dark skin, making a charming contrast with the egg-shell white of her thin, nicely filled sweater.

“Where’s Lily?” I asked Ursula.

“She couldn’t come.” Ursula’s fingers ran over the ribbons on my chest. “You didn’t write us about all those medals.”

“Why couldn’t Lily come?” I said Ursula stepped back.

“I’ll explain later. Aren’t you going to say hello to Miriam?”

I held out a hand to Miriam. She didn’t seem to notice it. She leaned against me and tilted her face and I brushed her mouth with mine. Her hands touched my arms and fell away quickly when I turned back to Ursula.

“Is that the way to kiss Miriam after two years, Alec?” Ursula boomed.

She was a big woman, large-boned rather than fleshy, handsome rather than pretty. She was feminine enough for most men’s taste, except when she was bossy and then she assumed the voice and manner of a back-slapping male. I’d never liked that in her, and it jarred me now, particularly.

“Has anything happened to Lily?” I persisted.

There was a silence overlapped by the roar of the departing train. Ursula and Miriam exchanged a glance. Then Ursula said, “She’s not home today,” and tucked a hand through my arm. “How about showing some interest in Miriam and me?”

“Why hasn’t Lily come to the station?” I said. “I phoned yesterday from New York that I’d be on this train.”

Ursula’s mouth made a crooked diagonal line, the way it did when she was annoyed. She didn’t have a chance to say whatever she intended to because just then Oliver Spencer came around the corner of the station house. He hurried toward me with outstretched hand.

He was a little man with a habitual stoop and a fringe of gray hair forming a halo about his bald pate. He owned the largest food market in West Amber, and he had a daughter with whom I used to neck and a son with whom I’d gone to high school.

“Well, well!” he said, pumping my hand. “I’ve heard you were coming home from India. Are you better?”

“I wasn’t wounded.” I picked up my bag.

“I know. I heard you were very sick. Some kind of mental—”

Ursula said quickly: “Doesn’t he look fine, Oliver? So ruddy and clear eyed.”

“Oh.” Mr. Spencer glanced at his watch. “You look fit all right, Alec, I’ve got to see about a shipment of eggs which came by express.” He shook my hand again and bustled off.

***

I walked between Ursula and Miriam down the length of the platform. None of us seemed to have any words. This wasn’t at all the way I had expected it.

When we reached the car, Ursula paused with her hand on the doorhandle and looked back. “I should have told Oliver that I’m calling off tonight’s game. Unless you’d like to take a hand at poker, Alec?”

“I expect to spend the evening with Lily, so go ahead and play,” I said.

Ursula slid behind the wheel without comment. We sat three in the front seat, I in the middle. The sedan still had its luster, inside and out, but it had trouble climbing the hill from the station to the town.

I leaned forward, listening to the motor. “It needs a tune-up. I’ll work on it tomorrow.”

“The car missed you almost as much as we did,” Ursula said.

That was all. I’d been gone twenty-two months. I had come home from the other side of the world, and conversation lagged after a couple of sentences about the car. I felt Miriam’s thigh against my right thigh and Ursula’s shoulder against my left shoulder, but we weren’t together at all.

We drove through the business section of West Amber. It was exactly the way I had left it: the sun baking the road and men in shirtsleeves lounging in the shade of storefronts and women in slacks wheeling baby carriages and children playing box games on the sidewalk and cars searching for parking space along the curb. Month after month, lying in a sticky bunk under netting or flying six miles above earth or brooding at the tailend of drinking too much, you saw this scene in your mind and your insides contracted at the thought of coming home to it or of never coming home to it or to anything, and sometimes you were sure that you would weep like a woman when you saw it again.

Here I was and it wasn’t like that. It wasn’t anything but a familiar scene.

“Look,” I said angrily. “When I was grounded, I wrote you that there was nothing much wrong with me, physically or mentally.”

“Of course,” Ursula said.

“Then what was Mr. Spencer talking about? He seemed to expect me to get off the train cutting paper dolls.”

“I suppose he heard—” she had trouble getting the rest of it out.

“That I was psychoneurotic?” I said. “So what? In the Air Forces that term covers almost anything. In my case it merely meant that the flight surgeon decided that I shouldn’t fly combat any more. Most men can’t fly combat to begin with. I’m at least as normal as they are.”

“Of course.”

“Then, damn it, don’t try to shield me from myself or from anybody else!”

***

URSULA took her eyes from the road, but not long enough for me to look into them. “What gives you the idea that anybody is trying to shield you from anything?”

“You do. I can take it. Go ahead and tell me that my wife walked out on me.”

“She didn’t,” Miriam said, her voice a little breathless.

“Then why wasn’t Lily at the station to meet me?”

“Alec, stop screaming,” Ursula said quietly.

I hadn’t realized that I was. I sank back between the shoulders of the two women and found that I was panting.

“All right, I screamed,” I said. “I suppose that proves that I’m a mental case.”

“Nobody thinks you are,” Ursula said. “Don’t be so sensitive.”

“I’m sorry,” I said. “There’s no reason why I should be sensitive. None at all. I’ve been away for a couple of years in a place I hated, doing what no man should be made to do. All that sustained a lot of us was the thought of getting home. Home to our wives those who had them. Now I’m home’ and my wife—”

Alec, listen.” Miriam touched my arm. “Lily—”

“She happens to be visiting relatives in New York.” Ursula cut in tartly. That’s why she wasn’t at the station ”

“Then why didn’t you let me know when I phoned you from New York? I could have met her there.”

“Because we wanted to see you,” Ursula said. “Lily would have kept you away from us for days. You’ll have time enough for her.”

That was all it was—jealousy. While I was away, they’d been kind enough to Lily whom I’d brought home to live with them; otherwise, Lily would have let me know in her far-between, carping letters. But now that I was back they resented another woman’s having predominant claim to me. Females must own a man, and both of them had owned me. They were on the defensive because they had played a shabby trick on me.

I said: “Ursula, please drive me back to the station.”

The car slowed down, but not enough.

“Can’t you wait at least a day to rush into her arms?” Ursula said grimly.

“She’s my wife.”

And Miriam and I are nothing to you.”

“You know that’s not true. Only—”

I couldn’t bring myself to put it into words for them. I had a right to Lily’s body and it was a beautiful body. Perhaps I hated her. I hardly knew her. There had been the letters, but before that there had been the three nights with her. I had come home for more of those nights. I had to have them.

I haven t any idea where she is staying in New York,” Ursula was saying.

You’ll have to wait till she comes back or phones.”

What was there to say? They had me blocked off. The car was laboring up Mandolin Hill Road.

“Here we are,” Ursula sang out abruptly cheerful.

***

The car swung into the semi-circular driveway and pulled up in front of the porch. The two-story frame house, painted ivory and trimmed royal blue, glistened in the sinking sunlight. The hedge needed cutting and the lawn was too high—another chore to keep me busy until Lily returned. If she did return.

I reached into the back seat of the sedan for my bag and went up the porch steps with Miriam. Ursula had gone ahead to unlock the door. There was a solemn and remote set to Miriam s angular face as she moved at my side, and no words for me.

“What’s wrong?” I said. “Have you been reading those sappy articles for civilians which tell you to treat returned veterans as if they hadn’t been away? Neither of you asked me a thing about myself.”

Miriam stopped on the top porch step. “It takes time for us to get used to you, Alec. You’re so different, so much older. Especially your eyes.”’

“I’m psychoneurotic,” I said. “A mental case. It shows through my eyes.”

“Please, Alec!” Miriam took my hand. Her palm was hot and moist. “Stop twisting everything we say. Stop feeling sorry for yourself.”

“That’s another symptom, feeling sorry for yourself.”

She yanked her hand away and strode angrily to the door. Then she turned, suddenly contrite. “Alec, remember this. Ursula and I love you more than anybody else does.”

“Meaning more than Lily,” I said. “What about her?”

Ursula was in the hall, looking out at us. “It’s after seven, Alec, and dinner is practically ready. You’ll want to wash up first.”

Miriam stepped into the hall and continued on to the kitchen. She hadn’t answered my question.

My room was waiting for me. When I closed the door behind me, the sense of being home came in a rush. These were the things I knew best: the walnut bed and dresser, the bookcases crammed with volumes on poker and chess and cryptography and mathematics, the stand containing thirty worn volumes of the Encyclopedia Britannica, the portable typewriter in its case, the tiny radio on the desk, the chess trophies on the corner shelf and the hand-carved ivory chessmen, the rifle and barbells on the wall and the two Hogarth prints.

I dropped my valise and took down my .22 repeating rifle from its pegs. It had been cleaned and oiled within the last week, probably by Ursula who knew something about guns and wasn’t a bad shot. I replaced the rifle and opened the closet door. My civilian suits hung there, cleaned and pressed. In the dresser I found shirts, socks, underwear, all freshly laundered.

It was all there for me to take up where I had left off.

I shed my uniform and left it on the floor and got into the checker rayon bathrobe and went into the bathroom for a shower. I returned and dressed in light gray slacks and navy blue shirt and powder blue knitted necktie and dark gray tweed jacket. Then I walked back and forth across the room, looking at myself in the dresser mirror every time I passed it. Little by little I got used to myself.

The clothes helped most. Home to clothes which were not. like everybody’s. Home to my own room. Home to Ursula and Miriam. They were swell women, and Lily wasn’t worth any part of them. But Lily was the woman I had come home to be with.



    
    \begin{ChapterStart}
    \vspace{3\nbs}
    \ChapterSubtitle[l]{Chapter ch10}
    \ChapterTitle[l]{ch10}
    \end{ChapterStart}

    \FirstLine{\noindent ### Chapter 10}

    
## Negotiation for a Name

I left the house through the back door when I heard some of the poker players at the front door. Ursula had asked me to go easy on using the car because she was short on gas coupons, but I didn’t intend to drive far.

The telephone directory said that Emil Schneider lived on Ivy Lane. There was a mailbox with his name on it, but it was lined up with eight or ten others at the head of the road. At the first house Mr. Rosenberg was watering his lawn. I stopped the car and asked him where Schneider lived.

Mr. Rosenberg peered at me through the deepening twilight, then turned off the water at the nozzle and came to the car. His son had been one of my close friends. “Alec Linn,” he said, recognizing me.

“I heard Dave was killed in Germany,” I said. “It was a great shock.”

“Yes.” The man had aged a great deal. “Well, Alec, I’m glad you came through all right.”

I felt a flash of resentment that was always with me now, before I realized that he meant the war and not the trial.

“Which house is Schneider’s?” I asked again.

“You’re going to visit Emil Schneider?” There was incredulity in his voice. He knew about Schneider and Lily, of course. Everybody did since the trial.

“Why not?” I said pugnaciously.

Mr. Rosenberg gave me a long solemn stare. Then he said: “Turn up the second driveway on the right and go as far as you can.”

All I saw when I came to the end of the second driveway was a shed garage. I had to get out of the car before I spotted the house a couple of hundred feet away. That was the closest you could drive to it. From there you walked up wooden steps with flimsy birch rails and weaved along a flagstone path through elaborate rock gardens and patches of sloping lawn. The Schneiders must have just come from a big city when they built this house. We small-town folk like to drive up to our front doors.

When I reached the porch, I looked back. I knew all the chaotic odds and ends of roads and lanes in West Amber. James Street and Ivy Lane ran more or less parallel, and from here it was a short walk down the driveway, across Ivy Lane, and along a footpath through the woods to the back of George Winkler’s house on James Street. Which meant also, approximately, to the back of the bungalow in which Lily had lived and died.

***

I rang the bell. There was no answer, but there was light in a second floor window, so I kept my finger on the button. Light went on in the downstairs hall. Emil Schneider opened the door.

“Linn,” he said and took a backward step as if afraid I would hit him. He was in his underwear shirt, and his long loose body was thinner than I had thought. His mustache was no longer streamlined and stubble deprived his face of its normal sleekness. “I want to talk to you,” I said.

“What about?” he asked nervously.

“Don’t be afraid. You weren’t the only one who made love to Lily. I need your help.”

“For what?”

“To find out who murdered her.”

He stood in the doorway looking at me speculatively. “Come in,” he said.

I followed him into the living room. Dust was an inch thick on the furniture. Ashtrays overflowed and the table was piled with old newspapers. This was a house in which there was no woman and the man had stopped caring about it.

Schneider ducked out of the room and returned with a bottle of whiskey and two glasses. I shook my head. He ignored the glasses and took a long pull directly from the bottle. Then he smacked his lips and said: “You were acquitted. What difference does it make to you who murdered her?”

“So you know I didn’t murder her?”

“What’s that?”

“Everybody else assumes that I did. Why don’t you?”

Schneider put down the bottle, eyed it longingly, decided not to, and sat down. “I take your word for it,” he said.

I walked to his chair and stood over him, fighting to keep my voice low and my hands steady. “You know I didn’t kill her because you did.”

“Oh, hell!” he said wearily. “Haven’t I been kicked around enough without you trying it?”

“You’ve been kicked around!”

“All right, both of us.” He seemed about to burst into tears. “Do you know what I was doing when you rang the bell? I was upstairs packing my clothes. I’m taking the seven-twelve train out of this damn town in the morning and never coming back. Yesterday I had divorce papers slapped on me. Eventually my wife would have forgotten and forgiven and would have come back to me with the children, but somebody sent her newspaper clippings of the trial and that finished me with her. Her lawyer who served the papers also told me to get out of this house. When I built it eight years ago I had a judgment against me and put the house in her name, so legally it’s all hers. On top of that, I was fired from my job. Those sanctimonious stuffed shirts at the bank don’t want a teller who was forced to admit in public that he was making love to a married woman. If I had more guts I’d put a bullet in my head.”

***

If he hadn’t had such an abundance of pity for himself, I might have pitied him a little. I knew what Lily could do to a man, and even now that she was dead she was not finished.

“Then why not tell me the truth?” I said.

“What truth?”

“Whatever it is. I can’t see you running to the district attorney when you heard Lily was murdered and telling him you had spoken to her on the phone. Why stick your neck out? Why put yourself in a position where your affair with Lily would probably be made public? It would have been smarter to have sat tight and not have mentioned that phone call at all.’

His attempt at laughter was between a sob and a giggle. “I learned what a mistake it is to be a public-spirited citizen.”

“You chose the lesser of the two evils,” I said. “You did phone Lily that night, but the truth doesn’t end there. She didn’t know that I was back; she asked you to see her that night, or consented to your coming over. You left for her bungalow at once. It took you less than five minutes to walk through the woods. What you did when you got there you know better than anybody else. Then you recalled that there are no private phones in this area. Any of the other seven parties on your line could have overheard you making the date with Lily. If you didn’t come forward to tell about that phone call, somebody else might, and it would look a lot better for you if you did it yourself.”

“And if anybody had heard me make a date to go there, where would I have stood?”

“Not worse off. The testimony of an eavesdropper and a gossip would not be worth much if you got in your version first.”

He studied a red and green pattern in the rug at his feet. “So I went to Lily’s that night, eh?”

“Nothing else explains the way you acted.”

Schneider rose for another slug of whiskey. He watched me as he drank, and the speculative gleam appeared again in his eyes. When he put down the bottle, he said: “How important is it to you to know who murdered Lily?”

“Very important.”

He nodded slowly, as if in agreement with himself. “I told you my wife gets the house. About every cent I have is in it—or every cent I thought I had. I played the stock market last winter and took a licking. Tomorrow morning I’m going to Chicago where I have a brother who’ll help me find a job. But I need money to keep me going until then.”

My stomach turned in disgust. “What do you want to sell me?”

“The name of the murderer.”

“It wasn’t you?”

His laughter jarred my nerves, “I’m not kidding. I need money badly and I have what you want.”

“How much?”

“Five thousand dollars.”

“What do I get for it?”

“A name.”

In other words, nothing. But I had less than nothing now.

“Will you go to the district attorney with me?” I asked.

He was back in the chair and had resumed his inspection of the rug. “Absolutely not.”

“What can I do with just a name?”

“That’s up to you. Whether you buy or not, I’m leaving on the seven- twelve train tomorrow morning.”

***

Five thousand dollars was thirty-five hundred more than I had to my name. But I could somehow get my hands on the rest if I had to.

“How do I know the name is the right one?” I said. “You might pluck it out of the air.”

He shrugged without looking at me. “It’s not easy for me to do this. You won’t believe me, but before I met Lily I was a pretty decent sort of guy. I’ll take your money, but I won’t lie to you.”

“Unless the right name is yours and that’s why you’re running away. What proof can you give me that you’re playing straight with me?”

“I can tell you how I know who murdered her.”

I walked over to the whiskey bottle, but kept my hand from reaching for it. “All right, tell me.”

Schneider leaned forward from his hips, the rug absorbing his attention. “You’re right about the phone call and why I told Hackett about it. Since my wife left me I’ve known that somebody has been listening in on my phone conversations, especially on incoming calls when they picked up the phone in the hope that it would be Lily. I’ve no idea who it is; likely some busybody, gossiping old woman. So Lily and I always pretended we were talking about houses she was interested in buying. Like when we spoke at ten o’clock that night. I didn’t phone her; she phoned me. I guess she was bored and wanted one fling with me before you came back. I’m not sure. She just said she wanted to discuss the house we’d looked at a couple of weeks ago and I knew what she meant and said I’d be right over.

I was so eager I didn’t even stop to put on a hat.”

He lifted his eyes and the bitterness in them startled me. “Damn her to hell! Both of us ought to be glad she’s dead. I don’t know how you feel, Linn, but I’m not glad. That’s why I damn her.” He shook himself. “You’re right, I cut through the woods to her bungalow. I approached from the back. I saw somebody come out through the back door. There was something queer about the way that person went to the road, going to the trees a short distance away and keeping in their shadow all the way. Then I lost sight of”—he hesitated—“the person. The car must have been parked a way down the road, headed toward town, because half a minute later I heard a motor start and a car leave.”

“A car dashed by me at the head of James Street.”

“It could have been that car. You must have been pretty close at that time, though I left before you got there. I was puzzled by the way that person had acted. I went in through the back door and across the kitchen and into the living room. I saw Lily on the floor. I bent over her and she was dead. Then I tyeat it.”

“Why didn’t you call the police?”

“Did you?”

“I was too stunned to think of it,” I said.

“I was stunned too, but not enough to lose my head. Look at what happened to you because the sheriff found you there with the body. Don’t you think I would have got that treatment if it had been me instead? My wife had left me because of Lily and now Lily was giving me the brush-off because you were coming home. What better motive could the district attorney ask for? I ducked out of there as fast as I could and kept going until I was home.”

***

I wasn’t much of a drinker, but now I couldn’t resist Schneider’s whiskey. I filled one of the two glasses. Some of the liquor slopped over my chin. The place had become a hothouse.

“Was it a man or woman?” I asked, putting down the glass.

He shook his head. “You have all the free information I’m giving you.”

“Did you recognize the person?”

“There wouldn’t be any point to this if I hadn’t. Light from the house flowed on the person when the person came out through the back door.”

“Do you know Don Yard and Bertha Kaleman?”

“Last spring I spent a weekend in New York and Lily met me there. That was why my wife left me. Somebody saw me with Lily and told her. This town is full of nosey, scandal hungry—” His mouth twisted. “Anyway, Lily and I went to a nightclub. A squat, very broad man came to our table and said hello to Lily. He didn’t once glance at me. After a minute he went back to his table where a hot-looking redhead was waiting for him. Lily told me that he was Don Yard, her former husband, and that the redhead was his current mistress, Bertha Kaleman.”

“Would you have recognized either Don Yard or Bertha Kaleman months later seeing them for only a moment?”

“I would have,” he said, “but it’s no use fishing for information. I told you enough to show you that I know more. Five thousand dollars will get you the sex and the name.”

“I could tell the district attorney what you just told me.”

Schneider snorted. He was looking at me more often now. He was gaining confidence, getting the feel of the business enterprise.

“I’ll say you lied.”

“The name won’t do me any good unless you back it up.”

“I couldn’t if I wanted to. I can’t prove I saw anybody. All I’ll get out of it is a perjury charge against me for having lied on the witness stand. Bring that five thousand dollars cash before midnight and the name is yours.”

“Cash?” I said. “How can I raise that much money at night? If you wait till the bank opens on Monday—”

“I’m leaving at seven-twelve tomorrow morning with or without the money. I don’t trust you not to stop a check after I tell you the name. That’s why it has to be cash. You have lots of rich friends in town. Your sister always keeps some cash in the house; she cashed a four hundred dollar check at the bank last Thursday. Then there’s Oliver Spencer. There’s no night depository at the bank, so he keeps the Saturday receipts in a wall safe at home until Monday morning. That’s generally around two thousand dollars, most of it in cash.”

“But five thousand dollars!”

“You can scrape the rest up elsewhere. That’s your problem.” He took another look at the rug design and added indifferently: “Take it or leave it.”

How would I figure his hand in a poker game? He’d be the eager kind in a bluff, a super-salesman if he had nothing of value. But he wasn’t playing it that way. He didn’t seem greatly interested in the pot.

“I’ll see what I can do,” I said.

He nodded vaguely, as if he had stopped caring one way or the other. He had gone down a long way from the solid citizen, the family man and home-owner I used to see in the bank cage. Lily had done that to him. He had sunk down among the worms, but there was enough man left in him to give himself an out. That part of him preferred that I turn his offer down and save what was left of his soul.

I decided to call him. Except that unlike poker I lost if he bluffed and won if he held a strong hand. And I wouldn’t be able to tell whether I’d won or lost. He’d give me a name, but how could I know it was the right one? It was a gamble any way you looked at it. I’d gambled before with a lot less at stake.

I said suddenly: “You can make more money by blackmailing the murderer.”

His eyes blazed. “Damn you! What do you think I am?”

There were a lot of words in the language for him, but I didn’t use any of them. “I’ll raise as much as I can,” I said and walked past his feet.

He didn’t speak until I was at the front door. Then he called after me: “Try to make it by midnight.”



    
    \begin{ChapterStart}
    \vspace{3\nbs}
    \ChapterSubtitle[l]{Chapter ch11}
    \ChapterTitle[l]{ch11}
    \end{ChapterStart}

    \FirstLine{\noindent ### Chapter 11}

    
## Poker

They were playing two-dollar stud, which was twice the usual limit. That was the Art Masterson influence. He was a traveling salesman for a dress jobber and had been a close friend of August Hennessey. Whenever he was within driving distance of West Amber on a Saturday night, he came in for a game. He was a fat man who liked to sit whenever possible, but there was nothing relaxed about his poker. He played a vigorous and nervous game, though basically not foolish.

“How’s the hero?” he said, reaching out a pudgy hand to me. His other hand waved toward a stranger. “Meet Dietz.”

Dietz nodded briefly. His face was a cipher with a cigar in it. Occasionally Masterson brought a friend, but never the same one. Or not a friend; merely a fellow salesman who liked a good game.

Dietz was backing an ace-king to the limit on the fourth card. George Winkler, high man with sixes showing, folded. Bevis Spencer and Kerry rode along on nothing apparent. Masterson raised. His cards showed no sense to the raise, but he was investing to drive out enough players to increase his chances if anything came along on the last turn. Kerry fled, but Bevis’ clung and Dietz merely called. Then Bevis paired with threes, which proved good enough in view of the fact that George had folded with his sixes.

“You deserve it, Spencer,” Masterson told Bevis graciously. He fingered his chips—two stacks of blues and one of whites. “Here’s your chance to take some of these away from me, Alec.”

“I’m not playing,” I said.

It was Bevis’ deal. He sat at the left of Miriam and placed the pack in front of her to cut. She did not see it; she was looking at me. She had only one blue chip left. It served her right for taking a hand in such fast company. Kerry sat on her right, and then Dietz, George Winkler, Oliver Spencer, Ursula, Masterson. There were two empty seats next to Masterson.

“Come on, Alec,” Oliver Spencer urged. “I haven’t played against you in years.”

He was trying to make peace. He looked at me with a rather anxious smile. I returned it pleasantly to show him that I didn’t bear a grudge because of what he’d said to me in his store that afternoon.

“I’ve got to leave right away,” I said. “I stopped off to borrow some money.”

Ursula rose.

“No, wait,” I told her. “I want to borrow from all of you. I need a lot and in cash.”

Ursula said: “I have a couple of hundred in cash.”

“Not enough. And a check won’t do because the bank is closed tomorrow.

***

They were all very still, watching me. I felt anger rise in me. If anybody else had said that, they would have thought nothing of it. Why was I different?

Kerry rose from his chair and started around the table. “What is it, Alec? What’s come up?”

“My God, do I have to—” My voice was too high. I lowered it. “I’ve got some money in a savings account and some in war bonds. The rest Ursula will make good.”

Kerry had stopped where he was, his hand on the back of George’s chair. It was unusual for a man to want a lot of cash on a Saturday night, but not so unusual that they had to stare at me like that.

“Ursula, you’ll give them your checks for their cash, won’t you?” I persisted. “I’ll pay you part of it back next week, but the rest will take time.” Ursula turned to Mr. Spencer on her left. “Oliver, how much have you got?”

He pulled out a roll that he could barely get two hands around. Evidently he had come here directly from the store. “How much do you want?” he asked me.

“You won’t have enough, but maybe we can get close to it if everybody here cashes Ursula’s checks. I knew Art Masterson always carried a wad. Five thousand dollars.”

It was as if a bucket of cold water had been poured down from the ceiling. Even Masterson and Dietz looked startled.

Ursula was the first to recover. She made her voice as casual as if she held a straight flush and was luring suckers. “If you need that much money, Alec, of course you can have it. What do you want it for?”

“I can’t tell you.”

George Winkler stood up and stepped around Kerry who hadn’t moved. “Let’s go upstairs and have a talk, Alec.”

“There’s nothing to talk about. Don’t any of you trust me?” Sweat trickled under my Basque shirt.

“George is your friend and lawyer,” Ursula said placatingly. “I’d certainly ask his advice if I considered spending that much money.”

With the back of my hand I wiped sweat from my upper lip. Confiding in George, or in any of them, would ruin whatever chance remained of borrowing the cash. From their point of view the deal was fantastic. Schneider couldn’t have a name worth buying for two cents because to them the name could only be my own.

“Ursula, I asked a simple favor,” I said, striving to keep my voice even. “I can’t tell you why I must have that cash tonight, but I need it desperately. Do I get it?”

“If you’ll tell George or me what you want it for.”

I was up against a wall. I opened my mouth. I felt I was going to shout and snapped it shut. I ran up the stairs. Feet came after me. In the hall I turned. George Winkler appeared through the door.

“Alec,” he panted, “if you’re thinking of leaving town—”

***

It should have occurred to me that they would pounce on that idea. I was a psychoneurotic. Instead of getting the rest I needed when I returned home, I’d gone through the terrific strain of a murder trial, and my questioning of them this afternoon had shown them that I was brooding over it. So now I was planning to cut and run. It was their duty to prevent me.

“I’m not leaving home,” I said. “If I do, it will be only for a few days to follow a trail.”

“Trail?”

“To Lily’s murderer, maybe.”

George’s small, sharp eyes probed me. “Five thousand dollars is a lot of money for a few days.”

“I don’t want it for that. Damn it, how can you expect me to be frank and open if you keep thinking—” I choked the rest off. I’d been shouting into his ear.

“Never mind,” I said and went upstairs.

I took a stinging cold shower. That washed off the sweat and loosened the nerves. I sat on the edge of the tub smoking a cigarette and let the air dry me. They were so sympathetic, so knowing, so smart. I grinned suddenly at the tiled wall. All right, I’d show them what smart really was. I dressed in fresh clothes and went down.

In the cardroom I found a studied attempt to forget the scene I’d made twenty minutes ago. Take me for granted, in stride, just another guy. They must have agreed among themselves that that was the proper treatment for somebody who was rocky on his mental base.

I sat down in the empty chair beside Bevis and spread a smile around the table. “I’m sorry I cut up, folks,” I said. “Let’s have a stack of chips.”

Ursula, as always, was the banker. She gave me a stack of blues and a stack of whites and marked the total down on a pad.

They were glad to have me. “Now we’ll see some action,” Masterson gloated, and Ursula, doubtless thought the game would be good for me.

The game ran into doldrums. It happens now and then for no reasons. The cards forgot how to match or did at the wrong time. There was too much private conversation around the table, a sure sign of lagging interest.

“Limit stud poker isn’t poker,” I said in disgust. “How about pepping it up with some table stakes?”

Masterson’s eyes glittered. “Now you’re talking.”

Nobody objected, though Ursula, who didn’t like the games to go too high in her house, stipulated that the stacks be limited to twenty-five dollars. Miriam wisely dropped out, leaving eight players.

***

At the end of a couple of rounds I owed Ursula for seven stacks of chips. Puzzled glances were directed at me.

Bevis asked: “What’s happened to your game while you were away?”

“Play your game,” I said. “I’ll play mine.”

It wasn’t that I was losing. It was the way I lost. I played high and wild, tapping an opponent with only angles to back me up. I took one pot and got caught on all the rest. A housewife would have shown better acumen in a penny-ante game.

Ursula came around to my chair and suggested that maybe I was too tired to play.

“Let me alone,” I said, and saw Dietz on the last turn when he had an obvious cinch on board. I pulled a case seven to make two pair and raked in the chips with a nasty chuckle. “I’d like to be tired like this all the time,” I told Ursula.

“Hell, I heard you were a player,” Dietz growled. “That was nothing but damn fool luck.”

Nobody else commented. The anxious shaded glances were again being directed at me. They were getting the idea. I was still obsessed by five thousand dollars and was making these frantic and impossible attempts to get it.

At midnight Masterson opened strongly with an ace on the first turn. Ursula folded. Oliver Spencer, showing a seven, hesitated and then saw George hung on with a four. Dietz and Kerry blew. Bevis Spencer kicked vigorously with a queen. And I, with an insignificant five showing, tagged along.

It was back to Masterson. He reraised Bevis’ reraise, and Mr. Spencer, no longer hesitant, kicked it once more. George decided that he was out of his depth and turned his nine down. Bevis saw.

It was my turn. There had been enough action to indicate what was what. If Masterson had held aces back to back, he would have sat tight, merely dragging along for the first couple of turns to keep the others in and make a pot for himself. I figured him for a picture against his ace, probably a king. Mr. Spencer very likely held a pair of sevens. He wouldn’t have reraised on less, and yet wasn’t too confident with a low pair against an open ace and an open queen. Bevis’ queen bothered me somewhat. I’d have to feel him out.

I raised. Mutters rose from those at the table who were out of the hand. This looked as if it would build up to something.

“Your backed fives don’t mean a thing, Alec,” Masterson said. “How many chips have you? I’m tapping you.”

I had the lowest stack on the table, seven dollars. Oliver and Bevis Spencer both matched the tap and I shoved in my stack, which ended the round.

***

Kerry was dealing. He slapped a nine down beside my five and scowled. He was obviously rooting for me; so were Miriam and Ursula. They had a notion that a nice pot would be good for my morale. Masterson bought a queen which Bevis needed, Mr. Spencer a jack, Bevis a tray.

Masterson fingered his chips and looked at the bare space behind my cards.

I sat back and smiled. “I’m getting a free ride for the rest of this hand. That’s the trouble with table stakes. Too bad the wraps aren’t off.”

“You mean no limit?” Mr. Spencer said. He turned up a corner of his hole card, as if to reassure himself, and shrugged. “It suits me. What do you say, Masterson?”

Masterson grinned lovingly down at his open ace and queen. “To me no limit means no limit. Let’s use the yellow chips at fifty bucks each.”

“I’m blowing anyway,” Bevis said, “so do what you like.”

Ursula protested, though not vigorously. Poker was in her blood. “But only this hand,” she conceded. “At the end of it you’ll settle up for your yellow chips and we’ll go back to table stakes.”

Masterson, still high man on board, chipped two yellows. Mr. Spencer, like me, had decided not to believe in Masterson’s ace-queen, but all the same he was shaky. He saw without re-raising.

I said, “I’m bluff-proof,” and upped it two hundred dollars.

In the sudden silence I could hear my heart thump, though I wasn’t nervous. I wasn’t sweating. This was poker. All the faces but two were featureless blobs circling the table, and the two that concerned me wore the blankness of long poker experience.

“Wait a minute,” Masterson said. “I expect to pay off at the end of this hand if I lose. Can you Alec?”

I told him that I had fifteen hundred dollars and that if it went higher Ursula would back me. Ursula wet her lower lip and gave a searching look and nodded.

Masterson said: “Good enough. Let’s play for real money. I’ll go two hundred bucks better.”

It was a good play by the high man on board who was certain the other two held small pairs. It should have chased us to cover.

Mr. Spencer pondered. Poker to him was a Saturday night’s relaxation at which he won or lost a top sum of one hundred dollars, usually a lot less. He was scared beneath that deadpan. I didn’t worry him; he couldn’t see me as more than fives against his sevens. But Masterson could be holding another queen out of sight or even have aces backed up from the first. On the other hand, it was just as likely that Masterson, with two cards still to be seen, was speculating on pairing an ace or king or queen.

Mr. Spencer asked Ursula for ten more yellow chips and counted out eight of them—four hundred dollars covering my re-raise and Masterson’s. Then he sat tight, holding his nerves together.

I raised it five hundred. Masterson looked momentarily shocked, which told me beyond doubt that he had only angles. He did no more than see me, and Mr. Spencer, afraid of Masterson and not at all of me, did the same.

Kerry had to be spoken to twice before he roused himself sufficiently to deal. All three of us bought. I caught a second open nine, Masterson a king, Mr. Spencer a second open jack.

***

The silence dissolved. Everybody but the three players started to talk it up to relieve the tension. I sat with a five and two nines, indicating two pair; Masterson, showing an ace, queen, king, indicated a high pair; Mr. Spencer’s seven and two jacks were obviously a pair of sevens and a pair of jacks.

Mr. Spencer, high now with an open pair of jacks, negligently bought more yellow chips and dropped ten of them into the pot. He didn’t expect a return. It was pure show, an expression of triumph.

I asked Ursula for more chips. She said in bewilderment: “Alec, do you know what you’re doing?”

“I’ve played poker before,” I said dryly. “I’ll either win or lose. May I have a thousand dollars worth? I’m kicking.”

The silence returned. Ursula’s hands shook as she pushed the chips to me. I continued them in motion into the pot. Masterson folded as silently as the Arabs folding their tents. His high pair had no meaning.

Mr. Spencer took at least a minute to make up his mind. The way he fumbled with his stacks of blue and white chips, which were practically worthless in this hand, showed how puzzled he was: if I hadn’t had fives backed up originally, why had I ridden along? To speculate. I’d bought a second pair, but so had he, and his two pairs were higher.

Slowly he nodded. “You’re bucking the odds without reason, to get the money you want. You’ve been playing like that all night. You’re not in the right frame of mind for poker. Let’s call this hand off and return all the money.”

He was being gracious. He could only see me taking one chance in twelve to draw another five or nine for a full house. The odds went higher because he had just as good chance to buy. Those weren’t odds a poker player in his right mind would accept if it cost as much as I was investing, except a player who was frantic for money and not in his senses. He—and they—had me all figured out. They were so damn smart.

“There’s no law against you going out,” I said.

He scowled angrily at me. He didn’t like this at all. He had too much money invested to be noble and let me take the pot, and at the same time there would be no triumph in winning from a crackpot. He counted out ten yellow chips. “If you insist on being foolish, I’ll just see you,” he said.

On the final turn Kerry gave me a four and Mr. Spencer a ten. He remained high man on board. He was elaborately kind, checking with a tired paternal smile. I wiped it off his face by chipping a thousand dollars.

“No!” Ursula said. “This is a ridiculous bluff. Oliver, the poor boy—”

“Chip or get out, Mr. Spencer,” I said.

The tired smile returned. I could see the noble gesture in the offing. He’d call the debt off when the hands were shown. It wasn’t ethical to take candy from a baby or a madman. He saw me.

I turned up my hole card. It was a nine.

***

For long seconds they were dazed.

Masterson leaned sideways, past Ursula, to make sure my three nines weren’t spots before his eyes. Mr. Spencer looked sick. He was a rich man, as rich men went in West Amber, but it was a lot of money to lose.

I raked in the chips and started to stack them—yellows and blues and whites, mostly yellow. I heard Dietz say in wonder: “Holy cats! You backed those first raises with only a nine in the hole and a five showing! Of all the dumb luck! ”

“Luck?” Kerry was washing the deck and grinning broadly. “You’ve just seen poker played, mister.”

“Like my missus plays it,” Dietz said sourly. “She goes in on an impossible nothing and sometimes she wins. That doesn’t make it poker.”

I had lost track of the number of chips in the pot, but there should be more than enough. I started to count them.

“You have to understand what was happening,” Kerry told Dietz. “Alec played for five thousand dollars. The small losses didn’t concern him; only the one big killing. Didn’t you notice the way he was playing, looking at one or two cards to see if he was in the right position? Sometimes he speculated wildly to put us off guard, to make us think he hadn’t any brains left. When he caught a nine in the hole and a five showing, he played the open five like a pair. Why not? What did he have to lose except for a few dollars to get a look at the next card? He’d tried it before and it hadn’t worked, but sooner or later it would. He pulled a nine to match his hole card and he was set! Did you notice how he sucked them in to remove the limit? That was part of the master plan. From that time on the odds were on his side. Mr. Spencer bought a second pair, but Alec caught a third nine, and then he really went to town on a cinch. Don’t tell me you’ve seen better poker.”

I was over five thousand in my counting. I heard Mr. Spencer ask: “How many yellows did I buy, Ursula?”

I looked up. He was writing a check.

“I’d like as much as possible of that in cash, if you don’t mind,” I said.

Mr. Spencer patted his dome. “My checks are good.”

“I know, but you have a lot of cash on you and I need it. It won’t make any difference to you.”

“I never pay out large sums in cash.” His hairless head dipped to the checkbook.

I turned to Masterson. “What about you, Art?”

“I’ve got around three hundred bucks, but I need it over the weekend,” Masterson said. “I’ll have to make it a check, too.”

I stood up and looked down at Ursula. She was sitting between Masterson and Mr. Spencer. She averted her eyes.

I felt the shakes coming. I ran my tongue over my lips, but my tongue was just as dry. Then Miriam was on one side of me and Kerry on the other. “Are you sick?” Miriam asked anxiously.

“I feel wonderful,” I said. “What’s better than to be with the people you can turn to when you need them?” There was dull silence behind me as I went up the stairs.



    
    \begin{ChapterStart}
    \vspace{3\nbs}
    \ChapterSubtitle[l]{Chapter ch12}
    \ChapterTitle[l]{ch12}
    \end{ChapterStart}

    \FirstLine{\noindent ### Chapter 12}

    
## Emil Schneider

It was after one o’clock when they departed. I watched them from the window of my room. Oliver and Bevis Spencer got into one car. Art Masterson and Dietz into another, and both cars drove off. Several minutes later Kerry Nugent and George Winkler and Miriam came out together. They stood at the foot of the porch steps. I tried to hear what they said, but all I could get out of the low jumble of words was my name.

There was a light knock on the door. I turned from the window and said, “Come in,” and Ursula entered.

Her right fist was filled with money. She dropped it on my dresser. “There was $7,173 in that pot,” she said.

I went to the dresser and pushed the money apart. There were a couple of fifties, but the rest was in twenties and tens and smaller stuff. “How much of it is this?”

“Every dollar in cash I had and could accumulate from the players—$2,235. Most of it came from Oliver Spencer. I hold checks for the balance due you.” In her hand she held a slip of paper from the pad. “Actually you made $4,131 profit: on that hand. The rest were your own chips which you owed me. Before that you’d lost $123, so there’s still $1,773 coming to you for a total amount of $4,008 in winnings.” She recited those overwhelming sums in the monotone of an accountant giving a report. Her face was unnaturally fleshy, lined and lumpy with an inner weariness.

“What made you change your mind?” I asked. “While I was raking in the pot you whispered to Art Masterson and Mr. Spencer not to pay off any of it in cash.”

“It’s what you said just before they left the cardroom—about people you could turn to when you needed them. That hurt more than you knew. Had I ever failed you, Alec?”

She had failed me in letting reason dominate faith in me. Yet at the same time she had backed me all the way, and objectively I should have been grateful. I was, objectively.

I put my hands on her shoulders. “You’ve always been swell to me.” Her eyes, somber and anxious were only a couple of inches lower than mine. “I’m praying that I’m not making a terrible mistake.”

“By letting me have the cash? Don’t worry. The worst I’ll do with it is throw it away.”

Ursula stepped away from my hands and started toward the door.

“One more favor,” I said. “Can you let me have your check for $2,765? That includes what you still owe me from the game, plus a $992 personal loan which I’ll repay you next week.” She turned with her hand on the knob. “Tonight?”

“I’d like it in ten minutes. And please make it out to cash.”

She stood there debating with herself. “If you wish,” she said listlessly and closed the door behind her.

I heard a car start up in the driveway and then another. George and Kerry were leaving. The downstairs door closed as Miriam came into the house.

***

I made a neat pile of the money and stuck it into the right pocket of my slacks. The wad bulged the material. Emil Schneider had said midnight, but he wasn’t leaving until morning. He would have to be satisfied with close to half of it in cash. It was a lot of money.

If he insisted on the full five thousand, I’d give him Ursula’s check, and he would have to accept my word that it wouldn’t be stopped at the bank Monday morning. I had done all I could. It was now up to him to take it or leave it.

Ursula had the check in her hand when I went down to the living room. Miriam was with her. They looked down at the bulge in my pants pocket. “I’d like to use the car, Ursula,” I said. “I’ll only do a few miles.”

“You’re driving to the station?” Miriam asked huskily.

“Look,” I said testily. “There’s nothing I have to run away from or want to. I’m going to see somebody in town and come right back.”

Ursula studied the check as if she were seeing it for the first time. “The cash in your pocket and this check makes an even five thousand dollars. Why that round sum? What are you going to do with it?”

“Something foolish, maybe.” I took the check from her lax fingers and slid it into my wallet. “Thanks a lot. May I have the car?”

“Anything you want,” Ursula said dully.

***

There was no moon. Low, threatening clouds blotted out the stars. I parked in front of the shed garage and cut the headlights. Immediately I was blacked out except for the lighted windows in Emil Schneider’s house a couple of hundred feet away. I took the flashlight out of the glove compartment and went up the wooden steps. The birch rails were white markers toward the house.

I had reached the flagstone walk when the porch light went on. The door opened. Schneider, still in his undershirt, came out to the head of the porch steps and peered to his left. “Linn?” he said. Then the corner of his eyes must have seen my light, for he started to turn to me. He never completed that turn.

The shot was louder than any cannon I had ever heard. It was the silence that had preceeded it and the unexpected quality of the sound. I felt myself jump, and for a moment I thought I was hit. I didn’t feel anything, but I had seen a ground crew sergeant have his arm blown off by a 37 mm cannon without feeling it until later.

The gun sounded again, not as loud this time because the startling effect was gone. I dropped flat on the field-stone. Nip strafers had conditioned my reflexes. Or, maybe I’d been hit. I wriggled. Nothing hurt.

The silence was back. I raised my head. Schneider was no longer on the porch. And I saw movement out from the left corner of the house—a momentary glimpse of something I knew was human only because it was tall and erect. Something like a long rod extended out from it and then merged with the shadow it made. A rifle.

I leaped up and ran forward, reckless with that mixture of fear and anger which makes heroes of men under fire. I saw Schneider then. Under the bright porch light he was a motionless splotch at the head of the steps. My throat uttered an animal cry that had no words in it. I veered toward the corner of the house.

The shape was gone in the darkness. I heard the familiar swish of brush when somebody crashes through it. The batteries of my flashlight were weak; the spray spread out a few feet and petered off. I ran, sweeping the feeble beam in an arc. I reached the brush and went through it. Then the light touched trees—big stuff, elms and oaks. The woods, I knew, ran down the side of the slope to the edge of Old Mill Road. There was no sound but the clamor of katydids.

Abruptly everything washed out of me but fear. I doused the light and in the darkness stumbled back to the house. When I reached the fringe of light flowing out from the porch and two side windows, I looked over my shoulder. There was no point to the gesture; only a black curtain lay behind me. I ducked low and raced around to the porch and up the steps.

Emil Schneider had not moved. He would never move. There was a hole in his turned-up cheek.

I had seen a lot of dead men, but this was like Lily being dead—intensely personal, as if death were reaching out from the flesh it had conquered and touching me. Blood rushed to my head as I bent over him. I yanked myself erect, but my head continued to whirl. I grabbed at one of the posts at the head of the steps and heard myself whimper.

“Don’t let it get you!” I said aloud. “Keep your head! ”

***

It went away, a little. I didn’t need the post. I pushed myself away from it, and I saw a light coming up the walk.

Panic hit me. I glanced in frenzy and then scooted into the house.

“What the hell, Alec!” a voice called.

Running feet slapped the flagstone. I hesitated in the hall and went back to the door on my toes and looked through the small rectangular door window.

Kerry Nugent had reached the foot of the porch steps. He stared at the thing on the porch and then came up slowly. “God!” he said, and raised his head. He couldn’t have missed seeing me go into the house. “Alec,” he said softly.

He hadn’t a rifle in his hand and there was nowhere in his summer uniform where he could hide one. I came out. Across the dead man we looked at each other.

Kerry had no words. The silence between us was beyond endurance. I said: “Funny you’re always right behind me when somebody is murdered.”

“Another one!” he said.

“What are you doing here?” I demanded.

Kerry came all the way on the porch and carefully skirted the body and stood beside me. “You acted queer,” he said. “About that five thousand bucks. When Ursula collected all the cash there was, it was plain she’d changed her mind and was going to give you the money. What did you want it for?”

“Schneider saw who murdered Lily. He was going to sell me the name.” Kerry gave me a quick, sidelong look. “The way you’ve been acting, we thought you might be planning to run away. There was no need; you’d been acquitted. Or maybe you’d do something else with the money that—that—”

“Say it. That a mental case like me might do. I had to be protected from myself—the way you’d all done it last time.”

Kerry didn’t bother to argue that. He said: “We decided that I park down the road and follow you if you came out.”

“Miriam again?”

“Miriam and George Winkler and I. We talked it over outside the house just before I left. So I trailed your car and saw you turn up Schneider’s driveway.”

***

I looked down toward the parking circle in front of the shed garage. It was too dark to see my car. “You didn’t follow me all the way,” I said. “I didn’t hear your car. You knew this was a dead-end driveway, so you parked on Ivy Lane and walked the rest of the way. You came to spy on me.”

His square jaw jutted “Sure I did. Why not? If you’re going to go around killing—”

“Damn it!” I screamed. “I didn’t! Now you’re condemning me without even hearing my side of it.”

“All right, I’m listening.”

I put a cigarette between my lips, but my shaking hands wouldn’t let me give myself a light. Kerry snapped on his lighter and held it for me. His hand was as nerveless as a rock. Between puffs I told him.

He listened with stolid patience. When I finished, he went to the dead man and knelt beside him. “One slug entered his side, below the left armpit. The other entered his cheek and went all the way through. I don’t know how he managed to scream before he died.”

“Scream?”

“I heard two shots while I was coming down the driveway and then a scream.”

“I screamed when I started to chase the murderer. It gushed out of me by itself.”

Kerry straightened up and stared at me. “There were only two shots, Alec?”

“Those were enough to kill him.”

“But they’re both in Schneider.” Kerry looked down the path and then turned to the left side of the house. “Why would anybody risk shooting Schneider just as you were coming up the path? And why didn’t he shoot you when you were chasing him? You were in the light of the porch for at least a few seconds and you said you chased him with your flash on. Why didn’t he take a shot at you then?”

There was an answer. There had to be, but I couldn’t straighten one out in my head. There was a sick, throbbing tiredness in back of my skull which wouldn’t let me think.

“How should I know why he acted like that?” My voice was high, strident. “I’ve got some question of my own. How the hell do you happen to show up every time somebody is murdered?”

“I told you.”

“And I don’t like it.”

He turned his head listening. It was only the sound of a distant car, probably on Old Mill Road.

Kerry said sharply, “Let’s get out of here!” and put his hard hand on my arm.

That was something I could agree with. Side by side we went down the walk and down the wooden steps. He held onto my arm. Like a policeman with his prisoner, I thought.

When we reached my car, I said: “Are you going to take me to the police in person or are you going to call them on the phone?”

“Don’t be a dope. I’ll help you go wherever you want.”

Savagely I wrenched open the car door. “I’m going home and I can get there without help.”

He was walking back to his car parked on Ivy Lane when I passed him on the driveway. He stepped into high grass to let my car go by. Halfway home I saw his headlights behind me. His car remained glued to my taillight the rest of the way home.



    
    \begin{ChapterStart}
    \vspace{3\nbs}
    \ChapterSubtitle[l]{Chapter ch13}
    \ChapterTitle[l]{ch13}
    \end{ChapterStart}

    \FirstLine{\noindent ### Chapter 13}

    
## Flight

Miriam and Ursula came out on the porch when they heard our cars. I remained seated behind the wheel without will to move. Kerry got out of his car and walked stiff-legged to my window, like a traffic cop about to hand out a ticket. He opened the door. “Come on,” he said.

I nodded sluggishly and cut the ignition and lights. Kerry helped me out as if I were sick. When my feet were on the ground, I shook him off and strode ahead. He caught up to me and walked at my side up the porch steps.

“What is it?” Ursula asked tightly.

“Emil Schneider was shot dead,” Kerry said.

Miriam made a thin moaning sound and crossed her arms over her breasts. Ursula stared with her mouth open rather foolishly.

“That was a delicately objective statement of fact, Kerry,” I said, and leaned against the house and laughed in a way that hurt my throat.

Ursula moved to my side. “Alec, for heaven’s sake!”

My laughter crumpled into ragged bits, which I coughed away. I said savagely: “‘Emil Schneider was shot dead.’ Not: ‘Somebody unknown shot Schneider.’ Not even: ‘Alec put a couple of slugs in him.’ That’s assumed. Somebody is murdered in West Amber, and who else would do it?”

“Did you?” Ursula whispered.

“Why bother asking?” I said. “You knew the answer right away.”

I pushed myself away from the wall and went into the house, slamming the door behind me. The handset phone was on the table in the hall. I picked it up.

The door flew open. Kerry came in fast. He wrapped his fingers around my right wrist and pushed the handset away from my mouth. “What do you think you’re going to do?” he demanded.

“Call the police. I’ve nothing to hide.”

“What’s the rush?” he said.

I shoved my free hand against his chest and strove to pull the mouthpiece up to my mouth. He was too strong. I twisted my body to him and jabbed my knee up. He grunted hollowly. His fingers loosened, but only slightly.

Abruptly the fight went out of me, though Kerry was the one who was hurt. The handset fell to the floor. Kerry released my wrist and crouched and put both hands down where my knee had struck him. His face was very white.

“Are you hurt bad?” I asked him contritely.

“Not much.” He became aware of Ursula and Miriam in the hall and raised his hands up to the pit of his stomach.

The phone was cackling on the floor. Ursula scooped it up, said, “Never mind, operator,” and hung up. Then she looked at me. “My God, Alec!” she said.

***

Miriam stood in the open doorway.

That pathetic sickness in her eyes was harder to take than horror would have been. I turned and went as far as the stairs and dropped down on the second one. I ran sweaty palms over my face.

“What do you want from me?” I said, and my voice was sobbing. “Are you trying to protect me the way you did the last time so everybody will believe I’m a murderer? I never harmed anybody in my life except indirectly in war.”

Kerry straightened up. Beads of perspiration stood out on his forehead, but he was taking the pain like the tight-lipped he-man he was. “I think,” he said slowly, “that we ought to call George Winkler at once.”

“Damn it, no lawyers!” I said. “They can burn me in the electric chair, but no lawyer is going to give me the works this time. I didn’t kill Schneider. Why would I? I needed him alive to buy a name from him. The name of the person who killed Lily.”

Miriam moved forward from the doorway. She sat down beside me on the stairs and took my hot hand between her cold palms. “Alec, what happened?”

I bowed my head and creased my brow as if in that way I could push the sluggishness out of my brain.

“I don’t know what happened,” I said. “I tried to think while driving home, but straight thoughts refuse to come. I must be too shocked, too tired. The things I saw and know are wrong, the way Lily’s murder was wrong. It’s like a mathematical equation which always gives the wrong answer no matter which way you work it. Always the same answer, with my name standing for X, and I’m the only one who knows that it can’t be right. But what’s wrong about it I can’t tell.”

I doubt if any of them followed me, but they listened patiently.

I looked up. “Here is what is true and definite. Schneider saw who murdered Lily. He visited her that night after he spoke to her on the phone. When he got there he saw somebody sneak out the back door, and he found Lily with the knife in her heart. He beat it to save his own neck and kept quiet about what he had seen. I doped out that that’s what must have happened. He admitted to me that he had seen the murderer’s face and would let me have the name for five thousand dollars.”

Ursula said: “Why didn’t you tell me?”

“Would you have let me have the money if I had? Would you have believed my story when you were convinced that the murderer’s name was my own? And if you did believe me, what would I be buying? Schneider refused to go to the district attorney with me. He might be the murderer himself or he might pull a name out of the air. It looked like a gold brick.”

***

Kerry was leaning against the wall, still in pain and not showing much of it. “That part sounds as bad as the rest,” he said. “Not to me; I just don’t know. I’m thinking what the police will think. Five thousand bucks for a name you couldn’t know was the right one and with which you couldn’t do anything if you had it.”

“Of course it was a gamble. A lead, maybe, a straw to clutch at—no more. A sucker’s buy, but not for me and not for the murderer.”

Miriam’s hands lightened over mine. “Don’t you see that Alec had to do it?” she told the others earnestly. “It means so much to him.”

“Thanks, Miriam.” I gave her a weak smile. “It was worth the money to me. I know now I would have received value from Schneider. He wasn’t the murderer, because the murderer murdered him. And he had the right name. That’s why he was shot. There’s a lot I can’t get straight in my head, but this much I can. Schneider needed money desperately. He’d lost his job at the bank; his wife was divorcing him; she owned the house; he was broke. If he’d sunk so low that he would take that kind of money from me, why not capitalize on what he’d seen by blackmailing the murderer? But’s that’s a dangerous business. A person who had murdered once will murder again to protect himself. Schneider was scared. He’d have been sure of collecting that five thousand dollars from me if he’d hung around town till the bank opened Monday morning, but nothing could stop him from taking the first train out tomorrow morning. Nothing but a couple of bullets.”

“And he was shot down before your eyes,” Kerry said.

I took my hand from between Miriam’s palms and set fire to a cigarette. I did not look at anybody.

“He was shot while you were with him?” Ursula asked thinly.

“It was about fifty feet away, on the walk to his house. He put on the porch light and looked to his left and said my name. Probably he heard somebody. He was expecting me. I chased the killer, but the night’s black and all I glimpsed was a shadow at the edge of the window light. It could have been either a man or woman.” I dragged at my cigarette. “It doesn’t make sense in several ways. Kerry knows why and the rest of you are figuring it out. There’s only one answer, but it’s the wrong one. I’m the only one who knows it’s wrong because I’m the only one who knows I didn’t kill him.”

Miriam reached for my hand. I wouldn’t let her take it. I stood up.

“I had no right to come back here,” I said. “If the police aren’t called, all you will be in this with me. Kerry, let me use the phone.”

He left the support of the wall and hooked his thumbs in his belt. You couldn’t tell by his face whether he was still in pain. “We four are the only ones who know you were at Schneider’s house tonight,” he said.

Ursula pounced on that. “Of course! There’s no hurry. It’s late and we’re tired. Suppose we all get a good night’s sleep and then we’ll see.”

***

I nodded slowly. I’d be helpless in jail, cut off from any opportunity to prove my innocence. Once in the hands of the police, I’d be bully and badgered by them and by my own lawyers, too. I’d gone through it once; it wasn’t anything to be faced again. I needed time to rest and to think in free air.

“Suits me,” I said and turned. Miriam shifted on the: step to give me room to pass her. I remembered something. “Did you and George send Kerry to follow me?”

“They didn’t send me,” Kerry said angrily. “The three of us decided it.”

“Did you?” I asked Miriam.

Her dark grave eyes studied me, not quite understanding. “Yes,” she said.

I went up the stairs. Somebody came behind me. Kerry entered my room a moment after I did. I sat down on the bed and looked at him. “What now?” I asked.

“Nothing but sleep,” he said. “Let me help you with your shoes.”

I told him I wasn’t an invalid and undressed myself. He got my pajamas out of the drawer and pulled down the bed covers. The close pal, the fellow officer, putting a drunken or ill or mentally deranged comrade to bed. I crawled in and said: “Aren’t you going to kiss me good-night?”

He grinned. “That’s the way I like to hear you talk.”

“How’s the groin?” I asked.

“Nasty for a minute or two. It’s all right now.”

“It was a dirty trick.”

“Forget it.”

He went to the door and put a finger on the light-button and turned. He looked at the wall, at my rifle hanging there.

I said: “Anyway, you know I didn’t kill him with that.”

He put out the light and closed the door behind him.

I lay in darkness and listened to him go down the stairs. There were low voices in the hall. After a while the voices went into the living room which was directly below my room. I got out of bed and put my ear against the floor, but distinct words didn’t come through. Without putting on a light, I groped my way to my shirt on a chair and fumbled out cigarettes and matches. I sat on the edge of the bed smoking, listening for Kerry’s car to depart. It didn’t.

***

I was on my fourth cigarette when I heard a car in the driveway. It was arriving, not leaving. I went to the window. George Winkler slammed the car door and ran up the porch steps. He was too big and clumsy to run. This was important. This was murder.

I tossed my cigarette out of the window and crossed the room in darkness and went out to the head of the stairs. All four were in the living room, keeping their voices low in order not to wake me. They thought they had me safely and snugly out of the way while they consulted a friend and lawyer about me. They didn’t need me there to hear any more of my side. They’d heard enough, and it was like the last time, an outrage to common sense. But I was a brother, a foster-cousin, a pal, so they had to do what they could for me.

I started down the stairs, my bare feet coming down hard, but making no sound. I was shaking worse than at any time I could remember. I gripped the bannister with both hands and clung. I mustn’t give into it. Bursting in on them in this condition would not help me.

“Are you sure nobody saw you or Alec?” George Winkler said in the living room.

I could hear them from here. The arched, doorless living room entrance was only five feet from the foot of the stairs. I leaned over the bannister, listening.

“It was after one-thirty,” Kerry was saying, “and dark as the inside of a pocket. There wasn’t a light in any house, and if anybody saw either of us they saw only a car pass or a shape walking behind a flashlight. Unless they happened to be looking at Schneider’s porch, but the house can’t be seen from the road or from any nearby house. It’s on a ledge at the end of five hundred feet of driveway and a couple of hundred feet past that.”

“But the shots,” George said.

“Two loud noises,” Kerry said.

“Maybe a truck backfiring on Old Mill Road, for what anybody who heard them knew. I wasn’t sure they were shots when I heard them, even though I was half-expecting something to happen. Besides, nobody came to the house or called out or showed a light. The body mightn’t be found for a week if we keep our mouths shut.”

I heard restless feet pace the floor. Then George said: “My God, do you folks realize what you’re doing? We’re accomplices after the fact. In the eyes of the law we’re as guilty as Alec, Kerry especially. He saw Alec and the murdered man and didn’t report it.”

“I’ll do more than that for Alec,” Kerry said.

“There’s nothing to point to Alec,” Ursula said. “All we have to do is say nothing.”

George snorted. “Do you think the police are dummies? As soon as they find Schneider with two bullets in him they’ll pile on Alec. He murdered his wife because she was unfaithful to him. He was acquitted by the jury not because there was any doubt of his guilt, but because of extenuating circumstances. Then a couple of days after he’s released the man with whom she had betrayed him in particular is found murdered. It’s a pattern you can’t get away from: his wife and her chief lover murdered. You don’t know how thoroughly the police work. They’ll find out about the poker game tonight. They’ll learn that Alec was desperate for five thousand dollars in cash and that he got a good part of it. They’ll established the time of death as not long afterward. They’ll conclude that Alec coldly planned the murder and needed getaway money.”

“No!” Miriam cried. “I can’t let you say that about Alec. He didn’t murder either of them.”

“How do you know?” George demanded.

“He said so.”

***

George delivered himself of another snort. “It’s your privilege to believe that. He almost had me believing that he was innocent of Lily’s murder, but now with Schneider shot—“

“Alec didn’t try to run away,” Ursula said. “He came home and wanted to call the police.”

“Because Kerry saw him,” George said. “The fight went out of him; he was ready to give up. You can never tell what anybody so emotionally unstable will do.”

“Hold it,” Kerry said. “You’ve got the arguments on your side, but—”

“My side!” George said. “I’m giving the police’s side. I’ll do all I can for him, but I can’t see where it will be enough.”

“We know you will,” Kerry said. “That’s why we phoned you to come right over. But Alec said he an idea Schneider was blackmailing somebody and the blackmailer killed him.”

“And you believed him?” George said.

There was a brief silence. Then Ursula said: “If the fight went out of Alec, why didn’t he confess?”

“How do I know?” George said. “You people are groping for anything at all to help you believe him. So am I. But look at the terrific coincidences. This afternoon Alec and I discussed laws of chance. He tried to show me mathematically that walking into Lily’s bungalow a few minutes after she was murdered and under circumstances that pointed directly to him as the murderer wasn’t such a great coincidence after all. I might concede that, though with reservations. But when it happens a second time—when Alec walks up to Schneider’s house at the precise moment when somebody is aiming a gun at Schneider, a man Alec has reason to hate, that becomes incredible.”

He paused. I could hear feet nervously prowling the floor—George’s feet, probably—and I heard nothing else. They hadn’t any answer for him. Neither had I.

“You can’t get around coincidences like that,” George went on. “Not one coincidence, and certainly not two. And look at the rest of it. The gunman hears and sees Alec’s car pull up and sees Alec come up to the house behind a flashlight. Does the gunman postpone the shooting? Does he run to cover? Does he take a shot at Alec first to protect himself? Not this gunman. He blandly goes ahead and kills Schneider as if he had all the world to himself. Then he runs. Alec, who’s supposed to have sense, chases him unarmed and with his flashlight on. Does the gunman turn around and shoot Alec who’s a clear target and perhaps has recognized him? Not this gunman. He’s every bit as thoughtless for his own safety as Alec is. And then Alec loses him and hurries back to the porch in time for Kerry to find him standing over the body.”

“The gun,” Miriam said. “Why didn’t Alec have a gun when Kerry saw him?”

“That’s right,” Kerry agreed. “His rifle is up in his room. I saw it when I went up there with him.”

“How do you know a rifle killed Schneider?”

***

“I heard the shots,” Kerry said. “I can tell the difference between a rifle and a small gun. A rifle booms, a handgun barks. And I’d say the rifle was bigger than Alec’s .22.”

“That doesn’t get us anywhere,” George said. “He wouldn’t use his own gun, especially not if he had planned this in advance. This is hunting country. Practically every house has a rifle of some sort. He could have stolen one or maybe driven to Trevan or somewhere this afternoon and bought one. Then, after shooting Schneider, he tossed it away in the brush. That’s another problem we have to face. It’s probably near there with his fingerprints all over it.”

“Probably” Miriam exclaimed in outrage.

“Miriam, you’ll wake Alec,” Ursula warned her.

“I don’t care. Why shouldn’t he be down here to defend himself? You’re fools—you especially, George. You’re supposed to be a lawyer. Can’t you get it into your head that if Alec had deliberately set out to murder somebody he would never concoct such ridiculous stories? He’s so clear-headed, so intelligent. He’d have a perfectly logical story to tell. The fact that what he told us doesn’t seem to make sense proves that he’s innocent.”

Leaning over the bannister in my pajamas, I smiled to the wall. Smart girl, Miriam. She’d got to the heart of it.

“I’d agree,” George said so softly that I had difficulty hearing his words, “if these weren’t crimes of passion which would make any man lose his head. I’d even agree if he were normal. We know he isn’t. Look at the way he’s acted since he was acquitted. Brooding, beating his brains out, speaking of nothing but Lily’s death. Isn’t it obvious that it was jealousy of Schneider that was gnawing at him?”

I swung from the bannister to go down to them. I forced myself to sit down and clasped my knees hard. My heart ticked off the long seconds of quiet in the living room. Even George’s pacing had ceased.

Miriam spoke, and her voice was low now and broken. “I don’t believe it. I can’t.”

“How are you going to get around facts and incredible coincidences?” George said. “But say we’re convinced he’s innocent—what do we do? I see only two alternatives, and neither of them is pleasant. We can sit tight and see what develops. Plenty will. The police will question Alec. Will he deny that he was near Schneider’s house tonight? Not in his state of mind. He’ll confess or more likely he’ll doggedly stick to the story he told you, the way he did to his equally absurd story of Lily’s murder. We four will be in hot water for having held out on the police. I’ll be lucky to get away with being disbarred. The other alternative is to call the police now. In either event, Alec will stand trial.”

“What will his chances be?” Kerry asked.

“Bad,” George said. “A clever lawyer like Magee can beat one murder, but two will have him licked. Alec’s been acquitted for Lily’s murder, so that can’t be brought into the trial, but the D.A. will find ways to link it up. Besides, it will be impossible to get a jury that isn’t aware that Alec had murdered his wife and that the man for whose murder he was being tried had been her lover. Frankly, I think the only thing we can hope for is to have him found insane and sent to a mental institution.”

Ursula said briskly: “You overlooked a third alternative. Alec can go to South America.”

***

I stood up and descended three steps and stopped. The staircase wobbled; the ceiling was coming down on me. My throat was raw with a scream that I would not let past my lips.

“Do you realize what you suggest?”

George said. “He’s murdered two people. I’m not saying he’s to blame. Let’s say the war twisted something in his brain. Murder has become a habit to express his emotions.”

That couldn’t be Ursula’s voice I heard, but it was. The deepness had gone out of it. She whimpered: “My God, George, you’re wrong!”

“Maybe I am,” George said. “I’d do anything for you, Ursula, but I can’t take this chance. I’m going home now. Do whatever you think best and I promise to keep my mouth shut. I’m sorry, but that’s as far as I can go.”

I raced up the stairs, running away, but in my pajamas I could go only as far as my room. When I had the door closed between me and them, I stood against it, breathing hard.

George’s car left, but Kerry didn’t. Under the floor I hear their voices still muttering. They were deciding my fate, sheltering me and protecting me, in their love offering me what was worse than jail or a mental institution or the electric chair, to be an eternal fugitive in a foreign land.

I put on the light and smoked a cigarette down to my fingers. There was a fourth alternative which had occurred to none of them. I put on my gray tweed suit and stuffed the $2,235 in cash I had won at poker into a pocket. My two windows opened out on the flat porch roof. I climbed out and slid down the post and landed among the hollyhocks.

An open living room window was almost within reach. I heard Kerry say miserably: “I’m willing to bet we can’t make him go away, but it’s worth a try. I can’t see anything else.”

A woman was sobbing. I looked through the window. Miriam sat on the couch with a handkerchief to her face and her shoulders shaking. Ursula wept also, but on her feet and silently. I had never thought that I would see Ursula weep.

I walked into the darkness.



    
    \begin{ChapterStart}
    \vspace{3\nbs}
    \ChapterSubtitle[l]{Chapter ch14}
    \ChapterTitle[l]{ch14}
    \end{ChapterStart}

    \FirstLine{\noindent ### Chapter 14}

    
## Shadows and Specters

Mango struck the roof and woke me. I lay listening for it to start rolling. For a long time now I had resented that huge mango tree above the barracks, as if its bombardment of the roof at night were the ultimate indignity of the conspiracy to deprive us of sleep. What with jackals howling and rats running through the thatched roof and sometimes snakes slithering on the beams and the eternal bugs and the outrageous wet summer heat, there was enough to keep you awake without a mango going through its routine of dropping to the roof with a bang, rolling down the incline with agonizing slowness, and then the dull plop when it finally hit the ground.

The routine paused unendurably after the first stage. My breath waited for the mango to start rolling. Light lay against my eyelids. It was day, and outside some of the men were calling to each other in shrill, oddly immature voices, and nearby an engine was warming. But what about the mango? It should have rolled to hell and back by now.

My arm flung out over the edge of the cot, and there was no mosquito net to impede it. I pried my eyes open and saw a flat plaster ceiling and came fully awake.

This wasn’t India. Mosquito nets were not considered essential on West Eighty-second Street; in New York City. Under my window a car motor sputtered and coughed and suddenly caught and roared. The shrill voices of boys playing on the sidewalk moved off. Again I heard the thump overhead. A second shoe, probably, dropped by a night-shift worker going to bed.

I dug my watch out from under the crumpled pillow and saw that it was past noon. Nine hours of sleep, but I felt unrested. The process of getting out of bed and dressing seemed complicated and unnecessary. This was Friday, the beginning of my fourth day in New York, and seven million people remained between myself and Don Yard. I had got no closer to him, and perhaps during this time the police had got closer to me.

They were after me. That was definite now. There had been a single paragraph deep inside yesterday’s World-Telegram. It had taken until Thursday for the news to reach New York, or more likely for the police to be ready to name names. All that those few lines had said was that Alexander Linn, former Air Force lieutenant, who last week had been tried and acquitted for the murder of his wife, was being sought in connection with the murder of Emil Schneider in West Amber.

So the police were looking for me and I was looking for a gambler named Don Yard and a woman named Bertha Kaleman and a man named Walter something. A merry-go-round on which the horses we sat on moved up and down and not at all forward. So far.

***

I climbed out of bed and shed my pajamas on the way to the shower. This was a deluxe apartment, with a private bathroom and a tiny alcove containing a midget refrigerator, a chipped porcelain sink and a two-burner gas range. I could afford it. I’d reached New York with better than two thousand dollars in cash, and in two sessions at downtown public poker clubs I’d increased it by two hundred dollars.

The shower was stinging cold. I came out of the bathroom dripping and opened the package of shirts and socks and underwear I’d bought last night. Enough to keep me supplied for a considerable stay. If the police would let me. If I didn’t happen to run into anybody who knew me.

When I was dressed, I stuck a cigarette between my lips and went downstairs a lot more jauntily than I felt. Mrs. Egan, the oversized landlady, was poring over the mail on the long mahogany table in the hall. She glanced sideways at me descending the stairs and returned her attention to the letters.

“Are you expecting mail, Mr. Berkowitz?” she asked.

Nobody was behind me or in the hall. I kept coming down. Mrs. Egan gave me a sharp look. “Doesn’t anybody write you, Mr. Berkowitz?”

She meant me. I was Berkowitz—Jeffery Berkowitz. I’d had a score of names since Sunday morning and I’d forgotten most of them. Tuesday I’d used another one to register at this rooming house; since then I hadn’t seen Mrs. Egan or been anywhere in New York where it had been necessary to give a name or be called by one, so it had eased out of my consciousness. This was a name I must not forget. This one must stick. When people said Berkowitz or Jeffery or Jeff, I had to respond in a split-second, the way I did to Linn or Alexander or Alec. Only I mustn’t respond at all to the names I’d had since birth. They were no longer mine. They were dangerous.

I clamped a grin on my face. “I sent my folks this address only a couple of days ago, and they’re all the way out in Utah.”

Mrs. Egan’s vast bosom heaved.

“The way this war separates families! I haven’t heard from my boy Peter in three weeks. I think I asked you when you first came. Didn’t you run across my Peter? He’s in the Pacific somewhere—a sergeant in the Air Force. In the ground crew. Peter Egan. He’s only twenty, but he looks—”

“No, ma’am,” I slipped in. “I was an Army private in the CBI theatre.”

“Isn’t that in the Pacific? My Peter is somewhere—”

“No, ma’am. CBI means China-Burma-India. That’s thousands of miles from where your son would be. And I was in the infantry while he’s in the Air Force.” I moved past her.

“Mr. Berkowitz,” she said, “you told me you were looking for a job. Mr. Marcus in 3-C works in a radar place in Long Island City. He says they’re looking for men to train and pay good wages while they learn.”

My hand turned the knob without opening the door. “I’m seeing a man in an advertising agency today. That’s the line I’d like to get into.” I pulled the door toward me.

“I hope you get it,” Mrs. Egan called after me.

“Thanks.”

***

It had turned cool in the first week of September. A pleasant breeze swirled from Central Park. It was a nice street, brownstone rooming houses on this side, swank apartment buildings on the other, and even the boys playing box-ball a few doors away were somewhat subdued about it, for boys. I started down the stoop and came to an abrupt halt and twisted halfway around. Almost I fled back into the house. A patrolman was coming down the street, peering at the numbers of the brownstone houses he passed.

He was coming for me. Somehow they had traced me.

I placed unsteady hands behind my back and stood poised between freedom and doom, my eyes only capable of motion as they shifted in their sockets to follow the inexorable approach of the cop. Somewhere along the line I’d made a mistake. Somewhere along that cautious, roundabout journey I had taken between three A.M. Sunday and 10 A.M. Tuesday the widely scattered links had come together.

From West Amber I had walked until daylight. The driver of a twelve-wheel trailer gave me a lift. He was burning up the road to make Buffalo by evening. I was bound there myself, but I left him at Trevan and took a lift southwest with a man in a Tennessee coupe. That day I took twenty-three short-hop lifts in all, weaving toward Rochester. Five miles outside the city I spent the night in a tourist cabin. Next day, Monday, six lifts brought me to Buffalo by noon. I hung around Buffalo until evening and took a sleeper to New York. Some time during the night the train passed through West Amber. In those two days I had not used the same name twice.

The cop was on the sidewalk directly below me. He looked up at me and then went on toward Columbus Avenue, still studying house numbers for whatever reason he had for doing it.

My insides fell back into place. Soundless laughter quivered in my throat. The links remained scattered. I had time, but time was all I had.

I walked to Central Park West and turned south. Today was Friday and the Hale County Weekly Star came out on Wednesday. It would have reached New York by now, if any copies for sale reached New York. There were newsstands on Times Square where I might get it.

“Shine ’em up, mister?”

He was a bright-looking black boy of ten or twelve. I leaned against a building bounding Columbus Circle and put a foot on his box.

***

If I were a cop and had a notion that Alec Linn had come to New York City, would it occur to me to have newsstands which sold out-of-town papers watched on the chance that Alec Linn would want to buy his local paper. The metropolitan papers had carried nothing about the murder except for those very few lines in the World-Telegram, but the Star would be full of it. Where else could Alec Linn, if he were in New York, get the information?

The polishing cloth gave a final flip to the toe of my second shoe. I handed the boy a quarter and waved the change aside. “How’d you like to earn a dollar” I asked.

The bright black eyes regarded me cautiously. “Doing what?”

“Do you know where there’s a newsstand which sells out-of-town papers?”

“Huh?”

“Newspapers from other cities.”

His gaze was openly suspicious. “Buy a paper for you for a buck, mister?”

“Look,” I said patiently. “I’m sure you know where Times Square is.”

“You bet. Forty-second Street.”

I wrote on a slip of paper: “An out- of-town newsstand—the Hale County Star.” Then I said: “Show this paper to somebody on Times Square. Maybe the first few people won’t know, but keep showing it till somebody tells you to go to a certain newsstand. If he has the newspaper, he’ll sell it to you.” I indicated a restaurant two doors away. “I’ll be in there waiting for you.”

He kept his doubting eyes fixed on me. “Suppose you ain’t there when I get back? I’ll be out the price of the paper and the carfare.”

“You’ll get ahead in life,” I told him. “I’ll give you a dollar now and a dollar when you come back.”

He snatched the dollar from my hand and scurried to the subway kiosk.

I’d eaten in that restaurant twice before. The plump waitress gave me a dimpled nod. I carried an Old Fashioned from the bar to the table and waited for my order and the return of the shoeshine boy. The afternoon was ahead of me and the night and days and nights after that. Yesterday and the day before I had tried to get to Don Yard through the only thing we had in common besides the fact that we had both been married to Lily, and that was poker. I’d played at two different poker clubs and had tried to make conversation about gamblers and slip Don Yard’s name in. No soap. At the first club only a couple of men had heard of him. He wasn’t, it appeared, a terrific big shot. At the second club, a swankier place, I’d been cut short twice by the house dealer because I’d held up the game with my talking.

I’d got a response from only one man—a tight-smiling, tight-playing man named Locust. “I was present last year when Yard dropped thirty grand on the cut of a card,” Locust had said. “Of course that doesn’t compare with what Arnold Rothstein used to bet on a card or win or lose in one night, but Yard will never be a Rothstein.”

“Do you know Yard well?” I asked.

Locust had shrugged delicately and had turned to the dealer who was growling at us. And later Locust had slipped away before I’d had a chance to speak to him in private.

My soup arrived. I picked up the spoon and put it down. There was a phone booth at the rear of the restaurant and a pile of directories on a shelf. Tuesday I had looked up Don or Donald or D. Yard in the Manhattan book and had drawn a blank. But Manhattan wasn’t all of New York. Most of the population lived in Brooklyn or the Bronx or Queens or Staten Island. Why not a gambler?

I went to the booth and looked in the other city directories. More blanks. He might live in the suburbs or at a hotel or have an unlisted number. I returned to my soup.

And if I found his number, what would I do about it? I could hardly ring him up on the phone or call on him in person and say: Mr. Yard, I am Alec Linn. You have no doubt heard of me in connection with a recent unpleasantness up at West Amber concerning Lily, and news may already have reached you of an even more recent unpleasantness concerning one of her lovers. We should get together. We have a great deal in common. We are the two men who made legitimate love to Lily. I need your help. Possibly you or Bertha Kalenan or a man named Walter or somebody else in your circle is involved in the murders of Lily and Schneider. If so, I’d like to know, please. And if not, do you or any of the others have knowledge or information which will help me find the murderer? I will greatly appreciate any cooperation you can give me in this matter.

And Don Yard would throw me out or turn me over to the police or kill me because to him I was the murderer myself, in person, of the woman he had loved.

So there I was, a fugitive in a restaurant in Columbus Circle, with the police of the nation looking for me to burn me in an electric chair or shut me away in a nuthouse, and no weapons to fight back with except what I had inside my head. Well, had I had more in India when we’d completed a tough bombing run and gas was running low and ten men were depending on my knowledge of navigation to take them home? Even that last time, when we’d got lost for a while and then had to step out on a cloud, I’d brought them to where we could reach ground with the loss of only one man and a couple of others banged up. Ezra Bilkin was killed, one out of eleven, when we might all have gone that way. I’d kicked myself around long enough. I’d done a good job bringing them back at all. I’d got a medal for it, hadn’t I? They don’t hand out medals for mistakes or failures unless you’re a lot higher up in the ladder than a first lieutenant.

I’d done it once. I’d brought ten out of eleven men home against terrific odds. I could bring myself home now.

***

I was up to my coffee and the shoe-shine boy hadn’t returned. It shouldn’t have taken him more than five minutes to get down to Times Square, ten to locate the newsstand and buy the paper, five minutes to return. Call it thirty minutes in all, and he was already gone forty-five.

Maybe they didn’t; have the Star and he decided to keep the dollar. Maybe—

Panic hit me in the pit of the stomach. It twisted me in the chair, toward the door. So I was smart! I was a brainy guy who’d sent a strange kid instead of going myself. But if the police were watching newsstands which sold out-of-town papers, they’d check on everybody who bought the Star. They’d compel the kid to take them to me.

I beckoned to the waitress for the check. She was busy at another table. She displayed her dimples for me, but she did not come over.

And then he was coming in, holding the shoeshine box in one hand and the paper under his other arm. He wasn’t followed in. But they might be outside, waiting for him to point me out to them.

He saw me at once. “Here you are, mister,” he said and spread the paper out in front of me on the table.

I almost jumped out of my skin. My face looked up at me from the paper. It was a two-column photo beside one of Emil Schneider. I was in uniform, wearing a careless smile and an overseas cap at a rakish angle. To my knowledge the same photo had appeared three times in the Star—before I’d been sent overseas, after I’d been indicted for the murder of Lily, and now.

“That it?” the boy asked anxiously.

“Sure,” I said. “Fine.” I folded the newspaper over my photo and fumbled my wallet out.

The boy’s bright black eyes were glued to the wallet. He hadn’t connected me up with the photo. I had been hundreds of years younger when it was taken and I had been in uniform and the Star’s halftones were always blurred.

“Boy!” he said. “Five bucks!”

I’d taken out a five by mistake. I shoved it into his hand. “I just came into a lot of money,” I explained weakly. “You’re a fine, intelligent boy and I want you to share a little of it.”

“Gee, thanks, mister!” he said ecstatically and raced out of the restaurant.

As soon as he was gone, I regretted having let him keep the five. Six dollars was too much for a nickel newspaper. He would talk about it and remember the name of the paper. I got out of there as quickly as I could.

***

I walked downtown on Ninth Avenue where there was less chance of meeting anybody who knew me. When I reached the middle fifties I stopped beside a cigar store and read the paper.

George Winkler had called the turn. In fact, the police had even more than he had anticipated.

At dawn Monday morning Emil Schneider had been found on his porch by the milkman who made deliveries every other day. Saturday afternoon Schneider had bought a railroad ticket to Chicago and had told the station agent that he planned to leave the following morning. Evidently it had been a sudden decision, for he hadn’t had a chance to tell the milkman to stop deliveries and give him a bill. The medical examiner had established the time of death as sometime Saturday night or Sunday morning.

Two .257 caliber bullets fired from a medium power rifle had entered his body.

There were witnesses—not to the murder itself, but they may as well have been as far as the district attorney was concerned.

First, Sheriff Owen Dowie. On Saturday afternoon I had stopped to speak to him on Division Street and had asked questions about Schneider and had made certain statements which had impressed him as having been threats against Schneider. He had been sufficiently worried by my attitude to have warned me to keep away from the man who had been my wife’s lover.

And Mr. Rosenberg, about whom I had completely forgotten. At about eight-thirty Saturday evening I had stopped at his house and inquired the direction to Schneider’s house. Mr. Rosenberg said that at the time he had thought it odd that I would want to call on my wife’s lover, but had given me the information. He had seen my car turn up Schneider’s driveway.

Finally Oliver Spencer—reluctantly, as the reporters put it, but compelled by a sense of duty. The others at the game—evidently Masterson and Dietz hadn’t been questioned—had tried to keep quiet about it, but when Mr. Spencer brought it out in the open, they were forced to string along. Mr. Spencer was the only one to insist that I had been frantic in my demand for five thousand dollars in cash. The others merely said that I’d asked for it, and that when I hadn’t got it I’d joined the game and won it.

What had I wanted the money for? Mr. Spencer didn’t know. Miriam thought I had mentioned something about needing a rest. Ursula was more definite. She stated that when she had gone up to my room with the money I had told her that I couldn’t face my fellow townsfolk and wanted to get away where nobody knew me. So early that morning I had left. She didn’t know where. I hadn’t told her. I’d been anxious to cut myself off from everybody who had heard of the murder and the trial.

District Attorney Hackett didn’t believe her. He said so in bold type. He said that I had planned to murder Schneider, and knowing that I wouldn’t be able to get away with this one, I had prepared getaway money in advance. He said that I had been desperate to get the money Saturday night because I had found out that Schneider was planning to slip out of my reach in the morning. He said that my flight was an admission of guilt. He said that Schneider was dead because a soft-hearted jury had permitted a glib New York lawyer to turn it aside from duty and let me to go free to kill again.

The police of three states were hunting for me.

I rolled the newspaper up and carried it as far as the corner and dropped it into a waste can.



    
    \begin{ChapterStart}
    \vspace{3\nbs}
    \ChapterSubtitle[l]{Chapter ch15}
    \ChapterTitle[l]{ch15}
    \end{ChapterStart}

    \FirstLine{\noindent ### Chapter 15}

    
## The Gamblers

It was one of the better public clubs.

It had carpeting instead of the usual battleship linoleum, and the voices of men were not boisterous in the air-conditioned atmosphere. At one corner leather lounge chairs fanned out from a tiny bar. In another corner stood the cashier’s cage. The rest consisted of cane-bottom chairs and five tables covered with white cloths. There were no line sheets tacked on the wall, no pool tables in the rear, no blackjack for suckers. This was strictly poker for gentlemen.

Only one table was working, a moderate-limit draw game. I watched it for a while, then went to the bar.

Willoughby, the proprietor, came to my side to say hello. He was the only man there in formal clothes. His thin, unbending frame ancl frozen face would have made him a natural for a butler’s job. “I see you are back, Mr.—” He let the pause hang delicately in the air.

“Berkowitz,” I said. “Jeffery Berkowitz.”

“Mr. Berkowitz, of course,” he said, as if I had told him my name when I had played there yesterday. “It is early, but there will be plenty of action soon. You do not care for draw?”

“I prefer table stakes stud.”

“Oh, yes, I remember. If you will make yourself comfortable for a little while—”

“Do you expect Locust today?” I asked.

“You are a friend of Mr. Locust?”

“No, but I played with him yesterday. I’d like to come up against him again.”

“Mr. Locust is an excellent player. He may be here today. He comes several times a week.”

“Who is he?” I said. “I’ve heard his name before, but I can’t place it.”

“Mr. Earl Locust is one of the automobile Locusts. He is well-known in poker circles. Pardon me.” He started to turn away.

I said. “Does Don Yard ever come here to play?”

Willoughby’s thin body stood a trifle straighter as he turned back to me. “We do not permit professional gamblers in this club, Mr. Berkowitz.”

“I’m glad to hear that,” I said, trying to sound happy about it. “I suppose you can’t be too careful in a place like this.”

“We protect our clientele, Mr. Berkowitz,” he said and moved off to the cashier’s cage.

Presently a couple of more stud players arrived, and with two shills and the house dealer participating a table stakes game was started. It dragged. Everybody played only cinches. Few hands ran out to the last turn.

***

I turned in my chair and saw Earl Locust speaking to Willoughby. They were cut out of the same pattern, those two, though you could tell at a glance that Willoughby was the butler and Locust the master. He wore a tan summer gabardine suit and a cocoa-brown gabardine hat and carried himself with careless urbanity. He was perhaps in his forties. His pinched face was vigorously tanned, but his eyes were unhealthily tired and gray ran in narrow streaks through his thinning brown hair.

Locust looked at me without expression, then said something to Willoughby and looked at me again. I sent him a little place with private booths. He sat nod and he received it with a lift of the back and regarded me quizzically, corners of his mouth. He slapped Willoughby’s shoulder lightly and came to the table.

“How are you?” he asked me.

“Fine,” I said. “I was hoping you’d show up. I want revenge for some pots you took from me yesterday.”

He pulled up a chair and stacked his chips. “You didn’t do so bad, Berkowitz.”

Yesterday I hadn’t mentioned my name to anybody here. That meant that Willoughby had told it to him. They had been discussing me.

The game stepped up. Outside players took the chairs of the shills and the dealer devoted himself wholly to warm cup. “I suppose Willoughby dealing. Action became brisk. These lads were skillful technicians and shrewd psychologists, and Earl Locust was the best of them. Both he and I were winners.

It was night when Locust cashed in his chips. A minute later I did the same. He left while I was assembling the money into my wallet. I shoved wallet and loose money into a pocket and hurried across the room and down “And yesterday you asked me questions the short flight of stairs. The street was empty.

“What’s your hurry, Berkowitz?” Locust said.

He was standing in shadow against the side of the doorway.

“I—I—” I drew in breath. “I wasn’t hurrying. The fact is that I’ve nowhere to go.”

“Then how about some coffee?”

I tried not to show my eagerness, cup I’d ordered. I was, I hoped, a “Good idea. Do you know any place portrait of a man making up his mind, around here? I’m pretty much of a Locust waited patiently with his tired stranger.”

He flagged a taxi. We drove three blocks and got out. He cupped my elbow in him palm, as if I were a woman or a child and steered me into a cozy little place with private booths. He sat back and regarded me quizzically.

“Tell me,” he said conversationally, “when you tapped into my open jacks and drove me out, did you hold anything in the hold?”

“No.”

He uttered a restrained, pleased laugh. “At the time I could have sworn you held queens. That’s a nice game of poker you play.”

“Thanks.”

The coffee and sandwiches arrived. Thoughtfully he stirred the sugar in his cut. “What are you after, Berkowitz?”

“After?” I said.

“What’s your interest in Don Yard?”

I wrapped my palms around the warm cup. “I suppose Willoughby told you I’d ask him if Yard came to the club.”

“And that you questioned him about me. You’re a stranger and you came to the club for the first time yesterday and you were an excellent player. Willoughby is afraid you might be a professional.”

“I’m no.”

Locust went on with out break. “And yesterday you ask me questions about Don Yard during the game. Personally, I don’t’ give a hand whether or not you’re professional. They play a more exciting game than amateurs. But I asked you want you’re after.”

***

I drank my coffee and set fire to a cigarette and pulled an ashtray toward me and looked around to see if the waitress was bringing the second cup I’d ordered. I was, I hoped, a portrait of a man making up his mind. Locust waited patiently with his tired eyes fixed on my face.

“I want to play poker with Don Yard,” I said.

“Why Yard in particular?”

“Not Yard and not in particular. For three years I’ve been in the infantry, one of the few American troops fighting in Burma. Five days ago I came out of an Army separation center. My home is in Utah. My father has a string of filling stations all over the state. I’m the only child and the business is waiting for me and so is the girl I’m going to marry. But here I am in New York with no responsibilities as yet and plenty of money. My mother left me a pile of securities, which is in the hands of a New York broker. I’m not saying I don’t want to go home to Utah and Marge and the business. I do. But here’s my chance for a fling, for doing for two weeks or a month what I’d rather do than almost anything else, and that’s play poker. The real stuff, I mean. The kind I thought I could find in New York.”

Locust nodded. “Check. Some men take a drink or sports or drugs or women. I’ll pass up the best of any of that for top-notch poker. But where does Yard come in?”

“I’m not sure. I’ve been going to poker clubs because I haven’t been able to find anything else. Willoughby’s is the best of them, but I object to the house rake-off. In the long run it puts impossible odds against the player. Besides, you don’t get the kind of action I’m after. I did very well tonight and ended up with only three hundred dollars. I can find bigger games than that in my home town. That’s where Don Yard comes in. We were discussing gamblers one night at a bull session in Calcutta, and Don Yard’s name came up. Actually I don’t know anything about him except that where he is there’s probably a big game.”

“And so you think you will get a real play for your money with professionals?”

“That’s it.”

Locust took a bite out of a ham sandwich and chewed it before he spoke. “Why are the lambs so eager for slaughter?”

“I’ve considered that,” I said, “but I think I can hold my own against anybody. As for crooked gamblers, card mechanics, I’ve been told that the real big-shot gamblers don’t go in for that sort of thing. They depend on their skill and on percentages. That’s true, isn’t it?”

***

He chewed some more. “In essence, yes. And you’re good enough from what I’ve seen of your playing. There will be a game at my house tomorrow night.”

“With Yard?” I blurted.

He gave me that tired, level stare. “What I mean is big-timers like Yard,” I added quickly. “The kind of men I’d like to have my fling against.”

“Yard will be there,” Locust said. “It’s a quiet, private game in my apartment.” He studied what was left of his sandwich. “We’re quite particular about who participates.”

“Oh.”

I didn’t have to act out disappointment. I felt it all the way down to my shoes.

Abruptly Locust discarded talk of poker. He had been an officer in the last war; he had spent a year in India and had visited Utah. Subtly he was pumping me, checking up on the story I’d told him. It was a good thing I had chosen the CBI theater as the one in which Jeffery Berkowitz had fought in, for I knew China and India and Burma, and after our second year at college Kerry and I had spent the summer driving tractors on the farm of a classmate in Utah. I passed Locust’s exam with flying colors.

We were finished eating before he returned to poker. “You strike me as a decent sort of chap. I can understand exactly how you feel. Poker’s in my blood too. I can invite you to tomorrow’s game. My recommendation will satisfy the other players.”

“Would you?” I said eagerly. “Those are fairly steep games.”

My face fell. “I hope to win, but I can’t possibly afford to lose more than twenty thousand dollars.”

Locust laughed. “You’ve been listening to wild tales. This is a friendly game. The stakes are only a thousand dollars each. We usually settle with checks, but I suggest that you, as a newcomer, bring cash.”

***

He was already there when I arrived at Locust’s apartment on Madison Avenue. At a glance I picked were in the oak-and-leather study him out of the four strange men who where the game was to be played.

A squat, broad man, Schneider had told me. But Don Yard was not particularly short. The effect was produced by the disproportionate breadth of his shoulders which provided a proper base for a beefy neck. Under an unruly mop of dark hair his face was as wide and craggy as a clenched fist. He exuded physical power, like a bull. A wide and craggy as a clenched fist. He was a lot older than I, at least twice my age.

His handshake was not hearty when Earl Locust introduced us. It was bonecrushing pressure applied and instantly relaxed. His incisive brown eyes, under shaggy, overhanging brows, slid quickly over me.

“I’ve seen you before,” he said.

My breath caught. How could I know that he hadn’t been among that blur of indistinguishable faces behind me during the trial?

“I’ve been in New York only six days since my childhood,” I told him.

“I guess lots of guys look alike,” he muttered and lost interest in me.

Locust cupped my elbow and led me to the three other men. They accepted me without comment and almost no conversation. We sat down at the table and bought chips from Locust who, as the host, was banker. The others bought on the cuff, but I conspicuously paid my thousand dollar stack in cash.

The game was table stakes draw. I had heard that on rare occasions it was played, but I never before participated in one. It was murder. My thousand dollars melted away.

I didn’t know whether the three other players were professionals like Yard or wealthy amateurs like Locust, but I was sure there wasn’t anything crooked. Cards couldn’t be marked in any way I couldn’t spot and I had taught myself to recognize and duplicate all the tricks of card mechanics, nothing like that was being pulled here. The cards didn’t fall right for me or I was being outplayed. These players weren’t Ursula or Oliver Spencer or Art Masterson. This was the big league.

I was halfway through my second stack of chips when a man and a woman came in. Yard greeted the woman as “Baby” and the other men called her Bertha. The man with her they called Walt. He pulled up a chair to the right of me and bought chips.

This was my lucky night, if not at cards. Here were all three of them. The first stage of my search was ended.

Walter was the man who had been at the roadhouse with Helen Spencer and Don Yard and Lily and again at the bungalow when Yard hit Oliver Spencer. A long face that was like a blank wall, Helen had described him. That was adequate. Add a chin like a spade to it and dull little eyes which showed as much emotion as a dead man’s. He didn’t have to change his face for poker.

***

Bertha Kaleman had hair of flame. It was too good to be natural. She reminded me of Lily because of that same brittle polish and that way of wearing a dress so that you were aware of every movement of her body. She had plenty of body. She was bigger than Lily had been, taller and fuller. But Lily had been beautiful where Bertha Kaleman only stepped up your blood pressure.

She was sitting next to Don Yard, looking at his cards. Her eyes lifted and caught me appraising her. Her red mouth threw me a sensuous smile. “Don’t I know you?” she said.

Not, ‘Do I know you?’ meaning I don’t know you and please introduce yourself. ‘Don’t I know you?’—as if mine was a face seen and lost among many other faces. Could she have been at the trial with Yard? No, Helen, who had met them both, would have recognized them and told me. I was frightening myself with shadows formed of words.

“I’d never have forgotten you if we’d met,” I said.

Locust mumbled a belated introduction.

“Berkowitz,” she said, leaning her arms on the table. “I bet your friends call you Berky.”

“Anything you call me is music,” I said.

“Look, the boy is gallant,” she said.

“Shut up and let’s play,” Don Yard growled.

On the next deal I caught a flush on a draw and tapped into Walter who was holding three eights. A player named Judson turned up a straight, and I made myself a nice pot.

But it was the last I won for a long time. My stack was sinking dangerously low and I had only a few hundred dollars left in my pocket. I played as tightly as I could, but that wouldn’t do it. There was too little money between myself and disaster.

Because if I dropped all my money here I was through. I could support myself by getting a job, any job, but I had no more taste for being a fugitive indefinitely than for burning in the chair or rotting in jail or going mad in a madhouse. I had to have money to remain in contact with Don Yard and Bertha and Walter. There was no way I could do it except through poker. I was in now. I was one of them. I’d be invited to other games in which Yard played. But not if I was broke. Not if I had a job where I would not make enough in a month to participate in a single one of these pots.

There was one way to get a stake. The ethics of it I could take in stride. These men were gamblers or rich and could afford to contribute to my attempt to regain my life. But it was highly dangerous. It wouldn’t be an exhibition before friends to show how clever I was with cards. These men knew all the tricks. You had to be better than good to fool them. Better than perfect.

***

The deck was three deals away from me. I sweated those deals out, throwing my hand in as soon as I glanced at it. Walter, on my right, brought out a new deck for his deal. I had two high pairs, but I went out, and raked in the discards and washed them. There was some hot action between a man named Silver and Yard and Locust, so that nobody paid attention to me shuffling and reshuffling the discards.

Yard won. I scooped up the remaining deck and shuffled some more.

“Come on, deal,” Judson said. “Do you want to wash the spots off them?” I placed the deck before Locust on my left. He cut it only in two, as he generally did. I’d banked on that. I dealt the way I always did, rapidly.

Yard said: “What’s the rush?”

I stopped. A vein pounded in my temple. “What’s that?” I asked.

“Take it easy,” Yard said. “We like the cards dealt slow.”

I forced a grin to the surface. “Oh, sure,” I said and continued to flick the cards out.

I sank back a little when the round was dealt. I didn’t have to look at my cards. There were three queens. I knew that a man named Smollens was holding three aces and Silver two high pair. The other hand had fallen any way they had come up.

Locust must have bought something good. He opened strong. Smollens was enthusiastic about his three aces and raised. Silver tagged along with his two pairs. Yard looked at his cards and then across the table at me and folded. So did Judson and Walter. I shoved in my stack. The three who were in saw me.

On the draw I dealt the cards straight except to myself. I gave myself a full house. None of the others helped. The pot was mine.

As I leaned forward to pull in the chips, my eyes tilted up to Yard’s craggy face. He was watching me flatly, without expression. My heart skipped a couple of beats. He said nothing. There was talk between Smollens and Silver about how neither of them had been able to fill all night. Locust was gathering up the cards. The game continued. I had my stake.

The gods of chance are not concerned with morals. From that deal on I won. At four o’clock in the morning I walked out with eleven thousand dollars in cash and checks.



    
    \begin{ChapterStart}
    \vspace{3\nbs}
    \ChapterSubtitle[l]{Chapter ch16}
    \ChapterTitle[l]{ch16}
    \end{ChapterStart}

    \FirstLine{\noindent ### Chapter 16}

    
## Don Yard

Walter was waiting for me on Madison Avenue. He said tonelessly: “Around the corner.”

Both hands were thrust deep into the pockets of his jacket. He had a gun in one of those hands or he would have been more polite. I looked up the street and then down the street. At that hour it was quiet enough to hear myself breathe.

“What’s around the corner?” I said. “Don. He wants to see you.”

We walked side by side to the end of the block and turned right. A snappy sedan stood twenty feet down from the corner. Bertha Kaleman sat behind the wheel. She said pleasantly: “Hello, Berky.”

The back door on the curb side opened. “Get in, kid,” Don Yard said out of the back seat darkness.

I sat beside him Walter followed me, climbing over Yard’s feet and mine, and sat at my left. Bertha had the front seat to herself, probably the first time in her life she was left sitting alone when there were men around. She drove west to Sixth Avenue, then south, then west again. Nobody said anything. I wondered why I wasn’t more scared. A week ago I would have tightened up and started to sweat and shake. Now I was chiefly curious.

Between Tenth and Eleventh Avenues Bertha stopped the car.

“Let’s have the dough, kid,” Don Yard said crisply.

They were shadows on either side of me, their thighs pressing against mine. Past Bertha’s left shoulder I saw the brightness of the Hudson River speedway. They could find better spots to kill a man.

“So it’s a holdup?” I said.

“You’re too smart for your pants,” Yard said. “You used the Haymarket Shuffle to give yourself that queen-high full.”

I laughed. The effect sounded all right to me. Just about enough derision. “Why didn’t you expose me during the game?”

“I had no dough in the pot. There’s more in it for me this way.”

“You’re crazy,” I said. “That was an honestly dealt hand. Why didn’t any of the others notice anything wrong?”

Bertha twisted around, putting her knees up on the seat and facing us. “Walt, did you notice the deal was phoney?”

“I wasn’t watching.”

“I was,” Bertha said. “I didn’t see him hold any fingers at the bottom of the deck.

“He’s smart,” Yard said. “Damn smart. Quick as lightning. But I’m smarter. I was brought up on the Haymarket Shuffle and I know when cards drop at the same time from the top and bottom of the deck, even if he doped out a new grip. Locust only cut the deck once, which made it snap for a good mechanic to stack them. And look how the cards fell, just right for him.” He turned his shadowed face to me. “You’re lucky I’m taking only the dough and leaving your health.”

***

It was my lucky night, all right. Whether I gave up the money or didn’t, my contact with Don Yard and Walter and Bertha was over almost before it started. Oh, I was smart as hell.

Walter said indifferently: “Take a look at this, kid.” The overhead light flashed on and off. There was a revolver on his knee.

“You win,” I said. I dug the roll of bills and checks out of my hip pocket.

“But two thousand dollars of that was money I started with. You saw me buy two stacks. You can have the rest.”

“I’ll take it all,” Yard said. He took it all.

“There are checks made out to my name.”

“Don’t let that worry you.” Yard pushed the door open. “Scram, mug. If I ever see you around again, you’ll be a mighty sick crook.”

I rose in a crouch to step over his feet. Bertha reached a hand over the back of the seat and put it on my arm. “Wait,” she said. “Don, you’re a dope.”

“That so?” Yard said indifferently. “Sit down, Berky,” she ordered me. I squeezed myself back between the two men. A match flared. Yard brought it up to his cigar. By its light I saw the money in his lap.

“You’re throwing away a gold mine,” Bertha said. “Those men tonight weren’t babes in the woods. How come none of them spotted a phoney deal. Walt’s eagle eyes didn’t. You said it yourself, Don—he pulled cards from the top and bottom at the same time like greased lightning. Could you do it as well?”

Yard grunted. The tiny glow at the end of his cigar made his face look like the side of a mountain seen at dusk.

“I bet you watched him pretty close after you spotted that trick deal,” she went on.

“What do you think?”

“Did you catch him at any other?”

“There weren’t any more.” A note of interest slipped into Yard’s voice. “I think I get it, baby.”

She tossed her hair which had turned black in the dimness of the car. “It’s about time. Berky was the big winner tonight and most of his pots came when he wasn’t dealing. He’s got to be mighty good to win from you sharks.

And just look at him. Did you ever see anybody nicer-looking? I mean clean and young, like he doesn’t know what day of the week it is. If he can take you like he did tonight, what do you think he can do to suckers? And you want to throw it away for a few crummy grand.”

“She’s got something there,” Walter said. “The kid’s born sucker-bait.” They were settling it all between themselves. I might have been part of the upholstery for all my opinion counted. That was all right with me.

Yard gave me a faceful of cigar smoke. “I thought I’d seen you around. Who are you, kid?”

Locust must have explained to Yard and the other players who I was when he had told them he had invited me.

I repeated the story.

When I finished, Yard said: “Let’s see your discharge papers.”

That was the basic weakness of my new identity. I told him that I’d mailed them to my father in Utah.

“Like hell you did,” Yard said. “You carry those on you all the time. What are you hiding, kid?”

***

I was silent, waiting for one of them to suggest it. Bertha did. She had enough brains for herself and the two men.

“Dishonorable discharge,” she said. “He was kicked out of the Army.”

I dropped my head. “I tore them up,” I said wearily. “They weren’t the kind I’d want to show. It was the same thing as happened tonight. I’d been winning a lot in barracks games and the Army rang in an expert on mechanics. He spotted me and I was sent home. I haven’t a rich father and no securities in my name. All I had left was twenty-five hundred bucks after blowing the rest away between India and here. I don’t know anything but cards and I tried to get in on the big-time, with guys like you. I was hitting the bottom of my stake and got scared I’d be cleaned out, so I pulled that one fancy deal.” I spread my hands. “There it is.”

They liked that story. It made us kindred souls. They could understand a lad like that and know how to work with him. An honest man would have been hazardous.

Yard removed his cigar from his mouth. “See me tomorrow. Maybe I can use you.”

I tapped the roll on his lap. “What about my dough?”

“Yours?” He chuckled dryly. Then he decided that that wasn’t the right tactic if he was to use me. “I’ll take care of it till tomorrow. Make it at two.” He told me a number and a street in Greenwich Village. “Now beat it.”

I climbed out over his feet. He followed me out and got in the front seat with Bertha. She raised a red-tipped hand to me and drove off.

***

The house was on the far side of Ninth Avenue. It was a hundred-year-old tenement which had been torn up inside and rebuilt and the shell had been covered with a veneer of gray brick. Don Yard occupied the street floor. He had an entrance all to himself under the side of the stoop.

I wondered why a man who won and lost thousands in one night would live in a dump. When I was inside, I saw an apartment which would have looked good on top of Park Avenue. The rooms were immense. The walls were hand-troweled light-green plaster. The living room rug was a pomegranate-red Persian with a grayish-green border. The furniture was subdued modern. Heavy drapes fortified by Venetian blinds barred the vulgar eyes of passersby from the street-level windows. It was a sunny afternoon, but soft, indirect lights were on. Twilight gloom would always prevail in these rooms, which wouldn’t matter to people who did most of their sleeping in daylight.

It was five minutes after two when Walter answered my ring. Bertha wore a black chiffon negligee which obviously was all between us and her lush flesh. Her long flame hair was a shawl over her shoulders. She gave me a hand to shake and a smile to absorb.

Yard sat in a semi-circle of Sunday papers scattered around his armchair. Powder-blue pajamas were covered by a red silk robe which hung open in front. His jowls were dark with stubble. He threw me a curt nod and dropped his eyes back to the sports section.

I advanced to the edge of the papers at his feet. “Before I hear your proposition,” I said, “I want my money back.”

“You stole it,” he muttered without looking up from the paper.

“Only a small part of it and not from you or Walter,” I said. “You two were out of that hand. The rest I won fair and square.”

Walter was at my side. I hadn’t heard him move over the rug. His hands were deep in his jacket pockets. “Should I throw him out, Don?” he asked lazily.

“We’ll see.” Yard dropped the paper to the floor and put his head against the back of the chair. “You had a good night’s sleep and now you’re feeling your oats. Is that it, kid?”

“I’ve had time to think it over, if that’s what you mean. I’m willing to work with you, but first I want my money.”

Bertha undulated around me and sat on the arm of Yard’s chair. One of her nicely rounded thighs showed. She didn’t cover it. “The boy has spirit,” she said.

“Yeah?” Negligently Yard pulled the negligee over the exposed thigh, then looked up at me through half-closed lids.

“So you heard about me in India?”

“From a former newspaper reporter during a bull session. I don’t remember his name.”

“What did you hear?”

“That you were one of the top gamblers in New York. Something like that. It was indefinite.”

“It was so indefinite that you came to New York looking for me,” he said. “You went around asking about me.”

“I suppose Locust told you.”

“Yeah.”

***

This was my opening. I said: “As a matter of fact, I’d forgotten about you until a couple of days ago. I picked up a paper in the subway. There was one of those full-page feature spreads about a murder. An Air Force navigator had been tried for the murder of a woman named Daisy—I think that was her name—and it said in the paper that you’d once been married to her.”

When I stopped speaking, I felt a vacuum of silence. Bertha glanced at Yard and then away. Yard’s eyes had opened all the way in a stare that smoldered.

“Lily,” Walter said softly at my side.

“That’s it, Lily,” I said. “I knew it was a flower.”

Yard’s eyes closed. “What was the name of the paper? I didn’t see anything about me in any paper.”

“It wasn’t a New York paper,” I said. “New Jersey or maybe Connecticut. I forget. Then somebody else was murdered, the lover of—” I stopped out of apparent delicacy for her former husband’s feeling.

“Go on,” Yard urged me quietly. “What else did the paper say?”

“The police are looking for the navigator. He’d been found innocent of Lily’s murder, but—”

Yard sat erect. “He wasn’t innocent. He was acquitted, that’s all. How’d they get my name into it? It wasn’t brought in at her trial.”

My bluff hand was being called. But it was unlikely that he would take the trouble to check up on every paper in New York and Connecticut and New Jersey to see if I’d lied. He would not think that there was any reason for me to lie.

“I guess you’re something of a public character,” I said. “Everything about you comes out sooner or later. But I was telling you how I came across your name a second time. What made me read that story was a picture of your former wife, Lily. She was something to look at. She had all the pin-up girls I ever saw beaten hollow.” A storm was gathering in his craggy face. I stirred it up. “I suppose a man like you who’s got a swell girl like Bertha, wouldn’t care if another man took Lily away from you. But I know if I had a girl like her—”

His beefy neck was as red as Bertha’s hair. He leaped to his feet. I started to duck to avoid the blow I thought I saw coming, but he only dropped his powerful right hand on my shoulder and squeezed. Then he shoved me aside and strode out of the room.

He still loved Lily. Even now that she was dead, Bertha was only a substitute for her.

Bertha slipped off the arm of the chair. Her head was dipped so that I could not clearly see her face. She went to a cigar stand and took a cigarette out of a hammered silver box and set fire to it with a table lighter.

I turned to Walter who stood regarding me with his dull little eyes. “What did I say?” I asked.

“Too damn much.”

“You mean what I said about his former wife who was murdered? I get it. He still cares for her.”

“You talk too damn much,” Walter said.

***

I started toward Bertha whose back was to me. Don Yard returned before I reached her. His color was high and his eyes bright. That must have been a pretty stiff drink he had gone for.

He said briskly: “Keep out of my personal life, kid, and we’ll get along. Here is the deal. I stake you and you get ten percent of the winnings.”

“Twenty-five percent,” I said.

Bertha emitted a low giggle. “Berky is all right, Don. You don’t scare him.”

He sent her an annoyed scowl and for a moment I was afraid she had queered it for me. But he shrugged and said: “Fifteen percent. We’ll see how you do on that. There’s a game tonight at the Hotel Tannor. Room 783. Be there at nine.”

“What kind of game?”

“Whatever kind of poker they ask for and whatever stakes they set. Those heels will play high. They all have plenty of dough in their pants and asking for somebody to take it away. I won’t be there because I’m too well known. If anybody asks you, tell them the first story you told us, about being a returned soldier with rich folks in Utah and so much dough you don’t know what to do with it. Walt will be playing and handling the kitty. It’s okay to know him, but only because he steered you up to the game. You’re a sucker like the others.”

“A house man for a floating game,” I said. “That suits me. What happens when I lose?”

“You don’t get your fifteen percent. Lose too much and it won’t pay me to do business with you. Win a lot and I’ll raise your percentage.” One corner of Yard’s mouth lifted. “The idea, kid, is not to lose.”

He left it at that. I had been afraid that he wanted me solely for my ability as a mechanic. That would have put me on the spot; one rigged deal had been enough to last me for the rest of my life. Yard was not a crook. He was a business man who made investments with the odds in his favor. He didn’t care how I won as long as I won and he made his eighty-five percent of the profits. He was right up there with the captains of high finance.

“Then it’s set,” I said, “except that you haven’t yet returned the money you took from me last night.”

Yard went out of the room and returned with a roll. It wasn’t my roll; there weren’t any checks in it. “Our agreement started as of last night,” he said. “You get your two grand back and fifteen percent of the rest. Any more arguments?”

I had none. He handed me thirty-three hundred and some odd dollars and Walter conducted me to the street door.

***

Four of the five strangers in the hotel room were visitors to New York who yearned for a fling at big-city poker. They were top-notch players in their home towns, but not quite adequate for a floating game organized by professionals. Some won and some lost, but both Walter and I won. Honestly. I ended up with three thousand dollars ahead. Walter with seven hundred dollars, not counting the kitty for playing host. A fair night’s work.

I waited for Walter a couple of blocks away. He picked me up in a taxi and we rode downtown to settle accounts with the boss. Between the Seventies and Fourteenth Street, not a word passed between us. He wasn’t a lad who worked his face except when essential. Even when sitting, his hands remained in his pockets, as if to hide them or warm them or hold onto something out of sight.

“Why did Don get sore at me when I mentioned his former wife?” I said suddenly.

“He wasn’t sore at you.” The words came out of the corner of the long, wide mouth; the face remained turned to the side window.

“Then what was he sore about?”

“Keep your trap shut about Lily.”

“Don strikes me being a pretty hard guy,” I persisted.

“Yeah.”

“Then why did he let another man take her away from him? Why didn’t he go after her if he cared so much for her?”

His face swiveled to me. “Let it lay, kid.”

“I only asked—”

“Let it lay.”

The taxi pulled up at the gray-brick house. No light showed through the Venetian blinds. Walter pressed the bell-button once. When there was no answer, he fished out a keyring.

“Do you live with Don? I asked. “I’ve got a place up in the Bronx, but I bunk here sometimes.”

A dim bulb glowed in the long hall. At the end of it was a back door leading out into a yard—the privilege of living on street-level. He waved to the living room on my left. “Wait in there. I’ll wake him up.”

I found the wall-switch and clicked it. A pre-dawn hush seeped into the house from the street. I crossed the room and took a cigarette from the hammered silver box. There was a bleached oak desk against the far wall. I had perhaps a minute to search it. What for? It was only in stories that there would be a letter from Lily saying: “If you don’t send me ten grand at once I’ll tell the police all I know, so you better pay up.” Or: “My husband is coming home any day now, probably this weekend, so don’t come sucking around me any more. You’re too jealous for a girl to live with. Alec is a hick and a dope, but he’s nice and maybe I’ll like it with him. If you don’t let me alone, I’ll tell him when he gets back.”

All right—fantastic. So were any other ideas I had which would work Don Yard into my equation. Here I was marking time with great vigor and getting nowhere in a hurry.

***

The cry wasn’t loud. Not the first one. The second was shriller, more urgent.

I started toward the hall door and then reversed myself. The door in the back wall of the living room led directly into the other rooms. I went through it and found myself in darkness except for light shining through a door at the other end. The lighted room was a small study, and to the left of it was the master bedroom.

Bertha Kaleman had both hands against Walter’s chest and was trying to push herself away from him. His arms were about her waist and his spade-chin was frantically nuzzling her bare shoulder.

“Cut it out!” I said.

Walter released her. His long face was very pale. His eyes had something in them now—glittering, feverish lights.

“Beat it!” he panted.

Agitatedly, Bertha pulled up the shoulderstrap of her yellow nightgown. She had just gotten out of bed, but the layer of paint was smooth and precise on her face. The toenails of her bare feet were scarlet.

“Never mind, Berky,” she said jerkily. “Don will kill him when he comes home.”

Walter thrust his hands into his pockets. He was trembling and I wasn’t. That was good to know.

“Beat it!” he said again.

I watched his hands. If he had a gun in either of those pockets, what would I do? This was absurd, unnecessary. I hadn’t come to New York to be killed to protect what honor the mistress of a gambler had. But here I was, stuck with the situation. I waited.

Bertha said hoarsely: “If you touch Berky, I’ll tell Don. I swear I will.”

“Yeah?” The blankness had whipped back into his face. His dead eyes looked at her and then at me. He went out through the other door, the one leading into the hall. I listened to him enter the living room.

“You were swell, Berky,” Bertha said. “He’s been waiting for a chance at me for a long time. That guy gives me the willies.”

There wasn’t much to that yellow nightgown of hers. Little purple forget-me-nots were scattered over it, and at the edge of the very low bodice was embroidered: “Forget Me Not.” Cute as hell. I could have thought of a more apt slogan.

She looked down at herself and brought her palms up over her breasts. “Why, I’m practically naked,” she said with a girlish giggle.

“Don’t mind me,” I said.

She took one hand from herself to pat my cheek. “You’re sweet, Berky.” The she ducked into the bedroom.

***

I returned to the living room.

Walter had placed a bottle and glasses on the white coffee table. “How about a snorter?” he asked almost amiably.

I shook my head.

He poured a short one and then said angrily: “What the hell can you expect when I meet her walking out of her room in only that yellow stuff? She’s always like that. She drives a guy nuts. You don’t have to make anything of it.”

“Why should I? Was Lily like that too? Was that why she and Don—”

The blank stare he gave me cut me off. “Why do you keep bringing up Lily?”

I laughed, probably too lightly. “I guess that newspaper account I read about her murder made me curious.”

“Well, don’t be.” He sat down at the coffee table and pulled a pocket chess set out of his inside breast pocket. “Do you play, kid? There’s no telling when Don will get back.”

I bit off the automatic reply that rose to my lips. Alec Linn was a chess player. “Never could get into it,” I said. “Poker is my only game.”

He set up a simple problem. It gave him trouble. After a while he looked up. “The way you ask about Lily, sometimes I think maybe you’re a cop.”

“Why would you be afraid of a cop?”

The empty stare lay flatly on me. “You’re too dumb to be a cop. Except you’re smart in poker and you play that too good for a cop.” And he moved the rook to KB5 instead of to KB6, which would have been proper.

Don Yard arrived with the dawn. He was counting out my fifteen percent when Bertha came in. She was wearing the black negligee over the yellow nightgown. Her flame hair hung down to her waist. She looked very good. What was it about unhandsome heels like Don Yard that could get women like Lily and Bertha?

Then I noticed Walter crouching over the chess set. He might merely have been studying the problem, but there was tension in the arc of his shoulders and his jutting jaw was ridged. He was afraid of Yard and afraid of what Bertha could do to him through Yard.

She ignored him. She went over to Yard and pecked his cheek. “How’d Berky make out tonight?”

Yard patted her hip. “Not bad. Three grand.”

“I told you he would. Don’t be long, Don. Good-night, Berky.” She undulated out of the room.

Walter’s body loosened. He sat back. That was that. I took my cut of the winnings and went home.



    
    \begin{ChapterStart}
    \vspace{3\nbs}
    \ChapterSubtitle[l]{Chapter ch17}
    \ChapterTitle[l]{ch17}
    \end{ChapterStart}

    \FirstLine{\noindent ### Chapter 17}

    
## Pursuit of an Equation

That week there were three more floating games in hotel rooms. I did very well in one and fair in the two others. But I got nowhere at all in the real purpose behind my playing. I saw Don Yard only at the pay-offs and then briefly. Tackling Walter, I had learned, was a waste of breath, if not hazardous. Bertha Kaleman seemed to be my best bet, but I couldn’t get her off by herself. Twice I phoned the unlisted number to ask her out to lunch. The first time there was no answer. The second time I hung up as soon as I heard Yard’s voice over the wire.

On a Tuesday, two weeks after I hit New York, I accepted Earl Locust’s invitation to take part in one of the murderous table stakes draw games he held in his study. Yard and Walter wouldn’t be there; they were busy on another project of theirs I knew nothing about. This was extra-curricular, winning or losing my own money. As it turned out, losing.

The game was an hour old when Robin Magee came in.

I suppose it was inevitable. If not Magee, it would have been somebody else. New York is immense, but even an out-of-towner like myself was always bumping into acquaintances. I tried to protect myself by keeping off the street as much as possible and depend on taxis for transportation. And now here was Robin Magee regarding me with that silky smile of his.

Locust rose from the table and introduced him to two players Magee had not met before. Then it was my turn. “Mac, I want you to meet Jeff Barcowitz who took us to the cleaners last game we had up here. He plays a mean poker. Jeff, shake hands with Robin Magee, the noted criminal lawyer.”

We shook hands. I doubt if there was any blood in mine. We looked into each other’s eyes. Magee broadened his smile and sat down beside me. The game resumed.

I’d been due for a streak of bad cards. It had started before Magee’s arrival, and the fact that he was now sitting beside me didn’t help my playing. I tightened up, supporting only cinches, and was lucky to get away with a loss of only a couple of thousand dollars. Or a couple of grand, as I was beginning to think of such an amount.

***

When the game ended, Magee and I left together. We walked a full block in the early morning quiet before either of us spoke. Then Magee chuckled.

“Berkowitz! That’s a little masterpiece. Who would think of Alexander Linn as Berkowitz, or the other way around?”

“What do you propose to do about it?” I said.

The chuckle died. He stopped walking and I stopped and we stood against a wall, two shadows; beyond the sickly rim of the nearest street light.

“I’m in a difficult position, my boy,” he said somberly. “My clients would lose confidence in me if I were to turn a former client of mine over to the police. Though you are wanted for an entirely different murder than the one for which I defended you, the fact remains that I was recently consulted concerning this second case.”

“By George Winkler?”

“He came to my office last week with Miriam Hennessey.”

“How is she?”

“Not well. Winkler told me later that she is eating her heart away. I feel sorry for that poor girl. A hardened cynic like me cannot understand why women are always falling irrevocably in love with the men who hurt them most.”

A milk wagon cluttered leisurely by. The horse appeared to be asleep on its feet.

“Miriam doesn’t love me like that,” I muttered.

“My boy, you are as blind as a bat if you really think that. It was plain to me in every word she said. Later, when I had a private talk, with Winkler, he said that if you had had an iota of sense you would have married Miriam in the first place and avoided all this trouble.”

The milk wagon turned into Third Avenue. The street was empty now except for the two of us standing against the wall.

“I suppose there’s no use telling you I didn’t kill Schneider,” I said.

He shrugged his aristocratically stooped shoulders. “I am a defense attorney, not a prosecutor or a juror. Winkler and Miriam came to New York to consult me as to your chances if you were caught or gave yourself up. Frankly, I would not give one chance in ten to your being acquitted this time. I believe, however, that I can get you committed to an institution.”

“I prefer the chair to a nuthouse!“

“Then you’ll get it.”

“I’m not giving myself up,” I said fiercely. “I’m fighting.”

He looked searchingly at me, though he could not see more than an outline of my face. “They tell me you are an intelligent young man, but you haven’t impressed me as such. Not before and during your trial and; certainly not now. Is this a rational way to hide out, going about openly in New York and participating in big poker games when the police are aware of your reputation as a poker player?”

“I’m looking for the murderer.”

“In New York?”

***

Evidently Locust hadn’t told him that I’d met Don Yard through him, and probably nobody but Walter and Bertha knew that I was working for Yard. Let him keep thinking that I was playing poker as the only way I had to make a living while in hiding.

“I’m working on an equation,” I muttered. “Some of the terms might be found here?”

“Terms?”

“For the hypotheses of a linear equation which I hope will tell me who murdered Lily and Schneider. Of course it can’t be strictly mathematical. There can’t be fixed values. But if I can find the right terms and then satisfy the equation—”

Magee said: “I’m sure that I can persuade a jury to find you psychologically irresponsible. This time we’ll bring in experts to testify.”

What I’d said about equation hadn’t gone through his ears. I was a crackpot raving wildly, a psychoneurotic in an advanced stage. We were back where we’d started from weeks ago. There was no common meeting ground, nothing I could say which he would understand or believe.

“Look,” I said. “I’m going to start walking away from here. Will I have to worry about the police coming after me?”

For a long minute he was a motionless and silent shadow against the wall. Then he said: “I cannot afford to be known as an attorney who hands his clients over to the police. I can only urgently advise you to give yourself up. There are ways of getting you out of a mental institution after a period of time.”

Everything emptied out of me. For the first time it struck me that he and all the others could be right. I was mad and had murdered two people without remembering. Or not a murderer and not completely insane, but sufficiently cracked to stake my life on an equation which was not mathematics or psychology or anything else in the books.

“I’ll be glad to accompany you to a police station,” Magee said gently.

I turned and walked away from him. In the breaking dawn I found myself at the East River. I turned back to the west. Men and women were leaving for work when I climbed the stoop of Mrs. Egan’s rooming house.

***

In the winter we needed blankets and moved in from the front porch and its protecting canopy to one of the three rooms into which our hut was divided. But the straw ticks continued to smell of mold and there were the little indefinable sounds which might be a scorpion crawling into your boots or a snake slithering inside or outside the wall. Just before sundown Kassim, my bearer, had bashed in the head of a deadly krait which had been within a yard of my leg. He had grinned delightedly as he had held it up by its tail. “Bas!” he had said. Finished! Meaning not the snake, but me if he, Kassim, hadn’t been quick enough.

I kicked the blanket off me. Winter or summer, India was no place for sleeping. I turned on my side, and there was a woman in bed with me. I smiled. This was what I had needed all along. I stretched out a hand to her. It froze in horror. Her hair consisted of peculiarly colorless kraits squirming toward me. The woman looked along her white passionate body and laughed at my terror. And her face was Lily’s face.

I tried to roll off the bed, but I could not move. Medusa turned her victims to stone. I lay with eyes closed on the damp sheet, waiting for Lily to dip her head closer and let the snakes get their venom into me. I heard her stir. I felt her hot flesh along my arm as she bent lower to bring the snakes into range. I screamed soundlessly and with a vast effort twisted my head to her.

The snakes were gone. Her hair was hair now, but it was not the platinum of Lily’s hair. It was black and very long and it lay spread over burnished tan flesh without quite covering it. Intense black eyes searched my face. The wide mouth was partly open—a mouth to be kissed.

“Miriam!” I cried joyfully.

There would be no snakes in Miriam’s hair, nothing to hurt me. I reached out my arms for her, and suddenly I was alone, tossing in my bed in Mrs. Egan’s rooming house. The sheet under me was damp. A high, hot sun poured in through the window and over my face and body.

My watch said two-thirty. It was afternoon, but I hadn’t slept enough. It was a long time since sleep had rested me. I turned my face away from the sun and closed my heavy lids.

***

There was a knock on the door.

I got out of bed, rubbed my eyes with my knuckles and then scratched my back. The knock was repeated. I went to the door and asked who it was. “Bertha.”

“Who?” I said stupidly.

“Bertha Kaleman. Is there any reason why I can’t be invited in?”

I turned the key and started to turn the knob before I realized that I wasn’t in proper attire to receive sleekly turned out female company. “Wait a minute,” I said. “I’m not dressed.”

The unlocked door was opening. “I don’t mind,” she said and came all the way in. She looked me over. “Why the fuss? You’re wearing pajamas.”

“It covers more than a sheer yellow nightgown with ‘Forget Me Not,’ embroidered on the bosom,” I said.

“I see you haven’t forgotten.”

“Do you think I could?”

She patted my face with a white-gloved hand. “You’re cute, Berky. No wonder I’ve fallen for you.”

I didn’t say anything. She strolled by me to the dresser. Her hat was built like a layer cake, the bottom layer black and the top and smaller one royal blue. Her black dress was also in tiers, the material folding out from her adequate hips and twice again before the hem. At the slashed bosom there was another tier, but that wasn’t the design of the dress. That was Bertha.

“Aren’t you glad to see me, Berky?” she said into the dresser mirror. She was removing the layer cake gingerly to preserve the complicated hairdo.

“It might be a prudish upbringing, but I was taught to entertain women with my clothes on, at least in the early stages.”

She smiled brightly into the mirror. I gathered up my clothes and carried them into the bathroom.

When I came out, her handbag and gloves were beside the layer cake on the dresser. She stood at the window, looking down into Eighty-Second Street. I went to her side.

“Sorry I and the room are such a mess,” I said. “I wasn’t prepared for company.”

“I dropped in for a moment to ask if you’d like to see a musical tonight. ‘High Feathers.’ Somebody gave Don four tickets. He was going to give two to Walt, but Walt has no girl friends, so I suggested that we take you. Don said all right.”

“And he sent you up here to tell me?”

“I imagine he assumed I’d phone you.”

“But you didn’t. You came in person.”

This time the hand was placed on my chest. The full, painted mouth was surprisingly close. I didn’t know it was going to be a kiss until it was an accomplished fact. Her hands slid up under my armpits. I held her tight. She felt very good. After a while I found myself looking at the artificial lashes at the edge of exotically darkened eyelids, and I told myself that I mustn’t like it too much. I must use it, but not let it run away with me.

***

I took my mouth and the rest of me away from her and went to the dresser and lighted a cigarette. Through the mirror I saw her undulate after me. The cigarette was plucked from my fingers. She brought it to where a smear of lipstick ruined the corner of her mouth.

“What’s the matter, Berky?” she said mockingly. “Didn’t you enjoy it?”

“I’m scared of Don,” I told her. “I don’t claim to be as tough as Walter, and he was scared sick last week when he thought you might tell Don that he’d forced himself on you.”

“You’re taking a big jump ahead. What do you think is going to follow that kiss?”

“I can hope, can’t I?” I sat down on the edge of the unmade bed and worked at looking miserable. “What would Don do if you left him for me?”

“You’ll never get a chance to find out. It was a nice kiss. Period. Don’s my man and I’m sticking to him. Don’t forget it.” She turned to the mirror to inspect her face. It needed refurbishing after that kiss. She took stuff out of her handbag.

“You’re safe now,” I said to her back, “with Lily dead.”

The lipstick paused at the corner of the mouth. “What do you mean by that crack?”

“Remember how Don almost took a sock at me when I talked about Lily and the second man she’d married? And Walter warned me to keep my mouth shut about her. Don can’t stand hearing about her. She’s dead, but he’s not got over being in love with her and never will.”

“Don’t try to get profound, Berky,” she said into the mirror.

“Don’s in love with a ghost,” I went on, “yet Walter was afraid of what Don would do to him for making love to you. It isn’t sense.”

She concentrated on restoring the lips. Then she said: “Don holds onto what he has. He has me.”

“And you hold onto what you have,” I said. “Even when Lily was alive she was a ghost haunting you. You were only sitting in for her. Don would have dropped you in a minute if she wanted to go back to him. She had you worried all the time.”

I expected Bertha to flare out and spew out angry words. She didn’t. She was changing a pin in her flame hair and looking at my reflection in the mirror. She only said: “You seem to know a hell of a lot, Berky.”

“I can add two and two.”

“Why all this talk? What’s it supposed to lead to?”

“I have to find out where I stand if you keep coming up to this room. I’m no hero. But if Don let somebody get away with marrying Lily when he still wanted her, why should he get tough with me about you?”

***

Her face was properly lacquered.

She turned and leaned languidly against the dresser, the three tiers of her dress jutting. “Don’s a funny guy,” she said thoughtfully. “He’s a poker player twenty-four hours a day, keeps things to himself. But when he breaks loose, he’s bad. I think that’s what made Lily divorce him. She liked money more than any woman I ever knew and Don has plenty. She liked Don, too. He’s a woman’s man, but never mind that. Lily gave him plenty of reason to get jealous and explode. Perhaps the reason she married that kid Linn in such a hurry was to get away from Don, even after she had the divorce. There’s something in having a legal husband that even Don would be afraid to buck. Yeah, she was scared of him. So is Walter and so are you and so am I. I’m coming up here for a few reasons. One of them is I don’t want to get you into trouble.”

“He didn’t do anything to Linn.”

“Like I said, that was marriage. Me coming up here is cheating on him. Besides, Linn went overseas a few days after he married Lily and didn’t come back till the night he murdered her, so you can’t tell what Don would have done.”

I had maneuvered up to it. I said: “Maybe it was Don who murdered Lily.”

“Two years after she married Linn?”

“It was the night Linn came back to live with Lily,” I said. “Don had been up to see her at a party only the week before. You were there. He must have got her off by himself and asked her again to come back to him. She turned him down. He couldn’t stand the thought of another man coming back to live with her. The next Saturday night he drove up there alone and killed her.”

Her artificial lashes almost covered her eyes. “How do you know so much about what happened?”

“That newspaper article was pretty detailed.”

“There was nothing at all about Don being married to Lily in that local paper he subscribes to. Nothing about me either or that party at Lily’s the week before.”

The air in the room was suddenly stifling. “What local paper?”

“A weekly sheet, the Hale County Star. That’s the upstate county where Lily was murdered. Don subscribed to it because it was the only paper that carried all the details of the murder and the trial. When it came in the mail, he’d spend hours sitting and reading the same words over and over. Then when he read that Alexander Linn was acquitted, I couldn’t talk to him for a whole day. I was; almost afraid he’d go up there and do his own executing, since the law wouldn’t. Maybe he would’ve if Linn hadn’t murdered Schneider only two days later and then taken a powder.”

***

That was why my face had been familiar to her and Yard when they’d met me. Twice within two months they’d seen my photo in the Star—an inaccurate and poorly reproduced halftone which had been like a face vaguely glimpsed in a crowd and remembered without detail. If they had seen my photo after I had come to them, after they had my living features before them, they would have made the connection.

I said: “The writer of the article I read must have scooped the local reporters on Don’s marriage to Lily, and the other dirt.”

That may have satisfied Bertha as to the source of my knowledge. I couldn’t tell. She appeared to have lost interest in the topic. She was facing the mirror and setting the layer cake on her flame hair.

“How can you be sure Don didn’t kill Lily?” I said.

“It’s plain that Linn killed her.”

“He never stopped insisting that he was innocent.”

The hat was precariously in place. She started to draw on her gloves. “Now I see what you’re after, Berky. If you can get Don jailed for murder, you can have me without worrying about him. You’re barking up the wrong tree. First of all, Don wouldn’t have harmed a hair of her head. Maybe he’d kill the guy who married her, but not her. Second, he was on Cape Cod the night she was murdered.”

“Are you sure?”

“I ought to be. I was with him. We drove out there Friday. We were to stay a week or so. Monday morning Walt phoned Don from New York that he’d just read a couple of lines in the morning paper that Lily had been murdered Saturday night. Don drove right back to New York, as if he could do anything about it, and then he called up the Hale County Star, and got the dope about the murder. Her death hit him like a ton of bricks.”

“Where was Walter that Saturday night?”

“How should I know? Somewhere around New York, I guess. Say, don’t tell me you got an idea that Walt killed her?”

“No,” I said. “All right, Don didn’t kill her. He wouldn’t harm her and he was on Cape Cod. But Schneider had been her lover. Maybe Don killed him.”

“Why only Schneider? Why not Linn, and while he was at it, all the others she’d ever messed around with?” She shook her head. “Berky, there’s nothing in it. You’re not going to get Don out of the way like that. Even if I thought you could, I wouldn’t play along. I’d keep my mouth shut.” She picked up her handbag. “I should’ve phoned you instead of coming here.”

I rose from the bed and put my hands on her shoulders and kissed her. It wasn’t like the first one. She broke it up after a couple of seconds.

“You’re a sweet boy, Berky,” she said listlessly, “but you spoiled it by talking too much. Your ticket for the show will be waiting for you at the box office.”

“I can’t make it,” I told her. Broadway and public places were out of bounds for me. I knew hundreds of airmen all over the world, and sooner or later some of those who survived would turn up in New York, which to a man back from action, meant bright lights and nightclubs and theaters.

Her eyes were screened by the artificial lashes which were red to match her hair. “You let a man kiss you and he gets sore at you,” she said. “You can never give a man enough.”

Here was the chance I’d been waiting for, to become socially friendly with the three of them. Bertha had kissed me and Yard had asked me to the theater. The ice was broken. I’d been gambling from the first and taken plenty of risk. Why not a little more? “All right, I’ll be there,” I said.

There was no farewell kiss. My cheek was patted with a gloved hand and then she was gone.

***

A bottle of milk was in the refrigerator. With a glass of it in my hand, I opened the top righthand drawer of the dresser and took out the sheets of yellow paper on which I had written variations of the equation. I pushed aside the embroidered scarf which covered the dresser top and spread out a fresh sheet of paper and wrote:

The following are the possible suspects: Miriam Hennessey, Ursula Hennessey, Kerry Nugent, George Winkler, Oliver Spencer, Bevis Spencer, Helen Spencer, Ogden Gar back, Don Yard, Walter.

I read over the list as I drank down the milk. Then I added, “Owen Dowie,” and under it I wrote:

Which of the suspects will satisfy the linear equation with a single unknown which results from the following hypotheses?

Let X=the Suspect.

Let OL=Opportunity to murder Lily.

Let KL=Knowledge of my movements in Lily’s murder.

Let ML=Motive for murdering Lily.

Let

I left it there and went into the bathroom to shave. I returned to the dresser and picked up my pen. There were eight terms in all. I didn’t finish the summary. I could see which was the only one of the ten X’s the equation could satisfy.

So what did I have? I was using a system of trial and error in a technique which required fixed values. I slammed down the pen. The point broke.

At once I regretted my rage. That pen had served me through college and in the Air Force. Many years ago Miriam had given it to me for a birthday present.



    
    \begin{ChapterStart}
    \vspace{3\nbs}
    \ChapterSubtitle[l]{Chapter ch18}
    \ChapterTitle[l]{ch18}
    \end{ChapterStart}

    \FirstLine{\noindent ### Chapter 18}

    
## The Pay-Off

I reached the theater after the first act curtain and in darkness sidled past Walter and Don Yard and Bertha and sat in the empty seat on her right. Her fingers groped for my hand and stroked it. I snatched my hand away. Was she crazy, with Don Yard sitting on the other side of her?

When the lights came on for the intermission, it seemed to me that the theater was filled with uniforms. You’d have thought the war wasn’t over. I dipped my face into my program.

“Anybody coming for a smoke?” Walter asked.

Yard and Bertha said that they would sit the intermission out. I said, “Same here,” and at the moment I heard my name called.

“Hey, Alec!” the voice shouted enthusiastically.

Clip Larsen was standing at the aisle seat four rows down and waving to me. I went hollow inside. I glanced to my left. Yard and Bertha sitting and Walter standing were watching Clip. When he saw that I wasn’t coming to him, he started to come to me.

I sprang up to intercept him before he got close enough for them to hear whatever he would say to me. I tripped over Yard’s feet. “Pardon me; there’s a friend of mine,” I panted. Walter pushed back against his raised seat to give me room to pass. I met Clip halfway.

One thing was certain: he didn’t know about Lily or that I was wanted by the police. Otherwise he wouldn’t have shaken my hand so heartily and introduced me to his mother and the girl he was engaged to marry. For half a dozen flights he had been bombardier in our B-29. Then he got dengue fever and after he recovered he’d been transferred to another ship. There was a silver bar on his shoulder now instead of a gold one.

For a small eternity I stood there at their seats shooting the breeze with him and his mother and his girl. Clip was leaving at two in the morning for a Florida base and the women felt bad about it, though he hoped to get his discharge before winter. I saw Walter return from the lobby. I kept talking, I don’t know about what, though , airmen are always full of gossip. When the curtain buzzer sounded, I had to go back.

Don Yard asked the inevitable question for the three of them as soon as I was settled in my seat. “What did that soldier call you?”

“Al,” I said. I leaned past Bertha, lowered my voice discreetly. “My name is really Alfred Jeffery. I dropped the Alfred and added the Berkowitz when I was kicked out of the Army. I became friendly with Lieutenant Larsen in a hospital in India. We both had dengue fever.”

It seemed to sound good enough to them. Yard nodded and watched the second act curtain rise.

***

As soon as the show ended I explained that I was going to the men’s room and left in a hurry. For a full ten minutes I stayed out of sight. By the time I came up the theater was empty. The three of them were waiting for me under the marquee.

“Walt and I have some business,” Yard said to me. “How about doing me a favor and taking Bertha home?”

I said I’d be glad to. Yard and Walter walked off and I left Bertha to hunt up a taxi in the after-theater rush. I walked from Forty-fifth Street to Forty-first before I managed to drag an empty one back with me.

In the taxi I took Bertha’s hand. She let it lie limp. She didn’t turn her face to me. With Yard beside her in the theater she had been anxious to touch me. How can you figure out a woman?

“What’s the matter?” I said. “Change your mind since this afternoon?”

Her voice was dull. “It’s the other way around. I found out I’ve got it bad for you.”

“And now you’re scared of Don?”

“More than you think.”

I didn’t like it. I had no desire to get in too deep with her now or to hurt her later. She was all right. I would have done better with her than with Lily, if I’d had any business with either of them or any woman like them.

We held emotionless hands until we reached the gray brick house. I paid off the driver and followed her to the private entrance under the stoop.

“Come in,” she said woodenly.

I tagged in after her. She switched on the living room light and dropped her cream-colored ermine wrap on a chair as if it were a rag. She wore a green strapless evening gown. Her figure had no trouble holding it up without visible supports. This was like a sophisticated comedy, the scene set for passion. I wasn’t particularly happy about it.

But she wasn’t having any just then. She went to the bleached oak desk and opened a lower drawer. A pile of newspapers lay in it. I knew then what was coming. I stood watching her, numb and without the ability to think. She selected one paper, looked at it, then walked over to me without sex in her movements. She dropped the paper on the coffee table.

“Alexander Linn,” she said.

***

I just stood there.

She said down to my photo in the newspaper: “That officer in the theater called you Alec. It was al, like you said, but Alec. Since Lily was murdered this paper has come to the house every Thursday and its full of the name of Alexander Linn, or Alec Lin as even the paper calls him mostly. And it came out at the trial that Linn was a crackerjack poker player and he’d been fighting in India. Like Jeff Berkowitz. And how did you know so much about Lily’s murder and why were you wanted to talk about her if you’d never known her? I sat between you and Don in the dark theater and tried to remember the picture of Alec Linn I’d seen in this paper. I couldn’t, but the first time I say you, you looked like somebody I’d seen before. During the show I was so scared I nearly died.”

I was silent.

Her head came up. “Get out! If I caught wise to you, Don will, too. He’ll kill you!” Her voice went strident. “Don’t just look at me. Start running.”

“I didn’t murder anybody,” I said.

She didn’t listen to me or hear me. She fell against me. I felt her mouth on my chin and on my cheeks and finally on my mouth. “I never though anybody could get inside me like you have,” she said. “But it’s too later. It was too late before I even set eyes on you.”

I dug my hands into her bare shoulders and held her a little way off. “Listen to me! I’m innocent!”

“I know you were acquitted of Lily’s murder, but the cops are after you for shooting Schneider.”

“Damn it, I never murdered anybody!” I yelled at her. “Why do you think I cam to New York to get close to Don Yard? He’s as dangerous to me as the police, but I’m grabbing at straws to clear myself. You’ve got to help me.”

We stood two feet apart with my hands on her shoulders connecting us.

“Berky, are you on the level?”

“I was framed for both murders.”

“Is that why you talked so much about it in your room this afternoon?” Light went out of her eyes. “You think Don killed Lily and then framed you because he hated you? But then he would have known you. How could he frame you without ever having see you before? And he was on Cape Cod with me that night.”

“Is that the truth?”

“I won’t alibi him. Not now. I’d see him in hell if that would give you to me.” Her body was hard against mine. “Berky, you’ve got to get out of town. Maybe we’ll meet somewhere. You can’t hang around waiting for Don and the cops.”

***

I held her close. This was the rottenest thing I had ever done, using her like this. I’d let her down as easily as I could, in a little while.

“There’s a chance you can help me,” I said into her hair. “I think I have what I need, but I’m not dead sure. Lily maintained some sort of contact with Don during the two years she lived in West Amber. He took her out on the Fourth of July and vested her a week before her murder, with you and Walter and others. I don’t know what I want or what there is, but maybe something she said to you that Don told you she told him.”

“Berky, that phone call! She eased herself away from me. My fingers had left marks on her bare shoulders. “That’s it!” she said excitedly. “Lily phoned Don a couple of nights before she was stabbed. She was afraid of somebody.”

“Who?”

“She didn’t say. It was Thursday night. I’m sure because we were packing to leave for Cape Cod next morning. I answered the phone and it was Lily and she asked for Don. We were in the bedroom. Don took the call on the extension in this room. I listened in on the bedroom phone. I wanted to hear whatever Lily had to say to him. I don’t have to tell you why.”

“Why was Lily threatened?”

“She didn’t get a chance to tell Don. All she said was that some crazy guy had said he’d kill her. She wanted Don and Walt to come up to West Amber and scare the daylights out of him. They could do it. Once somebody was hedging on a bet he’d lost to Don. Don and Walt took him for a drive in the car. They didn’t say much. They just let him look at a gun, like they did with you the time they took your roll. Then they drove him home. Don said the man’s legs were so weak he could hardly walk from the car across the sidewalk to his house. The next day he paid up. That was what Lily wanted Don to do to whoever she was scared of. Don was sore at her. I know why; he’d been angling to get her back and she’d turned him down. So he told Lily to get out of her own messes with her boy friends. Lily said this wasn’t anybody who had ever touched her or wanted to. Don said, ‘Well, this is the first guy I heard of who didn’t want to or you wouldn’t let,’ and he hung up on her.”

I sat down heavily and went about the business of lighting a cigarette in order to keep my hands occupied.

Bertha said: “Then when we heard Lily was murdered, Don assumed it was you she’d been scared of. So did I. But now it clears up. If she was afraid you’d kill her when you came home, why did she stay there in West Amber waiting for you? And wouldn’t she have told Don it was you? She spoke of this guy who was threatening her like he was nobody at all to her. That’s why I believe you. You didn’t do it. That other guy did.”

She came over to my chair and took the cigarette from me and killed it in an ashtray and dropped down on my lap. “Berky, darling, does anything I’ve told you help?”

“Yes.”

“You mean you know who did it?”

“Yes. But I haven’t the proof.”

“Oh, God!” She put her face into my shoulder. I could feel her breasts heave, but I couldn’t see if she was weeping soundlessly or breathing heavily. I held her tenderly. We sat like that, as if by inaction we could postpone reality.

***

I should leave now. There was nothing else for me here. I had what I had come after. All that was left for me was to take it to District Attorney Hackett and slap it on his desk.

I would say: Call off your bloodhounds. This shows who the murderer really is.

He would say: What’s all this arithmetic?

I would say: The evidence. A beautifully satisfied equation.

He would say: And you are convinced that this proves that X murdered Lily Linn and Emil Schneider?

I would say: I am.

He would say: Marvelous! This will revolutionize police science. All we will have to do in the future is to have X equal a formula and we’ll have our criminal.

I would say: Well, it’s not exactly as simple as that. The special set of circumstances in this case make it possible.

He would say: You disappoint me greatly. For moment I hoped for considerable unemployment of policemen. However, I thank you from the bottom of my heart for solving this case for me so simply.

I would say: My reason was wholly selfish.

He would say: All the same, I thank you. It will make me ashamed to collect my pay for the next month. I will merely have to read your equation to a jury and the conviction of X will follow automatically. Any little thing I can do for you to express my gratitude?

I would say: I don’t want anything except to be let alone and go home.

He would say: Now that’s unfortunate because that is the one thing I cannot do for you. It happens that I am placing you under arrest for the murder of Emil Schneider. My regret is that under law you cannot be tried a second time for the murder of Lily Linn, though that is really of small importance. We have not yet found a way of executing a man twice.

I would say: But my beautiful equation!

He would say—

She stirred on my lap, snuggled closer. Her mouth crawled up the side of my neck and along the ridge of my jaw. “Berky, what are we going to do?”

“You don’t want any part of me,” I told her. “They’re hunting me and if they catch me they’ll kill me. Hold onto what you have.”

“What have I got? I want you. Maybe we can work something out. There’s poker all over the world, and you can make a living everywhere they play it.”

So I had to let her have it. I was a coward. I turned my head away so that I would not have to look into her eyes, and I saw Don Yard.

***

He was standing just inside the door leading from the rest of the apartment. He’d come in soundlessly through the back door to spring his trap. One of us would have noticed him before this if he had been there long. But a single moment was enough.

He removed a dying cigar from his craggy face. “Very touching,” he observed.

Bertha’s head butted my chin in twisting toward the voice. “Oh, God!” she moaned. She sagged lower on me, suddenly dead weight.

Don Yard came toward us as wide and inexorable as a tank. His arms hung loosely with the short, strong fingers spread. Then Bertha’s hair hid him from me. It got in my mouth. I heard her say thinly, “Don!” but she hadn’t the strength to entangle herself from me. I slid my face free of her hair and saw him again.

He had reached the white coffee table and had picked up the Hale County Star. He looked at my photo.

“Is he Linn?” Walter asked.

Walter’s presence hadn’t registered before this. He’d been only the background, standing against the farther wall with both hands in his pockets and cigarette smoke weaving up over his long, impassive face.

“What do you think?” Yard said. “It bothered me where I’d seen him before. It was in this paper.” His voice was a tired monotone. “Alexander Linn, the rat who likes the women I like. He stuck a knife in Lily and now—”

The rest died off into a silence that said: And now Bertha. There was no answer I could make.

Casually he pulled a snub-nosed automatic pistol out of a pocket. “Get off him, Bertha.”

“No, Don!” She threw both arms around my neck, sitting high on me now. Her half-bared perfumed breasts pressed into my face. I couldn’t see or breathe. Suddenly I felt more ridiculous than afraid.

“Don, he didn’t kill Lily!” she said. “For heaven’s sake, give him a chance.”

“I gave him a chance to be alone with you,” Yard told her in that dead voice he was using tonight. “He took Lily and he took you and he killed Lily.”

“He didn’t kill Lily. Listen to me, Don. Remember that night Lily phoned you and said she was scared of somebody. It couldn’t have been Berky—Alec Linn.”

“Get off him!”

“No!”

Facing death was bad enough without the added indignity of a shield of half-naked flesh. I pulled her arms from about my neck. I strove to rise, taking her up with me. Resistance went out of her. She rose to her feet under her own power and took a couple of swaying steps to the side of the chair and then reached out to cling to its high back. I stood up as soon as I was free of her weight.

“If you kill me, you’ll let Lily’s murderer get away with it,” I said. “I’m the only one who knows, who he is.” The gun lay flat on Yard’s broad palm. His finger wound about the trigger. “You can’t say a thing I want to hear.”

Nothing. Convincing him that I hadn’t murdered Lily, if I could convince him, still left my marriage to Lily and my love-making with Bertha.

***

Walter had come across the room to Yard’s side. The cigarette bobbed with the motion of his lips. “Not here, Don.”

“You think I want to spoil a rug I paid three grand for?” Yard said. The gun cut a small arc in the air in front of my chest. “The car’s over on Ninth Avenue. We’ll go through the alley. Start moving, rat.”

This was the way it was done in the remote and not quite real world of the newspapers and the movies. A man steps into a car and rides a few miles to a desolate section of the suburbs and ends up with a bullet in him. Dying, or the anticipation of dying, wasn’t the hardest part of it. For two years I’d steeled myself, with a certain amount of fatalistic success, to accept the coming of death at any hour. But to die like this, pointlessly and without a chance to prove to Miriam and Ursula and the others who cared for me and had once respected me that I had never murdered, was bitterly unfair.

“I can give it to you here and we can carry you away,” Yard said. “Make up your mind.”

I started to shake with uncontrollable rage. Nearby somebody whimpered.

“Yellow,” Walter said scornfully. “How’d they ever let a yellow-belly like this in one of those big bombers?”

The whimpers were coming from me. I locked my lips against them and put my unsteady hands behind my back. “Walk,” Yard said.

Bertha was no longer clinging to the back of the chair. I hadn’t been aware of movement, but now I saw her standing against the bleached oak desk. There was nothing left of her lacquered face but splotches of garish paint.

“I’ll tell the cops you shot him,” she said.

Yard looked at her over his shoulder. “Maybe you’d like to come along?” She withered against the desk. “Don, please! I’ll be your slave. Don’t hurt him. Please!”

“Who wants you any more?” Yard looked at her thoughtfully, scratching his chin with the muzzle of the gun. “This is my job anyway. I can handle the kid alone. You stay here with her, Walt, and keep your eyes on her till I come back.”

He stepped around me and put his free hand at the small of my back and shoved. I stumbled forward, regained my balance, walked toward the door in the far wall of the room. Yard followed.

I passed within five feet of Bertha. She had her back to us now and was sagging broken and helpless over the desk. She did not turn her head for a last look at me. I stepped through the doorways

***

The shots were blurred coughing sounds, two or three or four of them overlapping. When I turned, Don Yard was already on the floor. A hand flapped weakly out to claw the edge of the Persian rug and suddenly died, becoming one with the inertness of the rest of that wide and solid mass of flesh and bones.

The gun was in Bertha’s hand. It was a tiny pearl-handled revolver held against her right hip. She leaned limply against the top open drawer of the desk, the drawer in which Don Yard must have kept that extra gun. Her mouth opened and closed and again opened and closed, but no sound came from her.

My eyes searched for Walter. He was coming slowly toward me with his hands still in his pockets. When he reached Yard, he squatted and took out both his hands. No gun was in either. He felt the motionless flesh. The veil remained over his eyes, but a corner of his mouth started to twitch as if fighting a smile. “You got him good, sweetheart,” he said and stood up.

He listened to the street outside and the house above us. The stillness maintained. It was doubtful if anybody had been awakened by the shots, or hearing them had recognized them as shots. The sound of a .22 revolver is no louder than a hand slapped sharply on a table.

Walter strolled leisurely over to Bertha and reached for the gun. She uttered a thin cry and pulled hand and gun behind her back.

“Better let me handle it, sweetheart,” he said. “You’re in trouble.”

She stared up into his face. Slowly, as if hypnotized by those empty eyes of his, she brought the gun back into view. He plucked if from her fingers and dropped it into a pocket. He put his face very close to hers and spoke. Those soft, slurred undertones did not quite carry to me.

I didn’t care what he was saying. It was up to me, to think faster and more accurately than ever before in my life. And I couldn’t. I was shaking hardly at all. Physically I was all right; I was holding together. But that might be only a hopeless acknowledgement of final defeat as I stood beside a dead body for the third time in two months. Three strikes and out. Bas!

Bertha and Walter were coming toward me. His arm was intimately, possessively, about her waist. Her stride was unhampered by fear or strain; her face was smoothly in place. The picture didn’t quite fit.

They skirted around the dead man. Bertha’s eyes were fixed straight ahead so that she wouldn’t have to look at him or at me. Walter stepped behind her to give her room to get through the door. He looked back at me.

“Let’s talk in the other room,” he said.

***

He was snapping the light on when I entered. It was the dining room. Bertha stood at the wall mirror pinning up her hair. It had become somewhat undone while she’d been sitting on my lap.

“Here’s the play, kid,” Walter said to me. “First the part that concerns me and Bertha. We didn’t come home from the show with Don. We went somewhere for drinks and maybe a party. He told us he had a date with somebody here at twelve, though he didn’t tell us who, and went right home. The idea is he wanted to be alone with whoever he was meeting. I’ve got plenty of pals around who’ll alibi Bertha and me, guys who owe me favors. In an hour or so I’ll get one of them to phone the police on a dial phone that he was passing this house and heard shots. He’ll say he doesn’t want to be bothered with red tape, so he won’t give his name. He’ll hang up and beat it. All right. The thing is, when the call comes in, Bertha and I will be where a hundred people will see us. Some of them will say we’ve been there since eleven-thirty.”

“I’m wanted for murder anyway, so one more won’t matter,” I said grimly. “I take the rap for Bertha. Is that it?”

“No, Berky,” Bertha said, giving her hair a final pat. “We’re all in this together and we’ll get out together.” Walter nodded. “That’s it, kid. All in together, like the three musketeers. If the cops pick you for this, you’d say Bertha did the shooting—”

“To save my life from Don. That’s legitimate.”

“Maybe a jury will believe that’s why she shot him, maybe it won’t. We don’t have to take the chance. If the cops pick you up, you’ll say it was Bertha and they’ll break her into confessing. They’ll do it; I know how those babies work. So the idea is to make sure the cops don’t pick you up for this. If they don’t you won’t go to them and squeal on Bertha because you’re wanted for another killing. There’s no chance of you and Bertha double crossing each other.”

“Where do you come in?” I said.

He sent a thin smile to Bertha. She looked down at the floor, then sat down at one of the dining table chairs and elaborately smoothed out the tablecloth. It cleared up then. She had made a deal with him. He had always wanted her, and now he could have her if he got us both out of this.

“I’m doing it for Bertha,” he told me with that ghost smile pasted to his long face. “Now start thinking, kid. How many people know you were working for Don?”

“Nobody, unless Earl Locust was told.”

“Not a chance. It was good business to keep the set-up under wraps. You met Don in his place, but so did a lot of other guys. How many people know Jeffery Berkowitz?”

“Locust and my landlady, Mrs. Egan. She knows even less about me than he does.”

“Good. What about you as Linn? Anybody know Linn is in New York?”

“Two people. One was Clip Larsen who saw me in the theater tonight. He has no idea that I’m wanted by the police, and he’s leaving for Florida in an hour or so.”

“You said two people.”

“Robin Magee is the other. He was my lawyer, and last night he met me in Locust’s apartment. He’s the one person outside of Bertha and you who knows that Jeffery Berkowitz is Alec Linn. I’m still his client in a way, so he promised not to tell the police.”

“I know Magee. One time he hid a guy out for a month till he got his defense all set. He’s a smart mouthpiece. You don’t have to worry about him. If he opens his mouth now, he makes himself an accessory because he kept quiet about seeing you last night. See what I’m getting at, kid? Bertha and I will get ourselves alibis. We can’t do the same thing for you because you’re hot to begin with. So we have to be careful nobody ties up Berkowitz and Linn and Don Yard or the cops will start getting ideas. Okay, then, you’re set. You check out of your room tomorrow morning and scram.”

***

Bertha stood up and looked through the doorway into the living room. One of Yard’s outstretched hands was visible to me and probably to her. She turned her head away, and she broke a little. “Hurry!” she said. “I can’t stand it.”

“A couple of minutes yet,” Walter said. “We’ve got to get rid of your fingerprints, kid. Mine and Bertha’s don’t count. We live here. But the cops have yours on file and might get an idea to check. Take your handkerchief and wipe everything you touched.”

I hated to go back into the living room, but I had to. I started with the doorknobs and ended with the chair in which Bertha had sat on my lap. Walter came in to fetch Bertha’s ermine cape and handbag and went out without a word to me. I was about to follow him when I remembered the light-switch. I wiped it.

Don Yard blocked my path back into the dining room. It was somewhat grisly the way we had to walk over him or sidle around his outstretched hands to pass from one room to the other. I stopped at his feet to look down at him. For the first time I saw the hole in the back of his head. Or rather where a small patch of the short hair at the nape was matted with blood. Only a tiny lead slug from an almost toy-like .22 revolver. Not as deep as a steak knife plunged into a woman’s heart or as efficient as a rifle shooting down a man on a porch at pointblank range, but it had sufficed.

A rifle, I thought. Three corpses and three different weapons, but only a rifle presented a serious problem after the act. The steak knife could be left in the body without danger to the killer after the print had been wiped off, the revolver Bertha had used could easily be disposed of by Walter, but a rifle had bulk and individuality. It was conspicuously connected to the owner, familiar possession, a piece of furniture decorating a wall.

“Aren’t you through, Berky?” Bertha called.

I roused myself. I stepped carefully around the outstretched hand clinging in death to the snub-nosed automatic and entered the dining room. Bertha was alone. I heard Walter opening and closing a drawer in another room. Probably he was after the large sums of cash a gambler like Yard would keep at hand in his apartment.

***

She had the ermine cape over her shoulders and was renewing her face with the contents of her handbag. “Are you set, Berky?” she asked with a pencil at her eyebrows. As if I were calling on her to take her out on a casual date.

She was working at being hard as nails. That was veneer, protective covering to survive or endure the narrow shadow world she lived in. For a few minutes tonight she had shed it for me, but Don Yard’s death had made her put it on again. Covered by that layer of toughness, she could take the killing of one man and bargaining her body to another man in stride. It held up the outer shell of her, whatever was happening inside of her.

I said: “I can’t let you do this for me.”

She zipped the handbag closed. “Say, you don’t think I shot Don because of you? You didn’t see me going for a gun until he told Walter to watch me. I was scared crazy of what he’d do to me when he got back.”

“Maybe,” I said.

“Oh, sure, the fact that he was taking you for a ride had something to do with it. But the real reason I did it was I was looking out for this little girl’s health.”

She sounded tougher than I had ever heard her. She was putting it on thick for my benefit.

“I mean I can’t let you go with Walter,” I said.

“He’s not such a bad guy. In my time I’ve been with worse.”

“Go away with me instead. That’s what you want to do, isn’t it?” I looked back at the gun clutched in Don Yard’s dead hand. “I’ll take care of Walter.”

Don’t talk like a kid. What do you want to do—kill him?”

“No.”

“Then what? You and me run out together? How far do you think we’ll get with Walt phoning the cops before we can go a block? Say you tie him up—the cops will still be after us, maybe a day behind, but after us. You think I want that when I can sit safe and snug with Walter. You’re poison to me, Berky.”

“Poison,” I said. “I did this to you. If I hadn’t come around, you would never have had to shoot Don.”

“What the hell, I asked for it. I threw myself at you.” She looked at me for a long time and a remote smile whisked over her red lips. “For a while I kidded myself, but I know better now. You never said you loved me. All you wanted out of me was dope on Lily’s murder.”

“I like you a lot.”

She kissed me. It was a brief, sisterly kiss. “That’s sweet, Berky. You think you have to ask me to go away with you because I saved your life. Only I’m going with Walter. That’s the only way it can work out for both of us.”

Abruptly her voice was harsh. She said, “Let’s go,” and swung away from me.

We found Walter in the kitchen. He stood at the sink, filling a glass with water. He drank it down without hurry-

“Bertha and I will go out first,” he said. “Wait a couple of minutes, kid. Don’s car is on Ninth Avenue. Don’t touch it. Keep going.”

Then they were gone through the back door. No farewell from Bertha. No final look back at me. She was gone with Walter.

I remained in the kitchen for the space of two hundred heartbeats. In the dark, littered yard I remembered that I had just closed the back door behind me. I returned and wiped both doorknobs and then groped my way up the narrow alley to Ninth Avenue.



    
    \begin{ChapterStart}
    \vspace{3\nbs}
    \ChapterSubtitle[l]{Chapter ch19}
    \ChapterTitle[l]{ch19}
    \end{ChapterStart}

    \FirstLine{\noindent ### Chapter 19}

    
## The Return

The white dress revealed her seated on the top porch step. Her hand lifted to the darker shadow that was her face. The cigarette glowed as she drew on it, momentarily highlighting her features in a soft blur.

It was forty-five hours since Don Yard had been killed. Next day I had checked out of the rooming house at noon—yesterday. I had ridden the subway to the north Bronx and thumbed the rest of the way. I could have made West Amber by this morning, but I had tried to time it to arrive under cover of darkness. That hadn’t quite worked out. My last lift, a salesman bound for Cleveland, would have passed through West Amber a couple of hours before twilight. I had dropped off three miles before the town and had hidden in woods until darkness permitted me to walk unseen.

The lights of the house were bright and hospitable and cozy. I hadn’t had real sleep in two nights, I was grimy and unshaven, I was still a fugitive—but I felt more secure and rested and looser inside than in months. In years. I was home. That first homecoming nearly two months ago had been a continuation of the horror, subdued usually, sometimes bursting in your brain like shrapnel, that was war and living in time of war. This was home now with all I wanted in it—a girl waiting for me on a porch step.

I cut across the lawn.

She saw me when I crossed the driveway. She stood up, poised on the step as if to plunge off it. I couldn’t have appeared more to her than a human form without identity or distinguishable shape, but suddenly she dropped her cigarette and raced down the steps. Then she was in my arms, and I was kissing her as I had never kissed any woman but Lily. Not even Lily, for there was a consuming yearning in the meeting of our mouths that was passion and beyond passion.

“Darling, I knew you’d come back,” Miriam said at last, the words rushing from her in a flow of emotion. “I’ve been waiting for you. Ursula has been keeping a lot of cash in the house for you. We’ll go away together. To South America. Anywhere you say.”

“West Amber is a good place to live,” I told her.

“Please, darling, don’t be flippant or stubborn now. I couldn’t stand it. And don’t go into the house. We’ll wait in the garden until they leave.”

I stiffened. “The police?”

“Almost as bad. Kerry and George Winkler are having a conference with Ursula. George just returned from New York. He wants to tell the police. I couldn’t stay in there listening to him. That’s why I was sitting outside.”

“George wants to tell the police what?”

“About Don Yard. He knows.”

***

I couldn’t see her face, but her voice was broken and hopeless. “For God’s sake, speak in a complete sentence. What do the police know?”

“Nothing. But George and Robin Magee know.”

The knots that had started to form untied. It was all right. I was safe and Bertha was safe. Or were we?

“I didn’t kill Yard or anybody else,” I said. “With luck, I’ll be able to clear this up tonight. Then we’ll get married and I’ll find a job and we’ll live decent, normal lives. I’ll work it out. Do you hear me?—I’ll work it out tonight.”

Her face was against my chest. “Darling, I love you so,” her voice came muffled. “I can’t think of anything except how much I want you.”

“If I ever stop being a model husband, remind me what a dope I’ve been all these years,” I said. “I love you. That’s rhetoric, but I’ll spend the rest of my life showing you what I mean. All I ask is that you have faith in me for just a little while longer.”

“Yes, Alec,” she replied with nothing at all in her voice.

It was unfair to demand too much from faith. I left it at that. I put my arm about her waist and turned her toward the house and together we went in.

Everything stopped dead when Miriam and I entered the living room. A ghost couldn’t have done it better.

Ursula was the first to move. She threw her arms about my neck, and words spewed from her in a half-sob. “Alec, I’ve money here, waiting for you. You’ll have to leave at once—tonight. Miriam insists on going with you, but she mustn’t. If you care for her at all, you can’t, let her throw herself away like this.”

“Don’t worry about either of us,” I said. I stepped around her and over to Kerry.

His two-month leave had done him a lot of good. He looked ruddy and fit, though he was bulging a trifle around the middle. He attempted a grin and gave it up and fumbled at his belt in embarrassment. I stuck out my hand. He looked down at it and wet his lips and then accepted it. His grip lacked its usual heartiness.

“How’s the boy?” I said.

“All right,” Kerry mumbled.

“There can’t be much of your leave left.”

“I’m reporting back in a couple of days. I’m hoping they’ll let me out soon.”

“How’s with you and Helen?”

“We’re engaged.”

It wasn’t a satisfactory dialogue. The words dribbled out of him as if there were no thought behind them.

***

George Winkler’s bear body was deep in a chair. From it he watched me angrily. I extended my hand to him. He ignored it, keeping his fists at his jowls.

“Don Yard was shot,” he said.

“I read about it in the paper,” I said. “There was little detail. I guess Don Yard wasn’t as big a big-shot as he liked to believe. Merely another gambler killed.”

“So you admit you knew him?”

“What did Magee tell you?”

“What the hell did you come back for?” George burst out. “Every time you kill somebody, you come running home. Why don’t you put a bullet in your head and let Ursula and Miriam alone?”

Behind me there was a gasp of horror—Miriam or Ursula. I didn’t look around.

“I’ll answer your questions after I get some details,” I said to George. “How much does Magee know?”

“Enough to phone me this morning to drop everything and rush to New York. He didn’t dare tell me over the phone. A few days ago he’d met you with a fast poker crowd in New York. You were going under the name of Berkowitz. After he read that Yard was murdered, he got in touch with the man whose house he met you in—Earl Locust. From Locust he learned that you had been anxious to meet Yard and that you had met him during a game in his place.”

“Is that all Locust told him?”

“He doesn’t know that Berkowitz is Alec Linn, if that’s any satisfaction to you. It isn’t to me.”

He was angry clear through. Angry with me for having put him in this position, and with himself because he hated to do what he felt he had to.

“It’s the pattern that gets you,” I said. “I killed Lily and am out to get everybody who ever made love to her. I suppose my next victim will be Bill Beaty when he comes back from the wars.”

“Alec!” Ursula cried. “Alec, don’t talk like that!”

“He’s stark, raving mad,” George said heavily.

“All right, so I’m mad,” I told him. “Even so, I’m entitled to details. The papers carried very few.”

George looked up at me with his lips drawn in. He pushed them out. “Magee has an in at police headquarters. The police received an anonymous phone call which they couldn’t trace that shots were heard in Yard’s apartment. I suppose you made that call. They found Yard dead with an unfired gun in his hand and with the doorknobs and light-switches and some of the furniture wiped clean of fingerprints. Yard had been at a show with Bertha Kaleman and Walter Herring. When the show was over, he told them he was to meet somebody at the apartment and hurried away without telling them who it was. They said he was pretty grim throughout the show. Magee and I know, of course, that it was you he had gone home to meet.”

“You don’t know anything,” I said. “Where were Bertha and Walter?”

“You can’t shunt it off on them. They proved they were at a party when you shot Don Yard.”

“What do the police think?”

“You were smarter this time than the first two times. You get that way with experience. After your fourth or fifth killing you’ll develop the art of murder to a high degree of perfection.”

***

The gasp came again. This time I knew it was Miriam, because Ursula said: “George, I won’t let you say such things in my house.”

“It’s my fault,” George told her bitterly. “I helped him get away with the first one. I pray to heaven he’s insane. It would be too terrible if he did these things while sane.”

He was scourging himself as well as me with words. If he’d talked less, he would be more dangerous at the moment.

“What does the shooting look like to the police?” I persisted.

Talk was what he wanted to get some of his bitterness out from inside him and into the open. “Like the murder of a gambler by somebody who owed him a lot of money. The police figure that Yard and the other man quarreled over money and that Yard drew his gun, but that he was beaten to the shot by the other man. Or that the murder was planned ahead and Yard was shot down from behind. All three bullets had come from behind—two in his back and one in the back of the skull. Yard probably drew his gun after he was hit the first time, or after he was dead the gun was placed in his hand by the murderer. The police suspect it was an underworld character because of the careful attention given to removing fingerprints and because Yard, who, it seems, never wore a gun, took care to wear one when he prepared to meet the killer,” George glowered. “You’ve always been a scholar, Alec. You learn quickly how to do such things right.”

“Do the police suspect the existence of Jeffery Berkowitz?”

“They don’t, according to Magee. Nor that Alec Linn was in New York that night.”

Walter had done a very nice job. He knew how to handle such matters thoroughly and convincingly. Possibly that was the perfection of experience George had mentioned.

“And Magee is frightened,” I said. “If I’m picked up for the shooting and it comes out that Magee had seen me the night before and had kept his mouth shut, his goose is cooked. That was why he sent the frantic call to you. He has to protect himself by protecting me.”

George thumped the arm of his chair. “I don’t give a damn what happens to Magee. I have a certain amount of responsibility to society, if nobody else has.”

Kerry said weakly: “There ought to be some way of handling it without giving Alec away to the police.”

“Not again!” George roared. “Every time we let him get away from us somebody else is murdered.”

“May I have a cigarette, George?” I asked quietly.

***

He stared at me in surprise, then lumbered up to his feet and took out a pack. I struck my own match because I knew that my hands would be steady. I was showing off. I should have been going to pieces, but George was the one who was shouting. What George had told me about the police had propped up a base under my feet. I had something to grip with my toes and stand erect. Whatever had been done to me by the war and Lily was over, past, finished—if I got safely through tonight.

“I didn’t shoot Yard,” I said. “He was killed in self-defense. Never mind why or by whom, but it wasn’t murder. I had to be sure that I was clear on the Yard shooting as far as the police are concerned, and now that I’m sure, I can go on. George, will it convince you that I didn’t kill Yard if I prove to you that I didn’t kill Lily and Schneider?”

I felt Miriam against my side. She reached for my hand and I squeezed hers. Ursula and Kerry watched me with a kind of breathless hope. And George, suddenly, looked more curious than angry.

“What sort of proof?” George asked skeptically.

“I’ve a beginning here—an equation worked out against each suspect.” Out of a pocket I pulled a sheet of paper torn from a notebook. “I wrote it out neatly this morning in an Albany beer joint.” I extended the paper to George.

“Mathematics!” he grunted. “So that’s all you have!” And he did not accept the paper.

Kerry came forward to take it from me. Once more he was the skipper, the man of action. Here was something tangible to be looked at, examined. They crowded about him, Miriam and Ursula on either side, and then even George. Kerry held the paper out so that they could all read the marks on paper which said:

Which of the suspects will satisfy the linear equation with a single unknown which results from the following hypotheses?

Let X = the Suspect.

Let KL = Knowledge of Alec’s movements before Lily’s murder.

Let KS = Knowledge of Alec’s movements before Schneider’s murder.

Let OL = Opportunity to murder Lily.

Let OS = Opportunity to murder Schneider.

Let ML = Motive to murder Lily.

Let MS = Motive to murder Schneider.

Let MA = Motive to frame Alec for both tnurders.

Therefore:

X = KL + KS + OL + OS + ML + MS + MA.

***

Kerry raised his eyes from the paper, frowned at me, and continued to read or reread it. Ursula appeared to be concentrating furiously, trying to make something out of what was beyond her scholastic knowledge. Miriam gave it up. She came to me and hugged my arm and waited for the answer which would mean life to both of us.

“Let me explain it,” I said into the silence. “I’m no longer bitter because you believed I was a murderer and that you still believe it. You, George, put it into words weeks ago—that the odds against me being so completely implicated in Lily’s murder through coincidence or accident were too great. When Schneider was murdered under similar circumstances, the odds were multiplied with each other and became overwhelming. As I knew that I was innocent, I concluded that both murders were deliberately planned to implicate me. There you have the terms, KL and KS, which limit X, the suspect, to those persons who were in this house during the card games on the two Saturday nights of the murders. Which brings us to OL and OS. Everybody who was here the night Lily was murdered could have beaten me to the bungalow. OS eliminated Miriam and Ursula; they were here when I drove away, but the rest had already departed. As for ML and MS—a number of you had reason to hate Lily or want her out of the way, and there could have been other motives that I was unaware of. ML was the least fixed of all the terms, just as MS was the most definitely fixed on the reasonable assumption that Schneider’s murder flowed out of Lily’s, that he was killed because he had seen who murdered Lily and tried to capitalize on it. In effect, ML and MS are connected terms. Satisfy the first and the second is automatically satisfied.”

“Symbols and words!” George exploded. “Anybody can write an equation.”

“Give Alec a chance,” Kerry snapped at him. He looked at me. “Where does the MA term come in. I can’t see it.”

“That was the crux of the equation,”

I said. “It had me stumped for a long time. MA flows inevitably from the other terms. Motive for mortally hurting me. Not killing me. George once pointed out to me that it would have been simpler to kill me than to erect an elaborate framework of guilt around me. Conceivably the killer slated me as the scapegoat, the fall-guy, but the delicate timing involved made it a lot more dangerous than killing me—or both Lily and me, if that was the idea—and then sitting tight. That’s still the most foolproof way to murder somebody—shoot or stab and go home. No, X was somebody who hated me, but had reason for not wanting me dead or not wanting me dead by his own hands.”

George sneered: “Q. E. D. But obviously you didn’t have confidence in this equation. You fled after you shot Schneider.”

“After Schneider was shot,” I corrected him mildly.

“You fled. An innocent man doesn’t.”

“He does if he has to depend only on himself to fight for his life. I had had a taste of what lawyers did to me once I was in jail. I needed time and I needed more information. I wasn’t dealing with fixed values, but I could try to fix them as nearly as possible.”

“How did you fix values by going to New York and killing Don Yard?”

***

I refused to let his anger get under my skin. “I went to New York because I had to go somewhere. It was as good as anywhere else, and Don Yard was there. I didn’t think that he—or even less likely, Bertha Kaleman—had murdered Lily. KL and KS excluded them, but not wholly. Coincidences occur, however improbable. They both had plenty of reason to do away with Lily, and at the time Yard was the only one I could think of who hated me enough to frame me. I had to be sure of facts concerning him before I could limit the equation to KL and KS. And I learned that he and Bertha were on Cape Cod the night of the murder. But that was only part of what I was after—the least part. Yard had maintained some sort of contact for all I knew, intimacy—with Lily while I was overseas, and there was a chance that he or Bertha or Walter could supply me with information which would help. I did get something from Bertha, a sort of proof of the equation. Somebody in West Amber threatened Lily’s life a couple of days before she was murdered—somebody who was not a lover or had any personal relationship with her.”

My throat was dry. I swallowed and sent a smile to Miriam. She leaned against me, hugging my arm. Her face, lifted to me, was filled by her grave black eyes.

“Who is it?” she whispered.

“That’s it—who?” George said. “And giving me a name won’t be enough. There’ll have to be a lot more than mathematical gymnastics to stop me from going to the police.”

All at once my new-found self-assurance was gone. I felt tired, drained of emotion. I had thought that Kerry would help me and that George, a lawyer, would advise me. But that was no good. I had to depend wholly on myself, as I had all along. I couldn’t be sure that the thing I needed was there, and I had to be before I brought the police in. It was still exclusively my job.

“I’d like you to try to work out that equation,” I said. “You should be able to do it with what explanation I’ve given you. If you get the same answer I did, it will be some sort of proof. Meanwhile, I’ll go upstairs to shave and change my clothes.”

George didn’t protest. He stood on one side of Kerry and Ursula on the other, all three absorbed in the paper. Miriam went upstairs to me.

“Listen—I’m sneaking out of the house,” I said to her when we stood in front of my door. “I don’t know how long I’ll be gone. Thirty minutes or a lot longer.”

She pressed against me. “Darling, can you really convince the police that you’re innocent?”

“I think so. There are two ways of doing it. I’ll try the easiest first. If I fail in both—”

“What will you do?”

“I don’t know. But I’ll come back here before I decide.”

“Who is it? Why can’t you tell me?” It would hurt her. Time enough for that later—if there would be a later for us.

“You’ll find out soon enough,” I said. “What I want you to do is watch George. Keep him here until I return. Don’t let him go to the police or phone them.”

“Ursula will help me if necessary. She could always handle him. He’s so devoted to her that he came here tonight to talk it over instead of going straight to the police. He’s going through hell wanting to do the right thing.”

“George is a good guy,” I said. “Now go downstairs and tell them I’m shaving and dressing.”

She would not release me. I kissed her and gently unwound her arms from about me and went into my room.

I stopped only to get a flashlight from my drawer. As I had several weeks ago, I climbed through a window and down the ground by way of the porch roof. Kerry’s and George’s cars were in the driveway. I could have borrowed either, but George would hear the motor and think I was fleeing a second time. It was a nice night for walking.



    
    \begin{ChapterStart}
    \vspace{3\nbs}
    \ChapterSubtitle[l]{Chapter ch2}
    \ChapterTitle[l]{ch2}
    \end{ChapterStart}

    \FirstLine{\noindent ### Chapter 2}

    
## Yesterday and Today

The dinette was separated from the living room by an arch and a step. When I raised my eyes from the table,

I looked directly at the large framed photo of August Hennessey above the redbrick fireplace. He had been a blonde giant, half-Irish, half-Swede, enough man for a woman like Ursula. In the photo he wore a grin as wide as his broad amiable face, and a yellow curl hung rakishly over his left eye. He could have passed for a romantic adventurer out of the more lurid pages of fiction. Actually he had been a salesman of glassware.

Ursula married him when she was eighteen. I was three at the time; our mother had had her two children fifteen years apart. Ursula had been visiting an aunt in New York City, and one day she led that huge man into the house by the hand and announced that they were on the way to Mexico for their honeymoon.

They visited us about once a year, but the first definite impression I have of August Hennessey was his excitement when I beat him at chess. I was seven, and he promised me that during his next visit he would teach me the fine points of poker. That was his game, he said—something like chess in that you must anticipate your opponent.

But when August Hennessey and Ursula came next year it was to attend Ma’s funeral. So there were no games, chess or cards. They wanted to take me back to live with them in the home they had built in the upstate New York town of West Amber, but Pa refused to let me go. He was a meek, taciturn trolley car conductor whom I’d never got close to, not even when I had become old enough to share his passion for chess, but now he needed me in his loneliness.

I never saw August Hennessey again. While on a selling trip to Buffalo, his car crashed into a truck and he was killed.

He left Ursula the house clear of mortgage and, like most salesmen, a substantial insurance policy. I don’t know how much it was, but enough for her to live in comfort and later support Miriam and me. For years there was a rumor in West Amber that had quite an income from her Saturday night poker games, but, of course, there was no kitty for the house. Though she was very good, the men who played with her were no slouches, and I doubt if she won much more than she lost.

Ursula wasn’t one to brood in an empty house. She could have married again; she had money and was young and attractive for a big woman. What she did was to pick out Miriam from an orphanage and adopt her as the daughter she had not had with August Hennessey.

I was ten when Pa died of pneumonia. Ursula drove all night from West Amber to Cleveland. Miriam was with her—a skinny, black-eyed girl of eight who terrified me by throwing her toothpick arms around my neck and kissing me and calling me her big brother. I pointed out to her that, if anything, she was my niece by adoption, but that actually she was nothing at all to me. I also told her that I disliked girls, especially being kissed by them. Which was true at the time, but not for long.

Ursula buried Pa and packed my things and the three of us drove east. When the car climbed Mandolin Hill and I saw the ivory-and-blue frame house, I felt that I was coming home, although I had never been there before. And in a few days it was home, and had been until four years ago when I had enlisted in the Air Force.

***

Now I was back, eating as I had so many hundreds of times in the pine-panelled dinette with Ursula and Miriam. But this was no longer all of my family. I had a wife who should be here or with whom I should be.

Abruptly I put down my knife and fork.

“Lily knew within a few days when I’d get in,” I said. “She’d hang around waiting for me. Or if she went away on a visit she’d leave word where I could find her.”

Ursula reached for my plate to fill it with a second helping of roast lamb and apricots, which nobody in the world could cook the way she did. “I’m not responsible for your wife, Alec. We’ll discuss her after dinner.”

“So there is something to discuss?” I said.

She dipped her head to slice the meat. Miriam ignored the food on her plate to watch her intently. Then Ursula handed my plate to me and said to Miriam: “What have we ever done to him to make him snap at every word we say? Does he think we murdered his precious wife and buried her in the cellar?” She turned an angry face to me. “There’s a great deal to discuss. You’ll have to start planning for jour future.”

It was an old trick of hers to turn my words back at me and making them mean something altogether different. “I never said you haven’t been swell to Lily and me. You’ve supported me for all those years and—”

“Pa left his insurance to you.”

“You saved it to see me through college. Then I brought my wife home, whom you’d never seen before, and you kept her here with you for two—”

“You’ve been supporting her through your allotments,” Ursula said. “I made sure that she contributed to household expenses. Do you know that Kerry Nugent is back?”

I finished the mashed potatoes in my mouth before answering. “He left later than I did, but he flew while I came by ship. Where is he stationed?”

“He’s home on sick leave,” Miriam told me. “He’s been here several times and he looks fine. When you called yesterday that you’d be home tonight, I tried to get in touch with him to tell him. But his mother said he’d gone to New York.”

“New York,” I muttered. Lily was in New York. Kerry had met Lily at the same party I had and he’d tried his best to make her. What the hell was the matter with me? Kerry wouldn’t pull anything like that. If he had had anything to do with Lily since his return, it would be to slap her ears off for the way she had treated me. Sure. That would be Kerry.

“He told us all about that last flight you two were on,” Miriam was saying.

The food turned to straw in my mouth. “Did he tell you how I smashed his ribs and killed Ezra Bilkin?”

They stared at me.

“What are you talking about?” Ursula said. “Who is Ezra Bilkin?”

“He was our flight engineer. He died because I lost my nerve and lost my way home.”

“Alec, that’s not what Kerry said.” Miriam leaned against the table, her grave black eyes filling her face. “Kerry was pilot; he ought to know. He said that after your radio was shot out no other navigator could have got the plane back to land. He said that if anybody else had been navigator everybody in the plane would have been killed.”

“Ezra Bilkin was a swell guy,” I said. “He had a wife and two kids in Indiana. It was my job to bring the ship home safely and I muffed it.”

“But you’d brought it back so many times before,” Miriam argued. “Kerry said you and he flew twenty-three combat missions before—”

“Before I lost my nerve,” I said. “Before flak shook us like a cork in a storm and a Zero planted a bullet an inch from my ear. Suppose the radio was gone? A schoolboy can take a B-29 home on a beam. I went to pieces. Why do you think they grounded me?”

***

URSULA reached across the table and firmly gripped my hand. “There’s no need to shout, Alec.”

So I was shouting again! I was shaking all over and the food I had eaten was bouncing in my stomach. I jerked my wrist away from Ursula and clutched the edge of the table with both hands. I’d never before had it as bad as this. Major Goldfarb, the chief flight surgeon, told me I’d be fine if I avoided undue excitement for a while, and that even if I got these attacks they were nothing to worry about. But Ursula and Miriam looked at me as if watching my dying convulsions.

“I shout,” I said bitterly, “so I’m a mental case.”

“Nobody ever said—” Miriam started to protest and then gave it up.

I pushed back my chair and stood up and went through the swinging door into the kitchen. Then Miriam called: “Alec, please come back and finish eating.”

“Let him go.” Ursula spoke in a half-whisper, but I heard it in the kitchen. “The trip upset him. He’ll be better after a rest.”

“Trip nothing,” Miriam said. “It’s Lily. Ursula, aren’t you going to—”

“Later,” Ursula said. “After he calms down.”

Their voices dropped still lower and I couldn’t distinguish words. I crossed the kitchen to the door at the other end and went up the hall. The doorbell rang before I reached the foot of the stairs. I opened the front door and three men entered. Two of them I knew—Oliver Spencer, who had been at the station when I arrived, and George Winkler.

Years ago Miriam and I decided that George Winkler would marry Ursula the moment she wanted him. He was a bear of a man, big and shaggy, with small bright eyes that crinkled readily and a broad nose that wrinkled when he was amused. He was the most prominent lawyer in West Amber, which sounded more imposing than it was. For a long time now his romance with Ursula had trickled on placidly, without change or interruption, if it was a romance. Perhaps Miriam and I were all wrong and he had dinner at our house once a week and was the mainstay of Ursula’s Saturday night poker games only because an aging bachelor needed an anchor somewhere.

He greeted me as a father might, pounding my upper arms with his large open hands.

“I see you were in a hurry to get back into civilian clothes,” Oliver Spencer observed. “How do you feel in them?”

“Fine.”

That’s the way it was when I came home from the last war,” George Winkler said. “I couldn’t wait to dress in something with color in it. Do you know Owen Dowie, Alec? I guess not. He comes from the other end of the county and was elected sheriff only last year.”

***

The third man stepped forward to shake my hand. Blame Western stories for the assumption that a sheriff is a gaunt, gimlet-eye man with a gun swinging from his hip. Sheriff Owen Dowie looked like a mild and somewhat scared bookkeeper. He had a pinched, rather intense face and pale, myopic eyes peering through thick shell-rimmed glasses.

“Heard a lot about you, Linn,” he said. “I’m told you can make the cards stand on end and call you brother.” Mr. Spencer chuckled. “I’ll lay you ten to one Alec can deal you any hand you ask for and you’ll never guess how he did it.”

“I’m out of practice,” I said.

“What’s the matter, didn’t you find suckers in the Air Force?” Dowie asked with a shadowy smile.

There was a silence. Then George handed Dowie a glare. “Alec never played a crooked hand in his life except to show how it was done.”

“I was only kidding,” Dowie said hastily. “I mean, a player like you should’ve had a good thing with all the poker played in the Army. One man I know came back from Italy with five thousand dollars in winnings.”

“I hardly played,” I said.

Mr. Spencer patted his bald head. “Well, I’ve been able to take you over now and then, Alec, and I’m looking forward to bucking you again tonight.”

“I’m sorry, but I’m not playing tonight,” I said.

Ursula came into the hall and apologized for having finished dinner so late. Would they wait downstairs in the cardroom until she and Miriam cleaned up? The three men went down the hall to the basement door under the staircase, and for the first time since my return I was alone with Ursula.

“All right, let’s discuss,” I said. Almost she seemed to squirm.

“There’s no rush and I can’t keep my guests waiting. Why not take a hand in the game? It will take your mind off—” She stopped.

“Off Lily?” I said. “No, thanks.”

***

I went outside. It was not quite dark yet and the moon was a squeezed lemon low in the sky above the lights of West Amber. I crossed the front of the house and walked up one of the gravel paths through the flowerbeds.

It was the middle of July now, but it had been as mild as this on those two early May nights when Lily and I had walked here. We’d had our arms about each other and now and then we stopped to kiss. We walked all the way to the rickety bench where the grape arbor used to be before it fell apart, and we sat there and necked like two high school kids.

The Lily I called her. She was like one, tall and slender and all of her so very white, even her hair which was platinum, and that night even her skirt and blouse were white. The smell of her hair was in my nostrils and my mouth tasted of the perfume that was in her lipstick. Her face was small, elfin, and lovelier in moonlight than I had ever seen it, and it was always turned up to me.

That was what I had wanted to come home to. Not the Lily who wrote bored and often nasty hitters, when she bothered to write at all, but the white Lily of that night. My ardent wife with the elfin face and the sweet eager body.

We had necked on the bench like two adolescents and then had gone up to our room. That was our last night together. There had been two other nights—the first in a New York hotel after we were married and the second here in my room.

Walking alone in the garden now, I had almost reached the bench. I stopped dead and turned to look at the house. That side of it was dark except for light in the kitchen and in the basement cardroom. There were three bedrooms upstairs—Ursula’s and Miriam’s and mine. Lily, of course, had been given my room when I left. It had been our room for two days and nights.

I found myself running back to the house. A minute or two would make no difference, but I ran and I was breathing hard when I burst into my room.

My room. A man’s room. Obviously not a room in which a woman lived.

I hadn’t guessed earlier that evening when I had changed my clothes. Because for so many years it had been exclusively my room, shared with nobody, it hadn’t occurred to me that it should have been different, that in twenty-two months it should have ceased to be a man’s room and should have become a woman’s room.

I tore open the closet door. No dress, no skirt, nothing that was a woman’s. No cosmetics and brushes and combs on the dresser and no women’s stocking and undergarments in the dresser.

Lily didn’t live in this room. She didn’t live anywhere in this house. They had lied to me. Ursula was afraid to tell me the truth, whatever it was.



    
    \begin{ChapterStart}
    \vspace{3\nbs}
    \ChapterSubtitle[l]{Chapter ch20}
    \ChapterTitle[l]{ch20}
    \end{ChapterStart}

    \FirstLine{\noindent ### Chapter 20}

    
## Q. E. D.

For a long minute I stood on the road recalling where each room in the sprawling house was located. The only light was in the largest of the wings, the living room. A voice drifted out to me, unctuous, modulated—a radio voice.

A lopsided moon provided too much light. I walked a short distance back the way I had come and squeezed through an opening in the boundary hedge. Even then the dark windows I wanted to reach were a good seventy feet away across a stretch of lawn broken only by a cluster of snowball shrubs. I made it in two spurts, the first to the shrubs, the second to the double casement windows.

The windows were open, but the double screens were fastened by a hook-and-eye. I didn’t have to go inside if I could see enough of the room from outside. I put my face against the screen. What moonlight trickled in made distorted shapes of the furniture and stopped short of the walls. I tried my flashlight, but the screen mesh restricted the beam. There was nothing to do but enter the room.

That was easy. The blade of my pocket knife slid between the screens and lifted the hook from the eye. I swung the screens inward and climbed through and swept my flashlight about.

This was the room they used to go to when they wanted to get away from the rest of the house, to study or hold bull or hen sessions. With considerable affectation they called it the library because of the two glass-door bookcases which held ancient and unread volumes. For the rest, they had put into it whatever furniture they had no use for elsewhere in the house: a scarred desk, a treadle sewing machine, a horsehair sofa, a cracking brown leather chair. On the wall hung a pair of boxing gloves, crossed sabers inherited from a remote Civil War ancestor, and a rifle on pegs.

The radio voice, strained by the length of the house, was overlapped by a woman calling from another room: “Where did you say you put it?” The reply was an indistinct mumble.

I snapped off the flashlight and stuck it into my hip pocket. I had found what I expected or hoped to find. The rest was up to the police.

I was halfway back to the window when the door behind me opened. I whirled. Moonlight was brighter in the room than it had appeared to be looking in from outside, and I could see the door closing and the white-shirted shape standing there with its back to me. In the sudden stifling stillness, I heard the thin grating sound of a key turning in the lock.

***

There was perhaps time to flee through the window, but the thought was discarded as soon as it crossed my mind. I could not give him the chance to dispose of the one physical link between himself and murder.

Then he was turned to me, his face a pale blob, and his hands were pale too in moonlight. The right hand was raised, holding something long and the lower half of it not as dull as the rest. A carving knife, probably—bigger and deadlier than the steak knife he had driven into Lily’s heart.

He must have seen me cutting across the lawn, and had gone outside to watch me enter this room or had listened at the door and heard me enter.

Within the drawing of a breath he cut the distance between us in half. It was not a large room, and now we stood facing each other across five feet of space. Too late to flee now. He would get me between the shoulders as I scrambled through the window. We were bound to this room, to settle it here within the next minute or two, the mathematics of our lives reduced to the lowest common denominator. He or I. He killed me or I killed him by turning him over to the police.

He leaped. I jerked myself sideways, away from the knife, and I felt the blade rip through jacket sleeve and shirt sleeve. There was no pain, but a scream, compounded of fear and shock and rage, tore from my throat.

The fury of the knife-thrust had taken him past me. Momentarily I was behind him. I managed to get my hands on his right arm, below the elbow. He wheeled, pushing his shoulder against me. My grip loosened, caught again at his wrist, and held on.

The next minute or hour or year lost detail. Always his breath was harsh in my face, groaning and swearing and sobbing as he strove to free the knife. My job was to wrench it from him or turn it against his own body. I could do neither.

We were on the floor then, myself on top clinging to his wrist, while he clawed my mouth and nose and chin with his free left hand. A chair was on my legs. We must have knocked it over when we had fallen, though I didn’t remember. Blood was in my mouth. I dropped flat on him and pushed my face into his chest to protect it. He pulled my hair, and I hear myself scream, and then I became aware of those other sounds.

***

They were on the other side of the locked door, impotently rattling the knob and pounding on the panel. They were shouting too, begging him to open the door, begging him to tell them what was happening.

Suddenly somebody was in the room with us. “Cut it out, you guys!” he said. Like an adult breaking up a fight between two kids. Oh, he was a comedian all right. Cut it out, he said, and so we would stop and straggle up to our feet and each accuse the other of having started it.

Except that he couldn’t stop unless with my death. He made a final effort, heaving up against me with all his strength, tearing at my hands on his wrist, trying to reverse the point of the knife and drive it into my body. “The knife!” I gasped.

I felt hands move down my arms and over my locked fingers and to that other hand and the knife in it. I lifted my head. Moonlight glinted on the captain’s bars on the newcomer’s shoulder.

Abruptly the man under me subsided. He was finished and he knew it.

“I’ve got it,” Kerry said. His shape rose and left us.

My hands were still a vise around that wrist which could no longer harm me. I loosened them and pushed myself up to my feet. He too was rising. His gangling form reached a sitting position and then remained like that, immobile and bowed in defeat.

Light flooded down from the ceiling. Kerry turned from the switch. “Did he hurt you?”

Three of my fingers fitted into the hole in my jacket sleeve where the blade had sliced through. My skin was unbroken. I had a headache and my clawed face burned, but I felt fine. “No major damage,” I said. “And this is one time I don’t complain because you were right behind me.”

He lifted his hand to look at the knife in it. “Only this time I came before a killing instead of after.”

Through the door panel an anxious voice shouted: “Kerry, is that you? Why don’t you let us in?”

For a moment Kerry looked as worn and shaken as if he had just stepped out of a bomber after a combat mission. It would be hard enough for me to face them and tell them, but Kerry was engaged to marry Helen. Then the square jaw set, the clear eyes hardened. He turned the key in the lock.

The door flew violently inward. Oliver and Helen Spencer burst into the room.

***

They stopped dead as if both were controlled by a single mechanism. Their eyes swept past Kerry, past me, and came to rest on Bevis sitting on the floor with his torso pushed into itself.

“What’s Alec doing here?” Helen said to nobody in particular. She detached herself from her father and dropped down beside her brother. “Bevis, what happened?”

Bevis Spencer’s somber, tragic face lifted dully and then sagged into his hands. “Let me alone!” he moaned. “Go away from me!”

“Are you hurt? What did he do to you?”

“Let me alone!”

Mr. Spencer was staring at the knife Kerry held along his thigh. “Kerry, you tell us,” he said quietly.

“I don’t know much more than you do, sir. You saw me come into the house while you and Helen were pounding on the door. I ran outside and climbed in through that window. I found them fighting and took the knife away from Bevis.”

“Bevis had the knife?” Mr. Spencer whispered.

“Yes.” Kerry made a feeble motion with his hand and the knife left it and fell on the desk.

“Whose knife is it?” Mr. Spencer said. “Did Alec bring it with him?”

It was time for me to get into it.

There was a frog in my throat. I cleared it and said: “It must have come from this house. Bevis tried to kill me with it.”

Helen laughed and stood up. It was a sound more dreadful than the moans trickling through Bevis’ fingers. “Oh, sure, everybody is always killing, except Alec,” she said. “He killed Lily and her lover, and now he wants Miriam and he came here to kill Bevis because Bevis asked her to marry him. He kills everybody who’s in his way.”

Oliver Spencer kept looking down at his son as if trying to recognize him. “Is that how it was, Bevis?”

Bevis lifted his head and dropped it. In that moment Mr. Spencer became an old man. He shuffled to the desk and stared down at the knife.

“What are we waiting for?” Helen said harshly. “I’m going to call the police.” She started toward the door, moving between Kerry and myself. Neither of us stirred.

“Wait,” her father said thinly. “This is our carving knife.”

She spun at the door. The peaches left her round face; the cream clotted. She dug a hand into Kerry’s arm. “Darling, what does all this mean?”

“Alec knows the answers,” he muttered.

I no longer felt fine. This was the business of the police, but it hadn’t worked out that way.

“Bevis murdered Lily and Emil Schneider,” I said.

***

Mr. Spencer turned from the knife on the desk and looked down at his son sitting broken and without protest on the floor. Bevis’ silence, his position, all of him, was a confession.

“Why do you say that, Alec?” Mr. Spencer asked me quietly.

“I’d rather save it for the police.”

“Let me hear it first.”

He was entitled to the facts, to know that there was no loophole that could save his son. I said: “Kerry, have you the equation?”

Kerry dug the folded sheet out of a pocket and gave it to me. When I turned from him, Mr. Spencer had dropped down on the horsehair sofa and was abstractedly patting his bald pate. His face was empty. He reached out a bony hand for the paper and fumbled out his reading glasses. Bevis’ moans had stopped, but his head remained bowed. Helen left Kerry and sat down beside her father and together they read the paper.

The silence was like physical pain. I felt that I had to break it or suffocate. I asked Kerry if the equation had brought him here.

“More or less.” He spoke in the husky undertone people use at funerals. “George and I worked it down to one of the Spencers. I went upstairs to ask you a couple of questions about it. Miriam tried to keep me back, so I suspected something was screwy. When I found you’d left the house, I thought that whatever you were up to I ought to be in on it. I tried this place first.”

“What’s George Winkler doing?”

“I left him talking with Ursula and Miriam. They’ll hold him there till you get back. He thinks I left to keep a date with Helen.”

Mr. Spencer looked up from the paper. “What does this gibberish mean?”

So I explained the hypotheses to him and Helen. They didn’t get it. I took out a second sheet on which I had worked out the equation against the various suspects and handed it to him.

“I had to limit my suspects in some way,” I said, “and at the last I considered only those who had been in

Ursula’s house the night Lily was murdered: Ursula, Miriam, Kerry, George Winkler, Owen Dowie, Helen, Bevis and you. Ursula and Miriam had knowledge of my movements before each murder and conceivably had reason to murder Lily. But OS and MA don’t cancel out against their names. They couldn’t have murdered Schneider; they’d been in the house when I’d left and I’d driven straight there. And, of course, I couldn’t see either of them wanting to hurt me.

“As for Sheriff Dowie, he wasn’t at the second poker game, so KS doesn’t cancel out, and, as far as I know, ML and MS and MA don’t. Kerry and George Winkler, both with full knowledge and opportunity, are left with ML and MS and MA. The motive was the tricky part, the most unfixed of the terms. There could be facts beyond my knowledge, but I had to go along with what I had.

“Finally you three Spencers. Helen saw her father knocked down by Don Yard because of Lily, and perhaps for that, and other reasons, hated her. It doesn’t matter. She wasn’t at the house just before Schneider’s murder, so KS doesn’t cancel, nor does MS. You, Mr. Spencer, and Bevis were in both poker games and had opportunity after each of them and resented or disliked or hated Lily because of what she was doing to Helen. Not strong motive for murder, but assuming it was sufficient, neither of you cancelled out MA. Why should you or Bevis, or anybody, want to frame me? That was the chief snag until a few days ago.”

***

It was awful the way they looked at Bevis and listened to me. I didn’t want any of it. I hadn’t from the beginning, and this, the end of it, was somehow the worst.

Kerry was sitting on the sofa beside Helen now. He held her hand. Dry-eyed, staring vacantly, she sagged against his side.

“Miriam,” Kerry said softly. “Is that it?”

I nodded. “I was that kind of fool. Or maybe not such a fool, because since childhood Miriam and I had lived like brother and sister. We’d been so close together, growing up together, that there was none of the newness and curiosity of a man suddenly coming upon a woman. Anyway, as far as I was concerned. After the trial Ursula said I ought to be spanked. I didn’t know what she meant, though just before that she’d told me. So had others, but it didn’t register. We loved each other, sure, like fond brother and sister. It was a stranger, Robin Magee, who opened my eyes. Suddenly I saw that Miriam loving me, wanting me, satisfied the equation.”

I laughed shortly, bitterly, through my nostrils. “That was the way love came to me first. Objectively—in effect, mathematically. That night I dreamed about her, and when I awoke I realized that all along Lily and every other woman I had known had only been substitutes for Miriam.”

Bevis stood up. Our eyes swirled to him, fixed on him. He clenched his fists and swayed.

“Don’t say anything,” his father warned him.

Bevis’ mouth went slack. He turned and bent his gangling body to straighten the overturned leather chair. We watched his awkward, uncoordinated movements. One foot of the heavy chair dropped on his toes. He cursed without passion, without feeling. Then he drew in a ragged sob and sank into the chair and lowered his head.

“Do you base your accusation on this nonsense?” Mr. Spencer said.

His face was suddenly sharp, calculating—the face of a shrewd business man who was weighing a deal.

“Perhaps it’s nonsense,” I told him. “It’s not real mathematics, but mathematics teaches you how to think straight. I could have worked it out another way, but because I’m a mathematician I used this method. It wasn’t meant to convince anybody but myself. That was enough. Bevis, loving Miriam, had powerful reason to want me out of the way. His name is the only one to cancel out MA, the only one that completely satisfies the equation. There’s a law called the economy of hypotheses. If you have a hypotheses that satisfies your case, you don’t go looking around for another hypotheses.”

“On paper,” Mr. Spencer said, fighting back now.

***

“It fits and nothing else does,” I said. “We can go back far to get to the beginning of it, probably to the time when we were high school kids and Bevis was in love with Miriam and Miriam with me and I, for a while, with Helen. Or start with Helen’s friendship with Lily and the inability of her father and brother to break it up. Or maybe it was the impact of Don Yard’s fist against your jaw, Mr. Spencer. Yard knocked you down, but it was Lily who was to blame. That was the ultimate outrage, the unendurable indignity to family pride. You could take it tight-lipped and hope to avoid scandal; but Bevis, brooding and intense, had to do something about it when Helen told him. Not kill Lily, of course. The motive, as I said, was weak—at that time. But he went to Lily and threatened her if she didn’t stop trying to make Helen a tramp like herself. Whatever he said frightened her so that she made a phone call to Don Yard for protection. She needn’t have worried. She would have been safe enough from Bevis if my homecoming had worked out a little differently.

“Bevis made considerable headway with Miriam while I was overseas. She liked him, and probably she would have married him if I hadn’t existed. I had a wife, I should have been out of the way, an obstacle removed, but it was obvious that the marriage couldn’t last. Sooner or later I would be free again. Miriam was waiting for that, Bevis thought, which was why she didn’t accept him. Maybe. My own opinion is that Miriam just didn’t care enough for him.

“He may have considered killing me when I returned. But he was no fool. To begin with, it would be too obvious. A pattern would be formed—a rival put out of the way. Even if he got away with murder, the truth would occur to Miriam or one of the others who knew how matters stood. And the chances were that my death would solve nothing, that she would remain faithful to my memory and not marry him or anybody. He himself would react that way to Miriam’s death, so why wouldn’t she to my death?

“No, there was no plan in advance. The idea must have come to him full-blown when I broke up the poker game and rushed to Lily’s bungalow. He had time to think of all its aspects while he drove his father home. It had everything; it was perfect. I’d be jailed or executed as a murderer. Miriam’s love for me would be destroyed. She’d accept him on the rebound. And the fact that it was Lily he had to kill in order to work it was a great deal in its favor.

“There was danger, delicate timing necessary, but he was a good gambler. Miriam was certainly worth any risk.

After he dropped his father off, he drove to Lily’s bungalow, beating me there with plenty of time to spare, entered through the back door, picked up a steak knife on the way through the kitchen plunged it into her heart, wiped the knife handle, drove on to Ursula’s house and was with Miriam when news of my arrest came. He couldn’t have known that Sheriff Dowie and then Kerry would go to the bungalow and find me there. But that hadn’t been necessary. There was enough circumstantial evidence against me without their arrival, just as there would be weeks later when he murdered Emil Schneider. But that was a bit of additional luck.”

“Luck! ” Bevis uttered a sound that was something like a laugh and something like a sob.

Helen’s face was pressed against Kerry’s shoulder. She slid it along his sleeve to face her brother. “Bevis, tell him he lies! Why don’t you tell him he lies?”

***

Again Bevis made that bitter, hopeless sound.

“Don’t say anything,” Mr. Spencer told him. He looked at me with that calculating expression he had assumed after the initial shock. “Is there anything more, Alec?”

“Not much,” I said. “He had luck then, but not later. I was acquitted. I was free in more ways than one. Free of Lily, free to marry Miriam. Bevis had outsmarted himself. And more bad luck—Schneider had seen him leave Lily’s bungalow. He needed money, and Bevis had to pay up. Then Schneider got scared. He knew that a man asks for death when he blackmails a killer. There was nothing left to hold him in West Amber, so he planned to get out of town as soon as possible. Then I came to him and he saw a chance to pick up more money before he left.

“I returned to the poker game to get the money. Bevis, of course, realized why I wanted five thousand dollars in cash. Only recently he had needed cash for the same purpose. He had to protect himself. Not only that, but he saw a chance to try again, to erect a frame around me once more. I didn’t leave the house until some time after the game broke up. Bevis was waiting at the side of Schneider’s porch. He saw my car pull up, saw me approach, and he called to Schneider who was in the house. Schneider, also having heard my car and thinking he heard me call his name, came out and died.”

“Nonsense!” Mr. Spencer said. “Complete and utter nonsense!” His narrow shoulders were straighten He managed to smile thinly to his son. “The fight has upset you, hasn’t it, Bevis? That’s why you’re acting like this. Alec comes here with a cock-and-bull story and not one scrap of real evidence.”

“God!” Bevis moaned. “Oh, God!” His head fell on his arms.

“I’m sorry,” I said, meaning it, sorrier for Mr. Spencer and Helen than I had ever been for myself. “There’s evidence and Bevis knows it. That rifle on the wall.”

They stared at it as if they had never seen it before. It hung where it had hung for years above the ornately carved mantle.

***

It was a Winchester bolt action rifle, a sweet job which could be used for woodchuck or deer, depending on the power of the cartridge. My .22 was no good for big game, and on the two or three occasions we had gone hunting together he had generously let me carry the rifle part of the time. I had never been lucky enough to spot a deer when I had it. For that matter, in all the years Bevis had had that rifle he had not been able to bring down a deer either. The rifle had never killed anything bigger than a raccoon until it had killed a man.

“Bevis, you fool!” Mr. Spencer shouted.

“What else could he do?” I said. “I more or less expected to find the rifle here because I expected Bevis to show sense. He had no chance to prepare a weapon when he decided to kill Schneider. He had to work fast before I reached Schneider’s house, so he grabbed the only weapon at hand. Naturally, after he’d fled from me, he’d watch the house from the woods to see what I would do. He saw Kerry come up less than a minute behind me. At once the rifle became a burden, a problem. He couldn’t leave it in the brush and later tell the police I had stolen it from him, for he couldn’t know that Kerry hadn’t been dose enough to see that I hadn’t shot Schneider. So he had to take it away with him.

“Then he learned that I had fled and that Kerry had seen nothing. It was too late to take the rifle to the scene of the crime and leave it there. He was stuck with it. For many years it had hung on that wall, a conspicuous and treasured possession. Questions might be asked if it disappeared the night Schneider was shot dead. He could not afford questions, anything at all pointing to him. And they were unnecessary. Replace it where it had always been, unnoticed by its very presence, and later, when the hunting season started, take it out and say he had lost it. The police were devoting themselves exclusively to the hunt for me. There was no hurry. And then suddenly he saw me cutting across the lawn and knew what I was after. This would be it.”

***

The three of them on the sofa kept staring at the rifle.

“And if Bevis hadn’t left it here?” Mr. Spencer muttered.

“I would have tried something else,” I said. “I suppose you and Helen hate me now, but I didn’t do this to Bevis. He did it to himself and to me and to you. I was prepared to be ruthless. I would have got him off alone, or with somebody’s help, and forced a confession out of him. You see how he’s gone to pieces. The intense kind like Bevis easily does. It wouldn’t have been nice, but nothing about this was nice.”

Mr. Spencer took a long time getting to his feet. He didn’t look at his son or at anybody. His slight body had shrunken to loose skin over a framework of bones. He shuffled past me and out of the room.

Bevis appeared to be asleep with his head on the arm of the leather chair. He wasn’t. Helen’s face, pressed against Kerry’s chest, wasn’t visible either. Gently Kerry stroked her hair.

“It’s not pleasant for you,” Kerry said to me. “You’d better go.”

“The rifle—”

“I’ll see that it’s here when the police come.”

I said: “I’m sorry.”

“Sure. We all are.”

I nodded and went out of the room. I didn’t have to call the police. Oliver Spencer was in the hall, crouched over the telephone table. “Is this the state police?” he said. “My son—”

I kept going. It was a clear, mild night. The high, lopsided moon had the sky to itself. I pulled my flashlight out of my hip pocket and walked home to Miriam.

#### THE END 


    
    \begin{ChapterStart}
    \vspace{3\nbs}
    \ChapterSubtitle[l]{Chapter ch3}
    \ChapterTitle[l]{ch3}
    \end{ChapterStart}

    \FirstLine{\noindent ### Chapter 3}

    
## Blood on the Lily

There were subdued voices in the living room. I went in there instead of down to the basement. On the couch in front of the dead fireplace Bevis Spencer was holding Miriam’s hand. He turned his head sharply when he heard me and dropped her hand in confusion and clambered up to his feet.

“It’s swell to see you back,” he said, grabbing my hand between his two palms.

My throat made polite noises of greeting.

Bevis was Oliver Spencer’s son, but in no way like his small, bald, go-getter father. He was a head taller and had a thick unruly mop of hair which would last forever and he was awkward and embarrassed with strangers. He had opened up only a little since our high school days when he’d been a moody, unsmiling boy who kept more or less to himself. The only reason I had come to know him fairly well was that in those days I had a crush on his sister Helen and spent considerable time in the Spencer house.

“I begged the Army doctors to pass me,” Bevis was saying, “but they refused. I’ll always feel I missed something big.”

“You missed something all right,” I told him, “You’ll never know how lucky you were. Do you mind if I speak to Miriam alone?”

She had been pushing back her hair from her brow and readjusting her bun. When I said that, her hands paused at the sides of her head, but she didn’t look up at me.

Bevis glowered at me. Glowering came easy to those deep-set eyes under overhanging brows. “I was on my way downstairs anyway,” he said without enthusiasm. “They have only four hands and I promised to make a fifth.”

I waited until he was out in the hall before I turned to Miriam. She spoke first.

“Bevis wants to marry me, Alec.”

That was why he had glowered at me. I’d be sore too if somebody had barged in at a time like that. I told her that I was sorry that I’d come in at the wrong moment.

“It doesn’t matter,” he said. “Bevis has been proposing to me about once a month during the last year. I’d just put him off again.”

“He’s too gloomy for you.”

He brightens up a lot when you know him well. He s intelligent and considerate and sweet.”

“So what’s wrong?”

Miriam didn’t answer. She stood up and went to the table and took a cigarette out of the translucent plastic cigarette-box. I noticed for the first time since my return that she’d put on weight. Just about enough weight. She had been too skinny since childhood and now her figure was pleasantly mature and her angular face no longer had that pinched look. She was a very attractive girl. Not beautiful like Lily or Helen Spencer and perhaps not even pretty if you took her features apart, but the black-eyed, dark-skinned animation of her face made up for it.

“I don’t know what to do, Alec,” she said after a pause.

***

It was my duty to be the sympathetic confidant and advisor and discuss her problem with her far into the night. I was the man of the household, the nearest thing to a father or brother she had. But I had a problem of my own which was a lot more urgent than hers. To me, anyway.

“Where s Lily?” I demanded.

She took the unlighted cigarette from her mouth. “Ursula told you.”

“Cut it out! ” I realized I had shouted that and lowered my voice. “Lily isn’t living in this house.”

She put a match to her cigarette. She had trouble getting a light. Then she said: “No, she isn’t.”

“What’s happened to her?”

“Don’t get frightened. She’s in the best of health as far as I know.”

I gripped her arm. She uttered a little exclamation and I loosened my fingers. “She walked out on me, didn’t she?”

“No, Alec, she didn’t. She—Ursula—” she drew in smoke and made a fresh start. “Ursula asked her to leave the house.”

“Why?”

“You’d better ask Ursula.”

“When did this happen?”

“Two weeks ago.”

Panting, I stepped back. It wasn’t Lily’s fault that she hadn’t met me at the station. That was good to know.

“Where’s Lily now?”

“She’s living somewhere right here in town.”

So Lily hadn’t walked out on me, as for almost a year now I had thought she might. She could have moved back to the bright lights and gayety of New York when Ursula had kicked her out. Instead she had remained in this dull town where I would be able to find her at once.

I struggled to keep my voice under control. “You mean to say that you didn’t let Lily know that I was coming home today?”

“Ursula wanted to have a talk with you before you saw Lily.”

“About why she threw Lily out of the house? Why did she?”

Miriam brought the cigarette up to her mouth. It trembled against her lips and she tossed it into the fireplace. She was nearly as upset as I was. “Ursula will tell you.”

“Why didn’t she tell me before this?”

“She wanted to break it gently.”

“Because she thinks I’m a mental case and wants to cushion the shock?”

***

Miriam put a hand on my chest.

Her dark, intense eyes were misty. “Please, Alec, try to understand how we feel. Ursula is anxious that you hear our side first. She planned to have a long talk with you after dinner, but then the men arrived for the poker game.”

“So poker is more important than I am? Is that it?”

“Alec, that’s not fair.”

“All right,” I conceded. “So she found excuses to put it off because she was afraid to tell me. She thinks I’m not in a fit mental state to take bad news. It must be pretty bad if Ursula kicked her out of the house. It was because of a man, wasn’t it?”

She reached for another cigarette, although she hadn’t taken more than a couple of puffs on the one she’d discarded. “Ursula will tell you all about it.”

“She had her chance. What’s Lily’s address?”

“Alec, please have a talk with Ursula first.”

“Where is Lily staying?” I yelled.

“Ursula asked me not to say anything until she—”

I swung away from her and went out to the hall and ran down the basement stairs.

The cardroom took up half the cellar and was more cozily furnished than any other room in the house. It had an oak floor and a beamed ceiling and was panelled in cedar. There was a tiny bar in one corner and a radio and a couple of leather lounge chairs in another and at the far end of the room a bridge table which was seldom used. The round poker table, big enough to seat ten players, dominated the room. The red leather chairs were custom- built for a person’s bones and the neon lights were as soft as a summer dawn.

Sheriff Owen Dowie was dealing when I burst in. They were playing stud and Dowie’s arm paused in midair as he was about to slide Bevis Spencer’s fourth card across the table. There must have been something in my face that made Dowie and Ursula and George Winkler, facing the door, stare at me. Oliver Spencer and Bevis Spencer turned in their chairs and added their stares to the others.

I said: “Ursula, where is Lily staying?”

She started to rise from her chair. “I told you—”

“You lied,” I said. “She’s in town.” Ursula rose all the way and flashed an angry look over my shoulder. Miriam had followed me down the stairs and stood behind me, just inside the doorway. Ursula turned back to the table. “I’m sorry, gentlemen, but you’ll have to excuse me for a few minutes.”

“I don’t want to break up the game,”

I said. “I only want to know where to find my wife.”

Ursula came around the chair and took my arm. “Let’s go upstairs, Alec,” she said in the gentle tone one uses to a sick child.

“Like hell!” I said. “All I want to hear from you is Lily’s address.”

“Alec, if you’ll listen—”

I shook her off and went up to the table. “Some of you men must know where Lily is. Bevis, you can tell me.”

Bevis dropped his eyes to his cards.

He had a seven and a queen showing, and the card which Dowie still held in his hand would be his second queen. Dowie was peering curiously at me through the thick lenses of his glasses George Winkler fumbled with a cigar wrapper. Oliver Spencer brushed back non-existent hair on his head.

“You’re all in on this!” I said. I was shouting. I’d been shouting since I entered the room. I didn’t care; I’d keep on shouting. “Don’t think I’m a dope. I’ve an idea what Lily was up to while I was away. It was between the lines of everything she wrote me. But now I’m back, and, by God—”

My voice choked off in my throat. This wasn’t any of their business. It was too personal to invite anybody in on my agony.

There was a brittle silence while I pulled out my handkerchief to wipe my sweating face. Then George Winkler sighed as loudly as a gust of wind sweeping through the room.

“Alec, she’s not worth it,” he said.

“None of your damn business!” I flung at him. “I don’t need any of you to tell me how to handle my wife. Where is she?”

Ursula was back at my side, clawing at my arm. “Alec, control yourself. If you’ll come upstairs with me for just five minutes—”

“Why should I control myself? Where’s my wife?”

Suddenly she looked very tired. Her broad shoulders drooped. “Have it your way. She s staying in a bungalow on James Street.”

“Thanks,” I said bitterly and turned to the door. Miriam was still standing there. She stepped out of the way and I brushed past her and raced out of the house.

A coupe was swinging into the driveway when I reached the road. Somebody called my name and the car stopped broadside to me. By the dash light I saw Kerry Nugent behind the wheel and Helen Spencer beside him.

They d come to visit me, of course. Kerry would ask me how I was and I would ask him how he was and he’d tell me the latest news about airmen we knew all over the world and whom you were always running into or hearing about and how soon he expected to return to active duty. I did not want any talk at all tonight except with Lily, and not too much talk with her.

I said, “I’ll see you tomorrow, Kerry,” and moved past the coupe and out to the road.

“Alec!”

I glanced back. Kerry had the door open and one foot out. I put my head down and walked as rapidly as I could without quite running, fleeing from the conspiracy to keep me from my wife’s arms.

***

At the bottom of the hill I paused.

James Street was two miles away, yet I’d run right past Ursula’s car which was still parked in the driveway. I was a guy in such a terrific hurry that I was taking ten times as long walking as it would to drive. I hadn’t even had sense to bring along a flashlight.

It would still save time to go back for the car, but some of them might be outside the house and I couldn’t face them again. Not now, anyway. I walked on and came to the fifteenth tee of the golf course where it angled toward the road. As kids we used to take a short cut diagonally across the course to the second hole, saving perhaps half a mile. I climbed the stone boundary wall and by the sickly light of the waning moon started across the fairway.

George Winkler had said that Lily wasn’t good enough for me or not worth it or something like that. Whatever it was, he had put in to indefinite language the vaguer fears caused by her letters or not hearing at all from her for weeks. Did George know or had he handed me town gossip? And would Ursula have kicked her out without certain knowledge?

Face it. Two years ago she was your wife for three days and nights and after that you were strangers who didn’t again see each other. While you yearned in loneliness and faithfulness at the other end of the world, she did not keep her bed always empty. What were you going to do about it?

Nothing.

Nothing tonight and perhaps nothing tomorrow or the day after that. I wanted her. I hated her and wanted her, as I had hated her and wanted her for a long time now, submerging dread and fear in the memory of those three nights and the never-ending hunger for her.

I’d use the Lily’s beautiful body. I’d satisfy my need for her. And after that—

I didn’t know. I didn’t want to know tonight.

At the second hole I came out on Old Mill Road at the edge of town. My watch said seven minutes to ten. Too early for her to be asleep. I would like that, finding her in bed and waking her with kisses.

But suppose she didn’t want you? Suppose her body had no more ardor for you than her letters had had?

By God, I’d take her anyway! Brutally. I’d hurt her. I’d—

I stopped to catch my breath. I hadn’t been walking rapidly since I had reached the bottom of the hill, but all the same I was drawing air into my lungs with labored gasps. I fumbled out a cigarette, lighted it, and the smoke felt good inside me. I walked along the clearly defined macadam road.

***

James Street. It wasn’t actually a street; there weren’t any in West Amber except for the three which comprised the business section. It was a tar road. Along either side houses were spotted at varying intervals. There was a mile of it, and like a sap I’d left the cardroom without asking just where on James Street Lily lived.

A car approached. I stood in the middle of the road and waved in the hope that whoever was in the car could tell me where to find her. The headlights bore down on me. The car did not slacken speed; it seemed determined to run me down. I jumped and felt the wind of its passing.

I cursed the driver for a madman, glad to be able to curse something, and went on. The mailboxes stood on the right side of the road. One of them should bear her name. I struck matches to read the boxes, passing up those in front of houses in which I knew she could not be because they were too large or I knew the people who lived in them.

Like George Winkler’s white brick house. When I passed it, light went on in an upstairs room. That would be Amy Winkler, George’s spinster sister who kept house for him. She knew everything there was to know in town. She would tell me, but she would also gush over me in greeting. I kept walking.

Two boxes farther on I came to it. The name JOSEPHS was crossed out with black crayon, and under it was waveringly printed: LILY LINN.

Linn. My name. The name I had given to my wife.

It was suddenly very good to be home. The fact remained that she had stayed on here in West Amber to wait for my return. That was all that counted. The rest was merely an inability to write letters as passionate as her embraces. She was waiting for me and I was coming to her.

It was a neat little stucco bungalow just off the road. Every window was lighted. With both hands I brushed back my wild hair and ran my handkerchief over my face and went up to the door. I flicked my tongue over my dry lips and knocked.

There was no sound in the house. She wouldn’t have gone out and left all the lights on. Perhaps she had dropped off to sleep while reading. The door was unlocked. I pushed it in.

The door opened directly into a small living room. I saw her at once. She was lying on the floor at the side of a reed couch.

I closed the door behind me and walked slowly into the room, not thinking, not feeling anything except some thing jumping in the emptiness inside of me.

She was on her side, turned away from me, her knees bent. I squatted beside her and put a hand on her shoulder.

Then I saw the knife in her breast. I reached over her shoulder and grabbed the handle and started to pull the knife out. It came out easily. It came halfway out, and suddenly I jerked my hand away from the knife as if it had turned glowing hot.

The truth hit me like a kick in the stomach. The Lily was dead. Nobody lives with a knife in his heart.

I straightened up beside her and at once my knees started to bend. I sank down on the couch. The room wavered.

I put my head in my hands, closing my eyes tight against the darkness of my palms.

***

It was very still. Somewhere a clock ticked. I dared take my hands from my face and the room had straightened out. My head remained bent; I was looking down at her legs.

One foot was covered by a satin mule. The other foot was bare and the mule which belonged to it lay behind her. She wore no stockings and the nails of her toes were scarlet. The first time I had ever seen her without any clothes on I had, oddly, noticed first of all her painted toenails. They had shocked me a little and excited me.

She was wearing the heavy silk hostess gown with the tawny-orange tiger lilies. I had bought it for her the day we were married. Lilies for Lily, I had said. A poor joke, perhaps, but the kind you delight in on honeymoons. During all the breakfasts I’d had with her, the first in the New York hotel room and the other two at home, she had worn it. The hem had hiked up on her body and her uppermost thigh was naked. It was a lovely thigh, smooth and rounded, and even now I could remember the firm warm feel of it.

Sitting bent over, that was all I could see of her. I didn’t want to see any more, but I had to. I raised my head an inch and let my eyes move up the crumpled form.

The knife protruded from the swell of her left breast. The dull black handle of a kitchen knife. There was a tiger lily on each breast, and I remembered telling her one morning that those two, and one other, were my favorite lilies of all the many on the hostess gown. The knife had pierced the one over her left breast, and it was no longer tawny-orange. Her life had dyed it red.

My eyes remained on the bloody lily, not wanting to look higher. A body is a body, living or dead, but a face—

I looked. Her face was no longer beautiful, no longer elfin—no longer anything but dead.

Then the shakes got me, as badly as that first time when we’d jumped from our gasless plane and I was bending over Kerry Nugent who was tangled in his parachute. Joe Klintcamp, the copilot, had slapped me and that had helped. There was nobody to slap me now. I dropped my head and shook and felt sweat cover me.

It didn’t last long. It was passing when I heard the door open. I looked up and Sheriff Owen Dowie stood in the doorway peering myopically at Lily.

He didn’t say anything at first. He closed the door and came forward and knelt beside her. Then he shifted his gaze up to me. His pinched face was sad.

“I’m sorry about this, Linn,” he said.

I hadn’t heard the sheriff’s car, but I heard the second car pull up. We both turned our heads, listening to a car door slam. Then there was a knock and a man said: “Alec?”

Dowie looked at me to see what I would do about it. I just sat there. The knock came again. “Lily?” Dowie said quietly: “Come in.”

It was Kerry Nugent. He gawked at Dowie in surprise and took a step or two into the room before he saw her. “Alec,” he said weakly. He came on to Dowie’s side. “Is she dead?”

“He stuck a knife in her heart,” Dowie said.

He? Whom did Dowie mean? I heard the clock. Why were they so very quiet?

“You don’t think—” I said. Their faces told me what they thought. “My God, I didn’t kill her!”

Kerry turned his eyes away from me. “I’m sorry, Linn,” Dowie said again.



    
    \begin{ChapterStart}
    \vspace{3\nbs}
    \ChapterSubtitle[l]{Chapter ch4}
    \ChapterTitle[l]{ch4}
    \end{ChapterStart}

    \FirstLine{\noindent ### Chapter 4}

    
## The Jail

Ursula stopped inside the cell and sent a startled, shocked glance over her shoulder at the door closing behind her. Then she kissed me quickly on the mouth and thrust a package into my hands, like an embarrassed little girl presenting a gift at a birthday party. “One of your chess sets,” she said.

“Thanks.” I tossed the package on the cot and led her to the wooden chair. “Sit on that. It’s softer than the cot.” Her big frame perched unrelaxed on the edge of the chair. She looked about. A second was enough to take in everything there was to be seen. She shuddered. “How horrible for you!”

“A bit cramped,” I conceded, “but the food is wholesome and the turnkey does little favors for me if I keep his palm greased. My chief complaint is that a vicious character like me is allowed only two visitors a week, not counting lawyers.”

“And they wouldn’t let me visit you at all the first week and they wouldn’t let Miriam come up with me.” Ursula removed a fleck of tobacco from her lip. “I’m glad that you’re taking it so well, Alec. I was afraid—I mean—”

“That I’d crack wide open?” I said. “That I was a psychoneurotic even before they tossed me in here and that I’d pound on the door and chew the mattress?”

“Alec, please! ”

“All right, I’m scared,” I said. “There’s still a trial between me and the death house, but the last time the district attorney talked to me he was already gloating over the two thousand volts of electricity which—”

“Alec!”

I stood up. I walked the five paces to the other end of the cell and lit a cigarette. I was sweating, but not much, and my hand was not unsteady. I returned to Ursula.

“I can take it,” I said. “It’s easier than sweating out a mission. The odds against me aren’t much greater than trying to get home after bombing Singapore. George Winkler hasn’t been around in a couple of days. What’s he doing about me?”

“I want to talk to you about that. We’ve retained a very big lawyer from New York named Robin Magee.”

“What’s the matter with George?”

“He will be associated with Magee. George does not think he has had enough experience in criminal law, and Robin Magee is supposed to be marvelous.”

“Expensive as hell, I bet.”

“Does that matter?”

“I’m grateful,” I said. “When I get out and start earning a living, I’ll pay you back. It’ll cost more than you think. Is George hiring private detectives?”

“Why private detectives?”

“To find the murderer, if possible. That’s what the police should be doing, but they’re so sure that I’m the one that they’re not bothering.”

She didn’t say anything. I sat down on the cot and stared at her. Her eyes lifted; there were tears in them.

“So you agree with the police,” I said slowly.

She propelled herself out of the chair and flung her arms about me. “Alec, it’s my fault. I should have had that talk with you as soon as you got off the train. I kept putting it off. I wanted you to rest after your journey. But I only made it worse for you. It must have been such a shock when Miriam blurted it out to you.”

“Let Miriam alone,” I said. “She didn’t tell me anything.”

She pressed her face against my arm. “I never liked Lily, especially not for you, but I should have let her stay in the house until you came home. It was only a couple of weeks.”

“Was Lily cutting up as badly as that?”

“There was gossip.”

“Is that all?” I said. “Nothing but gossip?”

“Well, she was seen drinking with a strange man.”

“Damn it!” I was on my feet, pacing the cell and slamming my left palm with my right fist. I wanted to break somebody’s neck, but I didn’t know whose. “Just gossip. Just drinking with another man. She didn’t walk out on me even after you kicked her out. She stayed around to wait for me.” Ursula’s jaw hung slack. “You thought from my actions that she was carrying on an affair with somebody else? Oh, God! If I’d minded my business—”

I swung to face her. “I wouldn’t have rushed off to kill her? You believe that, too!”

The door was flung open. The beanpole turnkey stood there. “What’s the yelling about?”

I took a deep breath. “I got excited. It’s nothing.”

“Well, the lady’s time is up. If you want any more visitors, Linn, keep it quiet.”

Ursula turned a rigidly set face up to me to be kissed. Her lips were cold. “Alec,” she said, “I know you’re not a murderer.”

“Sure.” I squeezed her arm. “Get the idea out of your head that Lily was murdered because of anything you did or didn’t do. It had nothing to do with you or me. Remind George Winkler about hiring private detectives.”

“Time’s up, lady,” the turnkey said. Ursula nodded dully and went.

***

I wasn’t sure what there should be about a lawyer to inspire confidence, but Robin Magee didn’t in me. He was a tall, slightly stooped man with hair graying aristocratically at the fringes, and he was dressed like an actor or an English politician. His breath smelled of very good liquor.

George Winkler sat on the chair and I on the cot. Robin Magee remained on his feet and tossed me a benign smile. “Now let’s hear your story, my boy.”

“I told it to the district attorney and I told it to George,” I said. “It hasn’t changed since then.”

Magee nodded amiably. “It may be that you will mention something in the retelling which you overlooked before.” So I told it again. There was surprisingly little. I had walked in and found Lily with a knife in her heart.

There was a silence when I finished. George looked up at Magee. The great criminal lawyer from New York smiled silkily.

“Why did you touch the knife handle?” he asked.

“I started to pull it out. Then I realized that she was already dead.”

“The knife was in her heart. Didn’t you see that?”

“My brain stopped working for a few seconds.”

Magee studied the ceiling. “They tell me you are a very bright young man. You must have known that it was not proper to touch a murder weapon, and that pulling the knife out wouldn’t have helped her even if she were still alive.”

“I wasn’t in a state to realize anything.”

“You were a soldier. You must have seen many dead people.”

He sounded a lot like District Attorney Hackett questioning me, but he was being paid to be on my side. I said: “I’ve seen dead men, but I’ve never before walked in on my wife with a knife in her heart. Did you?”

Magee chuckled. “Naturally not.”

“Then how the hell do you know how a man will act at such a time?”

“You were a navigator of a Superfortress,” he argued. “You were trained to keep your head in an emergency.”

“There was nothing in my training about how to act when I found my wife murdered. What are you getting at, anyway?”

George put in: “Didn’t the D. A. ask you those questions?”

“Yes.”

“We’ve got to know what you told him, don’t we?”

“That’s right,” I agreed. “Okay, throw ’em at me.”

Robin Magee dropped down on the cot. He did not hold my hand, but his manner was so intimate that he might have. His breath made me thirsty. “It looks bad, my boy. There’s no getting away from it. There are many witnesses that you left your house in a highly excited state of mind—unfortunately, one of them the sheriff. Then you were found bending over the body—”

“I was sitting on the couch.”

“You were sitting on the couch and she was at your feet. And the police laboratory found your fingerprints on the knife and no other prints.”

‘‘The murderer had wiped his off.”

“Let us grant that. This morning I had a talk with the D. A. He did not tell me m so many words, but he implied that he had even more definite proof than the medical examiner’s report that she had been alive a very short time before Sheriff Dowie found you there. He gave me this much to induce me to accept a second-degree plea.”

“Did you tell him to go to hell?”

“In effect, yes.”

Magee tapped my chest with his knuckles. “My boy, Winkler and I are going to get you off absolutely free. You may have to spend a month or two in a sanatorium, but even that will probably be unnecessary.”

***

It took time to penetrate. I looked from one pair of eyes to the other, both pairs watching me anxiously. Then I left the cot to get as far from them as I could, and turned at the wall.

“So that’s it,” I said. “I’m insane.”

“Not at all.” Magee’s honeyed words flowed to me. “Temporary insanity, perhaps. Extenuating circumstances, certainly.”

“In other words,” I said, “confess that I murdered Lily.”

George shambled his bear’s body over to me. The fatherly act. The close family friend. I knew what he’d say before he said it. “Alec, you know I’ll agree only to what’s best for you. Magee is one hundred percent right. He’s thought of the only way to get you off.”

“Get out!” I said. “Both of you!”

George’s small eyes blinked in pain. “Alec, as your attorneys—”

“I don’t want lawyers who are convinced I’m guilty. Get out!”

He turned to Magee for help, and the great lawyer gave me his smooth smile. “Naturally he’s upset, Winkler. He’ll be more reasonable tomorrow. Goodbye, my boy.”

“Don’t come back!” I flung after them.

For a long time I stood looking at the door after he closed it. My teeth were chattering. I gritted them and threw myself face down on the cot. The walls of the cell closed in on me. I could not breathe.

… They returned the next day. I had told the turnkey not to let them in, but he did. I lay on my cot and stared up at the ceiling and pretended not to hear them as they talked down at me. As a matter of fact, I heard very little of what they said. After a while they left.

***

Kerry Nugent came in briskly.

He grinned down at the chessboard on the chair.

“What’s the problem?” he asked.

“White to mate in four moves.”

He studied the board. “You’re even better than I think you are if you can do it.”

“I have plenty of time,” I said.

“I guess you have.” He sprawled down at the foot of the cot and nonchalantly dug out his pipe and pouch. It was all as casual as a visit to my bunk after we’d slept off a mission.

You may remember a photo in Life of a square-jawed, rugged-faced bomber pilot standing under a wing of a B-29 Superfortress in an India base. That was Captain Kerry Nugent. The moment the photographer’s eyes fell on him he tabbed Kerry as the idealized type of grim and nerveless American manhood who was bringing death and destruction to the enemy.

“How are the ribs?” I asked.

“As good as new, practically, except that I’m still taped up like a Prussian general in a corset.” His face clouded. “And stop talking like a heel about it. Miriam says you’re shooting a line that you lost the plane.”

“I did.”

“I’ll break your damn neck if you don’t cut it out.” Kerry sat up. “What the hell! I ought to have my head examined for bringing it up. How about a game of chess?”

“Your time will be up in a few minutes.”

“Not for Mrs. Nugent’s boy. I slipped the turnkey a five spot. That ought to be good for a couple of hours. Give me a knight and a bishop and I’ll try to make a game of it.”

I dumped the chessmen into their box. “I don’t want my mind taken off anything. I want to talk. How did you happen to show up at Lily’s place only a few minutes after I got there?”

His right cheek ticked. It was an old sign of annoyance.

“Why kick it around? Forget about her. She got what she deserved.”

“Why did you go to Lily’s?”

Noisily he drew flame into his pipe. “Miriam sent me. I’m a good soldier. When a lovely lady gives orders, I obey without question.”

“That’s no answer.”

“I suppose it isn’t,” Kerry said, suddenly grave. “I’m not quite straight on it myself. I’d gone to New York for a couple of days. I got home around eight and my mother told me that Miriam had phoned earlier; that you were expected home that day. I called up Helen and she’d heard from her father that you were back and she wanted to see you too, so after I ate I picked her up and we drove over.”

“You and Helen Spencer,” I said. “That’s something new.”

***

He grinned shyly. “About six days. Well, you know how you acted when we drove up and saw you coming out of the driveway. We drove on to the house and found everybody out on the porch talking about you.”

“So I broke up the card game?” I said.

“You sure did. I heard Mr. Spencer ask Ursula if she wanted to resume. She shook her head and stood looking out at the road. Then Miriam pulled me aside and told me what had happened. It didn’t sound like much to me. Why shouldn’t a guy raise hell when he isn’t told where his wife is? Why shouldn’t he dash off to her? But Miriam said I hadn’t seen the way you looked and acted. She was plenty worried. She gave me the idea that the place for you was a strait jacket, or anywhere but with a wife who’d been playing you dirty.”

He paused and then added: “That’s throwing it at you bluntly, but you wanted it.”

“You’re doing all right. Go on.”

“That’s about all. I argued for a while. You had every right to go dashing to your wife’s arms. Then Ursula joined the argument. I was your bosom pal; we’d faced life and death together; you’d resent my horning in on your affairs less than anybody else. That line. They weren’t sure what I should do when I caught up with you. I guess sort of hang around to keep you out of trouble and bring you back to the old homestead if Lily spat in your face, as they more or less expected. So I went.”

“Where was Sheriff Dowie?”

“He’d driven off. So had Bevis and his old man. Maybe George Winkler too; I don’t remember. But when I left, all I noticed on the porch were three women—Helen and Miriam and Ursula.”

“I walked all the way,” I said. “Why didn’t you reach Lily’s before I did?”

“That’s why I’m kicking myself now. But I didn’t hanker to barge in on you two in a fond embrace. My idea was to catch up to you on the road and maybe talk to you and calm you down or go to Lily’s with you if you didn’t object strenuously. So I crawled along the road in my car looking for you. When I reached the head of James Street, I decided that you couldn’t have walked there so soon or even run there and that I must have missed you on the way.”

“I cut across the golf course.”

“Hell! I should have thought of that. I drove back slowly, still looking for you. When I reached the bottom of Mandolin Hill, I turned around and tried again, figuring that somehow I’d passed you up in the darkness. This tune I went all the way to Lily’s. I saw a car parked outside—Dowie’s it turned out—and decided that if she was having company, why not me too?”

***

I sat on the other end of the cot fingering the chessmen in the box on my knees. It was a cheap wooden set. The black king had lost his cross and a white rook was minus a couple of turrets. Ursula had kept my good sets at home; this one would do for jail.

“Had you seen Lily before that?” I asked.

“Since my return home?” Kerry appeared to be smoking matches instead of tobacco. “Once. My second day in. Paid my respects to my pal’s wife. Told her you were okay and would be home any day. Drank a pair of cocktails she fixed and said so long. One of those duty calls. You know I didn’t care one bit for her.”

“You did that night we met her for the first time in New York.”

“Oh, then. She was a lass with plenty of pulchritude in a low-cut dress. Something to try to make. You did and I didn’t, so that was that. Besides, you married her, and that was okay by me, until what she did to you when we were in India. From then on she was marked lousy in my book.”

“How did she sound when you told her I was coming home?”

“Drop it, Alec. She was no good. It s not the thing to say about a pal’s wife, so in India I kept my Ups buttoned. But now I’m telling you.” He twisted on the cot to face me directly. This might hurt, but I got the impression that for all she cared you could rot in India.”

“It doesn’t hurt,” I said. “I’m sorry that she’s dead, but that’s all.”

“So forget her.”

“I doubt if the State of New York will let me.”

Kerry groped in his pockets for matches. I tossed him mine.

“I could hand you a lot of boloney about that,” he said after he had a light “but I won’t. I had a talk with George Winkler this morning. He’s plenty worried about the trial, especially because you refuse to be sensible.”

“I don’t want to discuss it.”

Winkler is one of the people who would give their right arm for you. He knows what he’s doing ”

“No.”

“Alec, you don’t know everything. If it came to taking a three-star fix, I’d back you against any navigator in the world. But this is law.”

“If that’s all you have to talk about,” I said, “you’d better go.”

His pipe stem made bubbling noises.

“I ought to push your teeth in,” he said after a while. “But it’s your life. Throw it away and the hell with you. Change your mind about a game?”

I submitted. We sat at either end of the cot with the board between us. I gave him a bishop and a knight, but ordinarily he needed more than that. Not today. I barely managed to squeeze out a draw.

***

Miriam came the day that Japan offered to surrender.

“Did you hear the news?” she said as soon as the door closed behind her.

“The turnkey told me. A lot of airmen I know are going to do a lot of celebrating at the thought of coming home.” I felt my teeth in my lower lip. “I was the lucky one. I beat them back to the peace and security and comfort of home.”

She moved all the way into the cell. She hadn’t very far to go. As usual, she wore white. She knew what it did for her black hair and eyes and dark skin. This time it was a white shantung suit with a black blouse. Very snappy. And her manner fitted the outfit. Calculatedly cool and casual. Chin up and a forced smile. Carefully ignore the locked door and the gray walls too close together and the single gash of window.

“The surrender isn’t official yet,” she said, still topically conversational. “Everybody is praying it will be.” She handed me a bulging paper bag. “Brownies which I baked for you this morning. With raisins, the way you like them.”

At the last sentence her voice betrayed her. It started to quiver, and suddenly tears brightened her eyes. She stepped past me and dropped limply into the chair.

Visits were too few and precious to be thrown away with weeping over me. I could do my own weeding when I was alone. So I sat down on the cot and tried to bring up a subject which would interest her more than I did. “Made up your mind about Bevis Spencer?”

“How can I think about it at a time like this?” She gave me her profile. She didn’t look well. Too much cheekbone showing. “Alec, why don’t you let your lawyers help you?”

“Because I don’t want lawyers who are convinced I’m guilty.”

“They don’t. George Winkler told me—”

“So he sent you to talk me into it?” I said.

“We all want to help you.”

“Sure,” I said. “Like you tried to shelter me from my wife.”

“That was a mistake. We should have overlooked everything Lily did until you got home.”

“And now you all believe I’m a murderer and you’re anxious to overlook that. You don’t shy away from me in horror. You’re loyal to the bitter end. And I’m not even grateful.”

“We don’t want gratitude. We want you back with us.”

There was no point in chasing the subject around. I dug into the paper bag and pulled out a brownie. Miriam couldn’t cook as well as Ursula—nobody could—but she could bake rings around her. “They’re wonderful,” I said and offered her the bag. “Thanks a lot.”

She took one and sat nibbling it. Her eyes were far away. “Alec,” she said remotely, “you must have loved Lily very much.”

“Did I?” I let the idea simmer before I spoke again. “I don’t think so. I hardly knew her. Six days in all, only three of them married to her. Those were fine days. I was going overseas, and I guess they would have been fine with almost any attractive woman with whom—” I stopped.

“I’m a big girl now,” Miriam said into her brownie. “I understand. Please go on.”

“Yes. And after that I was a million miles away and she was a woman to remember and dream about. If I was in love, it was with a memory.”

“Alec, did you kill her?”

***

You had to hand it to Miriam. She at least came right out with it. Nobody else had asked me except the district attorney and the police. The others skirted around it, pretending to assume that I hadn’t when they believed the opposite.

“No,” I said.

She nodded. “You wouldn’t lie about it if you had. Of course I believe you.” But what was believing? You thought you did and you said you did, but that was just emotion. Cold reason didn’t concur.

I took another brownie. “Did you send Kerry after me that night?”

“Yes. You frightened me. If Kerry had only met you! I mean, he would have gone there with you and he would have been a witness that you hadn’t done it. Or if George had met you on the way.”

“Did you send George also?”

“Only Kerry. The game broke up and we all stood on the porch for a few minutes. George had already driven away when Kerry left. The thing is that George lives on James Street, only a few hundred feet from Lily’s bungalow.”

“I saw him put a light on when I passed his house.”

“He was so close,” she said. “He blames himself now for not having waited for you on James Street or for not looking in on Lily. But how could he have known?”

“Did everybody leave the house after the game?”

“Well, Mr. Dowie was the first to drive away. Bevis drove his father home and came back a little later. Helen waited for Kerry to return. He was gone for more than an hour and then he brought the terrible news.” She put a hand on my knee. “Alec, why don’t you try to save yourself?”

“Back to the lawyers,” I grunted.

“They don’t want you to confess or anything like that. That would be stupid. George merely wants you to listen to him and Mr. Magee.”

“I’ve heard what they have to say.”

“You have to have a lawyer,” she said urgently. “I understand that you can’t plead guilty of murder even if you want to, so there has to be a trial and a defense. If George and Mr. Magee aren’t your lawyers, the court will assign one to you. Do you think you will be better off with a strange lawyer? George says that the district attorney is willing to accept a second-degree murder plea and that any other lawyer will snatch at it. George won’t and you know it. He won’t stop at anything to get you off free.”

She did it very well. George Winkler must have coached her.

“The district attorney asked me to testify against you,” she went on. “He has an idea that before you went down to the cardroom I told you Lily was unfaithful to you and that that was what made you so excited. I went to George and he told the district attorney that I would be a hostile witness and talked him out of a subpoena. So you see, George continues to work for you.”

“That’s why the court hasn’t assigned another lawyer to you.”

I sat hunched over on the cot with the bag of brownies between my palms. This was my third week of being shut in here with nothing to do but think.

I’d been dealt a busted straight and the state held a pat hand and it was a pot I had to take or die. I was scared. I hadn’t stopped being scared for a single minute since the state police lieutenant had told me I was under arrest.

The door creaked open. “Time’s up, lady,” the turnkey said.

Miriam sent him a frantic look, then turned her dark, wide eyes to me. “Alec, at least let George and Mir. Magee talk to you.”

I stood up and so did she and together we walked the few steps to the door.

“Alec, please!” she said.

“All right, send them over,” I told her wearily.

***

George Winkler said: “Get the idea out of your head that we think you’re guilty. It’s only that the evidence is overwhelmingly against you, and we decided that there is only one way to get you off.”

“By sending me to an insane asylum,” I said.

“Nonsense. You misunderstood Magee. Even if that were the only way, we couldn’t bring it off. The D.A. would stop us cold by bringing in psychologists.”

“How will confessing to a crime I didn’t commit help me?”

“That’s not what we want either. All we ask is that you don’t say anything before or during the trial. Sit tight. Leave it to us.”

“I’ll tell the truth over and over if I’m asked, by the D.A. or anybody else.”

Robin Magee left the wall against which he had been leaning. “By all means, my boy. The truth. In my twenty-two years of law practice I have lost only one client to the electric chair, and I must say he deserved it.”

“I’d as soon take the chair as a jail sentence.”

“You haven’t a thing to worry about,” Magee assured me placidly. “But I can promise to get you off free only if you don’t interfere with my line of defense.”

“Which is that I didn’t do it.”

“Certainly. Now, just relax. That’s all we ask of you.”

They prepared to leave, but I wasn’t through with them. “George, did you hire private detectives?”

He looked blank for a moment. “Oh, yes, detectives. That’s not necessary. Magee has a couple of top-notch investigators in his office.”

“Are they doing anything?” I asked. “Have they learned anything?”

“If anybody learns anything, they will,” Magee said. “They are digging into Lily’s life.”

I nodded. “That’s the right angle. Maybe it was a boyfriend who wanted her to go away with him when he heard I was coming home and he killed her when she turned him down.”

They were not impressed. Magee said: “That is a possibility, but the coincidence is somewhat too pat. I mean, the murder occurring only a few minutes after you arrived, after having left your house in a state of—uh—high excitement.”

We were back where we had been. They hadn’t changed their minds; they were merely applying soft-soap. And who could blame them? My hand couldn’t be worse and I couldn’t improve it.

“Listen,” I said. “A couple of days ago I remembered something. When I started down James Street that night, a car approached from the direction of Lily’s bungalow. I thought the driver might be able to tell me where she lived and I flagged him. The car was going very fast on that bumpy road. It didn’t even slow down.”

George Winkler’s bright little eyes glittered. “You think it was the murderer fleeing from the scene of the crime?”

“It could have been.”

“How do you know the car was coming from Lily’s house?” Magee argued. “You say you were at the head of the road. It might have been coming from any of the other houses or through from the other end of the road.”

“The car almost ran me down,” I said.

“A fleeing murderer would be panicky if anybody tried to stop him.”

“Did you see the license number? Do you remember the make of car?”

I shook my head. “The car meant nothing to me then. It might still mean nothing, but it’s worth looking into.”

“We certainly shall.” Robin Magee glanced at his watch. “Take it easy, my boy. I promise you again that you have nothing to worry about.”

More soft-soap. After they left, I stood at the window and thought about the car. It couldn’t be traced. It was just a car, any car, passing in the night, and there was only my word that it had passed.

My word wasn’t good for anything. To everybody it was the word of a murderer.



    
    \begin{ChapterStart}
    \vspace{3\nbs}
    \ChapterSubtitle[l]{Chapter ch5}
    \ChapterTitle[l]{ch5}
    \end{ChapterStart}

    \FirstLine{\noindent ### Chapter 5}

    
## For the People

Sunlight poured in through one of the tall courtroom windows and lay warmly over the long table at which I sat. When I held my hands in the dusty streamers, shadows formed on the table, and with my fingers I fashioned animated images of ducks and goblins.

George Winkler strode over from the jurybox where District Attorney Hackett and Robin Magee were examining talesmen.

“Don’t do that,” he said.

“Do what?” I asked.

“Those tricks with your fingers. You look too nonchalant. You might give the jury the impression that you’re hard-boiled.”

“What’s taking them so long?” George chuckled. “It’s a delight to see Magee work. He’s challenging everybody who hasn’t at least one son in the service. Wives of soldiers won’t do, obviously, because Lily was the wife of one. Magee insists on sons, parents of boys like you. Hackett is challenging those Magee accepts, but it’s a losing game for him because our armed forces are so large. They’re up to the twelfth juror. I realize this is dull, Alec, but whatever you do don’t act indifferent to the proceedings.”

I felt the grim and passionless bulks of my two nursemaids at either shoulder. They had guns on their hips and handcuffs dangling from their belts. They conducted me to and from my cell in the county jail and for lunch to the daytime cell behind the judge’s chamber. In the courtroom they were frozen beef planted behind my chair. Four walls made me feel freer than they did.

I said to George: “If I act the way I feel, I’ll start chewing the table.”

“That mightn’t be a bad idea,” George said.

County Judge Farrier leaned his gray mane toward the jury box. “Is the jury satisfactory to the defendant?”

Instantly the courtroom was silent. “The jury is satisfactory,” Magee said loudly.

“Is the jury satisfactory to the People?”

“The jury is satisfactory,” Hackett said.

“The jury will stand and be sworn.”

***

This was it. There was a stifling hush, as before a storm. I felt the emptiness in the pit of my stomach that I used to feel when we took off on a particularly tough mission with weather conditions uncertain. Only this would be tougher than the worst of them.

The clerk droned the roll. I twisted around in my chair. Between the solidity of the guards, I could see Ursula and Miriam in the front row of benches. Ursula caught my eyes and managed a wan, smiling-through-tears smile. Miriam’s profile was strained and tight as she listened intently to the jurors being sworn in.

Near the rear of the room Kerry Nugent and Helen Spencer sat together. Bevis Spencer came down the aisle and whispered briefly to his sister and Kerry.

“Quiet! ” the bailiff at the railing gate snapped. Bevis flushed and moved on to sit beside Miriam. There were others in the room I recognized, men and women I had known half of my life. It was quite a party.

I turned back to listen to the district attorney who was rising to address the jury. He was a plump, soft-spoken man whom I would have liked under different circumstances. He told the jury that for weeks and months I had planned the murder of my wife and that only a few hours after returning to West Amber I had rushed to the bungalow and plunged a knife into her heart.

Then it was Robin Magee’s turn. He had a voice that thundered and wept and sneered and whispered within a single sentence. He said that Lily had been an evil woman and that I was a decent and patriotic boy. And he sat down.

There wasn’t much for the first couple of hours. The county medical examiner said that a sharp instrument, commonly known as a steak knife, was driven into the body at a point approximately three and a half inches below the left shoulder blade anteriorally and penetrated the heart to a point an inch above the apex, imbedding itself directly between the fifth ttad sixth ribs without emerging from the body. In his opinion, she had not been dead more than an hour when he had first seen the body at eleven o’clock the night of the murder and his post mortem at Willlowby’s Funeral Parlor the following day had verified the time of death. Approximately, anyway.

Magee let him step down without cross-examination. A state police lieutenant told how he and Sheriff Dowie had taken me to the county jail.

A buxom woman named Mrs. Josephs stated that she had rented the bungalow to Mrs. Lily Linn on July 5th. The bungalow had come completely furnished, including a set of six steak knives, of which Exhibit A was one.

A sergeant from the State Bureau of Criminal Investigation testified that the only fingerprints on Exhibit A were mine.

***

Robin Magee rose to cross-examine for the first time. Languidly he sauntered to the witness stand; almost I expected him to suppress a yawn. He must have seen a great criminal lawyer act like that in a movie.

“Tell me, Sergeant,” he drawled, “did you test the five other steak knives in the kitchen for fingerprints?”

“Yes, sir, I did.”

“Why?”

“It’s routine. You can never tell what’ll be important, so we dust everything in the place.”

“Laudable diligence, I’m sure. Did you find fingerprints on the five steak knives in the kitchen drawer?”

“Well, on one was a smear of Mrs. Linn’s right thumb and on another a pretty good impression of—”

“No prints on the other three knives?”

“No, sir. Washing them had removed the prints.”

“Wouldn’t whoever dried the knives have replaced prints on them?”

“It depends on how they’re dried. Most women hold a few pieces of cutlery in a towel and take one end of the towel and dry them like that. Then they drop the cutlery into the drawer without having touched it except with the towel.”

“In short, there were probably no prints on the murder knife when the murderer removed it from the drawer?”

“Probably not. But as soon as he took the knife out of the drawer he put his prints on it.”

“But not if he wore gloves,” Magee said. “In that case the knife could have been plunged into Mrs. Linn’s heart without fingerprints appearing on it until the defendant, Alexander Linn, entered and touched the knife.”

“If he wore gloves, I guess so.”

“One more question, Sergeant. Could fingerprints have been wiped from the knife-handle while it protruded from Mrs. Linn’s breast?”

The silence in the courtroom deepened.

“You mean,” the sergeant said slowly, “take a rag or a handkerchief and wipe the prints off the knife-handle without removing the knife from the body?”

“Exactly.”

“I don’t see why not,” the sergeant said.

Magee strolled back to our table.

“That’s the way it happened,” I told him excitedly. “The murderer either wore gloves or wiped off the prints after he killed her.”

Magee shrugged. “At best it’s a negative point, but it helps confuse the jury.”

***

The sight of that sleek face streamlined by a narrow mustache made me think of bars at a bank teller’s window and a name-plate reading, “MR. E. SCHNEIDER,” and a mechanical remark about the weather as he passed Ursula’s bankbook back to me.

His loose-jointed body lounged in the witness chair as it had never dared in the stiff formality of the bank cage. His full name was Emil Schneider. He was thirty-seven years old and had worked at the West Amber National Bank for eleven years. He had become acquainted with Mrs. Lily Linn when she had come to the bank to cash her Army allotment checks.

“How well acquainted?” Hackett asked.

Schneider crossed his legs. The jurors leaned forward. Magee and George looked at each other. I felt my hands clench.

“It was like this,” Schneider said after a pause. “One morning a year ago last April Mrs. Linn came in to make a deposit and asked me casually if I knew of a house for sale in the neighborhood. She said if she found just the right thing she’d like to buy it and have it ready when her husband came home from the war. I was about to recommend a real estate broker, but suddenly I saw the chance to make a good commission for myself. I knew of a house for sale out in the Big Oaks section, so I told her I’d be glad to drive her around late that afternoon to take a look at it.”

“Was she interested in that house?”

“It was too big for her. I promised her that I’d keep my ears open. In a bank you hear a lot about houses for sale, and whenever I heard of one I took her to see it. But the ones we saw didn’t satisfy her or were too expensive.”

“What happened on the night of July 21st?”

“That morning I heard of a good buy near New Hollow. It was farther from town than Mrs. Linn wanted, but it was worth a try. I phoned her in the afternoon, but couldn’t get her home. Mrs. Hennessey told me she’d moved out that morning. In the evening I heard Mrs. Linn had rented the Josephs’ bungalow and had moved in already. I phoned her there.”

“At what time?”

“Around ten o’clock.”

“Can you be more exact?”

“I remember looking at my watch to see if it wasn’t too late to call her up. It was five to ten when I picked up the phone.”

“Who answered?”

“Mrs. Linn.”

I looked at George and then at Magee. Neither of them appeared to be greatly disturbed. In my statement to Hackett I had said that I had reached Lily’s house shortly after ten o’clock; besides, Dowie had found me there a few minutes later. Didn’t they realize how bad Schneider’s testimony was for me?

“You are absolutely certain that it was Mrs. Linn at the other end of the wire?” Hackett said.

Magee drawled: “Objection on the grounds that no proper foundation has been shown that he knew her voice.”

“He called her house, didn’t he?” Hackett said angrily.

“And spoke to a woman. That’s all that’s certain. It could have been any woman at the house at the time.” Hackett turned to the judge. “Let the counsel show no proper foundation in the cross-examination.”

“Suits me,” Magee said smugly and sat down.

“What was the substance of your telephone conversation?” Hackett asked Schneider.

“I told her about the New Hollow house and she asked me to pick her up next afternoon when I knocked off work.”

“What time was it when you hung up?”

“Five minutes later. Ten o’clock.”

“When did you next hear from or about Mrs. Linn?”

“Next morning I heard at the bank that she’d been murdered. During my lunch hour I went to you to tell you I’d talked to her the night before.”

“You were very cooperative, Mr. Schneider. That will be all.”

***

Robin Magee sauntered over to the witness. Almost he yawned in Schneider’s face. “You have a great deal of confidence in the accuracy of your watch.”

“It’s the one I’m wearing now. It never loses a minute.” Schneider smiled. “Besides, as soon as I hung up I went into the living room and turned on the radio for the ten o’clock news. It had just come on.”

“Was anybody with you when you made the phone call to Mrs. Linn?”

“I was alone at home.”

“You have a wife and two children, I understand.”

“Yes.”

“Were they at home at the time?”

“They were visiting my wife’s mother in Vermont.”

Magee rubbed his chin reflectively. “You say that Mrs. Linn consulted you about a house as long as fifteen months ago?”

“About that long ago.”

“And you showed her houses throughout that period?”

“Whenever I heard of something which might interest her.”

“Didn’t you get discouraged?”

“I stood to make from five hundred to a thousand dollars commission. That kind of money is encouraging.”

“Didn’t it seem to you excessively ardent salesmanship to see Mrs. Linn at least once a week for a period of fifteen months?”

Schneider uncrossed his legs. “It wasn’t that often.”

“Did your wife leave your bed and board last May and take your two children with her to her mother’s house in Vermont and not return to you?”

Schneider’s nonchalance was gone. It all belonged to Magee now. I heard a stirring behind me among the spectators.

“We had a fight,” Schneider said unhappily.

“Because of Lily Linn?”

“No. My wife and I hadn’t gotten along for years.”

“What were your relations with Mrs. Linn?”

“I—we—” Past Magee’s shoulder Schneider stared at me.

Everybody stared at me. I had risen from my chair and stood with my knuckles pressed down on the table. The hand of one of my nursemaids fell on my shoulder and pressed me down to the chair. I ran my fingers through my hair. I was trembling.

“What were your relations with Mrs. Linn?” Magee asked again.

Schneider sat back. “I don’t know what you mean. We had a business relationship, of course.”

“I am referring to an intimate personal relationship.”

Hackett leaped to his feet to make a vigorous objection.

“I wondered why you didn’t object before,” the judge said. He scowled down at Magee. “Objection sustained.”

Breezily Magee waved a hand. “I withdraw the question for the present. I reserve the right to recall Schneider for the defense if necessary.”

When Magee returned to the table, I leaned into his whiskey breath. “Schneider is lying,” I whispered. “Lily had no money to buy a house. She was always complaining in her letters about being broke. He murdered her. He lied about that phone call.”

Magee patted my arm. “You leave him to me. He’s going to be a mighty sick man before the trial is over.”

“You mean you can prove he murdered her?”

“Hardly that. But as one poker player to another, watch the way I’ll call the D. A.’s hand.”

***

There were four women on the jury and eight men, all middle-aged or older. I tried to make something out of their set, attentive faces, separating each from the other, but they remained a blurred entity. A dozen faceless people you could have picked out of any crowd, but they would decide whether I burned in the electric chair or rotted in prison or walked once more among my fellow men.

Hackett was reading from a sheet of paper. I yanked my mind away from the jury and heard:

“… don’t see why you can’t manage on the allotment you receive as the wife of a first lieutenant, plus whatever else I send you out of my pay. Unlike most other officers’ wives, you don’t even have to pay rent.”

“What’s that?” I asked George Winkler.

“Don’t you recognize your own letters?” George said. “They’re the ones you wrote to Lily from India. Hackett found them in her bungalow and is reading excerpts into the record.”

“Is he allowed to make public my private mail?”

George motioned to me to let him listen.

Hackett was reading:

“Can’t you ever write about anything but that you’re bored or haven’t enough money? You don’t even bother any more to throw in a casual mention at the end that you love me. Oh, I understand. It’s being separated for so long. It must be even harder on you than on me. We’ll make up for it when I get back.”

The next letter was dated a month later. It read:

“Needless to say, the news in your last letter that you drove all the way to New York with Billy—whoever or whatever he may be—to a night club and Started back to West Amber at four in the morning and drove till noon, has brightened my days and made my nights restful. Now I face each mission with stalwart heart, confident that my wife is loyally keeping the home fires burning.

I’m not jealous. I don’t expect you to shut yourself away from the world and make yourself sick pining away for me. I don’t mind your seeing other men now and then, though perhaps trips to New York with men named Billy is—

Damn it, I am jealous! Not because of what you write, but because of what you don’t. I assume that you are faithful, but you might make as much mention in your letters of me as you do of your Billys. You might even say that you miss me.”

I knocked my chair over as I jumped up. “You can’t read those!” I said. “They’re personal!”

I dashed for Hackett. He put the letters behind his back and tried to hold me off with his free hand. Then my two guards reached me and each grabbed one of my arms. All twelve jurors were on their feet to see better, and the courtroom was in chaos. The judge banged his gavel. I was sputtering, unable to find words to express my outrage.

***

The judge said something and Magee and George collected my arms from the guards and led me to the bar.

Gradually quiet resumed behind me. The judge gave me a lecture. He said that I was on trial for my life and that those letters were evidence and that he would not tolerate such conduct. I was wondering what more could be done to me, but he didn’t say. Magee led me back to my chair and I sat sweating and trembling and helpless.

Hackett had more. He read:

“Today everybody got mail from wives and sweethearts. There were letters for me, too—one from Miriam and one from George Winkler. The letter from Miriam was sent out airmail only eight days ago. She said that you were well. But no letter from you. None in seven weeks. Are you so busy with your various Billys that you can’t spare five minutes to drop a line to your husband? This morning Kerry Nugent asked me why I looked so glum lately. I told him that if I could get leave to go home and strangle my wife, I’d come back a new man.”

George sent a frown past me to Magee. “Not so good. Hackett is trying to show that while still in India Alec planned to murder her.”

“That’s only an expression men are always using about women, that they want to strangle them,” I said. “Can’t you make that clear?”

Magee patted my shoulder. “I’ll take care of those letters when the time comes. Have you the letters your wife sent you? They would be useful on our side.”

“They weren’t the kind of letters I’d want to keep,” I said fiercely. “And even if I had them, you wouldn’t get them.”

Hackett was still reading. I kept my eyes fixed on the top of the table. I was ashamed to look at anybody.

***

Bevis Spencer stumbled over the oath, sent a startled glance at me, sat uneasily at the edge of the chair.

“Is Bevis going to testify against me?” I asked George in surprise.

“Hackett subpoenaed him. He was unhappy about it and had an idea I could get him out of it. I couldn’t, of course.”

“You got Miriam off.”

“Hackett wasn’t keen on having her to begin with,” George said. “He decided that her testimony would work against him. As a matter of fact, Magee wants her for the defense, but she refuses to take the stand. I stopped Magee from getting tough with her.”

Bevis Spencer told how he had arrived for the Saturday night poker game half an hour after his father and George and Dowie. He had remained in the living room to spend a few minutes with Miriam. Then I had come in and asked to be alone with her.

“What was the defendant’s demeanor when he entered?” Hackett asked.

“Quiet, the way he usually was in all the years I’d known him. He looked a lot older, though, than two years ago. He didn’t get excited till later.”

“How much later?”

“Two or three minutes. I was in the hall, on the way downstairs to the card-room, when I heard my name and stopped to listen.” Bevis squirmed on the edge of the chair. “I’d just asked Miriam Hennessey to marry me, and I heard her tell Alec that. Alec has a lot of influence with her and I wanted to know how he felt about my marrying her.”

“We can understand a perfectly human action,” Hackett said graciously. “Was anything said about Lily Linn?”

“They started off by discussing me, then suddenly Alec wanted to know where Lily was staying. He said he’d found out that Lily wasn’t living with them any more and he got very excited.”

“By ‘them’ you refer to his sister, Ursula Hennessey, and her ward, Miriam Hennessey?’’

“Yes.”

“You stated that the defendant became very excited,” Hackett said. “What do you mean by that?”

“He started to shout.”

“What did he say?”

“I didn’t hear any more. I went downstairs to the cardroom because what I was listening to wasn’t any of my business.”

Bevis went on to tell about the scene in the cardroom when I burst in a few minutes later. He tried to make it sound mild, but Hackett wouldn’t let him. He said that after the game broke up he had gone out on the porch with the others. His sister Helen and Kerry Nugent had pulled up in a car.

“Did Captain Nugent remain long?” Hackett asked.

“He left a few minutes later in his car. I heard Miriam ask him to try to bring Alec back.”

“To prevent the defendant from going to his wife?”

Bevis’ somber face was miserable. “Well, Alec had become psychoneurotic in the Air Force and he—”

“Never mind that,” Hackett said quickly. “You’re not a psychiatrist.”

Magee chuckled. I wondered why.

“I was just telling you the reason they were worried about him,” Bevis said. “Alec’s nerves—”

“Never mind. I’m interested in the defendant’s actions when he learned where his wife was staying.”

“Well, he rushed out of the house.”

“Without pausing?”

“He was out of sight when I got to the porch.”

Hackett smiled and walked away. Magee smiled also and waved Bevis off the stand.

***

Sheriff Owen Dowie peered about myopically and then brought his pale eyes to rest on Hackett who was waiting for a reply to a question. Noisily Dowie cleared his throat. “Properly speaking, I didn’t arrest the defendant. I handed him over to Lieutenant Searson of the state police. I’m only a peace officer.”

“Tell us what happened the night of July 21st.”

“I’d been invited to play a little poker at Mrs. Ursula Hennessey’s house. Just a harmless little game in a private home between friends. Nobody could call it gambling.”

There were titters in the courtroom. The judge worked his gavel without passion.

Hackett said dryly: “Poker is a great American institution and I am sure the sheriff is as human as the rest of us.”

“Like I said, a friendly little game. Well, around eight o’clock that night Oliver Spencer phoned me and asked if I’d pick him up in my car on the way. His boy Bevis had taken the car over to the store; he was taking stock and would be late. So at eight-thirty I picked up Oliver Spencer. When we got to Mrs. Hennessey’s house, George Winkler had just arrived in his own car. We stood out on the driveway talking about Alec Linn, who’d just got home. Spencer and Winkler said Linn’s wife had been cutting up while he was away and they were worried he’d find out and—”

Magee made a motion to strike out. The judge granted it. Hatkett said to Dowie: “Please confine yourself to facts.”

“But everybody knew Lily Linn was running around with—”

“Please. Hearsay is not admissible.”

“Oh, all right. Then we went up to the house and the defendant himself opened the door for us. He didn’t want to come down to play with us. I’d heard that he was a whale of a poker player and I was anxious to play with him, but he didn’t want to. So the three of us went downstairs to the cardroom and played knock rummy till Mrs. Hennessey finished washing the dishes and then we started a four-handed game of stud. It wasn’t fun playing four hands, especially as Mrs. Hennessey, who’s a sharp player, wasn’t paying attention to her cards. Then around nine-thirty, or a little before, Bevis Spencer came down and joined the game. But it didn’t go much better because we could hear the defendant yelling—”

“Yelling?”

“Talking in a very loud voice. He was with Miriam Hennessey in the living room right over our heads.”

“Did you hear what the defendant said?”

“No. Just his voice, very loud. Suddenly he came dashing down into the room. His face was all twisted up and he yelled he wanted to know where his wife was. Mrs. Hennessey tried to quiet him, but he started swearing and yelled over and over he wanted to know where his wife was. It was mighty embarrassing for all of us.”

“Can you recall definite words that the defendant uttered?”

***

Dowie removed his glasses. “He said: ‘I know what Lily was up to while I was away.’ George Winkler tried to reason with him, and the defendant told him to mind his own—pardon me—damn business, and the defendant said: ‘I know exactly what I’m going to do to my wife.’”

“Are you sure those were his exact words?”

“Maybe he said he knew what to do with his wife instead of to his wife. I didn’t take what he said down in shorthand. Ask the others. They’ll agree I’m close enough. Then Mrs. Hennessey told the defendant that his wife lived on James Street, and without another word he ran out like there was a fire. We could hear him running in the upstairs hall and slam the front door.”

“What happened then?”

“Mrs. Hennessey didn’t want to play any more, and I guess the rest of us didn’t either. We all went outside. I told Oliver Spencer I’d drive him back to his house, but he said his son Bevis would drive him home, so I got in my car and drove off by myself.”

“What time was that?”

“A quarter to ten. I looked at my dash clock when I got into my car. The defendant had left around five minutes before. I drove home. My wife hadn’t come back from her knitting club—it was still early—so I drove down to town and had a beer at Lou’s on Division Street. All the time I was thinking about what had happened at the house. I’d heard the stories—” Dowie hesitated. “You don’t let me tell the stories I’d heard.”

“You can tell us why you went to Mrs. Linn’s bungalow.”

“It was on account of what had happened in the Hennessey house and what I knew about Mrs. Linn and that boy running out of the house like a wild man. I’m a peace officer. It’s a twenty-four-hour job. After a while I made up my mind that that boy, the defendant, wasn’t in a fit state to go busting in on his wife. I don’t say I had any idea of murder, but just the same I was worried. I figured there was no harm driving past Mrs. Linn’s bungalow. When I got there, all the lights were on, but it was very quiet. Everything looked all right. I drove up that little dirt driveway at the side of the house so I could turn around. But I didn’t back up. I could see through one of the windows into the living room, and there was the defendant sitting with his head in his hands. I watched him. He didn’t move. I got out of the car and walked to the window and saw a woman on the floor. I ran around to the front door and went in.”

“What time was this?”

“Fourteen minutes after ten.”

“According to Emil Schneider’s testimony, only fourteen minutes after Schneider had spoken to Mrs. Linn on the telephone. Did she appear to be dead?”

“I’m no doctor, but a knife was in her heart and she had no pulse I could find.”

“What did the defendant have to say?”

“He didn’t say anything at first. Captain Nugent came in right after me. Then the defendant said he didn’t kill her.”

“What else?”

“He just sat there on the couch, his feet almost touching the dead woman, till the state police came and Lieutenant Searson led him out of the bungalow. Then the defendant said: ‘My fingerprints are on that knife.’ I said: ‘What do you expect?’ He said: ‘You don’t understand. I started to pull the knife out without thinking.’ I said: ‘Do you want to make a statement now?’ He said: ‘I’ve made my statement. I didn’t kill her.’ ”

“Did you take a close look at the knife in the woman’s heart?”

“Sure. It was just like Dr. Abies said he found it.”

“How was that?”

“I couldn’t see the blade. The handle was in up to that gown she was wearing.”

“It wasn’t halfway out?”

“Well, maybe a little way out. There was that thick gown, like I said, between it and her skin. But not, not halfway out.”

“Thank you, Sheriff.” Hackett turned to Magee. “Your witness.”

***

Robin Magee tapped two fingers over a yawn. “Tell me, Sheriff, how long was it between the time you saw Lieutenant Linn on the couch and the time you entered the bungalow?”

“Maybe forty seconds. No, I’d say at least a minute. I cut the ignition and got out of the car and looked in the window and went around to the front door.”

“Had Lieutenant Linn moved in that interval of a minute?”

Hackett came forward. “Your honor, there has been no Lieutenant Linn mentioned. Whom does Magee mean?”

Magee beamed at the D.A. “You know very well I mean First Lieutenant Alexander Linn of the Army Air Force.”

“If by that he means the defendant, he is not in any way any longer connected with the Air Force.”

The judge clucked his tongue. “I don’t see where it matters. My father was called colonel for twenty-five years after he retired from the Army. Continue.”

What difference did it make whether I was electrocuted as Lieutenant Alexander Linn or as plain Alec Linn? None if the battle of legal wits concerned my guilt or innocence, but I was beginning to get the idea that that wasn’t the point at the trial at all.

Magee repeated: “Did Lieutenant Linn stir during the minute between the time when you first saw him and when you entered the bungalow?”

“Not a muscle.”

“Did you hear Captain Nugent’s car pull up into the driveway?”

“Yes.”

“How did Lieutenant Linn react when he heard that car?”

“He looked up and listened.”

“Let’s go back a few minutes. While you were in your car, after you had pulled into the driveway, was it possible for any person in the premises to leave without being seen by you?”

“He could have gone out the door on the other side of the bungalow—the door out of the kitchen.”

“But Lieutenant Linn remained in the bungalow?”

“He just sat there with his head in his hands.”

“That’s all.” Magee sauntered away. There was silence. Then Hackett stood up. “The People rest,” he said.



    
    \begin{ChapterStart}
    \vspace{3\nbs}
    \ChapterSubtitle[l]{Chapter ch6}
    \ChapterTitle[l]{ch6}
    \end{ChapterStart}

    \FirstLine{\noindent ### Chapter 6}

    
## For Me

Ursula wore a black-and-white print dress and a black cartwheel hat. She looked chic, as the fashion magazines put it, but not so obtrusively smart that the rather dowdy women jurors would resent her. There were tired and frightened lines at her eyes and mouth, but she strode to the witness stand like a man in a hurry. Her deep voice, though tense and somewhat subdued, filled the courtroom. She spoke of the early life of her kid brother.

I couldn’t see what that had to do with the case, but Robin Magee considered it important. Now that the witnesses were his, Magee assumed a different character. He was genial, sympathetic, a man to receive confidences.

“Alec was a shy and sensitive boy when he came to live with me,” Ursula said. “He never got over it. He became even more indrawn, if anything. He would have long spells of brooding, and suddenly, without warning, he would flare up into fits of temper.”

I stared at her, but she was careful not to look in my direction. She was lying about me. I’d been no different than any other boy, except perhaps that I’d been somewhat more even-tempered. The flare-ups came later, the night I’d returned home. Why was she trying to tear me down?

“Was he unfriendly?” Magee asked. “Not at all. Alec was—and is—the sweetest and most devoted boy I know. He has many friends. Everybody likes him. But he’s—well, he takes everything extremely seriously.”

They kicked my character around some more before they came to the point, which was Lily.

“One day Alec, brought that woman home,” Ursula said. “It happened so quickly, without preparation. She was beautiful, I suppose, but in a hard, flashy way. She claimed to be twenty- two; I’d put her age closer to thirty. But Alec was very much in love with her and was going overseas in a very few days, so Miriam and I accepted her as heartily as we could. Even when Alec asked us to let her live with us until he returned we put on a good face. There wasn’t anything we wouldn’t do for him.”

“How did you and Miriam get along with Mrs. Linn?”

“There was little friction, though she seldom helped with the housework. She went her way and we went ours.”

“Do you recall the conversation you had with Mrs. Linn on the morning of July 5th?”

“That was the morning I asked her to leave the house. Her conduct had become outrageous.”

“Please explain that.”

“The night before Miriam saw her disgustingly drunk in public with strange men. That was too much. I simply could not let her remain in my house.”

“What happened when Lieutenant Linn returned home on July 21st?”

“I didn’t break the news to him that I had asked his wife to leave the house. I told you how seriously he took everything. On top of that, he had been sent home because his nerves were very bad.”

***

HACKETT objected that she wasn’t an authority on nerves. The judge sustained him.

“Anyway,” Ursula said, “I kept putting off telling him. You must understand how I felt.”

“I’m sure we all do, Mrs. Hennessey,” Magee said sympathetically. He was through with her.

Ursula’s testimony hadn’t contributed a thing to show that I was innocent. It wasn’t meant to. The idea was to compress my character into a mold. I was one person to the prosecution and another to the defense. Take your pick—a coldly calculating killer or a sensitive, high-strung lad who had suddenly gone haywire through no fault of his own.

Hackett had his turn at the mold. “I understand, Mrs. Hennessey, that since childhood your brother has been something of a mathematical genius?”

“Alec is clever.”

“And that at his last two years at college he was the intercollegiate chess champion?”

“Yes.”

“And that he’s a poker player of extraordinary skill?”

Ursula frowned. “He is very good.”

“In short, his various laurels and honors show that in that science and in those games of skill which require a clear and exact mentality, a trained ability to think ahead and plan concisely and without emotion, your brother excels?”

Ursula opened her mouth, but Magee beat her to the punch. “Your honor, that’s not a question.”

“I phrased it as such,” Hackett retorted. “Mrs. Hennessey, as the sister with whom the defendant lived and who saw him through school, is certainly an authority on his scholastic abilities.”

They wrangled in front of the bench, but I didn’t hear any more of it. I turned furiously to George Winkler.

“You lied to me, both of you. Magee isn’t defending me. He’s trying to show that I wasn’t psychologically responsible, and Hackett is trying to show that I was. I haven’t yet heard Magee deny that I’ve murdered Lily.”

“He’s working every possible angle,” George whispered. “This is only one of them. Give him a chance.”

Ursula was stepping down from the stand. I subsided—at least to the extent that I kept my mouth shut.

***

He said that his name was Ogden Garback and that he worked in his father’s garage in West Amber. He was nineteen, a nice looking lad, slim-hipped and broad-shouldered. He grinned nervously and told Magee that he wasn’t in the Army because of a bad kidney.

“It was just before Christmas,” Garback said. “Last Christmas. It was after eight and I was locking the pumps when Mr. Schneider pulled up for gas and there was a woman with him who wasn’t Mrs. Schneider.”

“Emil Schneider who is employed at the West Amber National Bank?” Magee asked.

“Yes, sir. He gets all his gas at my father’s place.”

“Who was the woman in the car?”

“Mrs. Linn. Bill Beaty was hanging around waiting for me to close up so we could bowl a few games, and when they drove away Bill told me who she was.”

“When did you see Mrs. Linn again?”

“One afternoon just after Christmas. The garage is only a quarter of a mile from the station and she came walking up in the snow and said there was no taxi waiting and would I drive her home to Mandolin Hill. Pop told me to go ahead and I took the car, but I clean forgot to collect the dollar for driving her.”

“How did you do that?”

“I just forgot. I guess it was because she was talking so much, asking me all about myself. When I got back to the garage I remembered about the dollar and called her up and she said come around for it at nine. She was waiting out on the road when I got there. She said did I want to spend the dollar drinking beer with her and I said sure and we drove to Mike’s on New Hollow Road. Bill Beaty was there with a couple of fellas, and he came over to our table and said hello to her and we took turns dancing with her. Then I went out to the—the—” he looked out in embarrassment at the many eyes focused on him, “—well, I went out for a few minutes, and when I came back Mrs. Linn and Bill were gone.”

“Is the Bill you refer to William Beaty, now in the Navy?”

“Yes, sir.”

“When did you see Mrs. Linn again?”

“A couple of days later she called me up at the garage and said she was sorry she’d walked out on me and would I pick her up on the road at nine. So I did.”

“Where did you go with her?”

“No place. We drove for a while and then parked.”

“Did you make love to her?”

The kid flushed. “She let me kiss her a few times, then she said we should go home.”

“Did she say why?”

Garback looked as if he were searching for a hole to jump into. “She said I I was a nice boy and too young for her and I shouldn’t see her any more.”

Magee didn’t like that. He didn’t want to hear any good about Lily. “And you didn’t see her again?”

“I called her up a few times, but she always turned me down.”

“Did she continue to go out with William Beaty, if you know?”

“Yes, sir. Lots. I saw them together in Bill’s car.”

“What were his relations with her, if you know?”

***

The kid wiped his mouth. It was plain that he had had it bad for Lily, and I could understand that and how he must have felt when she had dropped him after a kiss or two and how it must have hit him to have known all the time that his friend Bill was having better luck with her. With my wife. That had been my wife they were speaking about.

“Well, sir,” Garback said, “Bill always told me every time—”

“No, give me your own observations.”

“Well, sir, you don’t think I was around to see—” Garback flushed. The boisterous laughter which swept the courtroom had to be silenced by the judge’s gavel. “One Saturday Bill stopped off at the garage for gas. Mrs. Linn was with him. While I was filling the tank he told me they were driving to New York and hoped they wouldn’t get back till next day. And next day, Sunday, they passed around twelve o’clock, coming back. They’d been together all the time, and they’d spent the night in a tourist cabin in—”

I was on my feet, trembling, pulling my arm out of George’s grip. “What has her personal life to do with this trial?” I shouted. “She’s dead. Why not let her alone?”

The guards shoved me down. Hackett said: “Your honor. I agree with the defendant. I’ve been giving Magee plenty of leeway, but I fail to see—”

“You’ll see if you give me a chance,” Magee told him.

They threw legal jargon at each other. I chewed on the knuckles of my clenched hands and felt all those eyes on my back.

***

Magee must have won his point because Hackett shrugged and walked away. The judge said to me “The defendant will please restrain himself.” He said it kindly. He was sorry for me; so was everybody else in the room. Not because I was being tried for murder, but because I had been married to a woman like Lily.

Magee was back at the witness stand. “Did Emil Schneider ever speak to you about Mrs. Linn?”

“Yes, sir.”

“What did he say?”

“Hey!” Hackett protested. Then he added: “Irrelevant and immaterial.” Magee turned to the bench. “Will the court take it subject to correction?” The judge looked interested. He nodded. Magee repeated the question.

“Well,” Garback said, “he came to the garage one afternoon and said he’d break my neck if I didn’t keep away from Mrs. Linn.”

“When was that?”

“About the beginning of February.”

I told him he was backing up the wrong alley, but he stayed sore and asked what did I know about Bill Beaty going around with her. I said to go ask Bill and he did and Bill told him to jump into a lake. Then a few weeks later Bill was drafted.”

“And that left the field clear to Schneider?”

“Move to strike out,” Hackett growled.

“I withdraw the question,” Magee said, grinning.

***

Helen Spencer had been the prettiest girl in high school. She had had a heart-shaped face and red cheeks and warm gray eyes and wavy honey-colored hair falling to her shoulders. She still had all of that, but she was trying to change it—to look worldly and sophisticated by aid of an up-swept hairdo and a long clinging dress and sleek make-up and scarlet fingernails. They didn’t do the trick. She was still a peaches-and-cream girl.

“Why, yes,” Helen said, “I became rather friendly with Lily Linn. I’m a friend of Miriam Hennessey and came to the house rather often, and Lily seemed to like me. She knew everything about the latest styles and took me shopping with her and helped me fix my hair and all that.”

“Did you go out on dates with her?” Magee asked.

Helen looked over his head. I turned and everybody else who was facing her turned. Oliver Spencer, her father, stood in the back of the courtroom with his hat in his hand. When he realized that he was suddenly the focal point of attention, he stopped chewing his lower lip and assumed a deadpan.

“Last month two men came up from New York to see her,” Helen said slowly. “Lily told me they were old friends and asked me to fill out the second couple. We went to a roadhouse in Trevan.”

“When was this?”

“On the Fourth of July. Lily got very drunk and started acting disgracefully, right out in public, letting this man Don who was with her make love to her and everything.”

“Everything?”

“Well, paw her. Even though he had been her husband—”

“Husband!” Magee exclaimed.

“Her first husband—the one she’d divorced.”

I found that I was sitting forward with my mouth hanging open. I closed it.

Magee glared at her. “Why didn’t you tell me when I spoke to you last week that Mrs. Linn had been married before?”

“Didn’t I?” Helen said serenely. “I thought I did when I told you about that night at the roadhouse.”

“You merely said that they were two men from New York.”

The judge rapped impatiently. “Please don’t argue with your witness, counselor.”

George Winkler growled in my ear: “You didn’t mention to us that Lily had been married and divorced. We could have brought her first husband up here and perhaps have got something out of him. Now it’s too late.”

“This is the first time I heard of him,” I said.

There was a great deal about Lily I hadn’t known and still didn’t. I yanked my attention back to Helen’s testimony.

“I never heard his second name,” she was saying. “Lily called him Don.”

“Short for Donald?”

“It may be. She just called him Don.”

“All right, go on.”

***

Helen glanced again at her father in the rear of the room and then quickly down at her knees. “Lily was very drunk and so was Don. He asked her to go away with him. She laughed and said they’d never get along together, but she didn’t stop him pawing her. Then Don said: ‘I’ll settle for a few days in Miami.’ Lily said: ‘What about that she-cat you’re living with? She’ll hardly approve.’ Don said he didn’t care what she thought. If she didn’t like it, she could—”

“Wait?” Magee broke in. “Who is this woman you’re talking about?” Helen raised her head and lowered it. “They didn’t mention her name. Then Lily told Don she couldn’t leave West Amber for even a day because she’d just got word from Alec that he was coming home. From now, she said, she’d be a good girl. And Don said: ‘And make a home for this hick and raise babies.’ Lily made a sour face. She said: ‘For my part, I’d rather have his insurance than him.’ ”

This was the worst kick of all, right in the pit of the stomach. All along I’d assumed that whatever she had done was out of boredom. But those others hadn’t even been substitutes for me. She’d preferred that I be killed in action so that she could collect my insurance.

I wanted to crawl away somewhere and lie alone with my face in my arms.

Helen was telling the rest of it. “Don and the other man—his name was Walter—laughed drunkenly, and Don said to Lily that she’d gotten herself stuck good and proper. That made Lily angry, but I’m sure that if she hadn’t been very drunk she wouldn’t have said what she did in front of me. She said how was she stuck if for two years she’d been living on an officer’s pay and in a fine home where she didn’t have to pay board? She said maybe she’d like Alec when he came home. She didn’t know; she’d even forgotten what he looked like. It would be stupid to walk out on him now. If she decided she didn’t want him when he came back, she was sure his sister, who had money, would buy her off to take her hooks out of him.”

“Are these Mrs. Linn’s exact words?”

“As far as I can remember. Lily went on like that. She said she was going to be a very good girl, at least till Alec came home, and that would be very soon. Then my brother and Miriam Hennessey came in and saw how Lily was carrying on with Don. Bevis insisted on taking me right home, and I was glad of it.”

“Did you see Mrs. Linn after that?”

“She called me up the next morning. She was sober and knew I was a friend of Alec and the family and realized she’d made a mistake blabbing in front of me. She asked me to come over so that she could explain.”

“Was she still living in Mrs. Hennessey’s house?”

“No, she’d left that morning and it was the bungalow on James Street. When I got there, Mr. Schneider was arguing with her.”

“What about?”

“They stopped when I came in, but I heard a little when I went up to the door. Lily told him that he mustn’t see her again because her husband was coming home. Mr. Schneider was begging her to go away with him.”

“Anything else?”

“Then I knocked and Mr. Schneider was very embarrassed when I came in. He left right away. Lily told me to forget what she’d said last night in the roadhouse. She had only been pulling Don’s leg, she said. I told her the truth had come out when she’d been drunk, and I walked out.”

“And that was the last time you saw her?”

Helen looked down at her crossed knees. “Yes,” she said.

***

Emil Schneider was limp and scared when he returned to the witness stand. Magee was merciless.

“You are aware that you are under oath?” Magee said crisply.

“Yes.”

“Did your wife leave you because of Lily Linn?”

“Yes.”

“When you phoned Mrs. Linn on the night of July 21st, you did not have real estate on your mind, did you?”

“I asked her if I could come over.”

“What did she say?”

“She said no. She said she’d made up her mind not to see other men until her husband came home. She said there was enough talk about her already.”

“Do you know whether she was aware that her husband had returned that evening?”

“She didn’t say anything about it.” Magee jabbed a finger at him. “I’ll ask the question I started to ask yesterday. Were you intimate with Mrs. Linn?”

“I—it wasn’t—”

“Answer me!”

There were no bones in Schneider’s loose body. “Only a few times.”

“In short, yes.”

“Yes,” Schneider whispered.

***

During the recess in my daytime cell in back of the courthouse, I said angrily: “So you’re not going to try to prove I’m innocent? You’re double crossing me!”

George Winkler sighed and moved in on me to apply soft-soap, but Magee waved him silent. He handed me his flask of very good rye. I took a deep slug and so did Magee, but George shook his head. Then Magee said: “All right, my boy, we’ll do whatever you say. What should we do?”

“You’re the lawyers,” I muttered.

“We’re not supermen. You heard the D. A.’s case. He showed that you were aware of your wife’s unfaithfulness—”

“I wasn’t. Not quite. You were the one who spent all morning giving me the motive to kill her.”

“—and that you rushed out of the house and that fifteen minutes after she was known to have been alive you were found sitting next to her murdered body and your prints were on the knife.”

“I explained about the fingerprints.”

“It was an explanation, the only one you could possibly make. Do you think the jury will believe you? Put yourself in their place. Would you have a shadow of doubt as to your guilt?”

I was boxed in. I groped for a way out. “The murderer is this man Don, her former husband, to whom she refused to go back.”

Derisively Magee lifted the corners of his mouth. “Yesterday you said it was Schneider.”

“Perhaps it is Schneider. He was madly in love with her and she wouldn’t go away with him.”

“Why leave out Ogden Garback whom she kissed and then dropped?” Magee shook his head. “The D. A. has made out an excellent circumstantial case. We can’t do a thing to counter it. Can we?”

I wanted more of Magee’s rye, but I didn’t ask for it. I set fire to a cigarette.

George said: “But Magee will get you off just the same. You ought to appreciate how clever he has been. Helen Spencer’s testimony was wonderful. Kerry Nugent’s will be even better.”

I ground the cigarette under my heel. “The unwritten law,” I said bitterly. “Temporary insanity. Extenuating circumstances. But not that I’m innocent. Neither of you believes that I didn’t do it.”

“Of course we do, my boy,” Magee lied suavely.

“Well, I’m going to tell the jury the truth when I’m on the stand.”

There was a silence. Then George said softly: “We’re not going to put you on the stand, Alec.”

“Why not? Afraid I’ll tell them I’m innocent?”

Magee handed me another of his pats on the shoulder. “You can’t help yourself and you might harm yourself. Everything is going line now. Why run the risk that you’ll antagonize the jury? You’ll be too pugnacious.”

“I don’t care.”

“My boy, do you want to live? No, you’ll live anyway; they won’t send you to the chair. Do you want to spend from twenty years to life in jail?”

The walls pressed so close that I couldn’t breathe. Through a window at my side I could see a patch of cloudless sky. I wished I were up there, not cooped up in a bomber, but flying free and alone at ten thousand feet in a single-seater.

“Have it your way,” I said listlessly.

***

Kerry Nugent made a splendid figure on the witness stand. He sat at ease in his natty uniform with the impressive double string of ribbons on his deep chest, but his rugged face was stiff with earnestness. He could utter sheer nonsense, and whatever he said would carry weight. I could see it in the faces of the jury, sense it among the spectators. Kerry wouldn’t let them convict me.

He told of our boyhood together. He made it sound idealistic, and maybe it was. A couple of clean, healthy, decent kids, devoted to their family—in my case, to my sister—and when they grew up responding at once to the call of duty when their nation was in danger.

“Alec and I went together to enlist in the Air Force,” Kerry said. “That was in our senior year at college.”

“What kind of boy was he?” Magee asked.

“Brighter than most. At school he took honors right and left, especially in math. I went out for athletics, but he was the studious type. High-strung. Thought a lot about everything. Used to he awake nights dreaming up new chess openings. Temperamental. Sometimes he’d almost bite my head off for no reason. Though I don’t mean that he wasn’t one swelI guy.”

“You two enlisted as soon as you graduated from college?”

“A couple of months before, but it wasn’t till fall of that year before we got our orders. Alec was a natural for a navigator. After we received our wings, we were put on B-17’s. We were about to be shipped to the European Theater when I was transferred to the new B-29’s. I pulled wires to get Alec assigned to my ship, so we stayed together. The B-29’s weren’t flying yet. They were just coming off the assembly lines in Kansas City and wee had to train for them and we—”

“Tell us about the weekend you and Lieutenant Linn spent in New York before flying to India.”

“We got ten days’ leave and spent the first few days of it at home. There was a party in Greenwich Village Alec and I were invited to. We decided to have a last fling and went down to New York by train and registered overnight at a New York hotel. At that party we met Lily Yard.”

“Yard?” Magee looked startled. “Don Yard? Was Yard her maiden or her married name?”

***

“I wouldn’t know. That’s the name she gave us. I saw her first and made a power dive for her. Alec was right behind me. She was worth rushing—tall and slim and platinum hair and a build you’d pin up in your locker any day. Alec won out. That was okay by me. There were plenty of other—” Kerry wet his lips. “Anyway, Alec didn’t go back to West Amber with me next day. He was at the Municipal Building with Lily getting a marriage license. That quick. But you know how it is. A man is going into combat and the odds are he won’t come back, especially if he’s in a bomber, so why not grab what you can while you can get it? I don’t mean he wasn’t crazy in love with her besides. As for Lily—well, you heard about her. She saw a chance to get herself an officer’s allotment and maybe his insurance. I saw her once more, when he brought her home. Then a couple of days later we flew to Africa and from there to India.”

“Did he talk to you about her?”

“Nothing else but. He called her the Lily. The Lily. She looked like one—the tall, white kind anyway—but that’s not all he meant. Pure as a Lily. Virginal. You know.”

“I know. We’ve heard a great deal of her purity this morning.” Magee sent a look at the jury and the jurors made their faces grimly angry. “When Lieutenant Linn reached his base in India, did he hear often from his wife?”

“That just it. He didn’t. You can’t imagine what letters meant to us unless you’ve been in it yourself, especially letters from the woman you love. She didn’t write often. You could hear Alec’s heart crack when he received a fistful of mail and there’d be nothing from her. She was the only one he wanted to hear from. Then after a year the letters she did write became worse than none at all. The other day you heard his answers to some of her letters. Those were just his answers. You should have seen the letters she wrote him.”

“Lieutenant Linn showed them to you?”

“Sure. We shared all our letters, no matter how intimate they were. You share death in the sky, so you share the rest of yourself on the ground. I don’t know why women do certain things. Maybe they’re bored, home alone. They blame the war and take it out on you. It wasn’t just Lily. I’ve seen wives and sweethearts do it to other men. Because you need them so much and think about them so much, they have a knife in you, and they take pleasure in twisting it. Not all women, not many, but more than you’d think. Lily was one—anyway, after the first year. It was a little thing she had to do, write more often and lie in her letters if she didn’t feel affectionate. But when Lily wrote, she twisted the knife all she could. Nagging about money, for instance. It got so bad that Alec went into a couple of those big crap games we have at the club, but crap is straight gambling and instead of making more money for Lily he dropped a couple of hundred bucks. He was even ready to play poker, but he didn’t.”

“It has been testified by Mrs. Hennessey that Lieutenant Linn is an exceptionally skillful poker player. If he wanted money, couldn’t he have made it that way?”

***

“Sure. The trouble is that most men play with a hunch and a prayer. They haven’t any idea of percentages or the fine points. The way Alec plays it, it’s not gambling. He got into a couple of small games and then stopped. It wasn’t fun for him. It was like taking candy from children.”

“Tell us more about Mrs. Linn’s letters.”

“It wasn’t only the nagging for money. She complained about everything. She hinted that there were other men in her life without coming right out and saying that she was sleeping—that she was actually being unfaithful to him.”

“How did Lieutenant Linn take it?”

“Bad. Lord knows all our nerves were jumpy enough, what with monsoons and heat and snakes and scorpions and bugs and hardly any white women and having your closest friends die all around you and being afraid all the time till it was a relief being up in the air and looking death right in the face. So you see, Alec had enough on his mind without worrying about his wife. He was navigating one of those huge crates over one of the toughest runs in the world. I never know how navigators do it. We pilots get the glamour. But we just have to be men without nerves and quick in a pinch for maybe a minute or two at a time. We have co-pilots and robots to relieve us. But a navigator sweats blood every second of a mission. That’s why you’ll find the high-strung, very bright boys are the navigators. Like Alec. They’ve got to keep themselves at a high pitch. And when on top of that they have something nagging on their minds all the time the way Alec had—well, it was plain murder on him. He got the jitters. He didn’t eat or sleep enough. Then we lost the ship on our twenty- third mission.”

“How did that happen?”

“One of those things. We were knocked about by flak over the target and then six Nip fighters hit us. By the time we chased them off they’d shot up the radio and almost got Alec. To add to our troubles, the weather turned to pea soup below us and we couldn’t see what was downstairs. And after a while Alec told me we were lost.”

“You were responsible for all the men on the plane, weren’t you, Captain.”

“I was the skipper.”

“Would you say, as an experienced skipper, knowing the men in your crew and responsible for them, that Lieutenant Linn’s ragged nerves caused him to make a serious mistake?”

“No. He—” Kerry saw Magee raise his eyebrows and he sat back. “Those things happen lots of times,” he said quietly. “A navigator isn’t God. He can be pretty near perfect—and Alec was tops—but worrying about a wife and being kicked around by flak and having your radio gone and being a thousand miles from the base—well, you see it, don’t you? We were flying at twenty thousand feet and couldn’t see our outboard engines. I tried to go below the weather, and we got a bad scare. Our altimeter showed fifteen thousand feet, but our positive altimeter only two thousand. We weren’t over water at all, as we should have been. There were mountains below us, at least thirteen thousand feet high, and if you know anything about—”

***

Magee yanked Kerry back to land.

“What happened finally?”

“Alec got us through. Anyway, , within a couple of hundred miles from our base. We were out of gas by then, so we had to hit the silk.”

“You were hurt when you jumped, Captain?”

“A couple of ribs bashed in. One of the men was killed, though. Sergeant Bilkin, a swell—”

“What happened to Lieutenant Linn?”

“Nothing physically. But he cracked wide open. He said it was his fault he lost the ship and Bilkin got killed. If it was anybody’s fault, I’d blame Lily Linn. The flight surgeon grounded him, of course.”

“And Lieutenant Linn was discharged as a psychoneurotic,” Magee said triumphantly.

“No. They don’t discharge a highly experienced officer so easily from the Air Force, especially if he’s so far away. He was assigned to ground duty.”

Magee frowned at him in annoyance.

“But soon after, he was given an honorable discharge, wasn’t he, Captain?”

“Six weeks later. The war in Europe had ended and they were being easier on discharges, and just about that time our entire wing of B-29’s was transferred to the Marianas. The shift was what really got Alec his discharge. The C.O. didn’t think it worthwhile sending a case like Alec to the Pacific.”

“So he sent him home to his wife?”

“That’s right, his wife.”

“But there’s no doubt that Lieutenant Linn was mentally hurt?” Magee persisted. “You were his skipper and his closest friend. You lived with him and faced death with him. How serious was the psychological harm—” Hackett protested vehemently. “Your honor, Captain Nugent is a pilot, certainly not a qualified psychiatrist.”

“Do you deny his qualification to testify that Lieutenant Linn was grounded for being psychoneurotic?” Magee retorted.

Hackett stood firm. “Your honor, why hasn’t the defense brought in qualified psychiatrists, to examine the defendant? Because Magee hasn’t dared to. He knows that the defendant is perfectly normal.”

“Sustained,” the judge said mildly. “Proceed.”

The words on that side of the room went on. Kerry told how Miriam and Ursula had asked him to go after me and how he had found me in Lily’s bungalow. After that Hackett had a go at him, but it didn’t take long and I was no longer listening. I felt as if I had been run through a wringer and tossed into a damp cellar where I would never regain my shape.

Then I heard Magee speaking loud and clear through the hush. “The defense rests.”



    
    \begin{ChapterStart}
    \vspace{3\nbs}
    \ChapterSubtitle[l]{Chapter ch7}
    \ChapterTitle[l]{ch7}
    \end{ChapterStart}

    \FirstLine{\noindent ### Chapter 7}

    
## The Verdict

I kept my eyes fixed on an inkspot on the table, but there was no getting away from Robin Magee’s voice.

I had to sit and take it, holding my shame inside of me.

“Look at him, ladies and gentlemen of the jury,” Magee orated, impaling me with a pointing finger. “Do you see a vicious murderer seated there? No! You see a boy like your own boys—a boy raised right here among you, who went to school with your children, who played with them and fought with them. A boy you would have welcomed to your home and have been glad to have your daughter know. A small-town boy, if you will, not worldly, perhaps, not hardened to recognize evil in its most insidious form. That, I submit, was his only crime.

“Had he ever before harmed a soul? The eminent prosecutor, much as he would have liked to, has not been able to indicate a single misstep, a single moral deficiency in his character. On the contrary, we have heard from Captain Nugent how this boy refused to take advantage of his comrades in arms through his superior poker skill, although that would have been perfectly legitimate. How many men could you find who would have acted with similar ethics? And yet the prosecutor has the hardiness to contend that this boy, with so highly developed a sense of fair play, could conceivably premeditate murder, no matter how great the provocation.

“I have never, in all my years of experience before the bar, represented a client of whom I think so highly. I’d be proud to call Lieutenant Alexander Linn my son. And when his country called its young men to arms, he was among the first to respond, enlisting of his own free will, ladies and gentlemen, in perhaps the most hazardous of the services. Need I dwell on how he distinguished himself? Need I read a list of his citations for heroism? Need I detail the incomparable courage he shared with all our airmen, with all our soldiers and sailors and marines, with your own sons and daughters who are or were in the service?”

“Ladies and gentlemen, Lieutenant Linn fought the good fight for his country. But always, riding in the bomber with him as he bent over his charts and maps, was his enemy. Not the Japs. They could be faced squarely and destroyed by guns and bombs. This enemy was more relentless, more heartless, and there was no defense against her, though she was fifteen thousand miles away. This enemy was the woman he had married.

“The Japs could not bring him down, but his wife did. He became a casualty of her assault. Psychoneurotic, the Army doctors call it. A big word meaning that he was wounded in the mind. He became subject to violent outbursts of temper, and the responsibility lay clearly with the woman who caused it.

“Need I tell you more about this woman? You have heard and your blood ran cold, as did mine. His hasty marriage to her may have been a foolishly impulsive act, but understandable when emotions are violently torn by war, when a boy is to go forth to battle not knowing whether he would ever again see his home and family and a desirable woman. He loved her for her face and figure and the mantle of deceptive virtue she wore to trick him. The Lily, he called her, no doubt for its age-old symbol of purity and loveliness. But if lily she were, it was no true lily, but a spider lily. You must have seen them in the woods, growing wild and untamed, the spiderwort called spider lily, which for all its superficial beauty has no right to the title. The lily—the spider lily. The lily who was a spider, weaving her insidious web to entangle whatever human fly came her way.

“How she must have laughed to herself at his-blind adoration—this big-city temptress, at least once married and divorced, eager for the embrace of any man who would have her—this vampire bringing her wickedness to the good people of West Amber, to corrupt its nineteen-year-old youths like Ogden Garback, to break up its homes like the Schneider home.

“Did she give a thought to the man with whom she had taken the sacred pledge to love and honor and who was hourly facing the untold dangers for the security of his country in a far and distant land? Yes, she thought of him. While in the arms of other men, she thought of the money she was receiving as the wife of an officer. While on drunken orgies, she thought of how convenient it would be if he were destroyed by a Japanese shell and his insurance would come to her.

“Ladies and gentlemen of the jury, murder was done. It was done to this fine, clean-cut, heroic young lieutenant of our glorious Army Air Force. This vile temptress from a big city, this unspeakably evil woman, murdered his mind and soul as surely as if.”

***

The jury was out eighteen minutes.

It filed back almost jauntily, and the foreman said happily: “We find the defendant not guilty.”

There was a rush for me. Hands grabbed my right hand to shake it. Robin Magee thumped my back and put his whiskey breath against my face. “We did it, my boy, didn’t we?”

District Attorney Hackett stood in front of me, extending a hand. “I tried my best, Linn, but I can’t say I’m sorry I took a licking this time.” Like a true sportsman. Like an opponent in a game of skill congratulating the victor.

“I didn’t kill her,” I said.

I spoke loudly, but nobody seemed to hear me, or else my words washed over them without meaning.

Soft arms were flung around my neck; Ursula held me close to her bosom. “Everything will be all right now,” she said. Then Miriam kissed me or wanted to, but her mouth sobbed against my cheek. Over her shoulder I saw Kerry grinning as he worked his way toward me.

George Winkler grabbed my arm. “Come over and thank the jury, Alec. They expect it.”

I tore away from him and stood panting against the table. “I didn’t kill her! Damn it, listen to me! I didn’t kill her!”

Abruptly the area around me became quiet. Magee reached out for my shoulder. I shook him off and pushed through the crowd about me, brushing Kerry aside and Ursula and others. I opened the rail gate and went up the aisle. People stood there. They moved back against the sides of the benches to make room for me to pass.

The stillness had spread throughout the room. I heard feet behind me, following. I did not turn.

“Can’t you let him alone!”’ Miriam’s voice burst out.

I reached the double doors. They were open. I went through them. My footsteps were the only ones to echo down the long courthouse hall.

It was late afternoon and the sun slanted down the broad courthouse steps. I could go where I pleased, do what I pleased, bio more bars. No four walls too close together.

I was a free man. I was a man everybody thought had murdered his wife.



    
    \begin{ChapterStart}
    \vspace{3\nbs}
    \ChapterSubtitle[l]{Chapter ch8}
    \ChapterTitle[l]{ch8}
    \end{ChapterStart}

    \FirstLine{\noindent ### Chapter 8}

    
## The Sympathetic People

That Saturday morning I took two showers. During the second shower I heard Ursula go out to the back yard and Miriam wasn’t home, so I walked naked up the hall to my room.

I sat at one of the windows and let what breeze there was wrap itself around my damp body.

It was now one o’clock, and since I’d awakened at eight I hadn’t dressed or left the room except to take the showers and at around ten to duck down to the kitchen in my pajamas for an orange and milk. The late August heat wasn’t responsible. It was nothing to what I had taken in my stride in India. I should be up and doing—sending out feelers for a job, tinkering with the car, mowing the lawn. I didn’t even read. I sat naked at the window.

At one-thirty Bevis Spencer’s sedan pulled into the driveway below my window. Sitting behind the wheel, he appeared to be as unclothed as I except for a rolled towel over his shoulders. The right front door opened and Miriam got out. She was wearing too little for anywhere but a public beach or the privacy of her own room. A white terrycloth robe was over her arm. In the back seat were Kerry Nugent and Helen Spencer, both wearing beach robes. They were home early, probably because this was Saturday and Bevis had to get back to his store.

Before the war we took those outings often—a bunch of young boys and girls piling into one or two cars and driving eleven miles in bathing suits to Corde Lake where the county had built a public beach. The girls put up picnic lunches, and if the weather was particularly hot we’d build a fire when it got dark and sprawl around it singing and telling stories and making mild love and occasionally drifting off two by two for another dip or to put ourselves beyond the revealing glow of the fire. In those days Helen Spencer had usually been my partner.

They had been fine times, and in India Kerry and I used to talk about them in fits of nostalgia and how one of the first things we’d do if and when we got back and if it was summer would be to go out to Corde Lake, I taking Lily and he any girl who was around and pleased him.

Well, we were back and Kerry had been out to Corde Lake with a girl and I was sitting at my window watching them return. Nobody had asked me to come along. I was out of it.

The car drove away. The semi-circular driveway lay empty under the fierce sun. I heard Miriam enter the house and stop to talk to Ursula in the downstairs hall.

Sitting became impossible. I walked back and forth across the room twice before I remembered that I wanted to dress. I put on ducks and a white Basque shirt and left the room.

Miriam was coming up the stairs. She looked wonderful in a bathing suit—or more likely she was one of the few women who, objectively, looked better without any clothes on. She had long thighs and no belly and her skin was burnished tawny-gold, contrasting with the very white of the two negligible strips of satin lastex.

I stood looking at her, waiting for her to speak first. She plucked uneasily at a loose thread in her terry-cloth robe over her arm. Then she said: “We’ve been swimming.”

“Obviously,” I said.

She attempted a smile. “I was going to ask you to come along, Alec, but you were still asleep when we left.”

“What you mean,” I said, “is that nobody wants a crazy murderer at a swimming party or at any other kind.”

I moved past her down the stairs. She remained silent. It wasn’t until I was turning into the living room that I heard her feet resume their ascent.

Ursula was in the dinette, setting the table for lunch. She studied my face anxiously and seemed pleased at what she saw. “You look rested, Alec. I thought sleep would be better for you than breakfast.”

I leaned against one of the posts of the dinette arch. “After I left the house that night—the night Lily was killed—what did you do?”

“For heaven’s sake, Alec, isn’t there anything pleasanter to talk about?”

“Ursula, where did you go after Kerry drove away to look for me?”

“I didn’t go anywhere.”

“All the men left,” I persisted. “You and Miriam and Helen remained in the house. Were you all together?” Ursula straightened a fork beside a plate and took time to reply. Evidently she decided to humor me. Somebody must have told her or she had read in a book that with people who were psychologically off balance the placating technique was in order. So she said amenably: “I went downstairs to the cardroom and counted out the chips at everybody’s place. You see, they’d left without figuring up. Then I played solitaire until Kerry came with the news that you had been arrested.”

I could have guessed that. When Ursula was upset or bored she played solitaire.

“Where was Miriam?” I asked.

“I believe she went upstairs to her room.”

“Was your car in the driveway all that time?”

“After Kerry returned with the terrible news. I could hardly think of putting the car away. It was outside all night.”

“Did you hear a car drive away when you were in the cardroom?”

The placating technique was too much for Ursula’s temperament. Her tone sharpened. “Alec, you’ve got to stop brooding over the—the death.”

“You can call it the murder,” I said. “That’s what it was.”

“What have these questions got to do with it?”

“You can’t solve an equation without knowing all the terms.”

***

Ursula went into the kitchen and seconds later returned with three glasses of tomato juice. “Will that bring her back to life?” she said irritably, as if there had been no break in the conversation. “You’ve been acquitted, so why worry about it?”

“I don’t like to be considered a killer,” I retorted. “Did you hear a car leave the house a few minutes after you went down to the cardroom?”

My voice had risen, and her last few sentences hadn’t been calmly spoken. It was becoming an argument, and that, in Ursula’s book, was bad for me.

She took a breath and returned to humoring me. “If I did hear a car, I didn’t pay attention. So many cars had been leaving at the time. I remember hearing a car arrive some time later and a man came into the house and call Miriam. I assume that it was Bevis because he was in the living room with Miriam when I came; up quite a while after that.” She couldn’t resist a mildly sarcastic dig. “I suppose all this is of vast importance.”

“Probably not,” I said.

I sat down at the table and drank my tomato juice. Ursula went through the swinging door. I heard Miriam come down the stairs and go up the hall into the kitchen. She and Ursula deliberately kept their voices low so that I wouldn’t be able to hear what they said. I took four soundless steps across the dinette and listened at the door.

Miriam was saying: “After what’s been done to him, he has all the right in the world to ask questions.”

“He’ll never get over it if he doesn’t get his mind on something else,” Ursula replied.

I was back in my chair before Miriam came into the dinette. Without looking at me, she took the chair opposite me and reached for her tomato juice. She was sore at me, she had a right to be.

“I’m sorry for what I said on the stairs,” I told her. “Naturally you wouldn’t want an odd man to horn in on your party of two couples.”

“All of us were anxious to have you with us,” Miriam insisted. The door swung inward and Ursula stuck her head through the opening. “You’ll have to blame me, Alec. Miriam was going to wake you, but I wouldn’t let her. You need all the rest you can get.” Her head disappeared.

I buttered a roll and took a bite. “So the night of the murder you went up to your room after the men drove away,” I said.

Ursula must have prepared Miriam for my questions. She replied readily: “I lay on my bed. I was—upset. Then Bevis returned after having driven his father home and called me from downstairs. I went down and we sat in the living room until Kerry arrived.”

“With Helen?”

“No. She was in the living room when I went upstairs, but I didn’t see her again until she came into the house with Kerry when he returned.”

“Did you hear a car drive away right after you went up to your room?”

“The road is so close that if I’d heard any I would have assumed it was passing,” Miriam’s dark eyes widened.

“Alec, you don’t think that any of us—” I saw how my questions were frightening her. The implication of them frightened me, too, but I had become used to it gradually. She was getting it suddenly, right between the eyes.

“Right now I’m after all the informational can get,” I told her. “Most of it won’t be of any use. None of it might.”

***

Miriam nodded in a remote sort of way. She wore black slacks and a yellow blouse. She looked almost as good in them as she had while practically undressed in that negligible white bathing suit. Bevis would get more than any man I knew when he got her.

“When are you and Bevis getting married?” I asked, trying to cheer up the talk.

“I’m not sure we will,” Miriam said unhappily.

“Is it fair to keep the poor guy dangling in midair?”

“I suppose not. But I couldn’t give him an answer while you were in jail. Now I have to face the problem again.”

“If you loved him enough, there wouldn’t be a problem.”

The kitchen door swung open and Ursula entered with three plates of creamed eggs on toast. “If Miriam is going to marry anybody, she should marry you,” she declared briskly.

I felt suddenly cold. I’d had a wife and there wasn’t anybody who didn’t think I’d killed her. I said bitterly: “Nobody wants to marry a man she believes likes to stick knives into his wives.”

Miriam gasped. I turned my head to her in surprise. She was on her feet. Her lips were quivering and her high boned cheeks had gone hollow. She strode out of the room.

“What’s come over her?” I asked Ursula.

Ursula dropped one of the plates in front of me so that it rattled. “I hope you choke on that! What fun do you get out of hurting her?”

“I didn’t say anything to her. It was to you.”

“She’s gone through hell since you came back from India. It’s not only that she loves you—”

“Everybody loves me,” I cut in. “Everybody tries to shelter me. I murdered my wife—or you’re all convinced I did—but I’m a sick man and must be treated like a child.”

Ursula plumped her big body down into her chair. “I wish you were a child so I could put you across my knees and spank you. Miriam is the only person who doesn’t doubt you at all.”

“But you do.”

Too late Ursula realized that she had brought it out into the open. But she didn’t hedge.

“I don’t know,” she gave it to me frankly. “It doesn’t make any difference to me one way or another as long as you’re a free man. Or to Miriam either. That’s why what you just said to her was so cruel.” Abruptly her voice and face softened. “Alec, why don’t you marry her? It’s the best thing you could do.”

“So that’s it,” I said. “You’re working to get my mind off Lily and the murder. You think marrying Miriam would do it. Besides, it would be cozy for you. It wouldn’t break up the household; it would keep us all together. How’ll you bully Miriam into it?”

“You idiot! She’ll marry you the moment you ask her.”

I started to eat. I was hungry. After a minute I said: “I suppose she would out of pity or because she thinks it her duty.”

“You’re so clever,” Ursula said scornfully. “You know everything.”

“I don’t know the one thing I must know, I said, “and that’s who murdered Lily.”

***

After lunch Ursula sent me downtown with a grocery list. She preferred to do her own shopping, but she was anxious to give me something to occupy me.

I drove downtown in the car. The curbs were jammed solid with Saturday afternoon shoppers. I rolled along Division Street, hunting for parking space close to Spencer’s Food Market. A small man was getting out of a car on the gutter side. I almost brushed him with my left front fender and stopped. “Mr. Dowie,” I called.

He peered at me through his thick lenses. “Well, Linn, I’m glad to see you’re a free man,” he said, somewhat embarrassed. “I hope you don’t hold anything against me.”

“You did your duty.”

Sheriff Dowie nodded. “I couldn’t do anything else. Fact is, I was sure you’d get off. Under the circumstances, no jury could find you guilty for what you’d done.”

I opened my mouth, but I didn’t say it. Repeating to everybody that I hadn’t killed her was a waste of breath.

“You’re sister is angry at me,” Dowie went on. “She told me so before the trial. Maybe I shouldn’t be sorry she isn’t inviting me to any more poker games. It was costing me too much money. Those two Spencers are sharks, and your sister and Winkler play a mean game. Well, glad you take it that way.” He started to move toward the rear of my car to get around it.

“Wait a minute,” I said. “Did the police check on Schneider’s phone call to my wife that night?”

Dowie came back to the window. “The telephone company can’t tell which of eight parties on a rural line makes the call.”

“But a call was made?”

“The company doesn’t make a record of a local call.”

“Then there’s only Schneider’s word that Lily was alive at ten o’clock,” I said.

Dowie leaned against the door to get a better perspective of my face. “What would Schneider gain by lying about the phone call? Don’t forget, he went to Hackett of his own free will to tell him he’d spoken to your wife at ten o’clock.”

“I’m not disputing that he spoke to her,” I said. “But I wonder if he didn’t speak to her in person at her bungalow at ten o’clock.”

Dowie missed it completely. “Look here, Linn, don’t you start up with Schneider because he fooled around with your wife.”

It wouldn’t do any good to suggest that perhaps Schneider had done something beside make love to Lily that night. So I said, “Don’t worry,” and put the car into gear. Dowie stepped back against his own car.

Ahead, the driver of a light truck was getting behind the wheel. I shot forward and eased in at the curb as the truck rolled out, and I was parked within a hundred feet of Spencer’s Food Market.

***

Bevis Spencer, in his long white coat, was behind one of the three checking counters. He was too busy to see me. I wheeled a basket wagon through the mob.

Somebody said: “Congratulations, Mr. Linn.” Les Shayne, who delivered Ursula’s laundry, was beaming at me.

I looked down at his gnarled outstretched hand and then up at his face. “Congratulations for what?”

He withdrew his hand. “They were taking five to one at the barber shop you’d be acquitted,” Les Shayne said cheerfully. “I took two bucks’ worth—on you, of course. I told the boys they’d never—”

I swung my wagon away from him and one of the wheels nicked a woman’s heel. She glared at me, but I had no apology for her. I hated her and Les Shayne and everybody in the world. The trial had been a circus, a sporting event on which to bet, and the fact that I had beaten the law was cause for congratulations. If they had despised me for a murderer it would have been easier to take.

I stopped beside the refrigerator and looked at the shopping list. It was shaking in my hand; the words on it blurred. I wiped mist out of my eyes. There was butter on the list. By the time I had it out of the refrigerator I wanted to kick myself. It wasn’t Shayne’s fault; he’d only tried to be nice to me, like everybody else. I looked around for him and saw Oliver Spencer coming toward me.

Mr. Spencer didn’t try to congratulate me. He said amiably: “I hope we’ll have some poker tonight with you, Alec.”

“I’m not sure I’ll play,” I muttered. “I hear Masterson will be there. He always makes a fast game.”

“I guess I have nothing else to do.” Then I remembered the thing I had to do twenty-four hours a day until the end, if I ever reached it. “Can I see you alone, Mr. Spencer?”

He looked at me speculatively, then nodded. I trundled my wagon to the rear of the store and followed him into his office.

“Why was Helen afraid of you when she was on the witness stand?” I asked.

Mr. Spencer pressed the back of his thighs against his cluttered desk and patted his bald pate. “Afraid of me?”

“That’s the idea I got. It seemed to me that she left out a lot in her testimony.”

He said sourly: “I’m glad she had the sense not to drag the Spencer name in the mud more than she did.”

“So she did leave a lot out?”

“My daughter is a good girl and that woman you married wasn’t.”

“And you objected to Helen’s being friends with Lily?”

“I intended to see that my daughter remained a good girl.” He was roused to anger. “There’s another thing. You and Helen used to go out together. I didn’t object at the time. I even looked forward to you—well, it would have been a good match.” Suddenly ill at ease, he fumbled at a button on his white coat. “I always liked you, Alec. I can’t excuse what you did to your wife, no matter what she was, but I’m glad you weren’t found guilty.” His eyes lifted defiantly. “But now I want you to stay away from Helen.”

I held myself together, fighting to keep my voice under control. “All that was when Helen and I were kids.”

“That may be. But now you have no wife and she is very pretty and perhaps you are still fond of each other. You’re not the man for her.”

“Afraid that it’ll become a habit?”

I said, feeling my voice go to pieces. “That any time I don’t like the way a wife of mine acts I’ll stick a knife into her?”

Mr. Spencer’s eyes dropped. I turned to the door.

“Alec,” he said gently, “perhaps I put it too crudely. I have nothing against you personally. But my only daughter—”

“I understand,” I said and kept going.



    
    \begin{ChapterStart}
    \vspace{3\nbs}
    \ChapterSubtitle[l]{Chapter ch9}
    \ChapterTitle[l]{ch9}
    \end{ChapterStart}

    \FirstLine{\noindent ### Chapter 9}

    
## Afternoon of a Mathematician

Bevis Spencer tapped on my shoulder while I was waiting with my basket wagon at one of the checking counters. “Hi, Alec.”

I nodded briskly

His somber face clouded. He leaned close to my ear. “I had to testify at your trial. I was subpoenaed. I tried to make whatever I had to say sound as good for you as I could.”

“You didn’t hurt me.”

“I tried not to,” he said.

A clerk at the vegetable section called him and he bustled off. Bevis was a different man in his father’s store. His awkwardness was left outside; he moved about with efficiency and assurance, supervising and trouble-shooting.

The unfamiliar business of ration stamps floored me at the checking counter, but the girl checker was pleasant and patient. I doubted if I’d ever seen her before, but she called me Mr. Linn and the girl at the next counter was giving me sidelong glances, Word must have passed around that I was in the store. I was the most notorious resident of West Amber.

While I was putting the bundles on the back seat of the sedan, I heard my name called. Bevis Spencer’s long body in its flapping white coat moved rapidly along the sunbaked sidewalk. I closed the car door and faced him.

“Alec,” he blurted, “you have no objection to me, have you?”

“As what?”

“As Miriam’s husband.”

“Her opinion is the one that counts.”

“I know, but—” Bevis regarded the tip of his cigarette. “Dad says he’ll make me an equal partner in the store when I marry. It’s a good store, the best in the country, and I’ll have a big income.”

“She’s the one to tell it to.”

“I’ve told her everything I can to convince her. I’ve never loved another woman and I never will. And she likes me. I’m sure she does.”

“If she didn’t, she would have turned you down the first time you asked her.”

“That is it,” he said eagerly. “She cares enough for me to be undecided. Maybe she doesn’t quite love me yet, but she will. I’ll make her love me.” His deep-set, intense eyes bored into me. “Alec, if you’ll tell her to marry me—”

“I’m not her master.”

“Of course I don’t mean tell her. Advise her. She thinks the world of your opinion. If you’d just sort of mention that you don’t object to me as her husband—”

“She’s got to make up her own mind.”

“I guess that’s so. But you don’t object to me, do you?”

“I don’t if Miriam doesn’t.” I laughed mirthlessly. “Speaking about objections, your father just got through telling me to stay away from Helen. He’s afraid I’ll marry her and then shove a knife into her.”

“So that’s why you were huffy in the store? I’m not responsible for Dad.”

“You think the same way. Everybody does.”

Bevis searched my face anxiously. “You’re not still in love with Helen?”

“You needn’t worry.”

“You have me all wrong. I’d back you to the limit with Helen. Dad’s crazy.”

“That was only kid stuff. Besides, she’s Kerry’s girl now.”

“Kerry is okay.” Bevis flipped away his cigarette. “Well back to work. See you at the poker game tonight.”

***

I headed toward home. When I came to the Old Mill Road intersection, I changed my mind and swung right. While I was at it, I might as well make a clean sweep of the Spencer family, and the hell with Oliver Spencer.

I hadn’t visited the Spencer home since my third year at college when Helen and I had started to drift away from each other. The house was a sprawling single story of many wings which had grown, with the prosperity of the food market, from a three-room box. Helen Spencer was on the side field-stone terrace. She reclined with a book on a chaise longue. She wore navy blue shorts and a flowered halter. No shoes or stockings.

“Did you hear the news?” she said excitedly. “Kerry wrote to his commanding officer for another thirty-day leave, and he got it.”

“That’s swell.” I sat down beside her fine bare legs. Her toenails were scarlet. “Did Lily teach you to paint them?”

“I got the idea from her.” She wiggled her toes. “Don’t you like it?”

“Not any more. What did you leave out from your testimony at the trial?”

“Nothing. Perhaps a few details.”

“What were they?”

Helen asked me for a cigarette and lay back puffing on it. She had an enticing body, as smoothly rounded as a baby’s without being tubby.

“Dad has a store and everybody knows him, so I didn’t want to say anything that would cause gossip,” she said presently. “This last year I went out several times with Lily on double dates. I don’t mean that Lily got men for me, except once, but sometimes Lily asked me to bring my date along, whoever it happened to be that night, and make a foursome.”

“Whom was she with?”

“Generally Bill Beaty until he was drafted, and twice this spring Emil Schneider after his wife left him.” She stirred her hips uneasily. “You must think I’m awful, Alec, going out on dates like that with your wife, but they seemed harmless. I didn’t know till the trial that she was actually having affairs with those men. And she was very sweet to me. I didn’t know a thing about clothes or makeup and what to wear with what till she taught me.”

I twisted around for another look at her toenails. “Lily was a glamour girl from New York,” I said. “She took you under her wing. She liked you or was bored and wanted somebody to work on. She dressed you and painted you.”

“Why shouldn’t I look smooth?” Helen said defensively.

“Smooth as lacquer.”

“You liked it in Lily.”

“Did I?” The question was to myself and the answer too complicated to think about.

Helen said: “I admired her and wanted to be like her. I mean so beautifully turned out and sophisticated. I don’t mean the other thing—with men. After what happened at the roadhouse on the Fourth of July I became disgusted with her.”

“What about that?”

***

“It was the way I told it at the trial. I didn’t keep any of that back; I wanted to help you. That was the first and only date I let Lily get for me. Her ex-husband, Don, drove up from New York with a man named Walter, and Lily asked me as a favor to come along for Walter. I didn’t like him at all. He had a long face that was as blank as a wall and small eyes that had no expression except when he looked at me and then they seemed to be undressing me. But I must say this for him—he never tried to make a pass at me. Well, you heard at the trial how Lily got very drunk and let Don paw her right in public and what she said about you. Then Bevis and Miriam came in for beer and dancing and they saw what was going on at our table and they took me home. I was glad to go, I can tell you. You can imagine the battle in our two houses that night. Bevis told Dad and he—”

“Two houses?” I said.

“Your house was the other one, of course. I learned from Bevis next day that Lily hadn’t come home till dawn and that Miriam had waited up for her and then Miriam had given it to her with both barrels. The argument woke Ursula. She came down and tried to make peace between them, but Miriam said she wouldn’t stay in the house with Lily and that one of them had to leave.”

“I had the impression that it was Ursula who had the fight with Lily.”

“No—Miriam. In fact, Ursula didn’t want to do anything about Lily till you returned. They were expecting you home in a week or two. But Miriam refused to listen. She started to pack up to leave, so what could Ursula do? Lily was kicked out, and that afternoon she rented the bungalow on James Street.”

There was a silence.

“Go on,” I urged.

“You heard at the trial how she phoned me that evening to try to explain away what she had said about you while drunk. I found Emil Schneider with her. That’s about all.”

“No, it isn’t,” I said. “You held back something important at the trial.”

Helen turned her eyes to the sun and closed them against the glare. Pearls of sweat were on her upper lip. “Don Yard hit Dad. You won’t tell anybody?”

“No.”

“It was my fault and Dad’s, too. I didn’t want to have anything more to do with Lily. Then one Saturday evening—it was a week before you came home—she phoned me. She said she had a party and needed more women and would I come. I was about to say no, but Dad was listening in on the extension in his room. He broke in and told me not to dare go, so—” She paused.

“So you went.”

Her round chin jutted. “I’m not a child. He and Bevis tried to make me use less makeup and objected to some of my clothes. Then this. Dad had no business ordering me around like that. Then he came out to the hall and ordered me again not to go. If he’d only told me in a different way—” Her bare shoulders shrugged. “I went without changing my clothes. There was quite a party when I reached the bungalow. Don and Walter and two other men had driven up from the city with a woman. Her name was Bertha—I never learned any of their second names. She was a stunning redhead who dressed beautifully, though she couldn’t hold a candle to Lily. She and Lily kept making catty remarks about each other. I think it was because of Don.”

“Why should Lily have cared?” I said. “She hadn’t wanted to go back to him.”

“You know how women are. Especially women like Lily who can get any man they want. She didn’t want Don, but she didn’t want anybody else to have him either. I think Bertha hated her because she was afraid Lily might decide to take him back some time Perhaps I’m wrong; it was only a notion I had. Actually I wasn’t there more than twenty minutes. Then Dad came in.”

***

Helen seemed suddenly cold in the hot sun. “It was awful. Nobody but Lily knew who he was, but there was something about the way he stood there looking at them and at all the liquor and then at me that made them very quiet. It would have been all right even then if Lily hadn’t opened her big mouth. She said: ‘Why don’t you let the kid have some fun, Pop?”

Dad went over to Lily and slapped her face. And Don hit Dad on the jaw.”

I rose to my feet and stood there rattling change in my pocket.

“Dad was knocked off his feet,” Helen went on. She was breathing hard, her full breasts rising and falling under the inadequate confines of the flowered halter. “I helped him up. I was sobbing when we left the bungalow. His car was outside. I drove. He sat beside me, holding his jaw, and didn’t say a word to me about what had happened. He never did until he heard that your lawyers wanted me to testify in court. He asked me not to mention that scene. You see how that would have started everybody in town talking and how bad it would have been for my reputation—my father getting in a fist fight over me. So I didn’t even tell your lawyers about that party.”

“Did you see Lily after that?”

“Of course not. You can imagine how I detested her.” She sat up and hugged her knees. “She deserved what happened to her.”

What she meant was that Lily had deserved what I had done to her. I had no desire to raise the issue. It never got anywhere with anybody.

I said: “The night Lily was killed you waited at my house for Kerry to return.”

“I thought it would be only a few minutes. It was more than au hour.”

“Were you in the house all that time?”

“For a while, but Miriam and Ursula left me alone in the living room and it was stuffy inside, so I went out in the garden to wait for Kerry.”

“Did you see or hear Ursula’s car leave or return or both? It was parked in the driveway.”

“I was sitting around the side of the house. I couldn’t see the driveway from there. I heard a car and thought it was Kerry’s and walked as far as the corner of the house. But it was Bevis so I went back to the chair.” She rested her chin on her knees and looked curiously up at me. “Why do you want to know all about these things? You needn’t be sorry that you—” Quickly she changed her mind about finishing it.

“I’m gathering material for a problem in mathematics,” I said.

“That’s over my head. What has mathematics got to do with Lily? Or is that supposed to be a wisecrack?”

I stood looking out to the road. It would be easy to drop it now. If I couldn’t face the people of West Amber, I could move to another part of the country. I had no ties, not even to my sister, when you reached my age.

Helen was telling me that she had beer on ice. I shook my head and said so long and went to my car. She stood up and watched me back the car to the road. I waved. She did not move a hand or smile. The last I saw of her she was still staring after me.

***

There was butter in the car worth a fortune in ration points and it was probably goo by now, but still I did not go home. I turned off on James Street.

George Winkler, his meaty, hairy torso stripped to the waist, was devoting the Saturday afternoon to mowing his lawn. He wiped sweat from his massive brow and asked how the boy was.

“Not bad for a murderer,” I told him.

“Oh, cut it out,” he said disgustedly. “Why don’t you go away somewhere for a few weeks?”

“What does Magee know about Lily’s former husband? When Kerry was on the witness stand and mentioned that Lily’s name had been Yard, Magee repeated the name Don Yard to himself.”

George rested his bulk on the handle of the mower. “Observant cuss, aren’t you? Don Yard is a notorious New York gambler. A very tough lad, I understand.”

“And Magee knows him?”

“Many people in New York know Don Yard or have heard of him. The fact that Lily had been married to him might have been useful at the trial to undermine her reputation further, but we’d only investigated her conduct in West Amber. We found enough for our purpose.”

“What do you know about a woman named Bertha in connection with Yard?”

“That must be Bertha Kaleman. After the trial Magee told me about Yard and mentioned her. She is probably Yard’s mistress.”

“And hated Lily because Yard wanted Lily to come back to him,” I said. “She could have come up to West Amber and murdered Lily. Or it was Don Yard because Lily refused to go back to him.”

“My God, Alec, that line of reasoning won’t leave out anybody who had ever known her. The lass inspired homicide.”

“Then there are Emil Schneider and Ogden Garback.”

“The kid’s out. We proved to our satisfaction that he hadn’t gone out with Lily since last winter.” George massaged the sweaty hair matting his chest. “Alec, you’re a mathematician.”

I laughed. “I’ve been brooding over the laws of probability.”

“That’s it—probability. Every time I see you and talk to you I’m convinced you didn’t do it. If you had, I think you’d admit it. But let’s be frank. It’s too much to believe that somebody stabbed her a minute or five minutes or ten minutes before you barged in there after events had fallen in such a way that all evidence inevitably would point to you. The odds against a coincidence like that are too great. Suppose I were to bet in a game that on the next hand I would pull a pat-hand straight flush. What odds would I ask?”

“One to 64,974.”

“I knew it was at least one to fifty thousand,” George said. “And if I pulled that straight flush, after making that bet, you’d call it a coincidence too far-fetched to be believed in.”

“I’d call it a violation of the laws of chance under the circumstances. I’d be sure you had stacked the cards, and that if you weren’t dealing, an accomplice of yours was.”

“And who,” George said softly, “would want to stack the cards against you?”

I thrust my hand deep into the pockets of my ducks to hide their trembling. “You’ve just made my point. The odds exclude everybody who wasn’t at Ursula’s house that night.” George left the support of the lawn mower and hitched his belt over his bulging belly. “What are you getting at?”

“That somebody knew that I was rushing to Lily in a high state of excitement. He or she saw an opportunity to murder Lily and frame me for the crime.”

***

George looked off to his right. Over the tops of the silver birches I saw the edge of a roof. Probably that was the bungalow in which Lily had lived. It was about the right distance and direction.

“Do you know what you’re implying?” George said, turning back to me.

I steeled myself to go on. “I walked to the bungalow. Everybody at the house had a chance to beat me to it. You, to begin with. You live only a few steps from the bungalow and you were home when I passed this house. Maybe you’d just returned from Lily’s.”

He open his mouth.

“I’m not being personal,” I cut him off. “I’m trying to set up a hypothesis. Kerry Nugent admits that he reached James Street before I did. Bevis Spencer drove his father home; he could have stopped off at Lily’s before going back to Miriam. Oliver Spencer could have walked from his house after Bevis dropped him off; it’s only half a mile and I had two miles to walk. Sheriff Dowie says he drove home and then left to drink beer, but he might have gone straight to Lily’s instead.”

“I was under the impression that mathematics was an exact science,” George said with heavy sarcasm.

“It’s good for any kind of straight thinking.” I forced myself to be ruthless. “Ursula was in the cardroom, Miriam lying down in her room, Helen Spencer around the side of the house in the garden. Any of them could have driven off in Ursula’s car which was parked in the driveway, gone to the bungalow and returned in ten minutes.”

“And the human equation?”

“You mean the motive? As you said, who didn’t have reason for killing Lily?”

“I didn’t.”

“That’s what you say. How do I know you didn’t or Kerry didn’t or even Dowie didn’t? I know why the others hated her.”

“I hated her, too,” George muttered, looking down with abstract interest at the blades of the lawn mower. “Because of what she did to you.” He lifted his shaggy head. “We started with poker. There is psychology as well as mathematics in poker and murder. Do you believe that any of the people you mentioned would be capable of deliberately framing you? Your mathematics means that or it means nothing.”

I was silent.

“I know what you’re thinking,” George went on. “At any rate, I’m thinking it. The murderer might frame you to divert suspicion from himself. But look at the risk involved. Look at the split-second timing necessary. Why not simply walk in on her one night and kill her and walk out with the reasonable assurance that her body will not be found for hours or perhaps days? By that time the trail, if any, would be cold. Too many people had a motive for the murderer to be suspected in particular. And what if he were suspected? There would be no evidence against him.” He shook his head. “No, Alec, you can’t have your killer an extremely clever and quick-thinking individual and at the same time a complete fool.”

***

He was right, of course. I had known it from the first. “I’m just starting to line up terms for my equation,” I said weakly.

George snorted. “Nuts to your mathematics! You’re not dealing with the set values of numbers, but with human beings.”

I took my hands out of my pockets They were clenched and hot. “I’m not finished. There are Schneider and Yard and Bertha Kaleman and perhaps somebody I never heard of.”

“And what’s become of your laws of chance?” George mocked.

“You were the one who suggested the fantastic odds of a straight flush. Why that hand? Why not the odds on a pair, which are little better than two to one? Or even three of a kind, forty-eight to one, which is not too improbable. Because, look. Let’s say that Lily intended to tell me something that was dangerous to somebody and the absolute necessity to kill her didn’t come to a head until I was actually in town. That lets us include Schneider and Yard and Bertha Kaleman and others. My homecoming made her murder inevitable that night. The killer had to wait for darkness; houses around the bungalow are so close that there was a risk he’d be seen coming or going by daylight. So he had to wait until nine-thirty or ten, when it was dark enough to shelter him. She was my wife; I was bound to visit her that night. In view of that there’s nothing so far-fetched in the fact that I walked in shortly after the murder.”

“You’re assuming—”

“I’m assuming unknown quantities until I learn their values.”

George Winkler studied me. “Alec, you’re not out for vengeance? You still don’t care for her?”

“Probably I’m conceited, but I don’t like to be thought of as a killer.”

“In any case,” George said, “the law has decided that the circumstances were extenuating.”

“To hell with the law!”

I went across the lawn to my car and drove home.



    

\scenebreak
\scenebreak
{\centering\textsc{the end}\par}

\clearpage

\null

\centering\textsc{www.TalesofMurder.com}\par

\vspace*{10\nbs}

%\centering\InlineImage[0, 3em]{/home/darkstar/dox/working-files/LaTeX/atticus.jpg}

TALES OF MURDER PRESS, LLC

\null

\scshape{675 TOWN CENTER BLVD
BLDG 1A STE 200 PMB 530
GARLAND, TEXAS 75040}

\null

\textit{atticus@talesofmurder.com}
\vfill


\end{document}
