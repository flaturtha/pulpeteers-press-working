% !TeX TS-program = LuaLaTeX
% !TeX encoding = UTF-8
\documentclass{novel}
%%% METADATA (FILE DATA):
\SetTitle{TITLE}
\SetAuthor{AUTHOR}
\SetPDFX{X-1a:2001}
\SetTrimSize{4.25in}{6.875in}
\SetMediaSize{4.5in}{7.12in}
\SetMargins{0.5in}{0.5in}{0.5in}{0.7in}
\SetParentFont{Libertinus Serif}
\SetFontSize{9.5pt}
\SetHeadFootStyle{5}
\SetHeadJump{1.5}
\SetFootJump{1.5}
\SetLooseHead{50}
\SetEmblems{}{} % Default blanks.
\SetHeadFont[\parentfontfeatures,Letters=SmallCaps,Scale=0.92]{\parentfontname}
\SetPageNumberStyle{\thepage}
\SetVersoHeadText{\theAuthor}
\SetRectoHeadText{\theTitle}
%%% CHAPTERS:
\SetChapterStartStyle{footer} % Equivalent to empty, when style has no footer.
\SetChapterStartHeight{10}
\SetChapterFont[Numbers=Lining,Scale=1.6]{\parentfontname}
\SetSubchFont[Numbers=Lining,Scale=1.2]{\parentfontname}
\SetScenebreakIndent{false}
%%% BEGIN DOCUMENT:
\begin{document}
\frontmatter
\thispagestyle{empty}
% Half-Title Page.
\begin{parascale}[2]
\vspace*{3\nbs}
\centering\charscale[0.75]{TITLE }\par
\centering\charscale[0.75]{TITLE LINE 2}\par
\centering{TITLE LINE 3}\par
\end{parascale}
\clearpage
\thispagestyle{empty}
\null % Necessary for blank page.
% Alternatively, List of Books.
\clearpage
\thispagestyle{empty}
% Title Page.
\begin{parascale}[4]
\centering\charscale[0.75]{TITLE }\par
\centering\charscale[0.75]{TITLE LINE 2}\par
\centering{TITLE LINE 3}\par
\end{parascale}
\vspace*{2\nbs}

\begin{parascale}[1]
\centering\textit{SERIES}\par
\vspace*{3\nbs}
\charscale[2]{AUTHOR}\par
\end{parascale}
\vfill
\begin{parascale}[1]
% \centering\InlineImage[0, 3em]{/home/darkstar/dox/working-files/LaTeX/atticus.jpg}

A Tales of Murder Press, LLC\par
\textit{GENRE} Novel\par
\end{parascale}
\clearpage
\thispagestyle{empty}
% Copyright Page.
\null\vfill
\allsmcp{First edition} PUBLICATION_DATE\par
\null\null
\allsmcp{ISBN}\par
\null\null
\vfill
\begin{adjustwidth}{3em}{3em}
\textit{This novel is in the public domain.} Certain \mbox{elements} in this edition are Copyright © COPYRIGHT_DATE Tales of Murder Press, LLC
\end{adjustwidth}
\clearpage
\thispagestyle{empty}
\clearpage % because ToC must start recto
\thispagestyle{empty}
\begin{toc}[0.5]{0em}
{\centering\charscale[1.25]{Contents}\par}
\null

%some kind of loop through chapters for TOC
\tocitem*[1]{CHAPTER_TITLE}{STARTING_PAGE}
\end{toc}
%end loop
\clearpage

\mainmatter
\cleartorecto
\thispagestyle{empty}

%some kind of loop through all chapters
\begin{ChapterStart}
\vspace{3\nbs}
\ChapterSubtitle[l]{Chapter 13}
\ChapterTitle[l]{## Flight}
\end{ChapterStart}
\FirstLine{\noindent ### Chapter 13

## Flight

Miriam and Ursula came out on the porch when they heard our cars. I remained seated behind the wheel without will to move. Kerry got out of his car and walked stiff-legged to my window, like a traffic cop about to hand out a ticket. He opened the door. “Come on,” he said.

I nodded sluggishly and cut the ignition and lights. Kerry helped me out as if I were sick. When my feet were on the ground, I shook him off and strode ahead. He caught up to me and walked at my side up the porch steps.

“What is it?” Ursula asked tightly.

“Emil Schneider was shot dead,” Kerry said.

Miriam made a thin moaning sound and crossed her arms over her breasts. Ursula stared with her mouth open rather foolishly.

“That was a delicately objective statement of fact, Kerry,” I said, and leaned against the house and laughed in a way that hurt my throat.

Ursula moved to my side. “Alec, for heaven’s sake!”

My laughter crumpled into ragged bits, which I coughed away. I said savagely: “‘Emil Schneider was shot dead.’ Not: ‘Somebody unknown shot Schneider.’ Not even: ‘Alec put a couple of slugs in him.’ That’s assumed. Somebody is murdered in West Amber, and who else would do it?”

“Did you?” Ursula whispered.

“Why bother asking?” I said. “You knew the answer right away.”

I pushed myself away from the wall and went into the house, slamming the door behind me. The handset phone was on the table in the hall. I picked it up.

The door flew open. Kerry came in fast. He wrapped his fingers around my right wrist and pushed the handset away from my mouth. “What do you think you’re going to do?” he demanded.

“Call the police. I’ve nothing to hide.”

“What’s the rush?” he said.

I shoved my free hand against his chest and strove to pull the mouthpiece up to my mouth. He was too strong. I twisted my body to him and jabbed my knee up. He grunted hollowly. His fingers loosened, but only slightly.

Abruptly the fight went out of me, though Kerry was the one who was hurt. The handset fell to the floor. Kerry released my wrist and crouched and put both hands down where my knee had struck him. His face was very white.

“Are you hurt bad?” I asked him contritely.

“Not much.” He became aware of Ursula and Miriam in the hall and raised his hands up to the pit of his stomach.

The phone was cackling on the floor. Ursula scooped it up, said, “Never mind, operator,” and hung up. Then she looked at me. “My God, Alec!” she said.

***

Miriam stood in the open doorway.

That pathetic sickness in her eyes was harder to take than horror would have been. I turned and went as far as the stairs and dropped down on the second one. I ran sweaty palms over my face.

“What do you want from me?” I said, and my voice was sobbing. “Are you trying to protect me the way you did the last time so everybody will believe I’m a murderer? I never harmed anybody in my life except indirectly in war.”

Kerry straightened up. Beads of perspiration stood out on his forehead, but he was taking the pain like the tight-lipped he-man he was. “I think,” he said slowly, “that we ought to call George Winkler at once.”

“Damn it, no lawyers!” I said. “They can burn me in the electric chair, but no lawyer is going to give me the works this time. I didn’t kill Schneider. Why would I? I needed him alive to buy a name from him. The name of the person who killed Lily.”

Miriam moved forward from the doorway. She sat down beside me on the stairs and took my hot hand between her cold palms. “Alec, what happened?”

I bowed my head and creased my brow as if in that way I could push the sluggishness out of my brain.

“I don’t know what happened,” I said. “I tried to think while driving home, but straight thoughts refuse to come. I must be too shocked, too tired. The things I saw and know are wrong, the way Lily’s murder was wrong. It’s like a mathematical equation which always gives the wrong answer no matter which way you work it. Always the same answer, with my name standing for X, and I’m the only one who knows that it can’t be right. But what’s wrong about it I can’t tell.”

I doubt if any of them followed me, but they listened patiently.

I looked up. “Here is what is true and definite. Schneider saw who murdered Lily. He visited her that night after he spoke to her on the phone. When he got there he saw somebody sneak out the back door, and he found Lily with the knife in her heart. He beat it to save his own neck and kept quiet about what he had seen. I doped out that that’s what must have happened. He admitted to me that he had seen the murderer’s face and would let me have the name for five thousand dollars.”

Ursula said: “Why didn’t you tell me?”

“Would you have let me have the money if I had? Would you have believed my story when you were convinced that the murderer’s name was my own? And if you did believe me, what would I be buying? Schneider refused to go to the district attorney with me. He might be the murderer himself or he might pull a name out of the air. It looked like a gold brick.”

***

Kerry was leaning against the wall, still in pain and not showing much of it. “That part sounds as bad as the rest,” he said. “Not to me; I just don’t know. I’m thinking what the police will think. Five thousand bucks for a name you couldn’t know was the right one and with which you couldn’t do anything if you had it.”

“Of course it was a gamble. A lead, maybe, a straw to clutch at—no more. A sucker’s buy, but not for me and not for the murderer.”

Miriam’s hands lightened over mine. “Don’t you see that Alec had to do it?” she told the others earnestly. “It means so much to him.”

“Thanks, Miriam.” I gave her a weak smile. “It was worth the money to me. I know now I would have received value from Schneider. He wasn’t the murderer, because the murderer murdered him. And he had the right name. That’s why he was shot. There’s a lot I can’t get straight in my head, but this much I can. Schneider needed money desperately. He’d lost his job at the bank; his wife was divorcing him; she owned the house; he was broke. If he’d sunk so low that he would take that kind of money from me, why not capitalize on what he’d seen by blackmailing the murderer? But’s that’s a dangerous business. A person who had murdered once will murder again to protect himself. Schneider was scared. He’d have been sure of collecting that five thousand dollars from me if he’d hung around town till the bank opened Monday morning, but nothing could stop him from taking the first train out tomorrow morning. Nothing but a couple of bullets.”

“And he was shot down before your eyes,” Kerry said.

I took my hand from between Miriam’s palms and set fire to a cigarette. I did not look at anybody.

“He was shot while you were with him?” Ursula asked thinly.

“It was about fifty feet away, on the walk to his house. He put on the porch light and looked to his left and said my name. Probably he heard somebody. He was expecting me. I chased the killer, but the night’s black and all I glimpsed was a shadow at the edge of the window light. It could have been either a man or woman.” I dragged at my cigarette. “It doesn’t make sense in several ways. Kerry knows why and the rest of you are figuring it out. There’s only one answer, but it’s the wrong one. I’m the only one who knows it’s wrong because I’m the only one who knows I didn’t kill him.”

Miriam reached for my hand. I wouldn’t let her take it. I stood up.

“I had no right to come back here,” I said. “If the police aren’t called, all you will be in this with me. Kerry, let me use the phone.”

He left the support of the wall and hooked his thumbs in his belt. You couldn’t tell by his face whether he was still in pain. “We four are the only ones who know you were at Schneider’s house tonight,” he said.

Ursula pounced on that. “Of course! There’s no hurry. It’s late and we’re tired. Suppose we all get a good night’s sleep and then we’ll see.”

***

I nodded slowly. I’d be helpless in jail, cut off from any opportunity to prove my innocence. Once in the hands of the police, I’d be bully and badgered by them and by my own lawyers, too. I’d gone through it once; it wasn’t anything to be faced again. I needed time to rest and to think in free air.

“Suits me,” I said and turned. Miriam shifted on the: step to give me room to pass her. I remembered something. “Did you and George send Kerry to follow me?”

“They didn’t send me,” Kerry said angrily. “The three of us decided it.”

“Did you?” I asked Miriam.

Her dark grave eyes studied me, not quite understanding. “Yes,” she said.

I went up the stairs. Somebody came behind me. Kerry entered my room a moment after I did. I sat down on the bed and looked at him. “What now?” I asked.

“Nothing but sleep,” he said. “Let me help you with your shoes.”

I told him I wasn’t an invalid and undressed myself. He got my pajamas out of the drawer and pulled down the bed covers. The close pal, the fellow officer, putting a drunken or ill or mentally deranged comrade to bed. I crawled in and said: “Aren’t you going to kiss me good-night?”

He grinned. “That’s the way I like to hear you talk.”

“How’s the groin?” I asked.

“Nasty for a minute or two. It’s all right now.”

“It was a dirty trick.”

“Forget it.”

He went to the door and put a finger on the light-button and turned. He looked at the wall, at my rifle hanging there.

I said: “Anyway, you know I didn’t kill him with that.”

He put out the light and closed the door behind him.

I lay in darkness and listened to him go down the stairs. There were low voices in the hall. After a while the voices went into the living room which was directly below my room. I got out of bed and put my ear against the floor, but distinct words didn’t come through. Without putting on a light, I groped my way to my shirt on a chair and fumbled out cigarettes and matches. I sat on the edge of the bed smoking, listening for Kerry’s car to depart. It didn’t.

***

I was on my fourth cigarette when I heard a car in the driveway. It was arriving, not leaving. I went to the window. George Winkler slammed the car door and ran up the porch steps. He was too big and clumsy to run. This was important. This was murder.

I tossed my cigarette out of the window and crossed the room in darkness and went out to the head of the stairs. All four were in the living room, keeping their voices low in order not to wake me. They thought they had me safely and snugly out of the way while they consulted a friend and lawyer about me. They didn’t need me there to hear any more of my side. They’d heard enough, and it was like the last time, an outrage to common sense. But I was a brother, a foster-cousin, a pal, so they had to do what they could for me.

I started down the stairs, my bare feet coming down hard, but making no sound. I was shaking worse than at any time I could remember. I gripped the bannister with both hands and clung. I mustn’t give into it. Bursting in on them in this condition would not help me.

“Are you sure nobody saw you or Alec?” George Winkler said in the living room.

I could hear them from here. The arched, doorless living room entrance was only five feet from the foot of the stairs. I leaned over the bannister, listening.

“It was after one-thirty,” Kerry was saying, “and dark as the inside of a pocket. There wasn’t a light in any house, and if anybody saw either of us they saw only a car pass or a shape walking behind a flashlight. Unless they happened to be looking at Schneider’s porch, but the house can’t be seen from the road or from any nearby house. It’s on a ledge at the end of five hundred feet of driveway and a couple of hundred feet past that.”

“But the shots,” George said.

“Two loud noises,” Kerry said.

“Maybe a truck backfiring on Old Mill Road, for what anybody who heard them knew. I wasn’t sure they were shots when I heard them, even though I was half-expecting something to happen. Besides, nobody came to the house or called out or showed a light. The body mightn’t be found for a week if we keep our mouths shut.”

I heard restless feet pace the floor. Then George said: “My God, do you folks realize what you’re doing? We’re accomplices after the fact. In the eyes of the law we’re as guilty as Alec, Kerry especially. He saw Alec and the murdered man and didn’t report it.”

“I’ll do more than that for Alec,” Kerry said.

“There’s nothing to point to Alec,” Ursula said. “All we have to do is say nothing.”

George snorted. “Do you think the police are dummies? As soon as they find Schneider with two bullets in him they’ll pile on Alec. He murdered his wife because she was unfaithful to him. He was acquitted by the jury not because there was any doubt of his guilt, but because of extenuating circumstances. Then a couple of days after he’s released the man with whom she had betrayed him in particular is found murdered. It’s a pattern you can’t get away from: his wife and her chief lover murdered. You don’t know how thoroughly the police work. They’ll find out about the poker game tonight. They’ll learn that Alec was desperate for five thousand dollars in cash and that he got a good part of it. They’ll established the time of death as not long afterward. They’ll conclude that Alec coldly planned the murder and needed getaway money.”

“No!” Miriam cried. “I can’t let you say that about Alec. He didn’t murder either of them.”

“How do you know?” George demanded.

“He said so.”

***

George delivered himself of another snort. “It’s your privilege to believe that. He almost had me believing that he was innocent of Lily’s murder, but now with Schneider shot—“

“Alec didn’t try to run away,” Ursula said. “He came home and wanted to call the police.”

“Because Kerry saw him,” George said. “The fight went out of him; he was ready to give up. You can never tell what anybody so emotionally unstable will do.”

“Hold it,” Kerry said. “You’ve got the arguments on your side, but—”

“My side!” George said. “I’m giving the police’s side. I’ll do all I can for him, but I can’t see where it will be enough.”

“We know you will,” Kerry said. “That’s why we phoned you to come right over. But Alec said he an idea Schneider was blackmailing somebody and the blackmailer killed him.”

“And you believed him?” George said.

There was a brief silence. Then Ursula said: “If the fight went out of Alec, why didn’t he confess?”

“How do I know?” George said. “You people are groping for anything at all to help you believe him. So am I. But look at the terrific coincidences. This afternoon Alec and I discussed laws of chance. He tried to show me mathematically that walking into Lily’s bungalow a few minutes after she was murdered and under circumstances that pointed directly to him as the murderer wasn’t such a great coincidence after all. I might concede that, though with reservations. But when it happens a second time—when Alec walks up to Schneider’s house at the precise moment when somebody is aiming a gun at Schneider, a man Alec has reason to hate, that becomes incredible.”

He paused. I could hear feet nervously prowling the floor—George’s feet, probably—and I heard nothing else. They hadn’t any answer for him. Neither had I.

“You can’t get around coincidences like that,” George went on. “Not one coincidence, and certainly not two. And look at the rest of it. The gunman hears and sees Alec’s car pull up and sees Alec come up to the house behind a flashlight. Does the gunman postpone the shooting? Does he run to cover? Does he take a shot at Alec first to protect himself? Not this gunman. He blandly goes ahead and kills Schneider as if he had all the world to himself. Then he runs. Alec, who’s supposed to have sense, chases him unarmed and with his flashlight on. Does the gunman turn around and shoot Alec who’s a clear target and perhaps has recognized him? Not this gunman. He’s every bit as thoughtless for his own safety as Alec is. And then Alec loses him and hurries back to the porch in time for Kerry to find him standing over the body.”

“The gun,” Miriam said. “Why didn’t Alec have a gun when Kerry saw him?”

“That’s right,” Kerry agreed. “His rifle is up in his room. I saw it when I went up there with him.”

“How do you know a rifle killed Schneider?”

***

“I heard the shots,” Kerry said. “I can tell the difference between a rifle and a small gun. A rifle booms, a handgun barks. And I’d say the rifle was bigger than Alec’s .22.”

“That doesn’t get us anywhere,” George said. “He wouldn’t use his own gun, especially not if he had planned this in advance. This is hunting country. Practically every house has a rifle of some sort. He could have stolen one or maybe driven to Trevan or somewhere this afternoon and bought one. Then, after shooting Schneider, he tossed it away in the brush. That’s another problem we have to face. It’s probably near there with his fingerprints all over it.”

“Probably” Miriam exclaimed in outrage.

“Miriam, you’ll wake Alec,” Ursula warned her.

“I don’t care. Why shouldn’t he be down here to defend himself? You’re fools—you especially, George. You’re supposed to be a lawyer. Can’t you get it into your head that if Alec had deliberately set out to murder somebody he would never concoct such ridiculous stories? He’s so clear-headed, so intelligent. He’d have a perfectly logical story to tell. The fact that what he told us doesn’t seem to make sense proves that he’s innocent.”

Leaning over the bannister in my pajamas, I smiled to the wall. Smart girl, Miriam. She’d got to the heart of it.

“I’d agree,” George said so softly that I had difficulty hearing his words, “if these weren’t crimes of passion which would make any man lose his head. I’d even agree if he were normal. We know he isn’t. Look at the way he’s acted since he was acquitted. Brooding, beating his brains out, speaking of nothing but Lily’s death. Isn’t it obvious that it was jealousy of Schneider that was gnawing at him?”

I swung from the bannister to go down to them. I forced myself to sit down and clasped my knees hard. My heart ticked off the long seconds of quiet in the living room. Even George’s pacing had ceased.

Miriam spoke, and her voice was low now and broken. “I don’t believe it. I can’t.”

“How are you going to get around facts and incredible coincidences?” George said. “But say we’re convinced he’s innocent—what do we do? I see only two alternatives, and neither of them is pleasant. We can sit tight and see what develops. Plenty will. The police will question Alec. Will he deny that he was near Schneider’s house tonight? Not in his state of mind. He’ll confess or more likely he’ll doggedly stick to the story he told you, the way he did to his equally absurd story of Lily’s murder. We four will be in hot water for having held out on the police. I’ll be lucky to get away with being disbarred. The other alternative is to call the police now. In either event, Alec will stand trial.”

“What will his chances be?” Kerry asked.

“Bad,” George said. “A clever lawyer like Magee can beat one murder, but two will have him licked. Alec’s been acquitted for Lily’s murder, so that can’t be brought into the trial, but the D.A. will find ways to link it up. Besides, it will be impossible to get a jury that isn’t aware that Alec had murdered his wife and that the man for whose murder he was being tried had been her lover. Frankly, I think the only thing we can hope for is to have him found insane and sent to a mental institution.”

Ursula said briskly: “You overlooked a third alternative. Alec can go to South America.”

***

I stood up and descended three steps and stopped. The staircase wobbled; the ceiling was coming down on me. My throat was raw with a scream that I would not let past my lips.

“Do you realize what you suggest?”

George said. “He’s murdered two people. I’m not saying he’s to blame. Let’s say the war twisted something in his brain. Murder has become a habit to express his emotions.”

That couldn’t be Ursula’s voice I heard, but it was. The deepness had gone out of it. She whimpered: “My God, George, you’re wrong!”

“Maybe I am,” George said. “I’d do anything for you, Ursula, but I can’t take this chance. I’m going home now. Do whatever you think best and I promise to keep my mouth shut. I’m sorry, but that’s as far as I can go.”

I raced up the stairs, running away, but in my pajamas I could go only as far as my room. When I had the door closed between me and them, I stood against it, breathing hard.

George’s car left, but Kerry didn’t. Under the floor I hear their voices still muttering. They were deciding my fate, sheltering me and protecting me, in their love offering me what was worse than jail or a mental institution or the electric chair, to be an eternal fugitive in a foreign land.

I put on the light and smoked a cigarette down to my fingers. There was a fourth alternative which had occurred to none of them. I put on my gray tweed suit and stuffed the $2,235 in cash I had won at poker into a pocket. My two windows opened out on the flat porch roof. I climbed out and slid down the post and landed among the hollyhocks.

An open living room window was almost within reach. I heard Kerry say miserably: “I’m willing to bet we can’t make him go away, but it’s worth a try. I can’t see anything else.”

A woman was sobbing. I looked through the window. Miriam sat on the couch with a handkerchief to her face and her shoulders shaking. Ursula wept also, but on her feet and silently. I had never thought that I would see Ursula weep.

I walked into the darkness.



\vspace{2\nbs}
\ChapterDeco[c1]{\decoglyph{e9665}}
\clearpage
\thispagestyle{empty}

%end chapter loop

\scenebreak
\scenebreak
{\centering\textsc{the end}\par}

\clearpage

\null

\centering\textsc{www.TalesofMurder.com}\par

\vspace*{10\nbs}

%\centering\InlineImage[0, 3em]{/home/darkstar/dox/working-files/LaTeX/atticus.jpg}

TALES OF MURDER PRESS, LLC

\null

\scshape{675 TOWN CENTER BLVD
BLDG 1A STE 200 PMB 530
GARLAND, TEXAS 75040}

\null

\textit{atticus@talesofmurder.com}
\vfill


\end{document}

