% !TeX TS-program = LuaLaTeX
% !TeX encoding = UTF-8
\documentclass{novel}
%%% METADATA (FILE DATA):
\SetTitle{TITLE}
\SetAuthor{AUTHOR}
\SetPDFX{X-1a:2001}
\SetTrimSize{4.25in}{6.875in}
\SetMediaSize{4.5in}{7.12in}
\SetMargins{0.5in}{0.5in}{0.5in}{0.7in}
\SetParentFont{Libertinus Serif}
\SetFontSize{9.5pt}
\SetHeadFootStyle{5}
\SetHeadJump{1.5}
\SetFootJump{1.5}
\SetLooseHead{50}
\SetEmblems{}{} % Default blanks.
\SetHeadFont[\parentfontfeatures,Letters=SmallCaps,Scale=0.92]{\parentfontname}
\SetPageNumberStyle{\thepage}
\SetVersoHeadText{\theAuthor}
\SetRectoHeadText{\theTitle}
%%% CHAPTERS:
\SetChapterStartStyle{footer} % Equivalent to empty, when style has no footer.
\SetChapterStartHeight{10}
\SetChapterFont[Numbers=Lining,Scale=1.6]{\parentfontname}
\SetSubchFont[Numbers=Lining,Scale=1.2]{\parentfontname}
\SetScenebreakIndent{false}
%%% BEGIN DOCUMENT:
\begin{document}
\frontmatter
\thispagestyle{empty}
% Half-Title Page.
\begin{parascale}[2]
\vspace*{3\nbs}
\centering\charscale[0.75]{TITLE }\par
\centering\charscale[0.75]{TITLE LINE 2}\par
\centering{TITLE LINE 3}\par
\end{parascale}
\clearpage
\thispagestyle{empty}
\null % Necessary for blank page.
% Alternatively, List of Books.
\clearpage
\thispagestyle{empty}
% Title Page.
\begin{parascale}[4]
\centering\charscale[0.75]{TITLE }\par
\centering\charscale[0.75]{TITLE LINE 2}\par
\centering{TITLE LINE 3}\par
\end{parascale}
\vspace*{2\nbs}

\begin{parascale}[1]
\centering\textit{SERIES}\par
\vspace*{3\nbs}
\charscale[2]{AUTHOR}\par
\end{parascale}
\vfill
\begin{parascale}[1]
% \centering\InlineImage[0, 3em]{/home/darkstar/dox/working-files/LaTeX/atticus.jpg}

A Tales of Murder Press, LLC\par
\textit{GENRE} Novel\par
\end{parascale}
\clearpage
\thispagestyle{empty}
% Copyright Page.
\null\vfill
\allsmcp{First edition} PUBLICATION_DATE\par
\null\null
\allsmcp{ISBN}\par
\null\null
\vfill
\begin{adjustwidth}{3em}{3em}
\textit{This novel is in the public domain.} Certain \mbox{elements} in this edition are Copyright © COPYRIGHT_DATE Tales of Murder Press, LLC
\end{adjustwidth}
\clearpage
\thispagestyle{empty}
\clearpage % because ToC must start recto
\thispagestyle{empty}
\begin{toc}[0.5]{0em}
{\centering\charscale[1.25]{Contents}\par}
\null

%some kind of loop through chapters for TOC
\tocitem*[1]{CHAPTER_TITLE}{STARTING_PAGE}
\end{toc}
%end loop
\clearpage

\mainmatter
\cleartorecto
\thispagestyle{empty}

%some kind of loop through all chapters
\begin{ChapterStart}
\vspace{3\nbs}
\ChapterSubtitle[l]{Chapter 12}
\ChapterTitle[l]{## Emil Schneider}
\end{ChapterStart}
\FirstLine{\noindent ### Chapter 12

## Emil Schneider

It was after one o’clock when they departed. I watched them from the window of my room. Oliver and Bevis Spencer got into one car. Art Masterson and Dietz into another, and both cars drove off. Several minutes later Kerry Nugent and George Winkler and Miriam came out together. They stood at the foot of the porch steps. I tried to hear what they said, but all I could get out of the low jumble of words was my name.

There was a light knock on the door. I turned from the window and said, “Come in,” and Ursula entered.

Her right fist was filled with money. She dropped it on my dresser. “There was $7,173 in that pot,” she said.

I went to the dresser and pushed the money apart. There were a couple of fifties, but the rest was in twenties and tens and smaller stuff. “How much of it is this?”

“Every dollar in cash I had and could accumulate from the players—$2,235. Most of it came from Oliver Spencer. I hold checks for the balance due you.” In her hand she held a slip of paper from the pad. “Actually you made $4,131 profit: on that hand. The rest were your own chips which you owed me. Before that you’d lost $123, so there’s still $1,773 coming to you for a total amount of $4,008 in winnings.” She recited those overwhelming sums in the monotone of an accountant giving a report. Her face was unnaturally fleshy, lined and lumpy with an inner weariness.

“What made you change your mind?” I asked. “While I was raking in the pot you whispered to Art Masterson and Mr. Spencer not to pay off any of it in cash.”

“It’s what you said just before they left the cardroom—about people you could turn to when you needed them. That hurt more than you knew. Had I ever failed you, Alec?”

She had failed me in letting reason dominate faith in me. Yet at the same time she had backed me all the way, and objectively I should have been grateful. I was, objectively.

I put my hands on her shoulders. “You’ve always been swell to me.” Her eyes, somber and anxious were only a couple of inches lower than mine. “I’m praying that I’m not making a terrible mistake.”

“By letting me have the cash? Don’t worry. The worst I’ll do with it is throw it away.”

Ursula stepped away from my hands and started toward the door.

“One more favor,” I said. “Can you let me have your check for $2,765? That includes what you still owe me from the game, plus a $992 personal loan which I’ll repay you next week.” She turned with her hand on the knob. “Tonight?”

“I’d like it in ten minutes. And please make it out to cash.”

She stood there debating with herself. “If you wish,” she said listlessly and closed the door behind her.

I heard a car start up in the driveway and then another. George and Kerry were leaving. The downstairs door closed as Miriam came into the house.

***

I made a neat pile of the money and stuck it into the right pocket of my slacks. The wad bulged the material. Emil Schneider had said midnight, but he wasn’t leaving until morning. He would have to be satisfied with close to half of it in cash. It was a lot of money.

If he insisted on the full five thousand, I’d give him Ursula’s check, and he would have to accept my word that it wouldn’t be stopped at the bank Monday morning. I had done all I could. It was now up to him to take it or leave it.

Ursula had the check in her hand when I went down to the living room. Miriam was with her. They looked down at the bulge in my pants pocket. “I’d like to use the car, Ursula,” I said. “I’ll only do a few miles.”

“You’re driving to the station?” Miriam asked huskily.

“Look,” I said testily. “There’s nothing I have to run away from or want to. I’m going to see somebody in town and come right back.”

Ursula studied the check as if she were seeing it for the first time. “The cash in your pocket and this check makes an even five thousand dollars. Why that round sum? What are you going to do with it?”

“Something foolish, maybe.” I took the check from her lax fingers and slid it into my wallet. “Thanks a lot. May I have the car?”

“Anything you want,” Ursula said dully.

***

There was no moon. Low, threatening clouds blotted out the stars. I parked in front of the shed garage and cut the headlights. Immediately I was blacked out except for the lighted windows in Emil Schneider’s house a couple of hundred feet away. I took the flashlight out of the glove compartment and went up the wooden steps. The birch rails were white markers toward the house.

I had reached the flagstone walk when the porch light went on. The door opened. Schneider, still in his undershirt, came out to the head of the porch steps and peered to his left. “Linn?” he said. Then the corner of his eyes must have seen my light, for he started to turn to me. He never completed that turn.

The shot was louder than any cannon I had ever heard. It was the silence that had preceeded it and the unexpected quality of the sound. I felt myself jump, and for a moment I thought I was hit. I didn’t feel anything, but I had seen a ground crew sergeant have his arm blown off by a 37 mm cannon without feeling it until later.

The gun sounded again, not as loud this time because the startling effect was gone. I dropped flat on the field-stone. Nip strafers had conditioned my reflexes. Or, maybe I’d been hit. I wriggled. Nothing hurt.

The silence was back. I raised my head. Schneider was no longer on the porch. And I saw movement out from the left corner of the house—a momentary glimpse of something I knew was human only because it was tall and erect. Something like a long rod extended out from it and then merged with the shadow it made. A rifle.

I leaped up and ran forward, reckless with that mixture of fear and anger which makes heroes of men under fire. I saw Schneider then. Under the bright porch light he was a motionless splotch at the head of the steps. My throat uttered an animal cry that had no words in it. I veered toward the corner of the house.

The shape was gone in the darkness. I heard the familiar swish of brush when somebody crashes through it. The batteries of my flashlight were weak; the spray spread out a few feet and petered off. I ran, sweeping the feeble beam in an arc. I reached the brush and went through it. Then the light touched trees—big stuff, elms and oaks. The woods, I knew, ran down the side of the slope to the edge of Old Mill Road. There was no sound but the clamor of katydids.

Abruptly everything washed out of me but fear. I doused the light and in the darkness stumbled back to the house. When I reached the fringe of light flowing out from the porch and two side windows, I looked over my shoulder. There was no point to the gesture; only a black curtain lay behind me. I ducked low and raced around to the porch and up the steps.

Emil Schneider had not moved. He would never move. There was a hole in his turned-up cheek.

I had seen a lot of dead men, but this was like Lily being dead—intensely personal, as if death were reaching out from the flesh it had conquered and touching me. Blood rushed to my head as I bent over him. I yanked myself erect, but my head continued to whirl. I grabbed at one of the posts at the head of the steps and heard myself whimper.

“Don’t let it get you!” I said aloud. “Keep your head! ”

***

It went away, a little. I didn’t need the post. I pushed myself away from it, and I saw a light coming up the walk.

Panic hit me. I glanced in frenzy and then scooted into the house.

“What the hell, Alec!” a voice called.

Running feet slapped the flagstone. I hesitated in the hall and went back to the door on my toes and looked through the small rectangular door window.

Kerry Nugent had reached the foot of the porch steps. He stared at the thing on the porch and then came up slowly. “God!” he said, and raised his head. He couldn’t have missed seeing me go into the house. “Alec,” he said softly.

He hadn’t a rifle in his hand and there was nowhere in his summer uniform where he could hide one. I came out. Across the dead man we looked at each other.

Kerry had no words. The silence between us was beyond endurance. I said: “Funny you’re always right behind me when somebody is murdered.”

“Another one!” he said.

“What are you doing here?” I demanded.

Kerry came all the way on the porch and carefully skirted the body and stood beside me. “You acted queer,” he said. “About that five thousand bucks. When Ursula collected all the cash there was, it was plain she’d changed her mind and was going to give you the money. What did you want it for?”

“Schneider saw who murdered Lily. He was going to sell me the name.” Kerry gave me a quick, sidelong look. “The way you’ve been acting, we thought you might be planning to run away. There was no need; you’d been acquitted. Or maybe you’d do something else with the money that—that—”

“Say it. That a mental case like me might do. I had to be protected from myself—the way you’d all done it last time.”

Kerry didn’t bother to argue that. He said: “We decided that I park down the road and follow you if you came out.”

“Miriam again?”

“Miriam and George Winkler and I. We talked it over outside the house just before I left. So I trailed your car and saw you turn up Schneider’s driveway.”

***

I looked down toward the parking circle in front of the shed garage. It was too dark to see my car. “You didn’t follow me all the way,” I said. “I didn’t hear your car. You knew this was a dead-end driveway, so you parked on Ivy Lane and walked the rest of the way. You came to spy on me.”

His square jaw jutted “Sure I did. Why not? If you’re going to go around killing—”

“Damn it!” I screamed. “I didn’t! Now you’re condemning me without even hearing my side of it.”

“All right, I’m listening.”

I put a cigarette between my lips, but my shaking hands wouldn’t let me give myself a light. Kerry snapped on his lighter and held it for me. His hand was as nerveless as a rock. Between puffs I told him.

He listened with stolid patience. When I finished, he went to the dead man and knelt beside him. “One slug entered his side, below the left armpit. The other entered his cheek and went all the way through. I don’t know how he managed to scream before he died.”

“Scream?”

“I heard two shots while I was coming down the driveway and then a scream.”

“I screamed when I started to chase the murderer. It gushed out of me by itself.”

Kerry straightened up and stared at me. “There were only two shots, Alec?”

“Those were enough to kill him.”

“But they’re both in Schneider.” Kerry looked down the path and then turned to the left side of the house. “Why would anybody risk shooting Schneider just as you were coming up the path? And why didn’t he shoot you when you were chasing him? You were in the light of the porch for at least a few seconds and you said you chased him with your flash on. Why didn’t he take a shot at you then?”

There was an answer. There had to be, but I couldn’t straighten one out in my head. There was a sick, throbbing tiredness in back of my skull which wouldn’t let me think.

“How should I know why he acted like that?” My voice was high, strident. “I’ve got some question of my own. How the hell do you happen to show up every time somebody is murdered?”

“I told you.”

“And I don’t like it.”

He turned his head listening. It was only the sound of a distant car, probably on Old Mill Road.

Kerry said sharply, “Let’s get out of here!” and put his hard hand on my arm.

That was something I could agree with. Side by side we went down the walk and down the wooden steps. He held onto my arm. Like a policeman with his prisoner, I thought.

When we reached my car, I said: “Are you going to take me to the police in person or are you going to call them on the phone?”

“Don’t be a dope. I’ll help you go wherever you want.”

Savagely I wrenched open the car door. “I’m going home and I can get there without help.”

He was walking back to his car parked on Ivy Lane when I passed him on the driveway. He stepped into high grass to let my car go by. Halfway home I saw his headlights behind me. His car remained glued to my taillight the rest of the way home.



\vspace{2\nbs}
\ChapterDeco[c1]{\decoglyph{e9665}}
\clearpage
\thispagestyle{empty}

%end chapter loop

\scenebreak
\scenebreak
{\centering\textsc{the end}\par}

\clearpage

\null

\centering\textsc{www.TalesofMurder.com}\par

\vspace*{10\nbs}

%\centering\InlineImage[0, 3em]{/home/darkstar/dox/working-files/LaTeX/atticus.jpg}

TALES OF MURDER PRESS, LLC

\null

\scshape{675 TOWN CENTER BLVD
BLDG 1A STE 200 PMB 530
GARLAND, TEXAS 75040}

\null

\textit{atticus@talesofmurder.com}
\vfill


\end{document}

