% !TeX TS-program = LuaLaTeX
% !TeX encoding = UTF-8
\documentclass{novel}
%%% METADATA (FILE DATA):
\SetTitle{TITLE}
\SetAuthor{AUTHOR}
\SetPDFX{X-1a:2001}
\SetTrimSize{4.25in}{6.875in}
\SetMediaSize{4.5in}{7.12in}
\SetMargins{0.5in}{0.5in}{0.5in}{0.7in}
\SetParentFont{Libertinus Serif}
\SetFontSize{9.5pt}
\SetHeadFootStyle{5}
\SetHeadJump{1.5}
\SetFootJump{1.5}
\SetLooseHead{50}
\SetEmblems{}{} % Default blanks.
\SetHeadFont[\parentfontfeatures,Letters=SmallCaps,Scale=0.92]{\parentfontname}
\SetPageNumberStyle{\thepage}
\SetVersoHeadText{\theAuthor}
\SetRectoHeadText{\theTitle}
%%% CHAPTERS:
\SetChapterStartStyle{footer} % Equivalent to empty, when style has no footer.
\SetChapterStartHeight{10}
\SetChapterFont[Numbers=Lining,Scale=1.6]{\parentfontname}
\SetSubchFont[Numbers=Lining,Scale=1.2]{\parentfontname}
\SetScenebreakIndent{false}
%%% BEGIN DOCUMENT:
\begin{document}
\frontmatter
\thispagestyle{empty}
% Half-Title Page.
\begin{parascale}[2]
\vspace*{3\nbs}
\centering\charscale[0.75]{TITLE }\par
\centering\charscale[0.75]{TITLE LINE 2}\par
\centering{TITLE LINE 3}\par
\end{parascale}
\clearpage
\thispagestyle{empty}
\null % Necessary for blank page.
% Alternatively, List of Books.
\clearpage
\thispagestyle{empty}
% Title Page.
\begin{parascale}[4]
\centering\charscale[0.75]{TITLE }\par
\centering\charscale[0.75]{TITLE LINE 2}\par
\centering{TITLE LINE 3}\par
\end{parascale}
\vspace*{2\nbs}

\begin{parascale}[1]
\centering\textit{SERIES}\par
\vspace*{3\nbs}
\charscale[2]{AUTHOR}\par
\end{parascale}
\vfill
\begin{parascale}[1]
% \centering\InlineImage[0, 3em]{/home/darkstar/dox/working-files/LaTeX/atticus.jpg}

A Tales of Murder Press, LLC\par
\textit{GENRE} Novel\par
\end{parascale}
\clearpage
\thispagestyle{empty}
% Copyright Page.
\null\vfill
\allsmcp{First edition} PUBLICATION_DATE\par
\null\null
\allsmcp{ISBN}\par
\null\null
\vfill
\begin{adjustwidth}{3em}{3em}
\textit{This novel is in the public domain.} Certain \mbox{elements} in this edition are Copyright © COPYRIGHT_DATE Tales of Murder Press, LLC
\end{adjustwidth}
\clearpage
\thispagestyle{empty}
\clearpage % because ToC must start recto
\thispagestyle{empty}
\begin{toc}[0.5]{0em}
{\centering\charscale[1.25]{Contents}\par}
\null

%some kind of loop through chapters for TOC
\tocitem*[1]{CHAPTER_TITLE}{STARTING_PAGE}
\end{toc}
%end loop
\clearpage

\mainmatter
\cleartorecto
\thispagestyle{empty}

%some kind of loop through all chapters
\begin{ChapterStart}
\vspace{3\nbs}
\ChapterSubtitle[l]{Chapter 20}
\ChapterTitle[l]{## Q. E. D.}
\end{ChapterStart}
\FirstLine{\noindent ### Chapter 20

## Q. E. D.

For a long minute I stood on the road recalling where each room in the sprawling house was located. The only light was in the largest of the wings, the living room. A voice drifted out to me, unctuous, modulated—a radio voice.

A lopsided moon provided too much light. I walked a short distance back the way I had come and squeezed through an opening in the boundary hedge. Even then the dark windows I wanted to reach were a good seventy feet away across a stretch of lawn broken only by a cluster of snowball shrubs. I made it in two spurts, the first to the shrubs, the second to the double casement windows.

The windows were open, but the double screens were fastened by a hook-and-eye. I didn’t have to go inside if I could see enough of the room from outside. I put my face against the screen. What moonlight trickled in made distorted shapes of the furniture and stopped short of the walls. I tried my flashlight, but the screen mesh restricted the beam. There was nothing to do but enter the room.

That was easy. The blade of my pocket knife slid between the screens and lifted the hook from the eye. I swung the screens inward and climbed through and swept my flashlight about.

This was the room they used to go to when they wanted to get away from the rest of the house, to study or hold bull or hen sessions. With considerable affectation they called it the library because of the two glass-door bookcases which held ancient and unread volumes. For the rest, they had put into it whatever furniture they had no use for elsewhere in the house: a scarred desk, a treadle sewing machine, a horsehair sofa, a cracking brown leather chair. On the wall hung a pair of boxing gloves, crossed sabers inherited from a remote Civil War ancestor, and a rifle on pegs.

The radio voice, strained by the length of the house, was overlapped by a woman calling from another room: “Where did you say you put it?” The reply was an indistinct mumble.

I snapped off the flashlight and stuck it into my hip pocket. I had found what I expected or hoped to find. The rest was up to the police.

I was halfway back to the window when the door behind me opened. I whirled. Moonlight was brighter in the room than it had appeared to be looking in from outside, and I could see the door closing and the white-shirted shape standing there with its back to me. In the sudden stifling stillness, I heard the thin grating sound of a key turning in the lock.

***

There was perhaps time to flee through the window, but the thought was discarded as soon as it crossed my mind. I could not give him the chance to dispose of the one physical link between himself and murder.

Then he was turned to me, his face a pale blob, and his hands were pale too in moonlight. The right hand was raised, holding something long and the lower half of it not as dull as the rest. A carving knife, probably—bigger and deadlier than the steak knife he had driven into Lily’s heart.

He must have seen me cutting across the lawn, and had gone outside to watch me enter this room or had listened at the door and heard me enter.

Within the drawing of a breath he cut the distance between us in half. It was not a large room, and now we stood facing each other across five feet of space. Too late to flee now. He would get me between the shoulders as I scrambled through the window. We were bound to this room, to settle it here within the next minute or two, the mathematics of our lives reduced to the lowest common denominator. He or I. He killed me or I killed him by turning him over to the police.

He leaped. I jerked myself sideways, away from the knife, and I felt the blade rip through jacket sleeve and shirt sleeve. There was no pain, but a scream, compounded of fear and shock and rage, tore from my throat.

The fury of the knife-thrust had taken him past me. Momentarily I was behind him. I managed to get my hands on his right arm, below the elbow. He wheeled, pushing his shoulder against me. My grip loosened, caught again at his wrist, and held on.

The next minute or hour or year lost detail. Always his breath was harsh in my face, groaning and swearing and sobbing as he strove to free the knife. My job was to wrench it from him or turn it against his own body. I could do neither.

We were on the floor then, myself on top clinging to his wrist, while he clawed my mouth and nose and chin with his free left hand. A chair was on my legs. We must have knocked it over when we had fallen, though I didn’t remember. Blood was in my mouth. I dropped flat on him and pushed my face into his chest to protect it. He pulled my hair, and I hear myself scream, and then I became aware of those other sounds.

***

They were on the other side of the locked door, impotently rattling the knob and pounding on the panel. They were shouting too, begging him to open the door, begging him to tell them what was happening.

Suddenly somebody was in the room with us. “Cut it out, you guys!” he said. Like an adult breaking up a fight between two kids. Oh, he was a comedian all right. Cut it out, he said, and so we would stop and straggle up to our feet and each accuse the other of having started it.

Except that he couldn’t stop unless with my death. He made a final effort, heaving up against me with all his strength, tearing at my hands on his wrist, trying to reverse the point of the knife and drive it into my body. “The knife!” I gasped.

I felt hands move down my arms and over my locked fingers and to that other hand and the knife in it. I lifted my head. Moonlight glinted on the captain’s bars on the newcomer’s shoulder.

Abruptly the man under me subsided. He was finished and he knew it.

“I’ve got it,” Kerry said. His shape rose and left us.

My hands were still a vise around that wrist which could no longer harm me. I loosened them and pushed myself up to my feet. He too was rising. His gangling form reached a sitting position and then remained like that, immobile and bowed in defeat.

Light flooded down from the ceiling. Kerry turned from the switch. “Did he hurt you?”

Three of my fingers fitted into the hole in my jacket sleeve where the blade had sliced through. My skin was unbroken. I had a headache and my clawed face burned, but I felt fine. “No major damage,” I said. “And this is one time I don’t complain because you were right behind me.”

He lifted his hand to look at the knife in it. “Only this time I came before a killing instead of after.”

Through the door panel an anxious voice shouted: “Kerry, is that you? Why don’t you let us in?”

For a moment Kerry looked as worn and shaken as if he had just stepped out of a bomber after a combat mission. It would be hard enough for me to face them and tell them, but Kerry was engaged to marry Helen. Then the square jaw set, the clear eyes hardened. He turned the key in the lock.

The door flew violently inward. Oliver and Helen Spencer burst into the room.

***

They stopped dead as if both were controlled by a single mechanism. Their eyes swept past Kerry, past me, and came to rest on Bevis sitting on the floor with his torso pushed into itself.

“What’s Alec doing here?” Helen said to nobody in particular. She detached herself from her father and dropped down beside her brother. “Bevis, what happened?”

Bevis Spencer’s somber, tragic face lifted dully and then sagged into his hands. “Let me alone!” he moaned. “Go away from me!”

“Are you hurt? What did he do to you?”

“Let me alone!”

Mr. Spencer was staring at the knife Kerry held along his thigh. “Kerry, you tell us,” he said quietly.

“I don’t know much more than you do, sir. You saw me come into the house while you and Helen were pounding on the door. I ran outside and climbed in through that window. I found them fighting and took the knife away from Bevis.”

“Bevis had the knife?” Mr. Spencer whispered.

“Yes.” Kerry made a feeble motion with his hand and the knife left it and fell on the desk.

“Whose knife is it?” Mr. Spencer said. “Did Alec bring it with him?”

It was time for me to get into it.

There was a frog in my throat. I cleared it and said: “It must have come from this house. Bevis tried to kill me with it.”

Helen laughed and stood up. It was a sound more dreadful than the moans trickling through Bevis’ fingers. “Oh, sure, everybody is always killing, except Alec,” she said. “He killed Lily and her lover, and now he wants Miriam and he came here to kill Bevis because Bevis asked her to marry him. He kills everybody who’s in his way.”

Oliver Spencer kept looking down at his son as if trying to recognize him. “Is that how it was, Bevis?”

Bevis lifted his head and dropped it. In that moment Mr. Spencer became an old man. He shuffled to the desk and stared down at the knife.

“What are we waiting for?” Helen said harshly. “I’m going to call the police.” She started toward the door, moving between Kerry and myself. Neither of us stirred.

“Wait,” her father said thinly. “This is our carving knife.”

She spun at the door. The peaches left her round face; the cream clotted. She dug a hand into Kerry’s arm. “Darling, what does all this mean?”

“Alec knows the answers,” he muttered.

I no longer felt fine. This was the business of the police, but it hadn’t worked out that way.

“Bevis murdered Lily and Emil Schneider,” I said.

***

Mr. Spencer turned from the knife on the desk and looked down at his son sitting broken and without protest on the floor. Bevis’ silence, his position, all of him, was a confession.

“Why do you say that, Alec?” Mr. Spencer asked me quietly.

“I’d rather save it for the police.”

“Let me hear it first.”

He was entitled to the facts, to know that there was no loophole that could save his son. I said: “Kerry, have you the equation?”

Kerry dug the folded sheet out of a pocket and gave it to me. When I turned from him, Mr. Spencer had dropped down on the horsehair sofa and was abstractedly patting his bald pate. His face was empty. He reached out a bony hand for the paper and fumbled out his reading glasses. Bevis’ moans had stopped, but his head remained bowed. Helen left Kerry and sat down beside her father and together they read the paper.

The silence was like physical pain. I felt that I had to break it or suffocate. I asked Kerry if the equation had brought him here.

“More or less.” He spoke in the husky undertone people use at funerals. “George and I worked it down to one of the Spencers. I went upstairs to ask you a couple of questions about it. Miriam tried to keep me back, so I suspected something was screwy. When I found you’d left the house, I thought that whatever you were up to I ought to be in on it. I tried this place first.”

“What’s George Winkler doing?”

“I left him talking with Ursula and Miriam. They’ll hold him there till you get back. He thinks I left to keep a date with Helen.”

Mr. Spencer looked up from the paper. “What does this gibberish mean?”

So I explained the hypotheses to him and Helen. They didn’t get it. I took out a second sheet on which I had worked out the equation against the various suspects and handed it to him.

“I had to limit my suspects in some way,” I said, “and at the last I considered only those who had been in

Ursula’s house the night Lily was murdered: Ursula, Miriam, Kerry, George Winkler, Owen Dowie, Helen, Bevis and you. Ursula and Miriam had knowledge of my movements before each murder and conceivably had reason to murder Lily. But OS and MA don’t cancel out against their names. They couldn’t have murdered Schneider; they’d been in the house when I’d left and I’d driven straight there. And, of course, I couldn’t see either of them wanting to hurt me.

“As for Sheriff Dowie, he wasn’t at the second poker game, so KS doesn’t cancel out, and, as far as I know, ML and MS and MA don’t. Kerry and George Winkler, both with full knowledge and opportunity, are left with ML and MS and MA. The motive was the tricky part, the most unfixed of the terms. There could be facts beyond my knowledge, but I had to go along with what I had.

“Finally you three Spencers. Helen saw her father knocked down by Don Yard because of Lily, and perhaps for that, and other reasons, hated her. It doesn’t matter. She wasn’t at the house just before Schneider’s murder, so KS doesn’t cancel, nor does MS. You, Mr. Spencer, and Bevis were in both poker games and had opportunity after each of them and resented or disliked or hated Lily because of what she was doing to Helen. Not strong motive for murder, but assuming it was sufficient, neither of you cancelled out MA. Why should you or Bevis, or anybody, want to frame me? That was the chief snag until a few days ago.”

***

It was awful the way they looked at Bevis and listened to me. I didn’t want any of it. I hadn’t from the beginning, and this, the end of it, was somehow the worst.

Kerry was sitting on the sofa beside Helen now. He held her hand. Dry-eyed, staring vacantly, she sagged against his side.

“Miriam,” Kerry said softly. “Is that it?”

I nodded. “I was that kind of fool. Or maybe not such a fool, because since childhood Miriam and I had lived like brother and sister. We’d been so close together, growing up together, that there was none of the newness and curiosity of a man suddenly coming upon a woman. Anyway, as far as I was concerned. After the trial Ursula said I ought to be spanked. I didn’t know what she meant, though just before that she’d told me. So had others, but it didn’t register. We loved each other, sure, like fond brother and sister. It was a stranger, Robin Magee, who opened my eyes. Suddenly I saw that Miriam loving me, wanting me, satisfied the equation.”

I laughed shortly, bitterly, through my nostrils. “That was the way love came to me first. Objectively—in effect, mathematically. That night I dreamed about her, and when I awoke I realized that all along Lily and every other woman I had known had only been substitutes for Miriam.”

Bevis stood up. Our eyes swirled to him, fixed on him. He clenched his fists and swayed.

“Don’t say anything,” his father warned him.

Bevis’ mouth went slack. He turned and bent his gangling body to straighten the overturned leather chair. We watched his awkward, uncoordinated movements. One foot of the heavy chair dropped on his toes. He cursed without passion, without feeling. Then he drew in a ragged sob and sank into the chair and lowered his head.

“Do you base your accusation on this nonsense?” Mr. Spencer said.

His face was suddenly sharp, calculating—the face of a shrewd business man who was weighing a deal.

“Perhaps it’s nonsense,” I told him. “It’s not real mathematics, but mathematics teaches you how to think straight. I could have worked it out another way, but because I’m a mathematician I used this method. It wasn’t meant to convince anybody but myself. That was enough. Bevis, loving Miriam, had powerful reason to want me out of the way. His name is the only one to cancel out MA, the only one that completely satisfies the equation. There’s a law called the economy of hypotheses. If you have a hypotheses that satisfies your case, you don’t go looking around for another hypotheses.”

“On paper,” Mr. Spencer said, fighting back now.

***

“It fits and nothing else does,” I said. “We can go back far to get to the beginning of it, probably to the time when we were high school kids and Bevis was in love with Miriam and Miriam with me and I, for a while, with Helen. Or start with Helen’s friendship with Lily and the inability of her father and brother to break it up. Or maybe it was the impact of Don Yard’s fist against your jaw, Mr. Spencer. Yard knocked you down, but it was Lily who was to blame. That was the ultimate outrage, the unendurable indignity to family pride. You could take it tight-lipped and hope to avoid scandal; but Bevis, brooding and intense, had to do something about it when Helen told him. Not kill Lily, of course. The motive, as I said, was weak—at that time. But he went to Lily and threatened her if she didn’t stop trying to make Helen a tramp like herself. Whatever he said frightened her so that she made a phone call to Don Yard for protection. She needn’t have worried. She would have been safe enough from Bevis if my homecoming had worked out a little differently.

“Bevis made considerable headway with Miriam while I was overseas. She liked him, and probably she would have married him if I hadn’t existed. I had a wife, I should have been out of the way, an obstacle removed, but it was obvious that the marriage couldn’t last. Sooner or later I would be free again. Miriam was waiting for that, Bevis thought, which was why she didn’t accept him. Maybe. My own opinion is that Miriam just didn’t care enough for him.

“He may have considered killing me when I returned. But he was no fool. To begin with, it would be too obvious. A pattern would be formed—a rival put out of the way. Even if he got away with murder, the truth would occur to Miriam or one of the others who knew how matters stood. And the chances were that my death would solve nothing, that she would remain faithful to my memory and not marry him or anybody. He himself would react that way to Miriam’s death, so why wouldn’t she to my death?

“No, there was no plan in advance. The idea must have come to him full-blown when I broke up the poker game and rushed to Lily’s bungalow. He had time to think of all its aspects while he drove his father home. It had everything; it was perfect. I’d be jailed or executed as a murderer. Miriam’s love for me would be destroyed. She’d accept him on the rebound. And the fact that it was Lily he had to kill in order to work it was a great deal in its favor.

“There was danger, delicate timing necessary, but he was a good gambler. Miriam was certainly worth any risk.

After he dropped his father off, he drove to Lily’s bungalow, beating me there with plenty of time to spare, entered through the back door, picked up a steak knife on the way through the kitchen plunged it into her heart, wiped the knife handle, drove on to Ursula’s house and was with Miriam when news of my arrest came. He couldn’t have known that Sheriff Dowie and then Kerry would go to the bungalow and find me there. But that hadn’t been necessary. There was enough circumstantial evidence against me without their arrival, just as there would be weeks later when he murdered Emil Schneider. But that was a bit of additional luck.”

“Luck! ” Bevis uttered a sound that was something like a laugh and something like a sob.

Helen’s face was pressed against Kerry’s shoulder. She slid it along his sleeve to face her brother. “Bevis, tell him he lies! Why don’t you tell him he lies?”

***

Again Bevis made that bitter, hopeless sound.

“Don’t say anything,” Mr. Spencer told him. He looked at me with that calculating expression he had assumed after the initial shock. “Is there anything more, Alec?”

“Not much,” I said. “He had luck then, but not later. I was acquitted. I was free in more ways than one. Free of Lily, free to marry Miriam. Bevis had outsmarted himself. And more bad luck—Schneider had seen him leave Lily’s bungalow. He needed money, and Bevis had to pay up. Then Schneider got scared. He knew that a man asks for death when he blackmails a killer. There was nothing left to hold him in West Amber, so he planned to get out of town as soon as possible. Then I came to him and he saw a chance to pick up more money before he left.

“I returned to the poker game to get the money. Bevis, of course, realized why I wanted five thousand dollars in cash. Only recently he had needed cash for the same purpose. He had to protect himself. Not only that, but he saw a chance to try again, to erect a frame around me once more. I didn’t leave the house until some time after the game broke up. Bevis was waiting at the side of Schneider’s porch. He saw my car pull up, saw me approach, and he called to Schneider who was in the house. Schneider, also having heard my car and thinking he heard me call his name, came out and died.”

“Nonsense!” Mr. Spencer said. “Complete and utter nonsense!” His narrow shoulders were straighten He managed to smile thinly to his son. “The fight has upset you, hasn’t it, Bevis? That’s why you’re acting like this. Alec comes here with a cock-and-bull story and not one scrap of real evidence.”

“God!” Bevis moaned. “Oh, God!” His head fell on his arms.

“I’m sorry,” I said, meaning it, sorrier for Mr. Spencer and Helen than I had ever been for myself. “There’s evidence and Bevis knows it. That rifle on the wall.”

They stared at it as if they had never seen it before. It hung where it had hung for years above the ornately carved mantle.

***

It was a Winchester bolt action rifle, a sweet job which could be used for woodchuck or deer, depending on the power of the cartridge. My .22 was no good for big game, and on the two or three occasions we had gone hunting together he had generously let me carry the rifle part of the time. I had never been lucky enough to spot a deer when I had it. For that matter, in all the years Bevis had had that rifle he had not been able to bring down a deer either. The rifle had never killed anything bigger than a raccoon until it had killed a man.

“Bevis, you fool!” Mr. Spencer shouted.

“What else could he do?” I said. “I more or less expected to find the rifle here because I expected Bevis to show sense. He had no chance to prepare a weapon when he decided to kill Schneider. He had to work fast before I reached Schneider’s house, so he grabbed the only weapon at hand. Naturally, after he’d fled from me, he’d watch the house from the woods to see what I would do. He saw Kerry come up less than a minute behind me. At once the rifle became a burden, a problem. He couldn’t leave it in the brush and later tell the police I had stolen it from him, for he couldn’t know that Kerry hadn’t been dose enough to see that I hadn’t shot Schneider. So he had to take it away with him.

“Then he learned that I had fled and that Kerry had seen nothing. It was too late to take the rifle to the scene of the crime and leave it there. He was stuck with it. For many years it had hung on that wall, a conspicuous and treasured possession. Questions might be asked if it disappeared the night Schneider was shot dead. He could not afford questions, anything at all pointing to him. And they were unnecessary. Replace it where it had always been, unnoticed by its very presence, and later, when the hunting season started, take it out and say he had lost it. The police were devoting themselves exclusively to the hunt for me. There was no hurry. And then suddenly he saw me cutting across the lawn and knew what I was after. This would be it.”

***

The three of them on the sofa kept staring at the rifle.

“And if Bevis hadn’t left it here?” Mr. Spencer muttered.

“I would have tried something else,” I said. “I suppose you and Helen hate me now, but I didn’t do this to Bevis. He did it to himself and to me and to you. I was prepared to be ruthless. I would have got him off alone, or with somebody’s help, and forced a confession out of him. You see how he’s gone to pieces. The intense kind like Bevis easily does. It wouldn’t have been nice, but nothing about this was nice.”

Mr. Spencer took a long time getting to his feet. He didn’t look at his son or at anybody. His slight body had shrunken to loose skin over a framework of bones. He shuffled past me and out of the room.

Bevis appeared to be asleep with his head on the arm of the leather chair. He wasn’t. Helen’s face, pressed against Kerry’s chest, wasn’t visible either. Gently Kerry stroked her hair.

“It’s not pleasant for you,” Kerry said to me. “You’d better go.”

“The rifle—”

“I’ll see that it’s here when the police come.”

I said: “I’m sorry.”

“Sure. We all are.”

I nodded and went out of the room. I didn’t have to call the police. Oliver Spencer was in the hall, crouched over the telephone table. “Is this the state police?” he said. “My son—”

I kept going. It was a clear, mild night. The high, lopsided moon had the sky to itself. I pulled my flashlight out of my hip pocket and walked home to Miriam.

#### THE END 


\vspace{2\nbs}
\ChapterDeco[c1]{\decoglyph{e9665}}
\clearpage
\thispagestyle{empty}

%end chapter loop

\scenebreak
\scenebreak
{\centering\textsc{the end}\par}

\clearpage

\null

\centering\textsc{www.TalesofMurder.com}\par

\vspace*{10\nbs}

%\centering\InlineImage[0, 3em]{/home/darkstar/dox/working-files/LaTeX/atticus.jpg}

TALES OF MURDER PRESS, LLC

\null

\scshape{675 TOWN CENTER BLVD
BLDG 1A STE 200 PMB 530
GARLAND, TEXAS 75040}

\null

\textit{atticus@talesofmurder.com}
\vfill


\end{document}

