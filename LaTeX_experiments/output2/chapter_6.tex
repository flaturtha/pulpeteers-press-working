% !TeX TS-program = LuaLaTeX
% !TeX encoding = UTF-8
\documentclass{novel}
%%% METADATA (FILE DATA):
\SetTitle{TITLE}
\SetAuthor{AUTHOR}
\SetPDFX{X-1a:2001}
\SetTrimSize{4.25in}{6.875in}
\SetMediaSize{4.5in}{7.12in}
\SetMargins{0.5in}{0.5in}{0.5in}{0.7in}
\SetParentFont{Libertinus Serif}
\SetFontSize{9.5pt}
\SetHeadFootStyle{5}
\SetHeadJump{1.5}
\SetFootJump{1.5}
\SetLooseHead{50}
\SetEmblems{}{} % Default blanks.
\SetHeadFont[\parentfontfeatures,Letters=SmallCaps,Scale=0.92]{\parentfontname}
\SetPageNumberStyle{\thepage}
\SetVersoHeadText{\theAuthor}
\SetRectoHeadText{\theTitle}
%%% CHAPTERS:
\SetChapterStartStyle{footer} % Equivalent to empty, when style has no footer.
\SetChapterStartHeight{10}
\SetChapterFont[Numbers=Lining,Scale=1.6]{\parentfontname}
\SetSubchFont[Numbers=Lining,Scale=1.2]{\parentfontname}
\SetScenebreakIndent{false}
%%% BEGIN DOCUMENT:
\begin{document}
\frontmatter
\thispagestyle{empty}
% Half-Title Page.
\begin{parascale}[2]
\vspace*{3\nbs}
\centering\charscale[0.75]{TITLE }\par
\centering\charscale[0.75]{TITLE LINE 2}\par
\centering{TITLE LINE 3}\par
\end{parascale}
\clearpage
\thispagestyle{empty}
\null % Necessary for blank page.
% Alternatively, List of Books.
\clearpage
\thispagestyle{empty}
% Title Page.
\begin{parascale}[4]
\centering\charscale[0.75]{TITLE }\par
\centering\charscale[0.75]{TITLE LINE 2}\par
\centering{TITLE LINE 3}\par
\end{parascale}
\vspace*{2\nbs}

\begin{parascale}[1]
\centering\textit{SERIES}\par
\vspace*{3\nbs}
\charscale[2]{AUTHOR}\par
\end{parascale}
\vfill
\begin{parascale}[1]
% \centering\InlineImage[0, 3em]{/home/darkstar/dox/working-files/LaTeX/atticus.jpg}

A Tales of Murder Press, LLC\par
\textit{GENRE} Novel\par
\end{parascale}
\clearpage
\thispagestyle{empty}
% Copyright Page.
\null\vfill
\allsmcp{First edition} PUBLICATION_DATE\par
\null\null
\allsmcp{ISBN}\par
\null\null
\vfill
\begin{adjustwidth}{3em}{3em}
\textit{This novel is in the public domain.} Certain \mbox{elements} in this edition are Copyright © COPYRIGHT_DATE Tales of Murder Press, LLC
\end{adjustwidth}
\clearpage
\thispagestyle{empty}
\clearpage % because ToC must start recto
\thispagestyle{empty}
\begin{toc}[0.5]{0em}
{\centering\charscale[1.25]{Contents}\par}
\null

%some kind of loop through chapters for TOC
\tocitem*[1]{CHAPTER_TITLE}{STARTING_PAGE}
\end{toc}
%end loop
\clearpage

\mainmatter
\cleartorecto
\thispagestyle{empty}

%some kind of loop through all chapters
\begin{ChapterStart}
\vspace{3\nbs}
\ChapterSubtitle[l]{Chapter 6}
\ChapterTitle[l]{## For Me}
\end{ChapterStart}
\FirstLine{\noindent ### Chapter 6

## For Me

Ursula wore a black-and-white print dress and a black cartwheel hat. She looked chic, as the fashion magazines put it, but not so obtrusively smart that the rather dowdy women jurors would resent her. There were tired and frightened lines at her eyes and mouth, but she strode to the witness stand like a man in a hurry. Her deep voice, though tense and somewhat subdued, filled the courtroom. She spoke of the early life of her kid brother.

I couldn’t see what that had to do with the case, but Robin Magee considered it important. Now that the witnesses were his, Magee assumed a different character. He was genial, sympathetic, a man to receive confidences.

“Alec was a shy and sensitive boy when he came to live with me,” Ursula said. “He never got over it. He became even more indrawn, if anything. He would have long spells of brooding, and suddenly, without warning, he would flare up into fits of temper.”

I stared at her, but she was careful not to look in my direction. She was lying about me. I’d been no different than any other boy, except perhaps that I’d been somewhat more even-tempered. The flare-ups came later, the night I’d returned home. Why was she trying to tear me down?

“Was he unfriendly?” Magee asked. “Not at all. Alec was—and is—the sweetest and most devoted boy I know. He has many friends. Everybody likes him. But he’s—well, he takes everything extremely seriously.”

They kicked my character around some more before they came to the point, which was Lily.

“One day Alec, brought that woman home,” Ursula said. “It happened so quickly, without preparation. She was beautiful, I suppose, but in a hard, flashy way. She claimed to be twenty- two; I’d put her age closer to thirty. But Alec was very much in love with her and was going overseas in a very few days, so Miriam and I accepted her as heartily as we could. Even when Alec asked us to let her live with us until he returned we put on a good face. There wasn’t anything we wouldn’t do for him.”

“How did you and Miriam get along with Mrs. Linn?”

“There was little friction, though she seldom helped with the housework. She went her way and we went ours.”

“Do you recall the conversation you had with Mrs. Linn on the morning of July 5th?”

“That was the morning I asked her to leave the house. Her conduct had become outrageous.”

“Please explain that.”

“The night before Miriam saw her disgustingly drunk in public with strange men. That was too much. I simply could not let her remain in my house.”

“What happened when Lieutenant Linn returned home on July 21st?”

“I didn’t break the news to him that I had asked his wife to leave the house. I told you how seriously he took everything. On top of that, he had been sent home because his nerves were very bad.”

***

HACKETT objected that she wasn’t an authority on nerves. The judge sustained him.

“Anyway,” Ursula said, “I kept putting off telling him. You must understand how I felt.”

“I’m sure we all do, Mrs. Hennessey,” Magee said sympathetically. He was through with her.

Ursula’s testimony hadn’t contributed a thing to show that I was innocent. It wasn’t meant to. The idea was to compress my character into a mold. I was one person to the prosecution and another to the defense. Take your pick—a coldly calculating killer or a sensitive, high-strung lad who had suddenly gone haywire through no fault of his own.

Hackett had his turn at the mold. “I understand, Mrs. Hennessey, that since childhood your brother has been something of a mathematical genius?”

“Alec is clever.”

“And that at his last two years at college he was the intercollegiate chess champion?”

“Yes.”

“And that he’s a poker player of extraordinary skill?”

Ursula frowned. “He is very good.”

“In short, his various laurels and honors show that in that science and in those games of skill which require a clear and exact mentality, a trained ability to think ahead and plan concisely and without emotion, your brother excels?”

Ursula opened her mouth, but Magee beat her to the punch. “Your honor, that’s not a question.”

“I phrased it as such,” Hackett retorted. “Mrs. Hennessey, as the sister with whom the defendant lived and who saw him through school, is certainly an authority on his scholastic abilities.”

They wrangled in front of the bench, but I didn’t hear any more of it. I turned furiously to George Winkler.

“You lied to me, both of you. Magee isn’t defending me. He’s trying to show that I wasn’t psychologically responsible, and Hackett is trying to show that I was. I haven’t yet heard Magee deny that I’ve murdered Lily.”

“He’s working every possible angle,” George whispered. “This is only one of them. Give him a chance.”

Ursula was stepping down from the stand. I subsided—at least to the extent that I kept my mouth shut.

***

He said that his name was Ogden Garback and that he worked in his father’s garage in West Amber. He was nineteen, a nice looking lad, slim-hipped and broad-shouldered. He grinned nervously and told Magee that he wasn’t in the Army because of a bad kidney.

“It was just before Christmas,” Garback said. “Last Christmas. It was after eight and I was locking the pumps when Mr. Schneider pulled up for gas and there was a woman with him who wasn’t Mrs. Schneider.”

“Emil Schneider who is employed at the West Amber National Bank?” Magee asked.

“Yes, sir. He gets all his gas at my father’s place.”

“Who was the woman in the car?”

“Mrs. Linn. Bill Beaty was hanging around waiting for me to close up so we could bowl a few games, and when they drove away Bill told me who she was.”

“When did you see Mrs. Linn again?”

“One afternoon just after Christmas. The garage is only a quarter of a mile from the station and she came walking up in the snow and said there was no taxi waiting and would I drive her home to Mandolin Hill. Pop told me to go ahead and I took the car, but I clean forgot to collect the dollar for driving her.”

“How did you do that?”

“I just forgot. I guess it was because she was talking so much, asking me all about myself. When I got back to the garage I remembered about the dollar and called her up and she said come around for it at nine. She was waiting out on the road when I got there. She said did I want to spend the dollar drinking beer with her and I said sure and we drove to Mike’s on New Hollow Road. Bill Beaty was there with a couple of fellas, and he came over to our table and said hello to her and we took turns dancing with her. Then I went out to the—the—” he looked out in embarrassment at the many eyes focused on him, “—well, I went out for a few minutes, and when I came back Mrs. Linn and Bill were gone.”

“Is the Bill you refer to William Beaty, now in the Navy?”

“Yes, sir.”

“When did you see Mrs. Linn again?”

“A couple of days later she called me up at the garage and said she was sorry she’d walked out on me and would I pick her up on the road at nine. So I did.”

“Where did you go with her?”

“No place. We drove for a while and then parked.”

“Did you make love to her?”

The kid flushed. “She let me kiss her a few times, then she said we should go home.”

“Did she say why?”

Garback looked as if he were searching for a hole to jump into. “She said I I was a nice boy and too young for her and I shouldn’t see her any more.”

Magee didn’t like that. He didn’t want to hear any good about Lily. “And you didn’t see her again?”

“I called her up a few times, but she always turned me down.”

“Did she continue to go out with William Beaty, if you know?”

“Yes, sir. Lots. I saw them together in Bill’s car.”

“What were his relations with her, if you know?”

***

The kid wiped his mouth. It was plain that he had had it bad for Lily, and I could understand that and how he must have felt when she had dropped him after a kiss or two and how it must have hit him to have known all the time that his friend Bill was having better luck with her. With my wife. That had been my wife they were speaking about.

“Well, sir,” Garback said, “Bill always told me every time—”

“No, give me your own observations.”

“Well, sir, you don’t think I was around to see—” Garback flushed. The boisterous laughter which swept the courtroom had to be silenced by the judge’s gavel. “One Saturday Bill stopped off at the garage for gas. Mrs. Linn was with him. While I was filling the tank he told me they were driving to New York and hoped they wouldn’t get back till next day. And next day, Sunday, they passed around twelve o’clock, coming back. They’d been together all the time, and they’d spent the night in a tourist cabin in—”

I was on my feet, trembling, pulling my arm out of George’s grip. “What has her personal life to do with this trial?” I shouted. “She’s dead. Why not let her alone?”

The guards shoved me down. Hackett said: “Your honor. I agree with the defendant. I’ve been giving Magee plenty of leeway, but I fail to see—”

“You’ll see if you give me a chance,” Magee told him.

They threw legal jargon at each other. I chewed on the knuckles of my clenched hands and felt all those eyes on my back.

***

Magee must have won his point because Hackett shrugged and walked away. The judge said to me “The defendant will please restrain himself.” He said it kindly. He was sorry for me; so was everybody else in the room. Not because I was being tried for murder, but because I had been married to a woman like Lily.

Magee was back at the witness stand. “Did Emil Schneider ever speak to you about Mrs. Linn?”

“Yes, sir.”

“What did he say?”

“Hey!” Hackett protested. Then he added: “Irrelevant and immaterial.” Magee turned to the bench. “Will the court take it subject to correction?” The judge looked interested. He nodded. Magee repeated the question.

“Well,” Garback said, “he came to the garage one afternoon and said he’d break my neck if I didn’t keep away from Mrs. Linn.”

“When was that?”

“About the beginning of February.”

I told him he was backing up the wrong alley, but he stayed sore and asked what did I know about Bill Beaty going around with her. I said to go ask Bill and he did and Bill told him to jump into a lake. Then a few weeks later Bill was drafted.”

“And that left the field clear to Schneider?”

“Move to strike out,” Hackett growled.

“I withdraw the question,” Magee said, grinning.

***

Helen Spencer had been the prettiest girl in high school. She had had a heart-shaped face and red cheeks and warm gray eyes and wavy honey-colored hair falling to her shoulders. She still had all of that, but she was trying to change it—to look worldly and sophisticated by aid of an up-swept hairdo and a long clinging dress and sleek make-up and scarlet fingernails. They didn’t do the trick. She was still a peaches-and-cream girl.

“Why, yes,” Helen said, “I became rather friendly with Lily Linn. I’m a friend of Miriam Hennessey and came to the house rather often, and Lily seemed to like me. She knew everything about the latest styles and took me shopping with her and helped me fix my hair and all that.”

“Did you go out on dates with her?” Magee asked.

Helen looked over his head. I turned and everybody else who was facing her turned. Oliver Spencer, her father, stood in the back of the courtroom with his hat in his hand. When he realized that he was suddenly the focal point of attention, he stopped chewing his lower lip and assumed a deadpan.

“Last month two men came up from New York to see her,” Helen said slowly. “Lily told me they were old friends and asked me to fill out the second couple. We went to a roadhouse in Trevan.”

“When was this?”

“On the Fourth of July. Lily got very drunk and started acting disgracefully, right out in public, letting this man Don who was with her make love to her and everything.”

“Everything?”

“Well, paw her. Even though he had been her husband—”

“Husband!” Magee exclaimed.

“Her first husband—the one she’d divorced.”

I found that I was sitting forward with my mouth hanging open. I closed it.

Magee glared at her. “Why didn’t you tell me when I spoke to you last week that Mrs. Linn had been married before?”

“Didn’t I?” Helen said serenely. “I thought I did when I told you about that night at the roadhouse.”

“You merely said that they were two men from New York.”

The judge rapped impatiently. “Please don’t argue with your witness, counselor.”

George Winkler growled in my ear: “You didn’t mention to us that Lily had been married and divorced. We could have brought her first husband up here and perhaps have got something out of him. Now it’s too late.”

“This is the first time I heard of him,” I said.

There was a great deal about Lily I hadn’t known and still didn’t. I yanked my attention back to Helen’s testimony.

“I never heard his second name,” she was saying. “Lily called him Don.”

“Short for Donald?”

“It may be. She just called him Don.”

“All right, go on.”

***

Helen glanced again at her father in the rear of the room and then quickly down at her knees. “Lily was very drunk and so was Don. He asked her to go away with him. She laughed and said they’d never get along together, but she didn’t stop him pawing her. Then Don said: ‘I’ll settle for a few days in Miami.’ Lily said: ‘What about that she-cat you’re living with? She’ll hardly approve.’ Don said he didn’t care what she thought. If she didn’t like it, she could—”

“Wait?” Magee broke in. “Who is this woman you’re talking about?” Helen raised her head and lowered it. “They didn’t mention her name. Then Lily told Don she couldn’t leave West Amber for even a day because she’d just got word from Alec that he was coming home. From now, she said, she’d be a good girl. And Don said: ‘And make a home for this hick and raise babies.’ Lily made a sour face. She said: ‘For my part, I’d rather have his insurance than him.’ ”

This was the worst kick of all, right in the pit of the stomach. All along I’d assumed that whatever she had done was out of boredom. But those others hadn’t even been substitutes for me. She’d preferred that I be killed in action so that she could collect my insurance.

I wanted to crawl away somewhere and lie alone with my face in my arms.

Helen was telling the rest of it. “Don and the other man—his name was Walter—laughed drunkenly, and Don said to Lily that she’d gotten herself stuck good and proper. That made Lily angry, but I’m sure that if she hadn’t been very drunk she wouldn’t have said what she did in front of me. She said how was she stuck if for two years she’d been living on an officer’s pay and in a fine home where she didn’t have to pay board? She said maybe she’d like Alec when he came home. She didn’t know; she’d even forgotten what he looked like. It would be stupid to walk out on him now. If she decided she didn’t want him when he came back, she was sure his sister, who had money, would buy her off to take her hooks out of him.”

“Are these Mrs. Linn’s exact words?”

“As far as I can remember. Lily went on like that. She said she was going to be a very good girl, at least till Alec came home, and that would be very soon. Then my brother and Miriam Hennessey came in and saw how Lily was carrying on with Don. Bevis insisted on taking me right home, and I was glad of it.”

“Did you see Mrs. Linn after that?”

“She called me up the next morning. She was sober and knew I was a friend of Alec and the family and realized she’d made a mistake blabbing in front of me. She asked me to come over so that she could explain.”

“Was she still living in Mrs. Hennessey’s house?”

“No, she’d left that morning and it was the bungalow on James Street. When I got there, Mr. Schneider was arguing with her.”

“What about?”

“They stopped when I came in, but I heard a little when I went up to the door. Lily told him that he mustn’t see her again because her husband was coming home. Mr. Schneider was begging her to go away with him.”

“Anything else?”

“Then I knocked and Mr. Schneider was very embarrassed when I came in. He left right away. Lily told me to forget what she’d said last night in the roadhouse. She had only been pulling Don’s leg, she said. I told her the truth had come out when she’d been drunk, and I walked out.”

“And that was the last time you saw her?”

Helen looked down at her crossed knees. “Yes,” she said.

***

Emil Schneider was limp and scared when he returned to the witness stand. Magee was merciless.

“You are aware that you are under oath?” Magee said crisply.

“Yes.”

“Did your wife leave you because of Lily Linn?”

“Yes.”

“When you phoned Mrs. Linn on the night of July 21st, you did not have real estate on your mind, did you?”

“I asked her if I could come over.”

“What did she say?”

“She said no. She said she’d made up her mind not to see other men until her husband came home. She said there was enough talk about her already.”

“Do you know whether she was aware that her husband had returned that evening?”

“She didn’t say anything about it.” Magee jabbed a finger at him. “I’ll ask the question I started to ask yesterday. Were you intimate with Mrs. Linn?”

“I—it wasn’t—”

“Answer me!”

There were no bones in Schneider’s loose body. “Only a few times.”

“In short, yes.”

“Yes,” Schneider whispered.

***

During the recess in my daytime cell in back of the courthouse, I said angrily: “So you’re not going to try to prove I’m innocent? You’re double crossing me!”

George Winkler sighed and moved in on me to apply soft-soap, but Magee waved him silent. He handed me his flask of very good rye. I took a deep slug and so did Magee, but George shook his head. Then Magee said: “All right, my boy, we’ll do whatever you say. What should we do?”

“You’re the lawyers,” I muttered.

“We’re not supermen. You heard the D. A.’s case. He showed that you were aware of your wife’s unfaithfulness—”

“I wasn’t. Not quite. You were the one who spent all morning giving me the motive to kill her.”

“—and that you rushed out of the house and that fifteen minutes after she was known to have been alive you were found sitting next to her murdered body and your prints were on the knife.”

“I explained about the fingerprints.”

“It was an explanation, the only one you could possibly make. Do you think the jury will believe you? Put yourself in their place. Would you have a shadow of doubt as to your guilt?”

I was boxed in. I groped for a way out. “The murderer is this man Don, her former husband, to whom she refused to go back.”

Derisively Magee lifted the corners of his mouth. “Yesterday you said it was Schneider.”

“Perhaps it is Schneider. He was madly in love with her and she wouldn’t go away with him.”

“Why leave out Ogden Garback whom she kissed and then dropped?” Magee shook his head. “The D. A. has made out an excellent circumstantial case. We can’t do a thing to counter it. Can we?”

I wanted more of Magee’s rye, but I didn’t ask for it. I set fire to a cigarette.

George said: “But Magee will get you off just the same. You ought to appreciate how clever he has been. Helen Spencer’s testimony was wonderful. Kerry Nugent’s will be even better.”

I ground the cigarette under my heel. “The unwritten law,” I said bitterly. “Temporary insanity. Extenuating circumstances. But not that I’m innocent. Neither of you believes that I didn’t do it.”

“Of course we do, my boy,” Magee lied suavely.

“Well, I’m going to tell the jury the truth when I’m on the stand.”

There was a silence. Then George said softly: “We’re not going to put you on the stand, Alec.”

“Why not? Afraid I’ll tell them I’m innocent?”

Magee handed me another of his pats on the shoulder. “You can’t help yourself and you might harm yourself. Everything is going line now. Why run the risk that you’ll antagonize the jury? You’ll be too pugnacious.”

“I don’t care.”

“My boy, do you want to live? No, you’ll live anyway; they won’t send you to the chair. Do you want to spend from twenty years to life in jail?”

The walls pressed so close that I couldn’t breathe. Through a window at my side I could see a patch of cloudless sky. I wished I were up there, not cooped up in a bomber, but flying free and alone at ten thousand feet in a single-seater.

“Have it your way,” I said listlessly.

***

Kerry Nugent made a splendid figure on the witness stand. He sat at ease in his natty uniform with the impressive double string of ribbons on his deep chest, but his rugged face was stiff with earnestness. He could utter sheer nonsense, and whatever he said would carry weight. I could see it in the faces of the jury, sense it among the spectators. Kerry wouldn’t let them convict me.

He told of our boyhood together. He made it sound idealistic, and maybe it was. A couple of clean, healthy, decent kids, devoted to their family—in my case, to my sister—and when they grew up responding at once to the call of duty when their nation was in danger.

“Alec and I went together to enlist in the Air Force,” Kerry said. “That was in our senior year at college.”

“What kind of boy was he?” Magee asked.

“Brighter than most. At school he took honors right and left, especially in math. I went out for athletics, but he was the studious type. High-strung. Thought a lot about everything. Used to he awake nights dreaming up new chess openings. Temperamental. Sometimes he’d almost bite my head off for no reason. Though I don’t mean that he wasn’t one swelI guy.”

“You two enlisted as soon as you graduated from college?”

“A couple of months before, but it wasn’t till fall of that year before we got our orders. Alec was a natural for a navigator. After we received our wings, we were put on B-17’s. We were about to be shipped to the European Theater when I was transferred to the new B-29’s. I pulled wires to get Alec assigned to my ship, so we stayed together. The B-29’s weren’t flying yet. They were just coming off the assembly lines in Kansas City and wee had to train for them and we—”

“Tell us about the weekend you and Lieutenant Linn spent in New York before flying to India.”

“We got ten days’ leave and spent the first few days of it at home. There was a party in Greenwich Village Alec and I were invited to. We decided to have a last fling and went down to New York by train and registered overnight at a New York hotel. At that party we met Lily Yard.”

“Yard?” Magee looked startled. “Don Yard? Was Yard her maiden or her married name?”

***

“I wouldn’t know. That’s the name she gave us. I saw her first and made a power dive for her. Alec was right behind me. She was worth rushing—tall and slim and platinum hair and a build you’d pin up in your locker any day. Alec won out. That was okay by me. There were plenty of other—” Kerry wet his lips. “Anyway, Alec didn’t go back to West Amber with me next day. He was at the Municipal Building with Lily getting a marriage license. That quick. But you know how it is. A man is going into combat and the odds are he won’t come back, especially if he’s in a bomber, so why not grab what you can while you can get it? I don’t mean he wasn’t crazy in love with her besides. As for Lily—well, you heard about her. She saw a chance to get herself an officer’s allotment and maybe his insurance. I saw her once more, when he brought her home. Then a couple of days later we flew to Africa and from there to India.”

“Did he talk to you about her?”

“Nothing else but. He called her the Lily. The Lily. She looked like one—the tall, white kind anyway—but that’s not all he meant. Pure as a Lily. Virginal. You know.”

“I know. We’ve heard a great deal of her purity this morning.” Magee sent a look at the jury and the jurors made their faces grimly angry. “When Lieutenant Linn reached his base in India, did he hear often from his wife?”

“That just it. He didn’t. You can’t imagine what letters meant to us unless you’ve been in it yourself, especially letters from the woman you love. She didn’t write often. You could hear Alec’s heart crack when he received a fistful of mail and there’d be nothing from her. She was the only one he wanted to hear from. Then after a year the letters she did write became worse than none at all. The other day you heard his answers to some of her letters. Those were just his answers. You should have seen the letters she wrote him.”

“Lieutenant Linn showed them to you?”

“Sure. We shared all our letters, no matter how intimate they were. You share death in the sky, so you share the rest of yourself on the ground. I don’t know why women do certain things. Maybe they’re bored, home alone. They blame the war and take it out on you. It wasn’t just Lily. I’ve seen wives and sweethearts do it to other men. Because you need them so much and think about them so much, they have a knife in you, and they take pleasure in twisting it. Not all women, not many, but more than you’d think. Lily was one—anyway, after the first year. It was a little thing she had to do, write more often and lie in her letters if she didn’t feel affectionate. But when Lily wrote, she twisted the knife all she could. Nagging about money, for instance. It got so bad that Alec went into a couple of those big crap games we have at the club, but crap is straight gambling and instead of making more money for Lily he dropped a couple of hundred bucks. He was even ready to play poker, but he didn’t.”

“It has been testified by Mrs. Hennessey that Lieutenant Linn is an exceptionally skillful poker player. If he wanted money, couldn’t he have made it that way?”

***

“Sure. The trouble is that most men play with a hunch and a prayer. They haven’t any idea of percentages or the fine points. The way Alec plays it, it’s not gambling. He got into a couple of small games and then stopped. It wasn’t fun for him. It was like taking candy from children.”

“Tell us more about Mrs. Linn’s letters.”

“It wasn’t only the nagging for money. She complained about everything. She hinted that there were other men in her life without coming right out and saying that she was sleeping—that she was actually being unfaithful to him.”

“How did Lieutenant Linn take it?”

“Bad. Lord knows all our nerves were jumpy enough, what with monsoons and heat and snakes and scorpions and bugs and hardly any white women and having your closest friends die all around you and being afraid all the time till it was a relief being up in the air and looking death right in the face. So you see, Alec had enough on his mind without worrying about his wife. He was navigating one of those huge crates over one of the toughest runs in the world. I never know how navigators do it. We pilots get the glamour. But we just have to be men without nerves and quick in a pinch for maybe a minute or two at a time. We have co-pilots and robots to relieve us. But a navigator sweats blood every second of a mission. That’s why you’ll find the high-strung, very bright boys are the navigators. Like Alec. They’ve got to keep themselves at a high pitch. And when on top of that they have something nagging on their minds all the time the way Alec had—well, it was plain murder on him. He got the jitters. He didn’t eat or sleep enough. Then we lost the ship on our twenty- third mission.”

“How did that happen?”

“One of those things. We were knocked about by flak over the target and then six Nip fighters hit us. By the time we chased them off they’d shot up the radio and almost got Alec. To add to our troubles, the weather turned to pea soup below us and we couldn’t see what was downstairs. And after a while Alec told me we were lost.”

“You were responsible for all the men on the plane, weren’t you, Captain.”

“I was the skipper.”

“Would you say, as an experienced skipper, knowing the men in your crew and responsible for them, that Lieutenant Linn’s ragged nerves caused him to make a serious mistake?”

“No. He—” Kerry saw Magee raise his eyebrows and he sat back. “Those things happen lots of times,” he said quietly. “A navigator isn’t God. He can be pretty near perfect—and Alec was tops—but worrying about a wife and being kicked around by flak and having your radio gone and being a thousand miles from the base—well, you see it, don’t you? We were flying at twenty thousand feet and couldn’t see our outboard engines. I tried to go below the weather, and we got a bad scare. Our altimeter showed fifteen thousand feet, but our positive altimeter only two thousand. We weren’t over water at all, as we should have been. There were mountains below us, at least thirteen thousand feet high, and if you know anything about—”

***

Magee yanked Kerry back to land.

“What happened finally?”

“Alec got us through. Anyway, , within a couple of hundred miles from our base. We were out of gas by then, so we had to hit the silk.”

“You were hurt when you jumped, Captain?”

“A couple of ribs bashed in. One of the men was killed, though. Sergeant Bilkin, a swell—”

“What happened to Lieutenant Linn?”

“Nothing physically. But he cracked wide open. He said it was his fault he lost the ship and Bilkin got killed. If it was anybody’s fault, I’d blame Lily Linn. The flight surgeon grounded him, of course.”

“And Lieutenant Linn was discharged as a psychoneurotic,” Magee said triumphantly.

“No. They don’t discharge a highly experienced officer so easily from the Air Force, especially if he’s so far away. He was assigned to ground duty.”

Magee frowned at him in annoyance.

“But soon after, he was given an honorable discharge, wasn’t he, Captain?”

“Six weeks later. The war in Europe had ended and they were being easier on discharges, and just about that time our entire wing of B-29’s was transferred to the Marianas. The shift was what really got Alec his discharge. The C.O. didn’t think it worthwhile sending a case like Alec to the Pacific.”

“So he sent him home to his wife?”

“That’s right, his wife.”

“But there’s no doubt that Lieutenant Linn was mentally hurt?” Magee persisted. “You were his skipper and his closest friend. You lived with him and faced death with him. How serious was the psychological harm—” Hackett protested vehemently. “Your honor, Captain Nugent is a pilot, certainly not a qualified psychiatrist.”

“Do you deny his qualification to testify that Lieutenant Linn was grounded for being psychoneurotic?” Magee retorted.

Hackett stood firm. “Your honor, why hasn’t the defense brought in qualified psychiatrists, to examine the defendant? Because Magee hasn’t dared to. He knows that the defendant is perfectly normal.”

“Sustained,” the judge said mildly. “Proceed.”

The words on that side of the room went on. Kerry told how Miriam and Ursula had asked him to go after me and how he had found me in Lily’s bungalow. After that Hackett had a go at him, but it didn’t take long and I was no longer listening. I felt as if I had been run through a wringer and tossed into a damp cellar where I would never regain my shape.

Then I heard Magee speaking loud and clear through the hush. “The defense rests.”



\vspace{2\nbs}
\ChapterDeco[c1]{\decoglyph{e9665}}
\clearpage
\thispagestyle{empty}

%end chapter loop

\scenebreak
\scenebreak
{\centering\textsc{the end}\par}

\clearpage

\null

\centering\textsc{www.TalesofMurder.com}\par

\vspace*{10\nbs}

%\centering\InlineImage[0, 3em]{/home/darkstar/dox/working-files/LaTeX/atticus.jpg}

TALES OF MURDER PRESS, LLC

\null

\scshape{675 TOWN CENTER BLVD
BLDG 1A STE 200 PMB 530
GARLAND, TEXAS 75040}

\null

\textit{atticus@talesofmurder.com}
\vfill


\end{document}

