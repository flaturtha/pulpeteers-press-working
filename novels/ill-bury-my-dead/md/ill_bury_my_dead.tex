% !TeX TS-program = LuaLaTeX
% !TeX encoding = UTF-8
\documentclass{novel}
%%% METADATA (FILE DATA):
\SetTitle{I'll Bury My Dead}
\SetAuthor{James Hadley Chase}
\SetPDFX{X-1a:2001}
\SetTrimSize{4.25in}{6.875in}
\SetMediaSize{4.5in}{7.12in}
\SetMargins{0.5in}{0.5in}{0.5in}{0.7in}
\SetParentFont{Libertinus Serif}
\SetFontSize{9.5pt}
\SetHeadFootStyle{5}
\SetHeadJump{1.5}
\SetFootJump{1.5}
\SetLooseHead{50}
\SetEmblems{}{} % Default blanks.
\SetHeadFont[\parentfontfeatures,Letters=SmallCaps,Scale=0.92]{\parentfontname}
\SetPageNumberStyle{\thepage}
\SetVersoHeadText{\theAuthor}
\SetRectoHeadText{\theTitle}
%%% CHAPTERS:
\SetChapterStartStyle{footer} % Equivalent to empty, when style has no footer.
\SetChapterStartHeight{10}
\SetChapterFont[Numbers=Lining,Scale=1.6]{\parentfontname}
\SetSubchFont[Numbers=Lining,Scale=1.2]{\parentfontname}
\SetScenebreakIndent{false}
%%% BEGIN DOCUMENT:
\begin{document}
\frontmatter
\thispagestyle{empty}
% Half-Title Page.
\begin{parascale}[2]
\vspace*{3\nbs}
\centering\charscale[0.75]{I'll Bury My Dead }\par
\centering\charscale[0.75]{I'll Bury My Dead LINE 2}\par
\centering{I'll Bury My Dead LINE 3}\par
\end{parascale}
\clearpage
\thispagestyle{empty}
\null % Necessary for blank page.
% Alternatively, List of Books.
\clearpage
\thispagestyle{empty}
% Title Page.
\begin{parascale}[4]
\centering\charscale[0.75]{I'll Bury My Dead }\par
\centering\charscale[0.75]{I'll Bury My Dead LINE 2}\par
\centering{I'll Bury My Dead LINE 3}\par
\end{parascale}
\vspace*{2\nbs}

\begin{parascale}[1]
\centering\textit{SERIES}\par
\vspace*{3\nbs}
\charscale[2]{James Hadley Chase}\par
\end{parascale}
\vfill
\begin{parascale}[1]
% \centering\InlineImage[0, 3em]{/home/darkstar/dox/working-files/LaTeX/atticus.jpg}

A Tales of Murder Press, LLC\par
\textit{Noir} Novel\par
\end{parascale}
\clearpage
\thispagestyle{empty}
% Copyright Page.
\null\vfill
\allsmcp{First edition} First published in 1958 by M. S. Mill. Second edition published in 1959 by Bantam Books. by Tales of Murder Press, LLC\par
\null\null
\allsmcp{ISBN}\par
\null\null
\vfill
\begin{adjustwidth}{3em}{3em}
\textit{This novel is in the public domain.} Certain \mbox{elements} in this edition are Copyright © 2024 Tales of Murder Press, LLC
\end{adjustwidth}
\clearpage
\thispagestyle{empty}
\clearpage % because ToC must start recto
\thispagestyle{empty}
\begin{toc}[0.5]{0em}
{\centering\charscale[1.25]{Contents}\par}
\null

\tocitem*[1]{1}{1}
\tocitem*[2]{2}{2}
\tocitem*[3]{3}{3}
\tocitem*[4]{4}{4}
\tocitem*[5]{5}{5}
\tocitem*[6]{6}{6}
\tocitem*[7]{7}{7}


\end{toc}
\clearpage

\mainmatter
\cleartorecto
\thispagestyle{empty}


\begin{ChapterStart}
\vspace{3\nbs}
\ChapterSubtitle[l]{Chapter ch1}
\ChapterTitle[l]{ch1}
\end{ChapterStart}
\FirstLine{\noindent Harry Vince came into the outer office, and hurriedly shut the door behind him, cutting off the uproar of men's voices, the sound of raucous laughter and the shuffling of many feet.}
    
"Sounds like a zoo in there, doesn't it? And—phew! it smells like one too," he said as he crossed the room to where Lois Marshall sat at the telephone switchboard. He carried a bottle of champagne and two glasses which he set down carefully on a near-by desk. "Mr. English says that you are to have some champagne. So here it is."

"I don't think I want any, thank you," Lois said, smiling at him. She was a trim, good looking girl around twenty six or seven, dark, with severe eyebrows, steady brown eyes and the minimum of make-up. "I'm not mad about the stuff—are you?"

"Only when someone else pays for it," Vince returned when he expertly broke the wire cage and thumbed over the cork. "Besides, this is an occasion. We don't win the Light Heavyweight Championship every day."

The cork sailed across the room with a resounding pop! and he hurriedly poured the foaming wine into the two glasses.

"Thank goodness we don't," Lois said. "How long do you think they are going to stay in there?"

"Until they are kicked out They haven't finished the whiskey yet." He handed her a glass. "Here's to Joe Ruthlin, the new Champ."

"Here's to Mr. English," Lois said quietly and raised her glass.

"Okay. Here's to Mr. English."

They drank, and Vince grimaced and lit a cigarette. "He has the magic touch. It doesn't seem to matter what he takes on: he has to make a success of it. Fight promotion this week; a circus last week, a musical the week before that. What's he going to do next?"

Lois laughed.

"He'll find something." She looked up at Vince, seeing a square shouldered man of medium height, around thirty-three, with a crew cut, pale brown eyes that looked worried and uneasy, a good mouth and chin and a straight narrow nose. "You've done pretty well for yourself, too, Harry."

He nodded.

"Thanks to Mr. English. You know, sometimes I just can't believe I'm his general manager. I can't make out why the devil he ever gave me the job."

"He has a good eye for talent," Lois said. "He didn't give you the job because he liked the way you wear your clothes, Harry."

Vince ran his fingers through his close-cut hair. "Look at the awful hours we keep." He glanced at his wrist watch. "Eleven fifteen. This shindig's going on until two o'clock at least." He finished his champagne, waved the bottle at Lois. "Have some more?"

"No, thank you. Does he seem to be enjoying himself?"

"You know what he's like. He's been standing around all evening watching the other guys drink. He acts as if he has just dropped in on somebody else's party."

The telephone buzzer sounded. Lois pushed in a plug and picked up the harness she had laid on the desk.

"English Promotions," she said. "Good evening."

She listened while Vince watched her.

"I'll ask him to speak to you, Lieutenant," she said, and laid down the harness. "Harry, would you tell Mr. English that Lieutenant Morilli of the Homicide Bureau is calling?"

"These coppers!" Vince said, grimacing. "Wants some favor, I bet. You don't want me to disturb Mr. English to talk to that chiseler, do you?"

She nodded, her eyes serious.

"Please tell him it's urgent, Harry."

He gave her a quick look, then slid off the desk.

"Okay."

He went across the big room and pushed open the door that led into Nick English's private office.

Lois said, "I'm getting Mr. English now."

At the other end of the line, Morilli grunted.

"Better get his car at the door, Miss Marshall," he said. "When he hears what I've got to tell him he'll want some fast action."

Lois thanked him, plugged in another line and told the garage attendant who answered to have Mr. English's car at the front entrance right away.

As she pulled out the plug, Nick English came out of his office, followed by Vince.

English was six foot three in his socks, and broad, giving the appearance of massiveness without fat. He was on the right side of forty, and his hair was gypsy black, cut short and inclined to curl. There were white streaks each side of his temples that helped to soften an otherwise hard and relentless face. He had a high broad forehead, a short blunt nose, a thin mouth and a square dimpled chin. His eyes were wide set, pale blue and piercing. He was good-looking without being handsome, and gave an immediate impression of granite hard strength.

Lois moved away from the switchboard, indicating a telephone on a near-by desk.

"Lieutenant Morilli is on that line, Mr. English."

English lifted the receiver. "What's on your mind, Lieutenant?"

Lois moved quietly over to Vince.

"Better get Chuck out here, Harry. I think he'll be needed."

Vince nodded.

Lois heard English say, "When did it happen?"

She looked anxiously at the big man as he leaned over the desk, frowning into space, his long fingers tapping on the blotter.

She had known Nick English now for five years. She had first met him after he had thrown up an engineering job in South America and had opened a small office in Chicago to promote a gyroscope compass he had invented to be used in petroleum drilling operations. He had engaged her to run the office while he had walked the streets in search of the necessary capital to manufacture the compass.

She had quickly learned that difficulties and disappointments only made English work harder. There had been times when she had gone without salary and he had gone without food. His optimism and determination had been infectious. She knew he must succeed. No one who worked as hard as he did could fail. But it had been a year of constant setbacks, and that forged a link between them that she had never forgotten, but at times, she wondered, if he had forgotten. Finally the compass had been financed and had proved a success. English had sold his invention for two hundred thousand dollars plus a royalty on future sales that still brought him in a comfortable income.

He had then looked around for other inventions to promote, and during the next three years he had built up a reputation as a man who could get money out of a stone. With his newly acquired capital, he branched out and went into the entertainment business, promoting small shows and night-club cabarets, and then moving on to bigger and more ambitious shows.

Money began to pour in, and he formed companies. More money poured in and he took over the lease of two theaters and a dozen night clubs. Later, when money became almost an embarrassment, he moved into the political field. It was his money that put Senator Henry Beaumont into power and was keeping him in office.

Looking at English now, Lois realized just how far he had come and what a power he had become and, woman-like, she had regretted his rise to a height where she could no longer be of real use to him, when she was just one of many who served him.

Vince came out of the inner office with Chuck Eagan who drove English's car and did any job that English wanted done without argument or question.

Chuck was a small, jockey-sized man in his late thirties. He had sandy colored hair, a red, freckled face, stony eyes and quick, smooth movements. A tuxedo didn't suit him.

"What's up?" he asked out of the side of bis mouth, edging up to Lois.

She shook her head at him.

English said into the telephone mouthpiece, "I'll be right over. Leave things as they are until I get there."

Chuck stifled a groan.

"The car?" he asked, looking at Lois.

"At the door," she told him.

English hung up. As he turned, the three stiffened slightly, their eyes on him, waiting for instructions. His solid, suntanned face told them nothing, but bis blue eyes were hard as he said, "Get the car, Chuck. I want to get out of here right away."

"It's waiting, boss," Chuck said. "Ill meet you downstairs." And he went out of the room.

"Let these wolves finish the case of Scotch, and then get rid of them," English said to Vince. "Tell them I've been called away."

"Yes, Mr. English," Vince said, and went into the inner office.

"Stick around, will you?" he said to Lois. "I may need you tonight. If you don't hear from me within an hour, go home."

"Yes." She looked searchingly at him. "Has something happened, Mr. English?"

He looked at her, then moving over to her, he put his hand on her hip and smiled.

"Did you ever meet my brother, Roy?"

She showed her surprise as she shook her head.

"You haven't missed anything." He gave her hip a little pat. "He's just shot himself."

She caught her breath sharply.

"Oh . . . I'm sorry."

"Save it," he said, throwing his coat over his arm and moving towards the door. "He doesn't deserve your sympathy and he wouldn't want mine. This could be messy. If the papers get it, stall them. Tell them you don't know where I am."



II

Chuck Eagan swung the big, glittering Cadillac into a downtown side street.

Halfway down the block on the right he saw two prowl cars parked outside a tall building that was in darkness, except for two lighted windows on the sixth floor.

He drew up behind the parked cars, cut the engine and got out as Nick English opened the rear door and untangled his long legs to the sidewalk.

Chuck looked enquiringly at him.

"Want me to come up, boss?"

"May as well. Keep in the background and keep your mouth shut"

English walked across the sidewalk to where two patrolmen stood either side of the entrance to the building. They both saluted.

"The Lieutenant's waiting for you, Mr. English," one of them said.

English nodded and walked into the dark lobby. He moved through a smell of garbage, faulty plumbing and the acid reek of stale perspiration. Facing the entrance was an ancient elevator scarcely big enough to hold four people. Chuck slid back the grill and followed English into the elevator. He thumbed the automatic button, and the cage started its jerky ascent.

English had left his overcoat in the car. He stood solidly on the balls of his feet, his hands thrust into the pockets of his tuxedo, a smoldering cigar between his teeth, his eyes brooding and cold.

Chuck glanced at him, then looked away.

Eventually the elevator jerked to a stop at the sixth floor and Chuck pulled back the grill.

English stepped into a dimly lit passage. Almost opposite him was an open door through which a light came, throwing a square of brightness on the dirty rubber floor of the passage. Further along the passage to the left was another door, showing a light through a frosted panel. To his right, at the end of the passage was yet another door without glass. A light showed under the gap between the bottom of the door and the floor.

Lieutenant Morilli came through the open doorway. He was a thickset man in his late forties. His lean face was pallid, and his small moustache looked startlingly black against his white complexion.

"Sorry to break up the party, Mr. English," he said, his voice pitched low, "but I thought you'd want to come down." He had the hushed, deferential manner of an undertaker dealing with a wealthy client. "A tough break."

English grunted.

"Who found him?"

"The janitor. He was checking to see if all the offices were locked. He called me, and I called you. I haven't been here myself much more than twenty minutes."

English made a sign to Chuck to stay where he was, and then walked into the shabby little outer office. Across the frosted panel of the door was the legend:

*The Alert Agency*

*Chief Investigator: Roy English*

The room contained a desk, a typist chair, a covered type-Writer, a filing cabinet and a strip of carpet. On the walls hung dusty handcuffs and faded testimonials in narrow black frames, some of them dated as far back as 1927.

"He's in the other room," Morilli said, following English into the outer office.

Two plain-clothes detectives stood around awkwardly. English nodded at them, then walked across the room and paused in the doorway that led to the inner office.

The room was a little larger than the outer office. A worn and dusty rug covered the floor. A big desk took up most of the room space. A shabby armchair stood near the desk.

English's eyes swept quickly over these details, noting with a little grimace the sordidness of the room.

His brother had been seated at the desk when he had died. He now lay across the desk, his head on the blotter, one arm hanging lifelessly, his fingers just touching the carpet. His head and face rested in a pool of blood that had conveniently dripped into the metal trash basket on the floor.

English looked at his brother for a few seconds, his face expressionless, his eyes brooding. Morilli watched him from the doorway.

English walked over to the desk, leaned forward to see the man's face more clearly. His shoe touched something hard, lying on the floor, and he glanced down. A .38 Police Special lay within a few inches of the lifeless fingers.

English stepped back.

"How long has he been dead?" he asked abruptly.

"A couple of hours, at a guess," Morilli told him. "No one heard the shot. There's a news service agency down the passage. The teletypers were working at the time, and the noise deadened the shot."

"That his gun?"

Morilli lifted his shoulders.

"It could be. He has a pistol permit. I'll have it checked." His eyes searched English's face. "I don't think that there's much doubt that it was suicide, Mr. English."

English moved around the room, his hands still in his pockets. The fragrant smell of his cigar followed him as he moved.

"What makes you say that?"

"Things I've heard. He was short of money."

English stopped walking up and down and fixed Morilli with his cold, hard eyes.

"Don't let me hold you up any longer, Lieutenant."

"I thought I'd wait until you came," Morilli said uncomfortably.

"I've seen all I want to see. Ill wait in the car. When you're through, let me know. I want to have a look at his papers."

"It could take an hour, Mr. English. Would you want to wait that long?"

English frowned. "Have you told his wife yet?" he asked, jerking his head at the still body across the desk.

"I've told no one but you, Mr. English. Would you like me to take care of his wife? I could send an officer."

English shook his head.

"I guess I'll see her." He hesitated, his frown deepening. "Maybe you don't know it, but Roy and I haven't exactly hit it off recently. I don't even know his home address."

"I've got it here," Morilli said, his face expressionless. He picked up a wallet on the desk. "I went through his pockets as a matter of form." He handed English a card. "Know where it is?"

English read the card.

"Chuck will." He flicked the card with his fingernail. "Did he have any money on him?"

"Four bucks," Morilli said.

English took the wallet from Morilli's hand, glanced into it, then put it in his pocket.

"I'll see his wife. Can you get one of your men to clean up here? I may be sending someone down to check his files."

"Ill fix it, Mr. English."

"So you heard he was short of money," English said. "How did you hear that, Lieutenant?"

Morilli scratched the side of his jaw; his dark eyes uneasy.

"The Commissioner mentioned it. He knew I knew him, and he told me to have a word with him. I was going to see him tomorrow."

"A word about what?"

Morilli looked away.

"He had been worrying people for money."

"What people?"

"Two or three clients he had worked for last year. They complained to the Commissioner. I'm sorry to tell you this, Mr. English, but he was going to lose his license."

English nodded his head. His eyes narrowed.

"So the Commissioner wanted you to talk to him. Why didn't the Commissioner speak to me instead of to you, Lieutenant?"

"I told him he should," Morilli said, a faint flush rising up his neck and flooding his pale face. "But he isn't an easy man to talk to."

English smiled suddenly; it wasn't a pleasant smile.

"Nor am I."

As English crossed to the door, Morilli went on, "I hear your boy won his fight. Congratulations."

English paused.

"That's right. By the way I told Vince to put a bet on for you. A hundred's brought you three. Look in tomorrow and see Vince." His eyes met Morilli's. "Okay?"

Morilli flushed.

"Why, that's pretty nice of you, Mr. English. I meant to lay a bet . . ."

"Yeah, but you didn't have the time. I know how it is. Well, I didn't forget you. Glad you won."

He walked into the outer office, and into the passage. He jerked his head at Chuck and stepped into the elevator.



III

English had only met Roy's wife once, and that casually at a cocktail party more than a year ago. He remembered he hadn't thought much of her, but was prepared to admit prejudice.

Chuck had turned off the main road, and was driving with easy assurance down an avenue lined on either side by small bungalows.

"This is the joint, boss," he said suddenly. "The white house by the lamppost."

He slowed down, swung the car to the curb and pulled up outside a small, white bungalow.

A light showed in one of the upper rooms through the drawn curtains.

English got out of the car, hunching his broad shoulders against the cold wind. He stood on the porch waiting, waiting, after he'd knocked on the door. Then he heard someone coming, and the door opened a few inches on the chain.

"Who is that?" a woman's voice asked sharply.

"Nick English," he returned.

"Who?" He caught the startled note in her voice.

"Roy's brother," he said, feeling a surge of irritation run through him at having to associate himself with Roy.

The chain slid back and the door opened and an overhead light flashed up.

Looking at Corinne, he found himself thinking she would probably look like this in thirty years' time. She was small and very blonde, and her body was pleasantly plump with provocative curves. She was wearing a rose pink silk wrap over black lounging pajamas. When she saw he was looking at her, her fingers went hastily to her corn-colored curls, patting them swiftly while she stared up at him with a surprised, rather vacant expression in her big eyes that reminded him of the eyes of a startled baby.

"Hello, Corinne," he said. "Can I come in?"

"Well, I don't know," she said. "Roy's not back yet I'm alone. Did you want to see him?"

He restrained his irritation with an effort.

"I think I had better come in," he said as gently as he could. "You'll catch cold standing there. I'm afraid I have some bad news."

"Oh?" Her eyes opened a trifle wider. "Hadn't you better see Roy? I don't think I want to hear any bad news. Roy doesn't like me to be worried."

He thought how typical that was of her. She could live in this smart litle bungalow, dress like a Hollywood starlet while Roy was apparently desperate for money, and could say without shame that he didn't want her to be worried.

"You'll catch cold," he said and moved forward, riding her back into the little lobby. He closed the door. "I'm afraid that this bad news is for you." "

He saw her face tighten with sudden fear, but before she could speak he went on, "Let's go in the living room and sit down for a moment."

She went past him into a long, low pitched room. The modern furniture was new and cheap looking, but it made a brave show. In the hard light of a standing lamp, he noticed her rose pink wrap was a Utile grubby at the collar and cuffs.

"I think we ought to wait until Roy comes in," she said, lacing and unlacing her small, plump fingers. He could see that she was desperately anxious to avoid any responsibility or to have to make any decision.

"It's because of Roy I've come," he said quietly, and turned to look at her. "Sit down, please. I wish I could spare you this, but you've got to know sooner or later."

"Oh!"

She sat down suddenly as if the strength had gone out of her legs, and her face went white under her careful make-up.

"Is—is he in trouble?" she asked.

He shook his head.

"No, he's not in trouble. It's worse than that." He wanted to be brutal and tell her that Roy was dead, but looking at the doll-like face, seeing the terror in the baby blue "eyes, the childish quivering of her lips, the sudden clenching of her fists, made it impossible for him to shock her with the straight facts.

"Is he hurt?" She met his eyes and flinched back as if he had threatened to hit her. "He's—not dead?"

"Yes, he's dead," English replied. "I'm sorry, Corinne. I wish I didn't have to tell you. If there's anything I can do ..."

"Dead?" she repeated. "He can't be dead!"

"Yes," English said.

"But he can't be dead," she repeated, her voice going shrill. "You're saying this to frighten me! You never did like me! Don't pretend you did. He can't be dead!"

"He shot himself," English said quietly.

She stared at him. He could see at once that she believed him. Her doll-like little face seemed to fall to pieces. She dropped back against the couch, her hand across her eyes.

He looked around the room, then crossed over to an elaborate cabinet that stood against the wall. He opened it and found an array of liquor bottles and glasses. He poured a shot of brandy into a glass and went over to her.

"Drink this."

He had to hold the glass to her lips, but she managed to get some of the brandy down before pushing his hand away.

"He shot himself?" she said, looking at him.

He nodded.

"Have you anyone who will stay with you tonight?" he asked, not liking the dazed horror in her eyes. "You can't be left here alone."

"But I am alone now," she said, and tears began to run down her face, smearing her make-up. "Oh, Roy! Roy! How could you do it? How could you leave me alone?"

It was the anguished cry of a child, and it disturbed English. He put his hand gently on her shoulder, but she threw it off so violently that he stepped back, startled.

"Why did he shoot himself," she demanded, looking up at him.

'Try to get it out of your mind for tonight," he said. "Would you like me to send someone to you? My secretary. . . ."

"I don't want your secretary!" She got unsteadily to her feet "And I don't want you! You killed Roy! If you had been a decent brother to him, he would never have done it!"

He was so surprised by the suddenness of this attack, and of her expression of hatred as she looked at him that he remained motionless, staring at her.

"You and your money!" she went on, her voice strident. "That's all you've ever thought about! You didn't care what happened to Roy. You didn't bother to find out how he was getting on! When he came to you for help, you threw him out! Now you've forced him to kill himself. Well, I hope you're satisfied! 1 hope you're happy you've saved some of your dirty money! Now, get out. Don't ever come here again. I hate you!"

"You mustn't talk like that," English said quietly. "It's quite untrue. If I had known Roy was in a jam, I would have helped him. I didn't know."

"You didn't care, you mean!" she cried shrilly. "You haven't spoken to him for six months. When he asked you for a loan, you told him you weren't giving him another cent. Do you call that helping him?"

"I've been helping Roy ever since he left college," English said, his voice hardening. "I thought it was high time he stood on his own feet. Did he expect me to keep him all his life?"

"Get out!" She stumbled to the door and threw it open. "Get out and stay out! And don't try to offer me any of your dirty money, because I won't take it! Now, get out!"

English lifted his heavy shoulders in a despairing shrug. He guessed that as soon as he had gone, she would collapse, and he was reluctant to leave her alone.

"Haven't you someone . . ." he began, but she broke in, screaming, "Get out! Get out! I don't want your filthy help or your sympathy. You're worse than a murderer. Get out!"

He saw it was hopeless to do anything, and he went past her into the lobby. As he opened the front door, he heard her sobbing, and he glanced back. She had thrown herself face down on the couch, her head in her arms.

He shook his head, hesitated, then opened the door and walked down the path to the car.



IV

Chuck swung the big car into the circular drive that led to an imposing apartment block overlooking the river, and brought it to a standstill before the entrance.

He got out and held the door open.

"I want to find out if my brother had a secretary or someone to help him in the office," English said as he got out of the car. "Go down to his office first thing in the morning and see if the janitor knows. I want her address. Be here not later than nine-thirty. We'll go and see her before we go to the office."

"Yes, boss," Chuck said dutifully. "I'll fix it. Anything else I can do?"

English gave him a quick smile.

"No. Go to bed and don't be late tomorrow."

He walked across to the entrance of the building, pushed against the revolving doors, nodded to the night porter who snapped to attention when he caught sight of him, and walked over to the elevator.

He thumbed a button, and leaned against the wall while the self-service elevator bore him swiftly and smoothly up fifteen floors to the penthouse apartment he had rented for Julie.

He walked down the walnut paneled corridor and paused outside the apartment. As he groped for his keys, his eyes shifted to the card in a chromium frame that was screwed on the door. It bore the single line of neat print: *Miss Julie Clair*.

He pushed the key into the lock, opened the door, and stepped into a small, lighted lobby. As he threw his hat and coat on a chair, the door opposite him opened and a girl stood framed in the doorway. She was tall and broad shouldered, with narrow hips and long legs. Her copper-colored hair was silky and dressed high on top of her small head. Her big almond-shaped eyes Were sea-green and glitteringly alive. She

was wearing olive green lounging pajamas with red piping, and her small feet were incased in high heeled red slippers.

Looking at her, English thought how very different she was from Corinne. How much more beautiful, and how much more character she showed in her face, which he considered to be more pleasing to his eyes than any other woman's he had met. Her make-up, even at this late hour, he thought, was a masterpiece of understatement. He knew she wore make-up, but he couldn't see where it began or left off.

"You're late, Nick," she said, smiling at him. "I was beginning to wonder if you were coming."

"Sorry, Julie," he returned, "but I've been held up."

He went over to her, put his hands on her hips and kissed her cheek.

"So Joey won his fight," she said, looking up at him. "You must be very pleased."

"Don't say you listened to the radio?" he said, leading her into the well appointed living room. A big fire burned brightly, and the shaded lamps made the atmosphere at once intimate and cosy.

"No, but I heard it on the news."

"You and Harry are a pair," he said, sinking into a big over-stuffed armchair and pulling her down on his knees. She curled up on his lap, slipping her arm around his neck, and resting her face against his cheek. "Believe it or not, although he handled most of the arrangements and worked like a dog for weeks, he stayed away from file fight. He's as squeamish as you are."

"I think fighting is a dirty business," she returned with a grimace. "I don't blame Harry for not going."

He stared at the bright flames and his hand stroked her silk clad thigh.

"Maybe it is, but there's a lot of money in it. Was the show all right?"

She shrugged. "I suppose so. They seemed to like it. I wasn't singing particularly well, but no one seemed to notice. Tell me about the fight, Nick."

"There's something else I have to tell you. Do you remember Roy?"

He felt her stiffen.

"Yes, of course. Why do you ask?"

"The fool shot himself tonight."

She half sat up, but he pulled her down against him again.

"Don't move, Julie."

"Is he dead?" she asked, her fingers gripping his arm.

"Yes, he's dead. That was one job he did manage to do efficiently."

She shivered.

"Don't talk like that, Nick. It's terrible. When did it happen?"

"About half past nine."

"But why did he . . . ?"

"Yeah, that puzzles me. Do you mind if I walk up and down? You're taking my mind off business." He lifted her, and got up, then set her gently in the chair, and moved over to the fireplace. "Why, Julie, you look pale."

"I suppose it's the shock. I wasn't expecting to hear anything like this. I don't know if you are upset, Nick, but if you are, I'm sorry."

"I'm not upset," English said, taking out his cigar case. "Maybe it was a shock, but I can't say I'm particularly sorry. Roy's been a damned nuisance ever since he was born. My old man and he were a pair. But I can't understand Roy's shooting himself. Morilli says he was short of money and was trying to raise the cash by threatening two or three of his old clients. He was going to lose his license. I would be willing to bet that Roy wouldn't have killed himself because of that. I shouldn't have believed that he would have had the nerve to kill himself no matter how bad a jam he was in. It's damned odd. Morilli says he's satisfied, but I still don't believe it"

Julie looked up quickly.

"But surely, Nick, if the police say so. . . ."

"Yeah, I know, but it foxes me. Why didn't he come to me if he was so hard up? Maybe I did throw him out last time, but that has never stopped him before. I've thrown him out lots of times and he's always come back."

"Perhaps he was too proud," Julie said quietly.

"Proud? Roy? Sweetheart, you can't have known Roy. He had a hide like an elephant. He'd take any insult so long as he got money out of me." English lit his cigar and began to move slowly about the room. "Why did the business fold up like that? When he got me to buy it for him, I took the trouble to investigate it pretty thoroughly. It was in the black then. It was an old established business. He couldn't have wrecked it so soon, unless he did it deliberately." He made an impatient gesture. "I was a fool to have had anything to do with it. I might have known he wouldn't have worked at it."

Julie watched him pace the floor.

"I've sent Lois to check up at his office," English went on. "She has a nose for that kind of thing."

"You sent Lois there tonight?" Julie said sharply.

"I phoned her on my way up here. I wanted her to have a look at the place before Corinne takes it into her head to go up there."

"You mean Lois is there now?"

English paused and looked at her, surprised at the sharpness of her tone.

"Yes, Harry's with her. She doesn't mind how late she works."

"Well, after all, it's nearly half past one. Couldn't it have waited until tomorrow?"

"Corinne might go up there," English said, frowning. He didn't like his orders questioned. "I want to know what Roy had been up to."

"I think she must be in love with you," Julie said, moving so that her back was turned to him.

"In love with me?" English said, startled. "Who? Corinne?"

"Lois. She acts as if she were your slave. No other girl would tolerate working for you, Nick."

English laughed.

"Nonsense. I pay her well. Besides, she isn't the kind of girl to fall in love with anyone."

"There's never been a girl who wouldn't fall in love if she's given a chance," Julie said quietly. "I should have thought that you would have had more insight, Nick, than to say a thing like that"

"Never mind Lois," English said a little impatiently. "We were talking about Roy. I went to see Corinne tonight."

"That was nice of you. I've never seen her. What's she like, Nick?"

"Blonde, plump and dumb-looking," English said, coming to sit on the arm of her armchair. "She told me that I was responsible for Roy's death and threw me out of the house."

"Nick!" Julie looked quickly at him, but was reassured by his smile.

"I guess she was hysterical, but to be on the safe side I got Sam Crail out of bed and sent him down to talk to her. I've got to be careful there isn't a stink about this business, Julie. I have a big deal cooking at the moment." His brown hand slid over her shoulder and his fingers gently stroked her throat. "In a few weeks, the Senator is going to break the news that I'm the man behind the new hospital. The committee knows, of course, but the press hasn't got it yet. The idea is to name the hospital after me."

"Name it after you?" Julie repeated blankly. "But why, for goodness sake?"

English grinned a little sheepishly.

"Sounds crazy, doesn't it. But I want it, Julie. I want it more than anything I've ever wanted." He got up and began to pace up and down. "I've made a fair success of my life, Julie. I started from scratch, and now I'm as good as the next man as regards money, but money isn't everything. If I drop dead this moment, no one would remember me in a week's time. It's the name people leave after them that counts. If the hospital were named after me—well, I guess I wouldn't be forgotten quite so easily. And then there's another thing— more important. I promised my mother I'd make a name for myself, and she believed me. She didn't live long enough to know that I had started on the way up. When she died, I was still fooling around with that compass and getting nowhere, but I told her that it was going to be a success, and I told her I was going way ahead, and she believed me. She would have got a big bang out of knowing the hospital is going to be named after me, and I'm soft in the head enough to think she'll still get a big bang out of it."

Julie listened in hypnotized silence. She had never had any idea that English could think and talk like this. She put her hand on his arm.

"If that's what you want, Nick, I want it, too."

"I guess that's right," he said, suddenly thoughtful. "But this business of Roy's might slap a lid on it."

"But why?"

"Believe it or not, Julie, it took me a hell of a time to persuade the commission to let me finance the hospital. You wouldn't believe that, would you?"

"What commission?"

"The City Planning Commission," he said patiently. "It's unbelievable what a bunch of stuffed-shirts they are. Although I bet their private lives wouldn't stand investigating, on the surface they are about the finest collection of plaster saints you've ever set eyes on. They didn't approve of me. Two of them even said I was a gangster. The Senator had to talk pretty sharply to them to get them to accept my money. At the time, nothing was said about the hospital being named after me. If it turns out that Roy was in trouble, that he blackmailed his clients, the chances of my name being used are as good as a snowball in hell. Morilli knows that. The Police Commissioner knows it, too. They'll expect to be taken care of if this is hushed up. But Corinne's the difficulty. She may try to cut off her nose to spite my face. If she lets on that I wouldn't finance Roy, and Roy was forced to raise money by blackmail, I shall be ruled out." He tossed the cigar into the fire and went on in a hushed voice, "Why couldn't the louse have shot himself next month when this was in the bag?"

Julie stood up.

"Let's go to bed, Nick," she said and slipped her arm through his. "Don't let's think any more about it tonight."

He gave her bottom an affectionate little pat.

"You're full of good ideas, Julie," he said. "We'll go to bed."



V

At the back of a modest walk-up apartment house on East Tenth Street, a small, shrub-infested garden ran down an alley hedged in on either side by a six foot brick wall.

For two hours, a man had been waiting in the alley, his eyes fixed on a lighted window on the third floor of the building.

He was a man of middle height, with broad and powerful shoulders. He wore a wide-brimmed brown slouch hat pulled down over his eyes, and in the dim light of the moon, only his thin lipped mouth and square shaped chin could be seen. The rest of his face was hidden by the black shadow cast by the hat brim. He was expensively dressed. His brown lounge suit, his white silk shirt and polka-dot bow tie gave him the appearance of a well-to-do operator, and once when he lifted his arm to consult a gold strap watch, he showed two inches of white shirt cuff, and the tail of a white silk handkerchief he wore tucked up his sleeve.

While he waited in the alley, he remained motionless. He chewed a stick of gum, his jaws moving rhythmically and continuously. His two hour vigil was conducted with the patience of a cat waiting for a mouse to appear.

A few minutes after midnight, the light in the third floor window suddenly went out and completed the darkness of the rest of the apartment house.

The man in the brown suit remained motionless. He leaned his broad shoulders against the brick wall, his hands thrust into his trouser pockets while he waited another half hour. Then, consulting his watch, he reached down into the darkness and picked up a coil of thin cord that lay near his feet. A heavy rubber covered hook was fastened to one end of the cord.

He swung himself over the wall and walked silently and rapidly up the cinder path that led through the garden to the back of the building.

The man in the brown suit paused under the swing-up end of the fire escape that was some five feet above his outstretched hand. He uncoiled the cord and tossed the hook into the air. The hook caught in the iron work of the escape, and held. He gently tightened his grip on the cord, then pulled. The end of the escape came down slowly and silently, and bumped on the ground. He climbed the escape, two steps at a time, without hesitation and without looking back to see if anyone happened to be watching him. He reached the third floor window that he had been watching for the past two hours, and saw with satisfaction that the window was open a few inches at the top and bottom. He noticed also the curtains across the windows were drawn. He knelt down by the window, his ear to the gap between the window frame and the sill, and listened. He remained like that for several minutes, then he put his fingers under the window frame and gently exerted pressure. The window moved up, inch by inch, making no sound.

The curtains hung well clear of the window, and he slid into the room without disturbing them. He turned and began to close the window, again moving it inch by inch, and again in silence. When the window was as he had found it, he straightened, turned and parted the curtains a few inches. He looked into darkness. The over-sweet smell of face powder, stale perfume and cosmetics told him he had made no mistake as to the room. He listened, and after a moment or so, he heard quick light breathing not far from where he stood.

He took out a pencil-thin flashlight, and shielding the bulb with his fingers, he switched the fight on. He saw a bed, a chair on which were some clothes, and a night table by the bed on which stood a small shaded lamp, a book and a clock.

The back of the bed was to the window. He could see the outline of a figure under the blankets. Hanging on the bedpost was a silk dressing gown.

Careful to keep the shielded light of his torch away from the sleeper in the bed, the man in the brown suit reached forward and gently pulled the silk cord of the dressing gown through its loops until he had disengaged it. He tested its strength, and, then satisfied, reached forward and picked up the book that was lying on the night table.

With the dressing gown cord and the flashlight in his left hand and the book in his right, he stepped behind the curtains again. He turned off the flashlight and slipped it into his pocket, then still keeping behind the curtains and holding one of them aside with his left hand, he tossed the book high and wide into the air.

The book landed on the polished boards of the floor. Coming down flat side up, it made a loud slapping noise that was intensified by the silence that brooded over the whole apartment.

The man in the brown suit closed the curtains and waited, his jaws moving rhythmically as he chewed. He heard the bed creak, and then a girl's voice said sharply, "who's that?"

He waited, without moving, his breathing normal, his head a little on one side as he listened.

The bedside lamp went on, sending a soft glow of light through the curtains. He opened them slightly so that he could see into the room.

A dark, slim girl in blue nylon nightdress was sitting up in the bed. She was looking towards the door, her hands clutching the blankets, and he could hear her rapid breathing.

Silently, he took one end of the dressing gown cord in his right hand, and the other end in his left. He turned sideways so that he could push aside the curtains.with his shoulder. He watched her, waiting.

She saw the book on the floor, and she looked quickly at the night table, and then back to the book again. Then she did what he was hoping she would do. She threw back the blankets and swung her feet to the floor, her hand reaching out for her dressing gown. She stood up and began to slide her arms into the sleeves of the dressing gown, turning her back on the window as she did so.

The man pushed aside the curtains with his shoulder and stepped silently into the room. With a movement too quick to follow, he whipped the cord over the girl's head, twisting it around her throat. He drove his knees into the small of her back, sending her down on her hands and knees. He dropped on her, flattening her to the floor. The cord bit into her throat, turning her wild scream into a thin, almost inaudible cry. He knelt on her shoulders and his two hands tightened the cord.

He remained like that, chewing steadily, and watching the convulsive heaving of her body and the feeble movement of her hands clawing at the carpet. He was careful, and kept the cord just tight enough to stop blood flowing to her head and air getting to her lungs. He had no difficulty in holding her down, and he saw with detached interest her movements were becoming less convulsed until only her muscles twitched in a reflex of agony.

He remained kneeling on her, the cord tight, for three or four minutes, then when he saw that there was no longer any movement, he carefully took the cord from her throat and turned her over on her back.

He frowned when he saw that a trickle of blood had run down one nostril and had made a smear on the rug. He put his finger on her eyeball, and when there was no answering flicker, he stood up and dusted his trouser knees while he looked quickly around the room.

He went to the door opposite the bed, opened it and looked into a small bathroom. He noted with a nod of his head the sturdy hook screwed to the back of the door.

He spent the next ten minutes or so arranging the room. His movements were unhurried and unruffled. When he had finished what he was doing, he surveyed the scene with quick, bright eyes that missed no detail. Then he turned off the lamp and went to the window. He opened it, turned to adjust the curtains, stepped out onto the fire escape and pulled down the window, leaving it as he had found it. Then he went quickly and silently down the fire escape to the darkness of the garden.

\vspace{2\nbs}
\ChapterDeco[c1]{\decoglyph{e9665}}
\clearpage
\thispagestyle{empty}

\begin{ChapterStart}
\vspace{3\nbs}
\ChapterSubtitle[l]{Chapter ch2}
\ChapterTitle[l]{ch2}
\end{ChapterStart}
\FirstLine{\noindent The following morning, a few minutes before nine-thirty, Chuck Eagan drove the Cadillac into the circular drive leading to Julie's Riverside apartment block. As he got out of the car, Nick English came through the revolving doors.}
    
"Morning, Chuck," English said as he got into the car. "What's the good word?"

"I went down and talked to the janitor like you said," Chuck said. "A Joe named Tom Calhoun. He seemed a helpful sort of a guy after I had let him see some dough. Your brother had a secretary. Her name's Mary Savitt and she's got an apartment on East Tenth Street."

"Okay," English said. "Let's go there. Snap it up, Chuck. I want to catch her before she leaves."

Chuck got into the Cadillac and set it in motion. While he drove rapidly through the congested traffic, English glanced at the newspapers he had brought down with him. All of them devoted considerable space to Roy's suicide, coupling his name with Nick's.

Morilli appeared to be earning his keep. He had given out that Roy had been overworking, and it was believed that he had shot himself in a fit of depression following a nervous breakdown. The story sounded a little thin, but English was satisfied that it would stand up so long as someone didn't come along to challenge it.

English wondered irritably if he were wasting his time going to see Mary Savitt. There was a lot to do. He had to see Senator Henry Beaumont and calm his fears. He had to have a word with the Police Commissioner. He had to talk to Sam Crail, his attorney, and then there were the news hounds to deal with. But he was pretty sure if anyone knew why Roy had killed himself this girl, Mary Savitt, would know.

Chuck said, "Here we are, boss. This joint on the left."

"Don't stop at the door," English said. "Drive on half a block, and well walk back."

Chuck did as he was told, then stopped the car. The two men got out.

"You'd better come with me," English said, and set off with long, quick strides to the brown-stone apartment house that Chuck had pointed out.

A row of mailboxes in the lobby, each with the owner's name on it, told English that Mary Savitt's apartment was on the third floor. The entrance to the apartments was guarded by a door by which was a row of buzzers. Chuck thumbed the third floor button and waited for the answering buzz. Nothing happened, and after pressing the buzzer three times, he looked over at English.

"I guess the nest's empty," he said.

"She's probably seen the newspapers and has gone down to the office," English returned, frowning.

At this moment, the door to the stairs opened and a girl came into the lobby. She was smartly dressed, and she looked sleepy and pale in the hard morning light. She stared at English, and her eyes opened wide. Her fingers went hastily to her hair, tucking in a stray curl under her hat. He raised his hat

"Pardon me, I was hoping to find Miss Savitt. She's out, I guess?"

"Oh no, she's not out, Mr. English," the girl said smiling. "It is Mr. English, isn't it?"

"That's right," English returned, holding his hat in his hand. "Clever of you to recognize me."

"Oh gee! I'd know you anywhere, Mr. English. I saw *The Moon Rides High* last week. I thought it was a terrific show."

"I'm glad," English said, and somehow he managed to convey that he was glad and that her opinion was something to cherish. "Maybe Miss Savitt's still asleep. I've buzzed her three times."

"Perhaps her buzzer is on the blink," the girl said. "I know she's in. Her milk's still outside the door and her newspaper's there, too. Besides, she never leaves before ten."

"Then I guess I'll go up and knock on the door," English said. "Thank you for your help."

"You're welcome, I'm sure."

He gave her a warm, friendly smile that left her looking a little dazed, and moved past her to the stairs, followed closely by Chuck.

A bottle of milk and a folded newspaper lay outside Mary Savitt's front door.

English jerked his head towards the door, and Chuck rapped sharply on it. No one answered. Again Chuck knocked, again no one answered.

"Think you could open the door, Chuck?" English asked, lowering his voice.

For a moment Chuck looked surprised, then he examined the lock.

"Nothing to it, but maybe she'll squawk for the cops."

"Go ahead and open it," English said.

Chuck took out a small metal lever from his pocket, inserted it into the lock, fiddled for a moment, then pushed open the door.

English stepped into a neatly kept living room.

"Is anyone here?" he called, raising his voice.

He waited in silence, then crossed the room and knocked on a door facing him, then opened the door and looked into a darkened room. Enough light filtered through the drawn curtains to show him that it was a bedroom. He looked toward the bed; it was empty, and the blankets were thrown back,

"I believe she's out," he said to Chuck.

"Maybe she's having a bath," Chuck said. "Want me to go and see?"

English ignored his eagerness and moved into the bedroom, turning on the light as he did so.

He came to an abrupt standstill.

To the right of the door leading to the bedroom was another door. Against this door, and hanging by a white silk cord which had been thrown over the top of the door and fastened to something on the other side, was the body of a dark haired girl in her early twenties. She was wearing a white silk dressing gown that hung open to show a blue nylon nightgown. What beauty she might have had was spoiled now by her waxen color and her swollen tongue that protruded from her open mouth. Dried blood made a red thread from her nose to her chin.

Chuck drew in a sharp breath.

"Holy mackerel! What did she want to do that for?" he said in a tight, low voice.

"She's been dead about seven hours, at a guess," he said. "This is getting complicated, Chuck."

Chuck came and stood at his side, his eyes on the dead girl.

"It sure is," he said. "We'd better get out of here, boss."

"Shut up, will you?" English snapped, and began to move around the room.

Chuck went over to the door and waited, his small, hard eyes on English.

"On the mantelpiece, boss," he said suddenly.

English looked at the mantelpiece. On it, was a silver framed photograph of his brother Roy. Written in ink across the lower part of the photograph in his brother's large sprawling handwriting, was "*Look at me sometimes, darling, and remember what we're going to be to each other. Roy.*"

English swore softly under his breath.

"So he had to fall in love with her!" He looked over at Chuck. "He's certain to have written to her. His kind always does. Get busy and see if you can find any letters."

Chuck went into action quickly and with professional thoroughness. In a very short time he had unearthed a bundle of letters done up in a blue ribbon which he handed to English, and then continued his search.

English glanced through the letters, recognizing his brother's handwriting. He had only to read two or three of them to know that Roy and Mary had been passionately in love with each other, and that Roy had been planning to leave Corinne and go away with Mary. He shoved the letters in his pocket as Chuck closed the last drawer.

"Take a look in the other room," English said, and when Chuck had left the bedroom he picked up the framed photograph of his brother and dropped it into his pocket.

Five minutes later, English and Chuck left the apartment.

"The office. And snap it up," English said as he climbed into the car. "And keep your mouth shut about this, Chuck."

Chuck inclined his head, slid under the driving wheel and sent the Cadillac shooting uptown.

The intercom on English's vast mahogany desk buzzed into life.

"Mr. Crail is here, Mr. English," Lois told him.

"Send him in, and when he's gone, come in yourself," English said.

A moment later, the door opened and Sam Crail came in. Crail was nearly as tall as English, and immensely fat. His hair was black and thick, and smoothly oiled. His complexion was pallid, and his eyes sharp and beady. His smooth, black jowels were blue with constant shaving, and his pudgy hands were hairy, his nails immaculately manicured.

He was the smartest attorney in town, and had handled all English's legal work ever since English had begun to climb.

"Hello, Nick," he said as he pulled up a chair. "This is a bad business."

English grunted, pushed his cigar box across the desk and eyed Crail speculatively.

"Did you see Corinne last night?"

"I saw her. She's going to make trouble, Nick."

"No, she isn't," English said shortly. "What do you imagine you're on my pay roll for? It's your job to stop her making trouble."

"What do you think I've been doing ever since I got there last night?" Crail said a little heatedly. "But she won't play. Her story is that Roy was in debt. He came to you for money and you threw him out."

English snorted.

"He came to me for a loan six months ago," he said. "That's not much of a story. Why didn't he shoot himself sooner?"

"She maintains that he came to you day before yesterday."

"Then she's lying."

"Roy told her he came to you."

"Then he was lying."

"Might be difficult to prove, Nick. The press is only waiting for something to break. She says because you wouldn't help him, he had to go to some of his old clients to raise the wind. One of them phoned the police. She says you told the Police Commissioner to withdraw Roy's license. With no future in front of him, he shot himself. Her story makes you directly responsible for his death."

English frowned.

"Did Roy tell her this or is she making it up on her own?"

"She says Roy told her, and that's the story she's going to tell the coroner. The inquest's in an hour, Nick."

"Yeah." English stood up and paced over to the window. "She doesn't like me, does she?"

"No, I guess she doesn't. She says her life's ruined, and she doesn't see why yours shouldn't be, either."



II

"The fool! Why does she think that my life would be ruined by a yarn like this?" English said. Crail shrugged.

"It wouldn't ruin you, Nick, but it would cause a stink. People think you are rolling in money. Public opinion is a dangerous thing to come up against. She says Roy wanted four thousand to get him out of bis mess. Four thousand wouldn't have scratched your pile. She could make it sound pretty cheap, Nick."

"He wanted ten thousand and he wouldn't tell me why," English said. "I turned him down because I thought it was time that he stopped sponging on me. Why the hell should I keep him and his wife?"

"Sure," Crail said, "but now he's shot himself, he gets the sympathy. This could kill the hospital idea, Nick. They are only waiting for an excuse to double-cross you."

"I know." English came back to the desk. "Now listen: the story is that Roy was overworking. The business was a failure. He tried to hold it together, but it was too much for him. Instead of coming to me, he tried to handle it himself, cracked under the strain and shot himself. That's the story I've given the press this morning, and that's the story you're going to give the coroner. Corinne will go with you and back it up."

Crail looked worried.

"She won't do it. I've talked to her, and I know. She's made up her mind to be difficult,"

"She'll do it," English returned, his voice hardening. "If she doesn't like that story, then I'll give the press another she'll like a lot less. Roy had a secretary—a girl named Mary Savitt. They were lovers. They planned to run away together, and leave Corinne out on a limb. Something went wrong; probably Roy couldn't get enough money to quit. Being the weakling he was, he shot himself. The girl must have gone to the office and found him. She went home and hanged herself."

Crail stared at him.

"Hanged herself?"

"Yes. I went to talk to her this morning, and found her dead. No one knows yet. Sooner or later, they'll find her, but I'm hoping the inquest will be over before they do."

"Did anyone see you there?" Crail asked anxiously.

"I was seen going up the stairs. My story is that I rang the bell, and getting no answer, assumed that she'd gone down to the office."

"Are you sure that they were lovers?"

English opened a drawer, took out the photograph he had found in Mary Savitt's bedroom, and pushed it across the desk. He tossed the bundle of letters into Crail's lap.

"There's all the proof. If Corinne thinks she can queer my pitch by telling a sniveling yarn like this, she's got another thing coming. Tell her to toe the line or this muck goes to the press."

"This is going to be a shock to her, Nick," Crail said slowly. "She was crazy about Roy."

English regarded him, his eyes hard.

"She doesn't have to know. Persuade her to toe the line if you're all that anxious to spare her feelings."

"I guess she'll have to see these letters," Crail said. "All the same, I don't like it."

"You don't have to do the job," English said. "I can always get another attorney, Sam."

Crail shrugged his fat shoulders.

"Oh, I'll do it," he said. "I wouldn't like to be as hard as you are, Nick."

"Skip the sentiment. Did Roy leave a will?"

"Yes. He left everything to Corinne. As far as I can see, it amounts to a flock of debts. He had a safe deposit box, and I hold the key. I haven't, had time yet to examine it, but I don't expect to find anything."

"Let me know how his estate stands before you tell Corinne," English said. "We could arrange to find an insurance policy. Fix it that she has a couple of hundred bucks a week for life. I'll pay."

Crail grinned.

"Who's going soft now?" he asked, getting to his feet.

"Get over to the coroner's office," English said curtly, "and make that story stand up."

"I'll make it stand up," Crail said.



III

A minute or so after Crail had gone, Lois left her desk, crossed the room to English's office door and tapped as she opened it.

English was staring at his cigar with cold, brooding eyes. He looked up and gave her a little nod.

"Come on in and sit down," he said and hunched his massive shoulders as he leaned across the desk. "What time did you get to bed this morning?"

Lois smiled as she pulled a chair up to the desk and sat down.

"It was after four, but I don't need much sleep."

"Nonsense. Of course you do. Go home after lunch, and go to bed."

"But really, Mr. English . . ." she began.

"That's an order," he broke in curtly. "Let the work wait You're always working. Let Harry do what's necessary. Now about last night. What did you think of the setup there?"

"Not much, Mr. English. I went through all the files. There's been no new business since last August."

English frowned.

"Are you sure? Let's see—I bought the business for him in March, didn't I?"

"Yes, Mr. English. I've found correspondence dated up to July 31st, but nothing since then."

"What was he doing then for the past nine months?"

Lois shook her head.

"The place might just as well have been closed. Nothing came in and nothing went out. At least, there are no copies of letters in the files."

English rubbed bis jaw thoughtfully.

"How about his cases? Did he keep any record of those?"

"He handled sixteen cases from April to the end of July. But after that there wasn't a thing."

"What about his books?"

"There was a set in the safe. I have copies if you would like to see them."

"What was his net average take?"

"Around seventy-five a week."

English grimaced.

"How did he manage to run a home like that on seventy-five a week? You mean to tell me that since August the business hasn't made a dime?"

"He may have kept another set of books, Mr. English, but according to the ones I found, nothing came in since August."

English shrugged.

"Well, okay. What else did you find?"

"There was a card index holder in his desk. It had a few blank cards in it. I have an idea that the cards that were in use have been taken away."

English studied her, his eyes interested.

"What makes you say that?"

"From the looks of the box. The bottom of it was very dusty, and by the marks in the dust it was pretty obvious that there had been a number of cards in the box. I'm just making a guess, but it did strike me that some of the cards had been recently removed."

"Maybe the box belonged to the previous owner."

"It looked new to me, Mr. English."

English pushed back his chair and stood up, his brows wrinkled into a frown.

"It's damned funny, isn't it?" he said after a long silence. "Well, all right, Lois, thanks a lot. Sorry to have kept you out of bed so late. Be a good girl and go home after lunch."

As Lois moved to the door, the buzzer on the desk sounded. English pressed down the switch.

"Lieutenant Morilli is here, Mr. English," the receptionist said. "He would like a word."

"Harry will see him," English said. "I'm going to lunch."

"He wants to see you, Mr. English. He says it's important."

English hesitated.

"Okay, send him in. I've still got ten minutes. Tell Chuck to have the car ready."

Lois opened the door, and stood aside to let Morilli enter the office.

"You've caught me at a bad time," English said as Lois went out, shutting the door behind her. "I've got to go out in five minutes. What's on your mind?"

"I thought I ought to have a word with you," Morilli said, coming over to the desk. "We've located your brother's secretary—a girl named Mary Savitt."

English looked at him, his darkly tanned face expressionless.

"So what?"

"She's dead."

English frowned and stared at Morilli who stared back at him.

"Dead? What—suicide?"

Morilli lifted his shoulders.

"It could be murder."



IV

For a long second, English stared at Morilli, then waved him to a chair.

"Sit down, and let's hear about it."

"I telephoned Mrs. English this morning," Morilli said, sitting down, "to find out if Mr. English had a secretary. She gave me the girl's name and address. I and a sergeant went down there. She has an apartment on East Tenth Street." He paused and looked hard at English.

"I know," English said, taking his cue from Morilli's look, "I went there myself this morning. I couldn't get an answer. I thought she must have gone down to the office."

Morilli nodded.

"That's right," he said. "Miss Hopper who lives in the apartment above Miss Savitt's said she had seen you."

"Well, go on," English said curtly. "What happened?"

"We didn't get an answer to our buzz. There was a bottle of milk and a newspaper outside the door, and that made me suspicious. We got the passkey and found her hanging on the bathroom door."

English pushed his cigar box across the desk after taking one himself.

"Go ahead and help yourself," he said. "What's this about murder?"

"On the face of it, it looked like suicide," Morilli said. "The

police surgeon said it was a typical suicide." He rubbed his

bony nose and added softly, "And he still thinks it's suicide."

Then he went on, "After the body was removed, I had a look around the room. I was on my own, Mr. English, and I made a discovery. Near the bed was a damp patch on the carpet as if it had been recently washed. When I examined it carefully, I found a small stain. I gave it a benzidine test. It was a bloodstain."

English took his cigar from between his lips and frowned at the glowing end.

"I don't reckon to be as smart as you, Lieutenant, but I fail to see how that makes it murder."

Morilli smiled.

"A faked suicide is often very difficult to spot, Mr. English," he said. "We're trained to look for the giveaway. That stain on the carpet was a pretty complete giveaway. You see, when I cut the girl down, I noticed that she had bled from the nose. There were no marks on her negligee, and I expected to find at least a drop or two of blood somewhere. Then I found a stain on the floor. That tells me she died on the floor, and not hanging from the door."

English pursed his lips.

"You mean she was strangled on the floor?"

"That's right. If someone surprised her, slipped the dressing gown cord around her throat and tightened it, she would have lost consciousness very quickly. She would have fallen face downward on the carpet; and while the killer was exerting pressure on the cord, it is likely she would bleed from the nose, making a stain on the carpet. Having killed her, it would be simple for him to string her up against the bathroom door to make it look like suicide."

English thought about this, then nodded.

"I guess that's right. So you think it's murder?"

"I won't swear to it, but how else did the stain get on the carpet?"

"You're sure it's blood?"

"No doubt about it."

English glanced at his wrist watch. He was already four minutes late for his appointment.

"Well, thanks for telling me, Lieutenant," he said. "This is unexpected. I don't know what to make of it. Maybe we can talk about it later on. Right now I have a date with the Senator." He got to his feet. "I've got to be running, along."

Morilli didn't move. He sat looking up at English, an odd expression in his eyes that English didn't like.

"What's on your mind?" English asked curtly.

"It's up to you, Mr. English, but I should have thought that you would have wanted to settle this business right now. I haven't put in my report yet, but I'll have to within the next half hour."

English frowned.

"What's your report got to do with me?"

"That's for you to say," Morilli returned, carefully. "I like to help you where I can, Mr. English. You've always been pretty good to me."

English had a sudden idea that there was something very wrong behind Morilli's visit.

He leaned forward and flicked down the intercom switch.

"Lois? Get hold of the Senator and tell him I'm going to be late. I shan't be with him until two o'clock."

"Yes, Mr. English."

He released the switch and sat down again.

"Go ahead, Lieutenant; keep talking," he said, his voice hard and quiet.

Morilli hitched his chair forward, and looking English straight in the face, said, "I don't have to tell you how the D.A. feels about Senator Beaumont. They've been sworn enemies ever since the Senator got into office. If the D.A. can do anything to discredit the Senator, he's going to do it. Everyone knows you're behind the Senator. If the D.A. can make things tough for you, he'll do it in the hope it'll eventually hit the Senator. If he can involve you in a scandal, he's not going to be too particular how he does it"

"For a Lieutenant of Homicide, you keep remarkably well informed about politics," English said. "All right, we'll take that as read. What has it got to do with Mary Savitt?"

"It could have plenty to do with her," Morilli said. "Doc Richards told me your brother died between nine and half past ten last night. He couldn't put it nearer than that. He says Mary Savitt died between ten o'clock and midnight Miss Hopper tells me she saw your brother leave Mary Savitt's apartment at nine forty-five last night. It's not going to take the D.A. long to arrive at the conclusion that these two had a suicide pact; that your brother murdered the girl, then went down to his office and shot himself. If he does arrive at that conclusion there's going to be quite a stink in the press, and it's going to come this way and bounce off you on to the Senator."

English sat still for a long moment, staring at Morilli, his eyes like granite.

"Why are you telling me all this, Lieutenant?" he asked at last.

Morilli lifted his shoulders; his small, dark eyes shifted away from English's face.

"No one but me knows it's murder, Mr. English. Doc Richards says it's suicide, but then he didn't see the stain on the carpet. If he knew about that, he'd change his mind, but he doesn't know, nor does the D.A."

"But they'll know when you have put in your report," English said.

"I guess they will unless I forget to mention the bloodstain."

English studied Morilli's white, expressionless face.

"There's Miss Hopper's evidence," he said. "You say she saw Roy leave the apartment. If she starts talking, the D.A. will investigate. He might even find the stain."

Morilli smiled.

"You don't have to worry about Miss Hopper," he said. "I've taken care of her. I happen to know what she does in her spare time. She wouldn't want to go on the stand. Some smart lawyer like Sam Crail might turn her inside out. I mentioned that fact to her. She isn't going to talk."

English leaned forward.

"You realize that the chances are a hundred to one that Roy killed the girl, don't you?" he said quietly. "If she was murdered, then someone is going to get away with it, if it wasn't Roy."

Morilli shrugged.

"It'll be your brother who murdered her if the D.A. hears about the stain, Mr. English. You can bet your bottom dollar on it. Either way, the killer gets away with it." He made a little gesture with his hand. "It's up to you. I'll put the stain in my report on your say-so, but since you've taken care of me in the past, I thought it was only right that I should give you a break when the chance came my way."

English looked at him.

"That's pretty nice of you, Lieutenant. I shan't forget it. Maybe it would be better to forget about the stain."

"Just as you say," Morilli said, getting to his feet. "Only too glad to be of help, Mr. English."

"Let me see," English said absently, "you have a bet to collect, haven't you? How much was it, Lieutenant?"

Morilli ran his thumb along his narrow, starkly black moustache before saying, "Five thousand, Mr. English."

English smiled.

"Was it as much as that?"

"I guess that was the sum," Morilli returned, his face expressionless.

"In that case, I'd better pay you. I always believe in paying my debts."

English stood up and went over to the wall safe. He opened it and took out two bundles of bills and tossed them on the desk. "I won't ask for a receipt."

"You won't need one," Morilli returned, picked up the two bundles, checked the amount with a quick flick of his fingers and crammed them away in his coat pockets.

"Of course, the D.A. might not trust your report," English said, going back to the desk and sitting down. "He might send up one of his people to check the room, and he might find the stain."

Morilli smiled.

"I like to kid myself that my service to you, Mr. English, is a pretty good one. The stain doesn't exist any more. I've fixed it." He moved over to the door. "Well, I'd better get over to the station house and write my report."

"So long, Lieutenant," English said. When Morilli had gone, English drew in a deep breath. "Well, I'll be damned!" he said softly. "The blackmailing sonofabitch!"



V

A few minutes after six o'clock, the same evening, and after English had finished dictating the last letter of the day, Lois put her head around the door to tell him Sam Crail was waiting, and wanted to see him.

English glanced at his wrist watch. He had promised to pick up Julie at half past six.

"Send him in," he said, "and go home yourself. You should have been gone hours ago."

"Yes, Mr. English," Lois said, and turned to beckon to Crail who was impatiently waiting behind the barrier.

"Come on in, Sam," English said as he caught sight of him. "You'd better ride down with me. I promised Julie I'd take her to a movie tonight."

"I don't imagine you'll want to go to any movie when you've heard what I'm going to tell you," Crail said, lowering his bulk into an armchair.

English stared at him.

"I've opened Roy's deposit box," Crail said. "There's twenty thousand dollars in it—in cash."

English gaped.

"Twenty thousand?"

"Yep, in hundred bills. How do you like that?"

"Where did he get it?"

Crail shook his head.

"Search me. I thought you'd want to know right away."

"Yes." English stood staring down at the carpet rubbing the back of his neck with his hand while his eyes brooded, then shrugging, he went over to the telephone, lifted it and said, "Get me Miss Clair's apartment, will you, Lois?"

Crail reached out and helped himself to a cigar.

"I could do with a drink if there's one within sight," he said. "I've had quite a day."

English motioned to the cabinet that stood against the wall.

"Help yourself." Then into the telephone, he went on, "Julie? I'm held up again. Yeah, I'm sorry but I can't make that movie. That's the way it is. Sam's just come in—about Roy. I'll tell you later. Sorry, Julie. I seem to be always standing you up. What are you going to do? Look, would you like , Harry to go with you?" He listened for a moment, frowning, then said, "Well, all right. I thought maybe you'd like a little company. I'll meet you at the club at nine. So long for now."

Crail passed him a whiskey and soda.

"You know your business best, Nick," he said, "but I'll be damned if I'd let an attractive girl like Julie go to the movies with Harry Vince—he's far too good-looking."

"Put your mind at rest, Sam. She isn't going with him. She prefers to wait until I can take her. What else did you find  in  the  deposit  box?"

Crail lit his cigar and blew carefully on the lighted end.

"Looks as if he was ready to skip," he said. "There were two airplane tickets to Los Angeles, the money, his will and a gold and platinum wedding ring."

"How the devil did he manage to lay his hands on all that money?" English asked.

"Why the devil did he commit suicide?" Crail said. 'That's the important question."

English nodded. He sat silent for several moments, then asked abruptly, "How did Corinne take it, Sam?"

Crail grimaced.

"It hit her hard, but she finally toed the line. I'm sorry for that girl. All right, she's dumb, but I didn't like telling her about Roy. But you'd better keep an eye on her. If she could do you dirt, she'd do it"

English lifted his broad shoulders.

"She and twenty thousand other people. So, what? Did the coroner take it all right?"

"Sure. All he wanted was a good reason, and I gave it to him—nervous depression brought on by overwork."

English reached forward and took a cigar.

"Mary Savitt was murdered, Sam."

Crail stiffened.

"What makes you say that?"

"I had a visit from Lieutenant Morilli. You know Morilli?" Crail nodded.

"He's worked it out as murder," English said, and went On to tell Crail about the bloodstain on the carpet.

"Was it Roy?" Crail asked, his fat face alarmed.

"Why do you say that?"

"I don't know," Crail returned, frowning. "The idea automatically jumped into my mind. Let me see: those two were lovers. They were going away. Maybe the girl suddenly decided it wasn't good enough. Roy was married. She would be left out on a limb. She says she's not going at the last moment. Roy loses his temper, and strangles her, then makes it look like suicide. He goes down to his office, gets cold feet, and shoots himself."

English smiled. His eyes turned frosty.

"You worked that one out fast enough."

"And so will the D.A.," Crail said slowly. "This is bad, Nick."

"Not as bad as it sounds. Morilli's agreed to keep his mouth shut. I gave him five thousand."

Crail whistled softly.

"That copper has big ideas."

"Anyone worth a damn has big ideas. He's pulled me out of a nasty jam."

"Do you think it was Roy?"

English shook his head.

"Not a chance. Roy wouldn't kill anyone. I knew him as well as I know myself. And another thing—Roy wouldn't kill himself either." He got to his feet and began to pace the floor. "If Mary Savitt was murdered, Roy was murdered too. How do you like that?"

"Why, that's crazy! The police say Roy shot himself. His prints . . ."

"Be your age, Sam. Someone faked Mary Savitt's suicide. Someone also faked Roy's suicide."

"Who would want to kill Roy?"

English spread out his hands.

"A lot of people, Sam. Roy wasn't an endearing type."

"That's right But who would want to kill the girl? Why the girl?"

"I don't know. Maybe Roy was blackmailing someone. Maybe Mary Savitt knew the details. They worked together in the office. Maybe the killer thought he'd be safe and wipe both out It could be, Sam."

Crail took a shot of whiskey.

"How about Corinne?" he asked. "The outraged wife angle. She has the motive if those two really were murdered."

English shook his head.

"No. Corinne wouldn't have had the strength to hoist that girl up against the bathroom door. It isn't the kind of setup a woman would tackle."

"Maybe she got someone to do it."

Again English shook his head.

"You're forgetting the twenty thousand. That could be blackmail money, Sam. Suppose Roy had been blackmailing someone in a big way, and decided to make a final killing before he went away. Suppose he turned the screw too far. Suppose the guy he was blackmailing decided he'd stop Roy once and for all, and while he was about it stop Mary Savitt too. If you're looking for a theory, try that one on for size."

Crail scratched the side of his fat neck with a carefully manicured fingernail.

"Are you going to talk to Morilli about this?"

"No. Do you think I want my brother branded as a blackmailer?"

Crail shrugged.

"Maybe the killer figured that the thing would be hushed up for just that reason. If he did, he's played it smart"

English showed his teeth in a mirthless smile.

"I wouldn't be surprised. Have you told Corinne about the money, Sam?"

Crail shook his head.

"I thought I'd better talk to you first."

"You did right. Sit on that money for a while. Keep it in the safe deposit box. In the meantime, go ahead with, that insurance idea of mine. See Corinne's fixed up, and let me know what I owe you. If that money turns out to be from blackmail, Corinne mustn't have anything to do with it."

"Okay, I'll fix it," Crail said. "One more thing, Nick. I've had an offer for the business. Four thousand—cash down! Want me to sell?"

English paused in his pacing and turned around.

"Who's the buyer?"

Crail shrugged.

"It's come through Hurst. He wouldn't say."

"He's a lawyer, isn't he?"

"That's what he calls himself. I have another name for him."

"Four thousand?"

"That's right. Corinne wants to sell."

"How does she know about it before I do?"

"Hurst went direct to her. He called at nine o'clock this morning. He didn't want to deal with me. Fortunately. She put him on to me. I told her to wait a few days. I said we were sure to get a better offer."

"Tell Hurst the business isn't for sale. I'll find a buyer for you, and the price is seven thousand. Tell Corinne, and give her your O.K. Do it first thing tomorrow morning."

"Who's the buyer?" Crail asked, staring.

"His name's Ed Leon. He'll call on you sometime tomorrow, give you his check, and all the details you want," English said. "And remember, Sam, I don't know Leon and he doesn't know me. Understand?"

"Now wait a minute, Nick. Don't keep me in the dark."

English came over and stood in front of Crail.

"Someone killed Roy. Someone wants to buy Roy's business in a hurry. I want to find out if the killer and the buyer are the same. Ed Leon's the guy to find out for me."

"Well, you know best, but what can you do if you find the killer?"

English's cold, brooding eyes stared at Crail for a long minute.

"This is a personal matter. Someone killed my brother. I don't like that. If the police can't take care of it, then I'll bury my own dead. That's what I can do about it."

Crail got to his feet.

"Watch out, Nick," he said seriously. "That kind of talk is dangerous. If you took my advice, you'd let it lie. You have too many commitments to start a caper like that. Let's face it. Roy didn't mean a thing to you. If you start to dig up his past, you may find something that you can't bury again. Suppose he was a blackmailer? Wouldn't it be better to forget about it? You've got the future to think of."

English slapped Crail on his broad back.

"I know you mean well, Sam, but even if Roy was a louse, he was my brother. No one's going to murder him and get away with it. I'll work it so that it remains a personal matter between me and the killer. Take care of Corinne, and I'll take care of Roy's killer."

When Crail had gone, English went into the outer office. Lois was still there, busily writing in English's appointment book.

"For the love of Pete! Didn't I tell you to go home hours ago?" English said, coming over to her.

"I guess you did, but I thought I'd stay on until you were ready to go."

"I don't know what I'd do without you, Lois," English said, standing beside her and looking down at her glossy, dark head.

She smiled, pleased.

"See if you can get Ed Leon. He's in Chicago some place. I don't know his number."

"I'll get him for you," Lois said, and turned briskly to the switchboard.

English went back to his office and closed the door. He began to pace up and down, his face thoughtful. Ten minutes later, the telephone rang.

"Mr. Leon's on the line now, Mr. English," Lois told him.

"Good girl. Put him on, will you?"

There was a click, and English said, "That you, Ed?"

"Well, if it isn't, some other louse is wearing my suit," a voice said in his ear. "What's on your mind?"

"I want you," English said. "Catch the first plane out tomorrow morning. I have a job that's right up your street."

"I don't want a job. I want to be left in peace," Leon said.

"I said I want you," English snapped. "This is a big job. Ed. Something right up your street or I wouldn't have called you. When you reach town, give me a call. I'll meet you somewhere. I don't want anyone to know you and I are working together. Do you understand?"

"Not a word," said Leon, sighing, "but if it's like that, I guess I'll have to do something about it. Is there any money in it for me?"

"Five grand," English said.

Leon gave a long, low whistle.

"The buzzing you hear in your ears is.my helicopter landing on your roof," he said excitedly, and hung up.

\vspace{2\nbs}
\ChapterDeco[c1]{\decoglyph{e9665}}
\clearpage
\thispagestyle{empty}

\begin{ChapterStart}
\vspace{3\nbs}
\ChapterSubtitle[l]{Chapter ch3}
\ChapterTitle[l]{ch3}
\end{ChapterStart}
\FirstLine{\noindent Julie had long ago learned never to keep English wait-ing, so she was dressed and ready to leave when he telephoned to tell her their movie date was off.}
    
When he hung up, she stood staring at her reflection in the mirror above the mantel. She told herself that she was looking her best, and the green scarf she had knotted at her throat, set off her eyes and her copper-colored hair with even more effect than she had imagined.

English had said that he would have dinner with her at the club at nine o'clock. She looked at her watch. It was now fifteen minutes past six. She had nearly three hours to wait. She dialed English's office. Lois answered, and Julie's mouth tightened. She disliked Lois intensely. Anyone could see that Lois was in love with English, except English himself, but then he would never notice a thing like that. The way Lois allowed English to make her his slave infuriated Julie. Besides, Julie knew Lois also disliked her. She was sure that Lois considered she wasn't good enough for English; and, whenever they met, Lois always seemed able to make Julie feel uncomfortable.

"Oh, Lois, this is Julie," Julie said brightly. "Is Harry there? I wanted some tickets for the show."

"Yes, he's here," Lois returned, her voice cold. "Hold on a moment, Miss Clair."

It was always "Miss Clair," although Julie had repeatedly asked Lois to call her by her first name.

There was a click on the line, and Harry's voice said "Hello, Julie. I was just going. Anything I can do?"

"I want two tickets for the show on Saturday, Harry," Julie said. "I was going to ask Nick to bring them, but our date's off. He won't be free until nine o'clock, and I'm meeting these people before then. Can you leave them at Nick's club, and I'll pick them up?"

"Of course. I'll be glad to. I'm on my way home now."

"Thanks, Harry," Julie said, and hung up.

Moving quickly, she picked up her handbag and gloves, and left the apartment. She rode down in the elevator, and asked the night porter to get her a taxi. While she waited, she lit a cigarette, and was annoyed to see that her hands were shaking.

"Where to, madam," the night porter asked, coming into the lobby.

"The Athletic Club."

The taxi made fast time through the evening traffic; and as the driver was about to turn into Western Avenue, Julie leaned forward, and said, "I've changed my mind. Drive me to Fifth and Tenth Street, please."

"Okay, miss," the driver said, and looked over his shoulder at her, grinning. "My old man always said it's because dames change their minds so often, they've got cleaner ones than us men."

"He's probably right," Julie said, and laughed.

After ten minutes' fast driving, the driver slowed down and pulled up.

"Here we are, miss."

Julie paid him, thanked him, and set off briskly along a quiet, dimly lit street that eventually led to the river. Every so often, she glanced over her shoulder, but the street was deserted, and she saw no one. Suddenly she slowed her pace, stopped and turned.

She looked quickly to right and left, and then up at the dark buildings opposite. Satisfied that no one was watching her, she went down a narrow, dimly lit alley that led to the waterfront.

A thin, white mist was coming off the river; and as she moved along in the dark shadows, a tug boat's siren hooted dismally from the other side of the river.

Again she paused. Again she looked right and left, then she stepped into the doorway of a tall, narrow building, pushed the door and stepped into a dark lobby. She moved without hesitation through the darkness as if she had been here so often she knew exactly where to go.

She heard a door open near her.

"Julie?"

"Yes."

She stepped into a dark room, and the door closed behind her. Then the light sprang up, and she turned, smiling, as Harry Vince caught her in his arms.

"What a bit of luck, darling," he said. "I was all set for a dull evening. I thought he was taking you to the movies."

She put her arms around his neck, and pressed her face against his.

"Sam turned up at the last moment," she said. "Oh, Harry, it seems such a long time. Kiss me."

Harry kissed her, holding her to him while his heart hammered against his side.

"We have such a little time, darling," she said pulling back and looking up at him. "Don't let's talk now. Don't let's waste a minute."

The clock on the mantel struck eight, and Julie sat up.

"Don't move, darling," Harry said out of the darkness, and his arm went around her. "You have an hour yet."

"No, only half an hour. I mustn't keep him waiting."

"Julie, we can't go on and on like this," Harry said, his face against hers. "Can't you talk to him? Can't you tell him you want to break off?"

He felt her stiffen, and there was a note of alarm in her voice when she said, "Nick would never give me up. You know that. Besides, how would we live? Don't let's go over this again. You know that we never get anywhere."

"But if he ever found out," Harry said, voicing a fear that had been haunting him for weeks, "he'd kill me. He wouldn't stop to think. He'd kill us first and let Crail fix it afterwards."

"Oh, darling, you're talking nonsense," Julie said, touching his face. "Nick wouldn't do a thing like that. He's far too wrapped up in his own career to risk ruining it. Why, he told me he wants the hospital named after him. Of course, he wouldn't."

Harry wasn't convinced.

"I'm not so sure. If he caught me . . ."

"But he won't—not now."

Harry half sat up.

"What do you mean—not now?"

"Now Roy's dead."

"What's Roy got to do with us?"

She hesitated, then speaking rapidly as if to force out the words, she said, "Roy knew. He'd been blackmailing me for the past six months."

Harry stiffened, and cold fear clutched at his heart He realized then how frightened he was of English.

He got off the bed and slipped on a dressing gown, then he turned on a lamp.

"Roy knew about us?" he repeated, and Julie saw that he had gone white.

She turned on her side, her hands covering her breasts.

"Yes, he knew."

Harry felt sick.

"When did it happen? You say he was blackmailing you?"

She nodded as she lit a cigarette.

"It's been hell, Harry. I thought I'd go out of my mind. One day, Roy came to see me. It must have been six or seven months ago. He said, 'You're going to come to my office every Friday with two hundred dollars, Julie. I can't make you come, of course, but I can tell Nick that you're having an affair with Harry Vince. Are you going to pay or am I going to tell Nick?' That's all there was to it. I was so frightened I didn't even ask him how he had found out. I said that I would pay, and every Friday since then I've paid."

"The louse!" Harry said furiously, clenching his fists. "So it is true. They're saying he was a blackmailer. The dirty, rotten louse!"

"You can't imagine how relieved I was when Nick told me that he had shot himself," Julie said. "It's been a nightmare these past months."

"Do you think he told Corinne?" Harry asked. "Suppose she goes to Nick?"

"Why should he tell her?" Julie said a little impatiently. "It wasn't anything to be proud of. Besides Corinne would have told Nick before now if Roy had told her. I'm sure only he and that girl of his knew, and they're both dead."

Harry began to pace up and down.

"Julie, can't we go away together?" he asked abruptly. "Must we go on taking chances? It's not as if he's married to you."

She paused, with one stocking on and the other in her hand.

"What would happen to us? Nick's so powerful. I'd never get another engagement, and you'd never get another job. He would see to that. He'd find us and he'd hound us for life. Be patient, Harry. Let's be thankful we can see each other like this from time to time. Something may turn up. Let's not take any chances."

"But this is dangerous," Harry said. "We're cheating him now, but if we went away together, he couldn't accuse us of that."

"He'd find us, Harry. He would never let me go."

"He's not God," Harry said. "I know he's powerful, but, damn it, he couldn't stop me earning a living. That's nonsense, Julie."

Julie slipped into her dress, put on her shoes and crossed over to the dressing table. She sat down and began to make up her face.

"Say something, Julie," Harry said anxiously. "Don't you see that this is even more dangerous than going away?"

She turned and faced him.

"All right, Harry. I'll tell you the truth. For weeks now I knew that we should have told Nick and gone away together, but I can't face it. I can't give up all the things that mean so much to me. I don't suppose you have ever thought what it means to me to have Nick behind me. If it wasn't for Nick, I wouldn't be singing at the best night club in town. I wouldn't have that lovely penthouse or all the clothes I've got. I wouldn't have accounts at all the big stores. If I walked out on Nick, I would walk into different life, and I wouldn't like it."

Harry winced, and sat down. He stared into the fire, his right fist grinding into the palm of his left hand.

"I see," he said in a flat, tired voice. "No, I hadn't thought of it quite like that, Julie."

"I want my cake and I want to eat it," Julie said, not looking at him. "I love you, Harry, more than any other man on earth. Sometimes I wish that I'd never met you. It would have saved so much pain and worry. But I did meet you, and I did fall in love with you, so there it is. You've got to take me as I am, or leave me. Now you know the truth, Harry, you'd better tell me whether you want to see me again. I wouldn't blame you if you now hate the sight of me, as I hate the sight of myself. I'm the worst kind of a bitch but I can't help it. I would do anything for you, except give up the life Nick gives me. I don't suppose you believe that, but I would. I would even keep away from you if you wanted me to, and that would be the hardest thing of all I'd do for you."

Harry got up and went over to her; bending, he lifted her face, and kissed her.

"I'm not going to give you up, Julie. You mean too much to me for that. All right, darling, we'll go" on as we've gone on. Perhaps one day something will turn up, and we can get out of this mess."

Julie got to her feet and clung to him.

"Darling Harry, I love you so, and I'll try to make you happy. Be patient. I'm sure it will come right in the end. Now, darling, I must go. I'll come again as soon as I can. Get my coat, will you? I'm going to be late if I don't hurry."

A few minutes later, Julie moved quietly to the mouth of the alley and looked quickly to the right and left The street was deserted. Moving forward briskly, she went in search of a taxi.

In a dark doorway, a youngish man in a brown suit and a brown slouch hat, stood with his back against a wall, watching her, his jaws moving slowly as he chewed. He remained in the shadows until she was out of sight, then he came out of the doorway and walked quickly towards the river, his lips pursed in a soundless whistle.



II

Ed Leon took possession of the Alert Agency two days after English had called him from Chicago.

Leon was tall and rangy, all legs and arms, and he had a deceptive appearance that led most people to assume that he was a harmless nitwit. He had a pleasant sun-tanned face, and at first glance you might have mistaken him for a not-too-prosperous farmer up for the day to see the sights of the city. He wore his clothes as if he had slept in them, not for one night but for many nights, and he had a habit of wearing an old tattered slouch hat far at the back of his head. No one would have believed that he was one of the smartest private eyes in the country.

As he wandered around the small, shabby office that had once belonged to Roy English, Leon wasn't pleased that he had allowed himself to be talked into taking this assignment The pay was good, but he didn't relish spending much time in these two rooms after the luxury of his air-conditioned office in Chicago.

He pulled at his long nose thoughtfully as he wandered around the room, his face tense, his eyes missing nothing. He spent two hours going through the files, examining drawers and cupboards, with the methodical care he had developed after years of experience that had taught him that nothing was unimportant, that there was a reason for everything; and that if you kept looking, that sooner or later, you would find something to interest you.

It was not until he had examined the fireplace that he made any worth-while discovery. He found a small object lodged in the chimney that made him raise his heavy eyebrows and take from his pocket a pencil-thin flashlight. It was a small microphone. The wires attached to it went through a crack in the chimney and into the outer office. He strolled into the outer office, and after a lengthy search, found the wires again neatly hidden between the floor boards. He traced them across the room to the door leading into the passage. From the look of the microphone, it had been installed for some time.

Leon wondered if the microphone was still alive, and if someone would be interested in listening to his conversations. At a more convenient time—when the building was closed for the night—he decided that he would make an attempt to trace the wires further.

English had told him that the janitor, Tom Calhoun, seemed cooperative, and Leon thought it might be an idea to go down and talk to him before settling down to a day's work.

He left the office, locking the door behind him, and took the elevator to the basement.

He found Tom Calhoun in the boiler room industriously carving a model boat from a chunk of soft wood, with the aid of a murderous looking pocket knife. He eyed Leon with mild interest and gave him a brief nod.

"Morning," he said. "Anything I can do for you?"

Leon hooked a chair towards him and folded his long length into it.

"I got an ulcer," he said. "At noon every day I give it a feed of whiskey. The trouble is I don't approve of drinking alone. Once a guy gets into the habit of secret drinking, he might just as well step into his box and let them screw him down. I thought maybe you might care to join me, but if you're a non-drinking man, just say the word and 111 go elsewhere."

Calhoun laid down the boat and sat forward.

"You've come to the right man, mister, but I wouldn't have thought whiskey would have done an ulcer much good."

Leon produced a half-pint flask of Johnny Walker and waved it in the air.

"A guy has got to show his independence," he said. "If I gave my ulcer what was good for it, it'd stay with me the rest of my days. The whiskey's good for me so I drink it. Got two glasses?"

Calhoun produced two paper cups from a shelf.

"Best I can do," he said apologetically after blowing the dust from them. He watched Leon pour two stiff shots, and eagerly took one of the cups and sniffed at it. "Good whiskey, mister. Here's to you," and he took a long pull, sighed, smacked his lips, wiped his mouth with the back of his hand and set the cup down.

Leon scarcely tasted his, but leaned forward to refill Calhoun's cup.

"I'm your new tenant," he said. "The name's Ed Leon. I've taken over the Alert Agency."

Calhoun looked surprised.

"Glad to know you. I'm Tom Calhoun. Alert Agency, huh? That's fast work."

"My mother was a fast woman," Leon said lightly. "It runs in the family." He frowned, shook his head, went on, "Business seems a little flat this morning."

"It'll pick up," Calhoun said encouragingly and took another drink. "I reckon that guy English knew what he was doing. He kept mighty busy. Why he shot himself beats me. Of course, that shooting might slow things down for you, but not for long."

Leon took out two cigarettes, rolled one across the table and lit the other.

"I was beginning to wonder if I'd been sold a dog. You really think that's a good business?"

"I'm sure of it," Calhoun said. "It stands to reason. Look at the people who went up to see him—as many as thirty people on some days."

"Well, well," Leon went on and lifted his feet on to the table. "So it looks like I've come into a good business. Who were these people who came to see English?"

Calhoun lifted his big, lumpy shoulders.

"I wouldn't know. Some of them would come every week. Some of them were trash, but most of them looked as if they had a sack of dough."

"Were you in the building when he knocked himself off?" Leon asked casually and leaned forward to fill Calhoun's paper cup again.

"Sure," Calhoun said. "Go easy on that stuff, mister."

"Don't tell me that a big boy like you can't drink a little Scotch," Leon said. "They tell me that he shot himself between nine and ten-thirty. Did anyone call on him around that time?"

Three people went up to the sixth floor. But I wouldn't know if they called on him. Why?"

"I'm always asking questions," Leon said, and closed his eyes. "I like the sound of my own voice. You should see the way the broads fall over when I whisper in their ears." He opened his eyes and stared at Calhoun. "Who were those three?"

'Two men and a girl," Calhoun told him. "I took them up to the sixth floor myself. I'd seen the girl before but not the two men."

"Who else is on the sixth floor?"

"Well, there's the Associated News Service. Maybe you've already heard their teletypers. Hell of a racket they make. Then there's your office, and then there's Miss Windsor."

"What's she do?"

"She's what they call a silhouette artist," Calhoun told him. "She cuts out your silhouette in paper, mounts and frames it. What else she does up there, I don't ask, but I do know she has only men clients."

Leon perked up, his eyes showing interest.

"Like that, is it?" he said. "And my next door neighbor. Well, well, maybe I'd better go along and let her look at my silhouette. She might even show me hers."

"She's a nice dish," Calhoun said, "but it's strictly for cash. No—I prefer to waste my money on horses, but it takes all types to make a world."

"Don't go philosophical on me," Leon said. "Let's get back to these two guys and a girl. They could have called on either Miss Windsor, this news service, or on English—that right?"

"The girl went to see English," Calhoun said. "I've seen her a number of times before."

"What's she look like?"

"She had sort of light brown hair, a good figure, and she was pretty enough to be in the movies."

"What a description! Do you realize that there are two million dames within a radius of thirty miles in this goddam city who look just like that? How was she dressed?"

"She was pretty smart," Calhoun said, screwing up his eyes as if he was trying to create a picture of the girl in his mind. "She wore a black coat with white wide lapels, black skirt, black and white gloves, and a black and white skull cap affair for a hat. And she had one of those charm bracelets. You know the ones—a gold chain with little charms hanging from it."

Leon nodded approvingly.

"Now you're talking. That's fine. You'll make a detective yet. Now about the two guys?"

"One of them was just a punk—a kid about eighteen. He had on a leather jacket and flannel trousers. He had a parcel under his arm. I have an idea that he was going to the news service, but the other guy was in the money. He was a young fella, around twenty-seven or eight, in a brown suit and a brown slouch hat I noticed that he wore his handkerchief in his sleeve. He was chewing gum. When a guy can afford clothes like that, he shouldn't chew gum."

Leon lowered his feet to the floor.

"Just to get the record straight, when did these people arrive —who came first?"

"The girl. Then the guy in the leather jacket. Then the guy in the brown suit."

"What time did the girl arrive?"

"It was nine-fifty," Calhoun said. "I know because she asked me the time."

"And the other two?"

"The fella in the leather jacket was waiting to go up as I came down from taking the girl up. The guy in the brown suit came along about fifteen minutes later."

"Did you see any of them leave?"

Calhoun shook his head.

"I take them up, but I don't reckon to bring them down. That's what they've got legs for."

"I guess that's right," Leon said, and stood up. "The self-service elevator wasn't working?"

"I lock it at seven o'clock. I like to know who comes into the building after that time."

Leon nodded again.

"Well, that's very interesting. You'd better keep what's left of that half-pint. If I took it up with me I'd be laying myself open to temptation. I guess I'd better go along and call on Miss Windsor. Nothing like being neighborly. Who knows? She might be lonely."

"That dame's never lonely," Calhoun said, "But watch it It's strictly for cash."

Leon propelled his lanky frame to the door.

"Not for me, brother," he said, pausing at the door. "I'm going to explain to her the principles of lend-lease."



III

As Leon stepped out of the elevator, he saw a short, shabby-looking man in a wrinkled overcoat and a dusty gray hat, knocking on his office door.

The man looked quickly over his shoulder as he heard the grill close. He was a man of about sixty, gray faced, tired-looking with a scrubby, gray moustache. He looked nervously at Leon as he wandered along the passage, then he rapped on the door again, and turned the knob. Finding the door locked, he backed away, obviously surprised.

"Hello, paL" Leon said, coming up to him. "Looking for me?"

The man gave Leon a quick look, and backed against the banister rail.

"No, thank you," he said. "I wanted to see Mr. English. Never mind. I'll come again. He doesn't seem to be in."

"Maybe I can do something," Leon said. "I'm looking after Mr. English's affairs at the moment." He took out his door key and pushed it into the lock. "Come on in."

"It's all right," the man said, and his tired, bloodshot eyes quivered. "I wanted to see Mr. English. It's a personal matter. Thanks all the same," and turning, he walked hurriedly towards the head of the stairs.

Leon started after him, then stopped as he remembered the hidden microphone in his office. That room wasn't the place to persuade someone to talk. He turned and ran for the elevator, stepped into it, and set the cage down to the ground floor.

As he entered the lobby, he could hear the shabby man running down the stairs. He had one more flight to go before he reached the lobby. Moving quickly, Leon went into the street and ducked into a near-by doorway.

He watched the shabby man come out into the spring sunshine and set off along the street. He moved slowly, his feet dragging, and he walked for some time toward Twenty-second Street.

Leon moved along behind him, taking care to keep out of sight. He saw the man pause outside a restaurant, hesitate, then walk in.

As Leon passed the restaurant, he glanced in. There were only three or four people in the place, and he spotted the man sitting at a table at the far end of the room.

Leon waited a few seconds, then pushed open the door and walked in.

The man glanced up, but didn't seem to recognize Leon. He was stirring a cup of coffee aimlessly, his face frowning and his eyes worried.

Leon inspected the other people in the place. There were two men at a table by the door; a girl reading a paper-backed book at a table near the counter, and a man hidden behind an open newspaper opposite where the shabby man was sitting.

Leon sat down at the table, and the man looked up and stared at him. Recognition swam into his eyes, and his face went a grayish white. He started to get up, then dropped back into his chair, nearly upsetting his coffee as he did so.

"Keep your shirt on," Leon said, and smiled. "I'm not going to bite you." He turned and waved to the girl behind the counter. "Bring me a cup of coffee, honey."

The shabby man ran his tongue over his dry lips.

"See here, mister," he said with feeble fierceness, "you've no right to follow me. Mr. English and me have a private deal on. It's none of your business."

"It is my business," Leon said. "I've taken over the agency. English isn't with us any more."

The man stared at him.

"I wasn't told," he muttered. "I've got nothing to say to you."

"I'm telling you," Leon said, and stirred his coffee. "I'm in charge now. Come on, what's it all about?"

"You mean you're taking the money in the future?"

"Don't I keep telling you?" Leon said roughly. "What do you want me to do—set it to music and sing it to you?"

"Where's Mr. English then?"

"He's gone to a warmer climate. Are you going to deal with me or do you want me to get tough?"

"That's all right," the shabby man said hurriedly. "I just didn't know." He took out a soiled envelope and slid it across the table. "Here it is. Now I've got to go."

"Sit still!" Leon snapped, and picked up the envelope. On it was scribbled: "*From Joe Hennessey. \$10.*"

"Are you Hennessey?" he asked.

The man nodded.

Leon ripped open the envelope and took out two five-dollar bills. He studied Hennessey for a moment.

"What's this for?" he asked at last.

"What do you mean? It's all right, isn't it?"

"Maybe. I wouldn't know. What are you giving me this for?"

Hennessey's face began to glisten with sweat.

"Give me back that money!" he said, keeping his voice low. "I knew you were a phoney. Give it back to mel"

Leon slid the money across the table.

"Relax. I don't want it," he said soothingly. "I just want to know why you're parting with this dough. From the look of you, you can't afford to give away ten bucks."

"I can't!" Hennessey said bitterly. He stared at the two bills lying before him, not touching them. "I'm not going to talk to you! I don't know who you are." He began to push back his chair.

"Take it easy," Leon said, and flicked one of his cards on to the table. "That's who I am, and I can help you if you'll let me."

"A copper!" Hennessey said when he had looked at the card. His eyes went dark with alarm. "There's nothing you can do for me, mister."

"Sit still!" Leon said, and leaning forward, went on, "English is dead. He shot himself three nights ago. Don't you read the newspapers?"

Hennessey stiffened, his fists clenched.

"He's dead, all right," Leon said. "Now, listen to me: I'm investigating his death. You can help me. Why are you paying him  money?"

Hennessey hesitated, then shook his head.

"It's nothing to do with you, mister," he said. "The less said about it, the better. I think I'll be getting along now."

"Wait a minute," Leon said, his voice hardening. "Do you want me to take you down to the station? You could be held as a material witness. You'd better talk, and talk fast. English was murdered!"

Hennessey went white again.

"You said he shot himself."

"Now I'm telling you he was murdered. Why were you paying him money?"

"He was blackmailing me," Hennessey blurted out. "I've paid him ten dollars a week for eleven months, and if he hadn't died, I would have gone on paying him."

"What had he got on you?"

Hennessey hesitated, then he said. "Something I did years ago—something bad. He was going to tell my wife."

"Were all the other people calling on English paying blackmail money?" Leon asked.

"I guess so. I never talked to any of them, but I've seen the same faces every time I went to that office."

"How did English contact you?" Leon said.

"A fella came to my shop. He told me he knew about what Td done, and if I didn't pay ten dollars a week he would tell my wife. He told me to take the money every Thursday to the Alert Agency, and that's what I did."

"It wasn't English?"

Hennessey shook his head.

"No, but English took the money. This other fella was the outside man. I reckon English was the boss."

"What was this man like?"

"A big, tough-looking guy. He had a scar from his right ear to his mouth; looked like an old razor wound, and he had a cast in his left eye. He was big and powerful, not the kind of fella you'd argue with."

"Let's have your address," Leon said. "I might want to talk to you again."

"I'm at 27 Eastern Street."

"Okay, pal, now relax. There's nothing for you to worry about. English is dead. Go home and forget about him and blackmail. Forget it ever happened."

"You mean I don't have to pay any more money?"

Leon reached out and patted his arm.

"No. If the tough guy shows up, stall him and tell me. Ill take care of him, and 111 see you're in the clear."

Hennessey got slowly to his feet He looked suddenly five years younger.

"You don't know what this means to me," he said, a break in his voice. "Ten dollars was skinning me."

"Consider it taken care of," Leon said. "I'm here to help you if you want help, and listen, I don't promise anything, but I may be able to get some of your money back for you. Ten dollars a week for eleven months. Was that it?"

Hennessey stared at him as if he couldn't believe his ears.

"Yes, that's right," he said hoarsely.

"Don't count on it," Leon said, "but 111 see what I can.do."

He got up, paid for the two coffees and went out the door with Hennessey.

The man who had been sitting at the table opposite Hennessey's and who had been hidden behind a newspaper, lowered the paper and looked after Leon, his jaws moving rhythmically as he chewed. He put the paper aside, and got up, crossed over to the counter and gave the girl a couple of nickels.

She smiled archly at him, impressed by his faultlessly fitting brown suit and the silk handkerchief he wore tucked in his sleeve.

He looked at her, and her smile faltered. She had never seen such eyes. They were amber-colored with small pupils, and the whites were the color of blue-white porcelain. They were as compelling and as expressionless as the eyes of an owl, and looking into them she felt a little chill run up her spine.

He watched her reaction with cat-like interest, then turned and moved briskly to the door.

He stood looking after Leon and Hennessey as they walked down the street together. Then he ran across the road to where a dusty, shabby Packard was parked. He got into the car, started the engine and waited.

He watched Hennessey and Leon pause for a moment at the corner. Leon shook hands with Hennessey, and then went off uptown. Hennessey walked away in the opposite direction.

The man in the brown suit shifted into gear and sent the car rolling slowly after Hennessey.

Hennessey walked with a light step. He was anxious to get back to his shop.

The man in the brown suit drove along near the curb, his amber-colored eyes fixed on Hennessey's distant back. He drove patiently, keeping out of the way of the faster traffic, and every now and then he looked searchingly at the numbers of the stores as if he were hunting for a particular number to explain his slow crawl.

At the end of the street, there was a narrow alley. It was an alley dwarfed by high warehouses, and even in daylight, it was shadowy and dark. Few people used it, but to save his legs, Hennessey always went home that way.

As Hennessey began to walk down the long, narrow alley, he heard a car behind him, and looking round sharply, he saw the Packard.

No cars ever came this way. The alley was far too narrow. There was only a foot clearance each side of the car's fender. Hennessey realized that the car was coming after him, and fear clutched at his heart, for a moment paralyzing him.

He stood in the middle of the alley, hesitating, looking frantically to right and left. Ahead of him, some two hundred feet, was an archway leading to a courtyard. The archway was too narrow for a car. He began to run toward the archway, his old blue overcoat slapping, and his breath rattling at the back of his throat. He was too old and stiff to make much headway, but he did his best.

The man in the brown suit pushed down on the gas pedal and sent the Packard surging forward. For a few seconds, the running, stumbling man and the swiftly moving car seemed to remain equi-distant. Hennessey looked over his shoulder. He saw the car rushing down on him. He cried out in fear and desperation as he made a frantic effort to reach the archway. He was within ten yards of it when the car hit him.

It hit him the way a charging bull hits a matador. It threw him high into the air and forward, so that he came down on his back within a few yards of the car.

The man in the brown suit slammed on his brakes and stopped within a yard of Hennessey who turned his head to stare at the car, seeing only the two wheels and the dusty hood. A thin trickle of blood ran out of his mouth, and he felt a terrible pain tearing at his chest.

The man in the brown suit glanced into the driving mirror. He could see the dim length of the alley stretching out behind him. It was empty and silent. He shifted gears and reversed the car, stopping it when it was some twenty to thirty feet from where Hennessey was lying, then he shifted into second, let in the clutch and sent the car forward slowly, leaning out of the window so that he could see what he was doing. Hennessey screamed wildly as the car came toward him.

The man in the brown suit moved the steering wheel a trifle. He leaned far out of the car. Hennessey looked up into the big amber-colored eyes that were as indifferent to him and as expressionless as the headlamps of the car. The left front wheel went over Hennessey's upturned face. Keeping his course, the man in the brown suit felt the rear wheel lift and thud down, and he gave a pleased little nod. He slightly increased his speed, reached the end of the alley, swung into the main street and headed uptown.



IV

Nick English paced the floor of his office, his hands clasped behind his back, his chin down, his face hard and frowning.

The time was six minutes after seven. Everyone, including Lois, had gone home, and only he and Ed Leon remained in the office.

He listened to Leon's report with growing alarm, although he didn't reveal the fact to Leon.

Leon lolled in an armchair, his long fingers laced around one knee, his hat rested on the back of his head, and he talked in a low voice, marshaling his facts and bringing them out clearly.

"Well, I guess that's about all," he wound up. "I don't know how you feel about Hennessey, Nick, but I gave him a hint that he might get some of his money back. He's been bled for close to five hundred dollars."

"I'll write a check," English said, and moved over to the 52



desk. "I just can't believe it! Organized blackmail! Who's this

man with the scar?"	\-'

"I don't know, but I'll find out if you want me to. From what Hennessey said, he was just Roy's stooge."

"I don't believe that either. Roy hadn't it in him to organize a racket like this. If anyone was the stooge, he was the one."

Leon didn't say anything.

"If this gets out, Ed, I'm sunk," English went on. "But these people should be found and paid back. This man with the scar should be put out of business. Maybe he was the one who shot Roy."

"I've checked on that angle," Leon said. "Three people went up to the sixth floor around the time Roy was supposed to have shot himself. Two men and a girl. The girl was the only one that Calhoun was sure had gone to see English. The other two called on the News Service Agency. I checked on them."

English frowned.

"Funny time to call, wasn't it?"

"That's what I thought, but the manager of the News Service said that they never close and people come in at all hours. Still, this guy might be worth checking on. He might have gone first to this news service and then along to shoot Roy. It would have given him an alibi if Calhoun had reported his presence to Morilli."

"Is it likely a killer would have used the elevator?" English said. "I doubt if he or the girl shot Roy. The killer wouldn't want to be seen. He would slip into the building and walk upstairs."

"Maybe," Leon said, "but on the other hand he might be a smooth operator, and anticipate that that was what people would think. He might figure that he would be unlikely to be suspected if he used the elevator and let Calhoun have a good look at him so long as he could prove he had been to the news service."

English nodded.

"Yes, that's smart thinking. You'd better see if you can find out something about him. Have you got a description?"

"Yep, and a good one. He's around twenty-seven or \-eight, and he wore a brown suit and a brown hat. He carries a silk handkerchief tucked up his sleeve, and he chews gum. But for all that, it won't be easy to find him."

"Think so?" English said. "I think I can give you his name and tell you where he lives right now. If I'm not mistaken, his name's Roger Sherman and he lives in Crown Court"

Leon stared at him.

"A friend of yours?"

English shook his head.

"No, I haven't even spoken to him, but I've seen him often enough. He has an apartment on the same floor as mine. The description fits him like a glove."

"What does he do for a living?"

"I don't know. Nothing as far as I can see. He's a dilettante. He's interested in art and music. You'll always find him at previews of fashionable galleries, and he has a box for all important concerts. I might have a talk with him myself. I can't imagine that he even knew Roy—let alone wanted to shoot him, but he might know something. I think you can leave him to me."

Leon nodded, slowly got to his feet and stretched.

"Well, I guess I'll get along. I want to find somewhere to sleep. The hotel I'm staying at gives me the creeps. The room I've got is so small, I have to use a folding toothbrush."

"What about the girl who called on Roy? You haven't told me about her yet," English said.

"According to Calhoun, she was good-looking enough to be in the movies," Leon said, stubbing out bis cigarette. "He said she was wearing a black and white small cap, a black suit with wide, white lapels, black and white gloves, and a charm bracelet."

English paused in his pacing and looked sharply at Leon.

"A charm bracelet?"

"That's right—a gold chain with little charms hanging from it."

"Well, I'll be doubled damned!" English said under his breath, and he ran his fingers through his hair.

"Don't say you know her too?"

"I don't know. I might. I'll let you know, Ed. Here, wait a minute, let me write a check for Hennessey. Cash it yourself and give him the money, and don't let him know where it comes from."

"I'll do just that thing."

Leon waited until English had written the check, slipped it into his pocket and made for the door.

"I guess I'll go back to the office and see if I can find out where those microphone wires lead to," he said. "If I make a startling discovery, I'll phone you. Where will you be?"

"Phone me at my apartment after midnight," English said, glancing at his watch. "Or maybe you'd better leave it until tomorrow morning."

"Ill do that," Leon said. "So long for now."

When he had gone, English turned off the lights, put on his overcoat and went down to where Chuck was waiting with the car.

"Miss Clair's apartment," English said curtly.

"Want the evening paper, boss?" Chuck asked, offering it

"Thanks," English returned, got into the car and turned on the reading lamp. He read through the paper as Chuck drove towards Riverside Drive. A small item caught his eye. He read it, frowning. Read it again, then said, "Get me to a telephone quickly, Chuck."

"One just ahead," Chuck said, swung over to the curb, and pulled up outside a drugstore.

English got out of the car and hurried across the sidewalk to a row of phone booths. He called the Alert Agency.

Leon answered.

"I've only just this second got in," he said, startled to hear English's voice so soon.

"That old man you were telling me about—was his name Joe Hennessey?"

"That's right. Why?"

"27 Eastern Street?"

"Yep."

"He's dead. It's in the paper. He was killed by a hit-and-run artist in an alley that's barred to traffic."

"For crying out loud!"

"Listen, Ed, I don't like the sound of this. It may be a coincidence but I don't think so. It seems to me that you two were seen together, and someone decided that Hennessey talked too much. Check into this business."

"Okay, I'll do that right away," Leon said. "Where can I call you if anything goes wrong?"

"I'll be with Miss Clair," English said. He gave Leon Julie's number, and hung up.

Ten minutes later he was letting himself into Julie's apartment He stood in the lobby, frowning.

"Julie?"

There was no answer, and taking off his hat and coat, he went into the living room. He crossed over to the bedroom, pushed out the door and turned on the lights.

He stood looking around the room, then walked across to the big built-in wardrobe, opened the double doors and glanced in. Among the many dresses, suits and coats hanging in an orderly line, he spotted the black suit with the wide white lapels. Above it, on a shelf was a pair of black and white gloves, and a small black and white hat.

He closed the doors, stroked his jaw thoughtfully and returned to the sitting room. He stirred the fire, went over to the liquor cabinet and poured himself a rye. Then he sat down before the fire, lit a cigarette and waited, his eyes brooding and cold.

Some ten minutes later, he heard Julie come in.

"Oh, Nick!" she said as she opened the sitting room door. "Have you been waiting long? I had a rehearsal, and there was some dope who couldn't get anything right. I'm sorry I'm late."

English got up and kissed her.

"That's all right. How are you, Julie? You're looking pretty good."

"I'm fine, but tired," Julie said, taking off her coat and sinking into an armchair. "I'm dying for a drink. Would you get me a martini?"

He began to mix the martini, shooting a searching glance at her from time to time. He thought she looked tired, and the usual sparkle in her eyes was missing.

"What's been happening to you?" she asked, leaning back and closing her eyes. "Have you had a good day?"

"Oh, all right," English said, and came over and gave her the martini. "I hope that's not too dry."

"It's perfect," Julie said, drank half of the martini, sighed and put down the glass. "What are you doing tonight?"

"I'm afraid I have a date in about an hour," English said. "Something important. Sorry, Julie."

"Oh, well, never mind. I don't have to be at the club until ten-thirty. I'll take a bath and a snooze. I don't feel like having dinner. I'll have something when I get back."

English moved slowly over to the fire.

"Julie, why did you go and see Roy the night he died?" he asked quietly.

He saw her stiffen and go white. She looked at him, her eyes opening wide, and it startled him to see the utter fear in her eyes. "Now, look, Julie," he went on, "you mustn't ever be frightened of me. I know you went there, and I want to know why, but that doesn't mean you have to be frightened."

"No, I—I suppose not," Julie said huskily, and made an effort to control herself. How much did he know? she asked herself, her mind cold with panic. Did he know about Harry? Was this only a shot in the dark? "You frightened me, Nick. I didn't think anyone knew about that."

He smiled.

"No one does except me. Was Roy blackmailing you?"

For a moment, Julie thought she was going to faint.

"I found out this afternoon that Roy had been blackmailing a number of people," English went on. "You were seen going to the sixth floor, and I recognized the description of that suit. I wondered if you, too, were paying Roy money."

Was that all he knew? she wondered, her tongue touching her dry lips.

"Yes, he was blackmailing me," she said, and her mind darted about trying to think of a reason that he would believe.

"For God's sake!" English exclaimed. "Why didn't you tell me? I would have broken his neck!"

"I didn't want to tell you. I was too ashamed of myself."

"You needn't have told me why he was blackmailing you. I don't want you to tell me now. All I'm interested in is the fact that he was blackmailing you."

Julie went limp. He didn't know! The relief was so great, she wanted to cry.

"He's been blackmailing me for the past six months," she said. "I had to go to his office and pay him two hundred dollars every week."

"You should have told me," English said, his face hard. "The little rat! I knew he was a weakling and a louse, but I never realized he had sunk as low as that. Julie, for God's sake, don't hide things like that from me again. I could have fixed Roy in a moment."

"I couldn't tell you," Julie said. "But I want to tell you now."

She realized she had to tell him some story. If she didn't, sooner or later he would become suspicious of her. He might even have her watched. She wasn't deceiving herself that his present sympathetic attitude would last. She knew him too well for that. She remembered sharing a room with a girl years ago in Boston. She remembered what had happened to the girl, and unable to think of a convincing story, she decided to borrow from the girl's experience.

"You don't have to tell me anything," English said, and came over and sat on the arm of her chair. He put his arm around her shoulders. "Is there anything I can do to help?"

"Not now. It's past history," Julie said. "It was when I was in Boston—years ago. I was only seventeen, and I was hard up. I got an audition. It came out of the blue just when I thought I would have to give up and go home. I had nothing decent to wear. I knew if I went as I was, I wouldn't get the job. The woman who kept the boarding house always kept money in the house. I stole it. I thought I would be able to get it back before she found out, but she caught me in the act She sent for the police, and I was given a week in jaiL"

English patted her shoulder.

"You needn't have told me that, Julie. So what? Most of us have done something at some time or the other that could have landed us in jail if we were caught. You were unlucky! Do you mean to tell me that Roy was blackmailing you for that?"

"He threatened to tell the papers. I would have lost my job, and then they would have got at you through me, Nick."

English's eyes hardened.

"I guess that's right. Does anyone else know about this?"

She shook her head.

"Then we'll forget it. How much did you pay Roy in all?"

"I don't want to discuss that part of it," Julie said quickly.

"Nonsense! I intend to return the money to you. How much was it?"

"Please, Nick, I don't want you to do that"

"What was it—five thousand?"

"Yes, about that, but I won't take it. I mean that. It's nothing to do with you. I've paid, and I've forgotten about it."

"Well see," English said and stood up. "Julie, when you went up there, was Roy alive?"

She nodded.

"Yes, he was alive."

"You realize, don't you, that a few minutes after you had gone, he died?"

Again she nodded, and her hands turned into fists.

"Would you say that he looked like a man who was about to commit suicide?"

"Oh, no. He was smiling and joking. He even tried to make a pass at me. It was the first time I had been alone with him in the office. Usually, the girl was there, too."

English's mouth tightened.

"What happened?"

"He tried to kiss me, but I got out of his way. I gave him the money and left."

"You gave him the money? Two hundred dollars?"

"Yes."

"You're sure about that, Julie? It's important."

"Yes, I gave it to him."

"It wasn't found. He had only four dollars on him. Lois went through the office very carefully. She didn't find any money anywhere."

"Well, I gave it to him. He put it on his desk and put a paper weight on top of it."

English stroked his jaw, his eyes brooding.

"I think that that about clinches it," he said, half to himself. "Roy was murdered."

Julie closed her eyes.

"Did you see anyone or hear anything when you were up there?" English went on, watching her.

"No, nothing. Only the machines in the office along the passage. They were making a lot of noise."

"Well, someone shot him and took the money," English said. "It didn't walk out of the office on its own. Someone took it."

"What will happen, Nick?" she asked, her eyes scared.

"I have a man working on it," English said, tossing his cigarette into the fire. "There's nothing for you to worry about, Julie. No one knows you went up there, and no one is going to know. You can forget about it."

"But if someone murdered him, shouldn't the police be told?"

"If it gets out that Roy ran an organized blackmail racket, I'm sunk," English said quietly. "I'm not telling the police a thing. It's up to them to find out for themselves. My man may find the killer, and if he does, we shall have to decide what to do with them. There's nothing for you to worry about in any way." He went over to her and took her hand in his. "Now I've got to run along, Julie. Have a rest and forget about this. I'll see you tomorrow."

"Yes, Nick."

She got up and went with him into the lobby. While he was putting on his coat, she stood near him, watching him, her eyes uneasy.

"Nick, wouldn't it be better if you forget all this yourself? Must you hunt for this man? If you did find him, you couldn't hand him over to the police. He might talk and give Roy away."

English smiled at her.

"Don't worry your head about that. I have to find him first. Roy may have been a louse and a rat, but no one's going to murder one of my family and get away with it. I'll think of a way of fixing this guy when I've found him." He kissed her and patted her hip. "Don't worry."

He went down to where Chuck was waiting patiently.

As Chuck drove rapidly through the dark streets, English sat still, his face thoughtful, his mind busy.

He went straight up to his apartment and tossed his coat to Ushu, his Filipino boy.

"Anyone waiting to see me?" he asked.

"No, sir."

"No calls?"

"No, sir."

English nodded and went into his study. He sat down at the desk and reached for a cigar. When he had lit it, he sat thinking for a few minutes, then he picked up the telephone.

"Get me Police Captain O'Brien, Police Headquarters, Boston," he told the girl at the switchboard.

"Yes, Mr. English."

He hung up, and got to his feet, and began to pace slowly up and down. After a little delay, the telephone rang.

"Hello, Mr. English. Well, well, you are a stranger," O'Brien's voice boomed over the line.

"Hello, Tom. How are you?"

"I'm fine. How's yourself?"

"Oh, I'm alive. I was expecting you at the fight. Why didn't you come?"

"You know how it is. I got a couple of murders on my hands right now. Glad your boy won. Seems like a good scrap."

"It was all right. Look, Tom, I want a quick favor."

"Anything you say, Mr. English."

"Some eight years ago, a girl named Julie Clair was arrested for stealing money from her landlady. She drew a week in jail. Can you check that?"

"I guess so," O'Brien returned. "Give me three minutes."

English sat on the edge of the desk, swinging his leg, his eyes brooding, cigar smoke drifting past his face.

In less than three minutes, O'Brien came on the line again.

"No one of that name was arrested, Mr. English. We have no record of her." \- Nick's face hardened.

"Any record of any girl arrested for stealing money from her landlady about that time?"

"I'll see," O'Brien said, and there was a long pause. Then he said, "A girl named Doris Caspary—she got a week in jail because she had been shoplifting the previous month."

English remembered Julie had once mentioned sharing rooms with a girl named Doris Caspary. Once he had heard a name, he never forgot it.

"Julie Clair was a witness for the defense," O'Brien went on. "But she wasn't charged."

"Thanks, O'Brien, I must have got my facts muddled," English returned. "Don't forget to let me know when you are coming to town."

He hung up and frowned down at the carpet. He had had an idea that Julie had been lying the moment she had started telling him the story of the theft.

"Now, I wonder what you've been up to, Julie?" he said half aloud, as he walked toward the apartment door.

\vspace{2\nbs}
\ChapterDeco[c1]{\decoglyph{e9665}}
\clearpage
\thispagestyle{empty}

\begin{ChapterStart}
\vspace{3\nbs}
\ChapterSubtitle[l]{Chapter ch4}
\ChapterTitle[l]{ch4}
\end{ChapterStart}
\FirstLine{\noindent The elevator that was nearly opposite English's door was coming up. It stopped, and a youngish man in a well-cut brown suit, a white silk handkerchief tucked up his sleeve and a brown slouch hat set squarely on his head, got out. He gave English a quick, searching glance, then began to move along the passage to the other apartment that was at the end of the passage.}
    
"Mr. Sherman," English said quietly.

The man in the brown suit paused. He had the most extraordinary eyes English had ever seen—they were amber-colored with huge pupils, and they were as expressionless as two yellow buttons.

"Why, yes, I'm Sherman," he said. His voice was low pitched and musical, and he smiled at English, showing small, white teeth. "Did you want me? It's Nick English, isn't it?"

"Yes, that's right." He went on to Sherman, "I did want a word with you. Perhaps you'd care to step in for a moment?"

"I wonder if you'd mind coming along to my apartment?" Sherman said. "I'm expecting a telephone call and it's important."

"Certainly," English replied and closed the front door, moving along the passage at Sherman's side.

Sherman unlocked his door, reached forward and turned on the light, then stood aside.

"Please go ahead, Mr. English."

English walked into an ornate lobby that seemed full of flowers. He turned and watched Sherman close the door. Sherman hung his hat on the rack, ran a small, white hand over his flaxen hair, and opened the door facing him. He reached in and pressed light buttons, and lights sprang up in the room.

He stood aside, motioning English to enter.

It took a lot to startle English but this room brought him to an abrupt halt, and he stood staring around, his face clearly showing his astonishment.

It was a big room. English was aware first of a feeling of space—a vast stretch of polished floor spread out before him. There was no carpet or rugs to break up mat stretch of flooring. It seemed to go on and on until it finished up against long black velvet drapes that covered the windows.

A white corded settee and two white lounge chairs cringed in the empty space. In the alcove by the window, stood a grand piano. There was a big fireplace where a log fire burned brightly, and on either side of it stood six-foot-high black candles with small electric lamps imitating candle flames.

"As a showman, Mr. English, you should appreciate this room," Sherman said, moving over to the fireplace. "At least, it is original, isn't it? Of course, not many people would care to live in it, but then I'm not like most people."

"I agree with you," said English drily. "But I mustn't keep you. I wanted to ask if you called on the News Service Agency at 1356 Seventh Street on the seventeenth of this month."

Sherman slowly unpeeled the wrapping on the gum package, his expressionless eyes on English's face.

"I believe I did," he said. "I can't be sure if I went there on the seventeenth, but it was some evening this week. It could be the seventeenth, come to think of it."

"I have a reason for asking," English said. "You went there about ten-fifteen?"

"It is possible. I didn't particularly notice."

"At about that time, my brother committed suicide," English said, his eyes on Sherman's face. "He shot himself."

Sherman lifted his eyebrows.

"How very unpleasant for you," he said, took a piece of gum from the package and put it into his mouth. "I'm sorry."

"Did you hear a shot when you were in the building?"

"So it was a shot," Sherman said. "I did hear something and it crossed my mind that it was a shot, but I finally decided it must have been a car backfiring."

"Where were you when you heard the shot?"

"I was coming up in the elevator."

"Did you see anyone in the sixth floor passage or coming out of my brother's office?"

"Had your brother an office on the sixth floor?" Sherman asked. "There is a Detective Agency and the News Service Agency on that floor, if I remember rightly. Where would your brother's office be?"

"He owned the Detective Agency."

Sherman's jaws moved rhythmically.

"Did he? That's interesting. I had no idea your brother was a detective," he said, and his tone implied that he didn't think anything of detectives.

"Did you see anyone near my brother's office?" English repeated.

Sherman frowned.

"Why, yes. Come to think of it, I did, I saw a girl up there. She was wearing a rather smart black and white outfit. I remember thinking for the type of girl she so obviously was, she had a flair for clothes."

With an expressionless face, English asked, "And what type of girl was she, Mr. Sherman?"

Sherman smiled.

"The type of girl who wouldn't have too many ethics."

English's eyes were cold and hard as he said, "And this girl was in the passage when you came up in the elevator?"

"That's right. She was walking away from the agency, making for the stairs."

"You saw no one else?"

"No."

"How long would you say was it that you heard the shot and saw the girl?"

"About five or six seconds."

"Well, thanks," English said, suddenly realizing where Sherman's answers were leading. "I guess I won't keep you any longer. You've told me all that I wanted to know."

"That's fine," Sherman said. "I suppose your brother did commit suicide, Mr. English?"

"That's what I said," English returned curtly.

"Yes, so you did. But detectives do appear to lead dangerous lives. I wonder if your brother discovered something unpleasant about this girl, and she shot him to silence him."

English smiled bleakly.

"My brother shot himself, Mr. Sherman.."

Sherman nodded.

"Well, you know best, Mr. English. I wonder what the girl was doing in your brother's office. He must have shot himself while she was actually in the room. Of course, the girl could be easily traced," he went on absently. "I should imagine she worked in some night club. She looked the type." He ran his fingers through his flaxen hair, ruffling it so that he suddenly appeared almost boyish as he smiled at English. "I'm an artist, Mr. English. You wouldn't know that, of course, but I'm rather clever at creating a likeness. It would be a very easy task for me to provide the police with a picture of this girl. Do you think I should do that?"

"The police are satisfied that my brother shot himself," English said quietly. "I don't think you need bother to supply them with a picture. Thank you for your help."

"Only too glad," Sherman said. He continued to chew, his hands in his pockets, his face lit by a smile. "As a matter of fact, I had been hoping to have the opportunity of talking to you. After all, you are quite a celebrity.

"I suppose I am," English said, moving toward the door. "Good night, Mr. Sherman."

"I guess if the police knew about Miss Clair, it might be very awkward for her, and unpleasant for you," Sherman said, raising his voice slightly. "After all, she did have a very good reason for shooting your brother, didn't she?"

"Miss—who?" English asked, politely interested.

"Julie Clair, your mistress," Sherman returned. "Her motive and my evidence could put her in jail for quite a long time. She might even go to the chair, although if she flashed her legs at the jury, she would probably avoid that. But she would get at least ten years. You wouldn't like that, would you, Mr. English?"



II

There was a pause while the two men looked at each other, then English came back slowly to the center of the room.

"No," he said, speaking quietly. "I shouldn't like that. Are you quite sure the girl you saw was Miss Clair?"

Sherman made a little gesture' of impatience with his hand. "I know you are a very busy man," he said, "but you-might feel inclined to discuss the situation now rather than later, but please yourself. I'm in no hurry."

"What is there to discuss?" English said.

"Wouldn't it save time if we stopped behaving like a couple of clubmen at a social gathering?" Sherman said sharply. "I own a piece of information and I am prepared to sell it to you. That's what there is to discuss."

"I see," English said, raising his eyebrows. "This is a surprise. I was wondering if you would have the nerve to try to blackmail me." Sherman smiled.

\*To me, Mr. English, you are just a rich man. Your importance and fame leave me indifferent. You have the money and I have the information. I can either sell it to you or to Miss Clair. I would prefer to sell it to you as I would be able to ask a much higher price; but if you are not inclined to make a deal, I must go to her."

"I was under the impression that you already have dealings with her," English said mildly. "She has been paying you two hundred dollars a week, hasn't she?"

Sherman's eyes blinked, then he smiled.

"I don't usually betray a buyer's confidence, but as she has obviously told you about it, then I see no harm in saying that we have a modest deal on together, but this new proposition would be a much larger deal, and it would be a cash payment."

"I don't think she could pay."

"Possibly not, then perhaps you would come to her assistance."

"What do you want for your information?" English asked.

"From you, I should think a fair price would be two hundred and fifty thousand in cash," Sherman said. "From her, I don't suppose I could get more than fifty thousand. But if I sold to her, I couldn't guarantee that the press wouldn't discover your brother was a professional blackmailer. For the larger sum, I should be able to guarantee it"

English's face was expressionless, his eyes unworried.

"How did Roy happen to get mixed up with you?" he asked.

Sherman leant his shoulders against the mantel while he studied English, a slightly puzzled expression in his eyes.

"Need we go into that?" he said. "We are discussing a deal if I may take your mind back to business."

"There's plenty of time to talk about that," English returned airily. "How did Roy happen to get mixed up with you?"

Sherman hesitated, then shrugging his shoulders, he said, "Your brother was anxious to make some easy money. His agency was a convenient place for my clients to go and settle their accounts with me without causing embarrassment to either side. I paid your brother well. He collected 10 per cent of the gross."

"I see," said English. "And he decided that 10 per cent wasn't enough. He attempted to help himself. Probably he held out on you. He was planning to go away with his secretary, Mary Savitt, and no doubt he was anxious to lay his hands on a getaway stake. I assume you found out that he was cheating you, and you decided to teach him a lesson. On the night of the seventeenth, you went to his office, shot him through the head with his own gun, impressed his fingerprints on the gun butt and collected the card index containing the names of your customers before leaving. Am I right?"

Sherman continued to smile, but his eyes were now wary.

"I believe something like that did happen," he said. "Naturally, you wouldn't expect me to swear to it before a jury, but between ourselves, since we are talking off the record, something very much like that did happen."

English nodded, and blew smoke toward the ceiling.

"You then went to Mary Savitt's apartment. You strangled her and strung her up against the bathroom door. I assume you silenced her because she knew what Roy had been doing and could have told the police that you had the motive for murdering him."

"You appear to keep yourself very well informed, Mr. English," Sherman said.

"During the late afternoon," Mr. English went on, "a man named Hennessey called at the Alert Agency to pay his dues. He met the present occupant who persuaded him to talk. Somehow you managed to overhear the conversation, and you murdered Hennessey by running him down in your car."

There was a long pause of silence while Sherman studied English. His smile was fixed now, and his eyes were uneasy.

"All this is very interesting, Mr. English," he said at last, "but suppose we get back to our business deal. I have an appointment in half an hour."

English smiled.

"You don't really imagine you can blackmail me, do you?" he asked.

"Yes, I see no reason why not," Sherman returned, his voice hardening. "It would be no hardship for you to find a quarter of a million. The advantages of paying are considerable. Up to now, you have made a big impression on this city. You are anxious to have the hospital named after you. You have done the city a lot of good. It would be a pity to spoil your good name because you happen to have a brother who failed to live up to your own high standards. I think you would be extremely foolish not to make a deal with me."

"But I don't have to make a deal with you," English said mildly, "It is you who have to try to make a deal with me."

"What do you mean?" Sherman asked, frowning.

"I should have thought it is obvious. Within the past three days, you have murdered three people. I hold your life in my hands."

Sherman made an impatient gesture.

"Surely that is an exaggeration. There is a considerable difference in making a guess and proving it."

"I won't need to prove it. You will have to prove you didn't kill these people."

"I'm afraid you're wasting time," Sherman said sharply. "Are you going to buy my information or do I have to go to your mistress?"

English laughed.

"I had the mistaken idea that when I found the man who murdered my brother, I was going to take the law into my own hands. At the back of my mind, I was prepared to shoot him. I know that my brother was a weak, gutless fool, but in my family we have a tradition. We bury our own dead. So I had made up my mind that I would find Roy's murderer myself and deal with him." He leaned forward to flick ash into the fire. "Well, I have found him but the circumstances have changed. I have also discovered that my brother was not only a cheap cheat but he was also a blackmailer, and to me, Mr. Sherman, a blackmailer is lower than any other form of life. If you hadn't killed him, then I should have. In fact, Mr. Sherman, I am moderately grateful to you for ridding me of Roy."

Sherman's face was now set, and his yellow eyes gleamed.

"All this is very interesting, but it doesn't answer my question. Are you paying me or do I have to go to your mistress?"

"I'm certainly not paying you," English said, "and Miss Clair isn't paying you either."

"Then you give me no other alternative but to go elsewhere with my information," Sherman said.

"Nor will you take your information elsewhere," English returned. "Up to now you've been blackmailing people who don't know how to hit back. I do."

"That remains to be seen," Sherman said.

"You don't seem to realize what you've taken on by trying to blackmail me," English said, stretching out his long legs. "I have a lot of money and a lot of influence. I have many useful friends. When dealing with a blackmailer, I should not hesitate to throw aside all scruples. I have already told you— I don't regard a blackmailer as a human being. I would treat him as I would treat a rat that happens to find its way into my room. I would exterminate him without mercy and by any means, and that is what I am prepared to do with you. At the moment, I've no evidence against you that would stand up in court, but in two or three days I shall have the evidence. I admit that it would cost money, but then I have money. Having got my evidence, perjured or legal, I shall then talk to the judge who will try you. I know all the judges in the city, and they are all anxious to do me a favor. I will arrange to see the jury before they try you, and I will promise them a reward if they bring in a guilty verdict. Once you are arrested, Mr. Sherman, I guarantee you will be dead in a few months. Make no mistake about that"

"You don't think you can scare me, do you?" Sherman said. "I make a point always to call a bluff."

"There comes a time when you can call a bluff once too often," English returned. "I admit if I handed you over to the police, it wouldn't be possible to keep the shabby news that my brother was a blackmailer out of the papers. I admit that I would cook my own goose in this city by having you arrested, but rather than submit to blackmail or let Miss Clair submit to blackmail, I shan't hesitate to go after you; and once I do go after you, no power on earth can save you from the electric chair." He got up abruptly and began to pace up and down, his hands clasped behind his back, his face thoughtful. "I can't allow you to remain in the city, nor can I allow you to continue to levy blackmail. I am going to make you a proposal. It doesn't suit me at the moment to hand you over to the police. Instead, you are to leave town by the end of the week. You are not to return. You are to give up your blackmailing activities. If you don't leave, I shall hand you over to the police. If you think I'm bluffing, go ahead and stay in this apartment and see what happens to you. I promise you, if it's its last thing I do, I'll have you in the electric chair within six months. If this apartment isn't empty by Saturday night, you will be arrested on Sunday morning. I shall not warn you again. Get out of town by Saturday night or take the consequences."

Sherman had gone pale, and his yellow eyes showed his suppressed fury.

"A war is never won until the last battle, Mr. English," he said, his voice unsteady.

English looked at him and made a grimace of disgust.

"This happens to be the last battle," he said, opened the front door and walked slowly down the passage.

\vspace{2\nbs}
\ChapterDeco[c1]{\decoglyph{e9665}}
\clearpage
\thispagestyle{empty}

\begin{ChapterStart}
\vspace{3\nbs}
\ChapterSubtitle[l]{Chapter ch5}
\ChapterTitle[l]{ch5}
\end{ChapterStart}
\FirstLine{\noindent Corinne English carried the coffeepot into the living room and set it on the table. As she sat down, she yawned and ran her fingers through her blonde hair.}
    
The time was twenty minutes past eleven in the morning, and the bright sunshine made her feel jaded. She poured the coffee into a cup, and then, after only a momentary hesitation, she got up and went over to the liquor cabinet for a bottle of brandy.

Since Roy's death, she had been drinking heavily. The lonely house, her brooding thoughts about Roy and Mary Savitt, and her hatred of Nick English so preyed on her mind that she turned automatically to brandy to "deaden her suffering," as she put it to herself. She brought the bottle to the table, poured a liberal shot into the coffee, and sat down again. Her eyes hardened as she thought of English. She hated him as she didn't think it possible to hate anyone. His threat to hand over Roy's letters to the press filled her with vindictive fury.

She finished her coffee, got up and took a glass from the liquor cabinet and half filled it with brandy.

"May as well get soused as sit here and think about that bastard," she said aloud. "I've nothing to do until lunch time. And when lunch time comes, I shan't want any lunch. So what the hell?"

As she was about to sit down, the musical chimes at the front door sounded.

"Oh, damn," she said crossly. "That'll be Hetty."

She went across the sitting room into the lobby, and opened the front door.

A youngish man stood on the step. He raised his brown slouch hat, showing thick, flaxen hair that looked like burnished silver in the sunlight. He smiled at Corinne, his jaws moving rhythmically as he chewed, his amber-colored eyes sliding over her plump little figure like a caress.

"Mrs. English?"

"Yes, but I—I don't receive callers at this hour. Who are you?"

"My name is Roger Sherman, Mrs. English. I'm an" old friend of Roy's."

"Oh!" Corinne stepped back. "Perhaps you had better come in. The place is in a ghastly mess. My maid hasn't come yet I was just having breakfast."

Sherman stepped into the lobby, and closed the door.

"Please don't be embarrassed," he said, and gave her a charming smile. "I should have called you on the telephone first. I do hope you will forgive me."

Corinne was in a flutter. Roy had never mentioned Roger Sherman to her, but it was obvious this man was wealthy. She had caught a glimpse of a big shiny Cadillac at the door, and his clothes and manner impressed her.

"Please go into the living room. I won't be a moment," she said, and retreated hurriedly into her bedroom, shutting the door.

Sherman walked into the living room and looked around with a slight wrinkling of his nose. He saw the bottle of brandy, and the glass, and nodded to himself. He went over and stood by the electric heater, his hands in his pockets, his jaws moving slowly. He remained like that for over a quarter of an hour, his blank expression masking his impatience.

Corinne came in, still flustered. She had put on make-up, and had changed into a lilac-colored wrap which she kept for best occasions.

"I'm sorry to have kept you waiting," she said, closing the door. "But I had to make myself look a little presentable."

"Why, you look charming," Sherman said, smiling at her. "So you are Roy's wife. He often talked about you, saving how pretty you are, and now I've seen you for myself, I can endorse that."

It seemed a long time to Corinne since anyone had paid her a compliment, and for a moment she forgot how Roy had betrayed her, and the memory of their past happiness brought tears to her eyes.

"Roy never mentioned you," she said, touching her eyes with her handkerchief. "You say you were a friend of his?"

"We were very old friends. I was shocked to hear of bis death. I would have come to you sooner only I have been out of town. I can't say how sorry I am."

"Please don't talk about it," Corinne said, and sat down. "I don't think 111 ever get over the disgrace."

"You mustn't  say things like that,"  Sherman said gently.

"After all, it wasn't Roy's fault. I suppose you know that his brother was at the bottom of the whole thing?"

Corinne stiffened.

"How do you know?"

Sherman's eyes went to the bottle of brandy.

"Would it be rude of me to ask if I might have a drink? I like a drink at this time of the morning, but perhaps you wouldn't approve?"

"Oh, yes," Corinne said. "Please help yourself. I don't mind in the least."

Sherman went over to the cabinet for a brandy glass. He poured brandy into it, and then appeared to notice Corinne's empty glass for the first time.

"May I give you a drink, too, Mrs. English?"

Corinne hesitated. She didn't want this presentable young man to think that she was in the habit of drinking brandy in the morning, but she wanted a drink badly.

"Well, perhaps a small one. I'm not feeling very bright this morning."

"I'm sorry to hear that," Sherman returned, poured brandy generously into her glass and gave it to her. "I hope this won't be the last time we meet," he went on, and saluted her with his glass.

Corinne drank half the brandy while Sherman scarcely touched his.

"You were talking about Nick English," Corinne said. "How do you know he was at the bottom of Roy's death?"

"Roy told, me," Sherman said, and sat down beside Corinne.

"What did he tell you?" Corinne demanded.

"He told me about the money," Sherman said. "You know about that, of course?"

"What money?"

"Why, the twenty thousand dollars Roy meant you to have," Sherman returned, lifting his eyebrows. "Surely your attorney has given it to you?"

Corinne's big blue eyes opened wide.

"Twenty thousand dollars?" she repeated. "I don't know anything about it."

"But surely you've been left something? Forgive me for appearing curious but, after all, I was Roy's best friend, and I feel that I should see that his wife has been properly provided for."

"Oh, thank you," Corinne said, nearly dissolving into tears. "You don't know how lonely I've been. Of course, Sam Crail has been kind, but he is very busy. After all, it's not as if he was a friend. He was only Roy's attorney."

"He's Nick English's attorney, too," Sherman said.

Corinne stiffened.

"He is? I didn't know that But it doesn't matter, does it? He wouldn't tell that man anything, would he?"

"He's on English's pay roll," Sherman said. "It's no secret. He does exactly what English tells him."

"Oh!" Corinne's face flushed. "What am I going to do? I wouldn't have had him in the house if I had known."

"May I ask what you've got to live on?" Sherman said, leaning forward and looking at her intently.

"Roy left an annuity. I'm to have two hundred dollars a week for life," Corinne said.

"And nothing has been said about the twenty thousand?"

"No. This is the first time I have heard of it. What twenty thousand?"

"You know about- Mary Savitt, I suppose?"

Corinne looked away.

"Yes, I know about her. How Roy could have done such a thing . . ."

"Some men get carried away by unscrupulous women," Sherman said, shaking his head. "And she was unscrupulous, Mrs. English. It wouldn't have lasted."

Corinne put her hand on his.

"Thank you for saying that. That's what I've been telling myself. Roy couldn't have gone off and left me. I know he would have come back."

"He didn't forget you. He provided for you. He told me so. He brought off a deal which netted him twenty thousand. He intended to give you the money when he went off with Mary Savitt."

"Roy made twenty thousand," Corinne said, startled. "Why, I can't believe it. Roy never made any money, ever."

"Strictly speaking, it was rather a shady deal," Sherman said. "Apparently, Nick English was handling it. Roy happened to call on the same client on another matter, and the client confused Roy with Nick. Roy didn't enlighten him, and pulled off the deal. Nick English was so angry, he called in the police. They were on their way when Roy got into a panic and shot himself."

"Oh!" Corinne said, and leaned back, closing her eyes. "You mean that that man was going to have his own brother arrested?"

"I'm afraid so. Roy had put the money in a safe deposit and had given Crail the key. Grail was to give you the money. As you haven't gotten it, it would seem pretty obvious that English had instructed Crail to hand the money to him."

Corinne sat bolt upright, her eyes furious.

"D'you mean he's stolen the money from me?"

Sherman lifted his shoulders.

"It looks like it, but neither you nor I have any proof that the money even exists."

Corinne took a long pull at her glass. The brandy she had already drunk before Sherman arrived was beginning to have an effect on her, and she felt a little dizzy, and very reckless.

"Well, he's not going to get away with it. I'll fix that louse." She jumped to her feet. "I'll make him suffer for this!"

"I can understand your feelings," Sherman said, watching her narrowly, "but how do you propose to do it? He is an extremely powerful and influential individual."

"I'll think of some way," Corinne said, and moving a little unsteadily across the room, she refilled her glass, slopping the brandy on the carpet as she did so.

"Perhaps I could help you," Sherman said, getting to his feet.

She turned and leaned against the liquor cabinet, staring at him.

"Can you?" she asked. "How?"

"It wouldn't be possible to get the money out of him, but if you want to make him suffer . . ."

"That's what I do want! Do you know how I can do it?"

"Yes, but it'll depend on you whether you succeed or not You know Julie Clair?"

"I don't know her," Corinne said, "but I know of her. She's bis mistress, isn't she?"

"And English is crazy about her. I happen to know that she is having an affair with his general manager, a fellow named Harry Vince."

Corinne stood very still, looking at Sherman, her eyes gleaming.

"Are you sure?" she said. "Are you absolutely sure?"

"She goes to Vince's apartment whenever English has a business date. I've seen her go there."

"This is what I've been waiting for," Corinne said, and moved unsteadily back to the settee. "Oh! Now I'll make him suffer. If only he could find them together!"

"That could be arranged," Sherman said. "He happens to be dining tonight with Senator Beaumont at the Silver Tower. She's bound to go to her lover. Why don't you go along and tell him?"

"Will you come with me?" Corinne asked, her face lighting up with a cruel little smile.

Sherman shook his head.

"That's not possible. I have an engagement for tonight. English will show up about eight-thirty. If you get there by nine, it will be time enough."

'Til be there," she said, clenching her fists. "Ill make a scene he and his swank friends won't forget in a hurry."

Sherman smiled.

"I thought you'd make good use of the information."

She suddenly looked at him curiously.

"Why did you tell me? Have you something to settle with his as well?"

"If I had," Sherman said smoothly, "I would do my own dirty work. I happen to be angry about the way he has treated you. I felt I had to give you a weapon, and I've given it to you."

Corinne smiled at him.

"I'm grateful." She crossed her legs, letting the wrap fall away a little, to show her knees. "I can't say how grateful lam."

"There's just one thing I'd ask you to do," Sherman said, his eyes straying to her knees. "When you have told him, will you telephone me?"

"Why, of course."

He took out a card and gave it to her.

"You'll find me at this number after nine o'clock. Will you telephone me immediately after you have spoken to him? I want to know what he does. It's important. Can I rely on you?"

"Of course," she repeated, taking the card. "I'll call you just after nine."

"Thank you." He looked around for his hat, and suddenly she couldn't bear the thought of his leaving. Not since she had first met Roy had a man had such a strong attraction for her. "Well, I'll be running along," he went on. "May I come and see you again?"

"Yes, I wish you would. You can't imagine how lonely I get Roy and I were always around together when he wasn't at work, and I miss him terribly."

The amber-colored eyes dwelt on her face speculatively.

Then I'll fix it soon," Sherman said and moved to the door.



II

Lois Marshall was just finishing dictating a batch of cables to her stenographer when Ed Leon pushed open the office door and came in.

He lifted his hat.

"Mr. English around?"

"Yes, he's expecting you," Lois told him. "Would you sit down for a moment? Mr. Crail's not here yet."

Leon lowered his long frame into an easy chair, and groped for a pack of Camels. In a few moments, the door opened and Crail came in. His freshly shaven face was thoughtful as he nodded to Lois, and waved a plump hand at Leon.

"You look as if you'd had a pretty good breakfast," Leon said enviously. "Or is that bulge under your vest just part of your scenery?"

Crail ran his hand over his paunch, and smiled smugly.

"It's part of my goodwill," he returned. "If I had a frame like yours, I'd go out of business. No one trusts a thin man these days." He looked over at Lois. "Mr. English ready for me?"

"I think so," Lois said, picking up the telephone, "Mr. Crail and Mr. Leon are waiting, Mr. English," she said, and nodded

at the two men. "Will you go in, please."

Leon levered himself out of the chair and followed Crail into English's office.

English was sitting at his desk. Harry Vince was crossing the room to the door, a pile of papers in his hand. Harry nodded to Crail, looked sharply at Leon, and went out.

"Who's that guy?" Leon asked, dropping into a chair.

"Don't you know Harry?" English said. "He's my general manager, and a damned fine one at that."

"What's new, Nick?" Crail asked, sitting down. "I can't stay long. I'm in court at ten-thirty."

"I've found the guy who murdered Roy," English said quietly.

"You have?" Crail sat up. "Well, for God's sake! That's fast work."

English nodded over to Leon.

"He may not look it, but he happens to be a fast worker."

"Sherman?" Leon asked.

"Yes."

English went on to tell them of the conversation he had had with Sherman the previous night.

"Three murders?" Crail said, his eyes opening. "He admitted them?"

"He didn't deny them," English returned.

"Well, I'll be damned! I'd like to see the D.A.'s face when you tell him," Crail said, and rubbed his hands together. "He hasn't even connected the three killings with the same man."

"I'm not telling the D.A.," English said, paused to light his cigar, and as he waved out the match, he went on, "It's up to him to find the killer. I'm not anxious to tell the world my brother was a blackmailer. I've given Sherman until Saturday to get out of town."

Crail looked quickly at Leon who stared back at him with an expression of complete indifference.

"You can't do it, Nick," Crail said sharply. "It'll make you an accessory after the fact. Damn it! It would make me an. accessory, too."

"That's one of the drawbacks of working for me," English said, and smiled.

"Do you think Sherman would go?" Leon broke in.

"He'd be a fool if he didn't. I hold all the cards. He didnt strike me as a fool. But I want you to take care of him, Ed. Sit on his tail. Don't lose him for a second. I want him under your eye day and night until he leaves town on Saturday."

Leon nodded.

"You don't mean to tell me you are going to let him get away with three murders?" Crail said, horrified.

"He's already got away with them," English returned. "I haven't any evidence that'd stand up in court. If he double-crosses me, I'll manufacture some evidence, but not until."

"What do you mean—manufacture evidence?" Crail asked, his eyebrows climbing.

"I'll explain that when and if I have to," English said. "If this fellow double-crosses me, he's going to the chair, and you and I are putting him in the chair."

"That'll be something for you to dream about," Leon said to Crail, and grinned. "What do you make of Sherman?" he went on to English.

"I'd say he was crazy," English said soberly. "I mean that He's about as dangerous as a rattlesnake. There's a chance he'll try to pull a fast one. He might even arrange for me to be bis fourth victim. I've put into writing the whole of our conversation, and I have it here." He slid an envelope across the desk. "I want you to take care of this, Sam. If anything should happen to me, give it to Morilli."

Crail looked startled.

"You're not serious, are you?"

"I'm very serious, Sam. I've told Chuck to carry a gun, and not let me out of his sight for a moment, and I want you to swing into action any moment, Sam. If Sherman tries anything funny, I mean to send him to the chair."

Crail shook his head as he got to his feet.

"We'll cross that bridge when we come to it. I've got to get going, Nick. See you later." Crail nodded to Leon and left the office.

English glanced at his watch. "Well, I've got work to do, Ed. Will you get after Sherman? From now on until Saturday, I don't want him out of your sight. It's important. Don't let him give you the slip."

"Ill take care of him," Leon said. "I know what I meant to tell you—I tracked down those mike wires in Roy's office. They lead into an office on the same floor, owned by a silhouette artist. A woman named Gloria Windsor."

"Think she's one of the gang?" English asked, not particularly interested.

"Must be. It's my guess she fingered Roy. She must have heard Roy and the Savitt girl planning to pull out. Those two must have made their plans in the office, not knowing the mike was in the chimney to pick up every word they said. You can bet that's how Sherman found out that Roy was cheating on him."

"Well, it's done now," English said, shrugging. "I'm content to get rid of Sherman. When he's gone, the rest of them will be like a body without a head."

Leon got to his feet.

"Let's hope so. I'll keep tabs on Sherman. If he looks like starting anything, I'll call you."

"Thanks, Ed. So long for now."

After Leon had gone, English immersed himself in the routine paper work that came to his desk every day. He worked quickly and methodically, his mind concentrated on the worK before him.

Lois found him hard at it a few minutes to lunch time. She came in and put another pile of papers in his In-tray. He glanced up and smiled at her.

"Did you remember to book a table at the Silver Tower for tonight?" he asked.

"Yes, for eight-thirty."

"Might have known you wouldn't forget. I don't believe you've ever forgotten anything to do with my business since we hooked up together. That's quite a record."

"That's what you pay me for," Lois said lightly.

"I guess so," English said, and frowned, "but I bet not many secretaries give the service that you do. Let's see—you've been with me for five years, haven't you?"

Lois smiled.

"Yes. It'll be five years exactly on Saturday."

"Is that right? How did you remember that?"

"I have a good memory for dates."

"Saturday, eh?" English said. "Well, damn it! We should celebrate. We've come pretty far in five years, haven't we? Tell you what! I'll take you out to dinner on Saturday. We'll celebrate the firm's fifth birthday. What do you say?"

A faint flush came to Lois' face. She hesitated, then said quickly, "I don't think I can manage Saturday night, Mr. English. I have a date."

English studied her, noticing her flush deepen.

"Okay. Lois, if you can't put your boy friend off, you can't. Well, maybe some other night. I'll see what I can fix."

"It's nothing to do with a boy friend," Lois exclaimed with a vehemence that startled English. "I just happen to be busy that night," and she went out of the office, closing the door sharply behind her.

English frowned down at his blotter, puzzled.

"Well, that takes care of that," he said to himself. "And Julie says the girl's in love with me. Well, what do you know? Won't even accept an invitation to dinner. Is that what Julie calls love?"

Some ten minutes later, he put down his pen and walked over to where his hat and coat were hanging. He was struggling into his coat when a tap came on the door, and Julie came in.

"Why, hello, Julie," he said, straightening his coat. "What brings you here?"

Julie reached up and gave him a quick kiss.

"I want some money," she said. "I'm lunching with Joyce Gibbons, and I've come out without my purse."

"I wish I could join you," English said regretfully, taking out his billfold. "Will fifty hold you?"

"That's plenty, darling. And you can drive me downtown, if you like."

"Where are you lunching?" English asked, reaching for his hat.

"Marcel's."

"Right. It's on my way. Come on, then. Let's get going."

They walked into the outer office. Harry Vince came in at this moment. He gave Julia a quick, uneasy look, then stood aside.

"Hello, Harry," Julie said gaily. "I know what I want you to do for me."

"Yes, Julie?" Harry said, stiffly.

His tone made Lois look up sharply. She was sitting at her desk by the window, unnoticed by either Julie or Harry.

"I want two more tickets for the show. It's for tonight," Julie said. "Can you get them for me?"

"Why, yes," Harry said, changing color.

"Hey!" English said, with a grin. "Don't ruin me, Julie. I can't give too many tickets away."

"Now, don't be a tightwad," Julie said, linking her arm in his. "You know people expect me to give them tickets for all your shows."

"See what you can do for her, Harry," English said. "What she says goes, it seems."

"Yes, Mr. English," Harry said huskily.

"Aren't you dining with that dreary old Senator tonight?". Julie said as she led English across the office. "What time are you meeting him?"

"Eight-thirty," English said. "I won't be seeing you tonight, Julie."

He followed her into the passage.

Harry stood motionless, looking after them. There was an expression on his face that startled Lois. She watched him, and when he went abruptly out of the office, she felt a little chill of apprehension run through her.



Ill

As English handed his hat and coat to the check-girl at the Silver Tower restaurant and was about to move to the washroom, he saw Senator Beaumont come in.

"Hello, there, Senator," he said.

Senator Henry Beaumont was sixty-five years old—small, wiry and thin. He was a man of insatiable ambition. When English first met him, Beaumont had been Commissioner of Highways, attached to the Democratic machine ruling Chicago at that time. Beaumont had introduced him to his circle of wealthy businessmen. It was through Beaumont's introductions that English financed his gyroscope compass.

When English finally settled in Essex City, he remembered Beaumont and wrote to him, offering to finance him if he cared to run for the post of county judge. Beaumont jumped at the offer, and with English's money behind him, he was elected.

English was quick to realize that as his business expanded and his kingdom grew, it was essential to have a powerful friend in the political machine. Although Beaumont was no ball of fire, he was at least sharply aware of his debt to English, and was willing to pull strings when English wanted them pulled.

The next move, English had decided, was to get Beaumont elected Senator. The opposition was stiff, but again with English's money and coupled with his ruthless determination, Beaumont became Senator. Now, he was to come up for reelection in another six months' time, and English knew Beaumont was uneasy as to what the results would be.

"How are you, Nick?" Beaumont asked, shaking hands.

"I'm fine. I was just going to have a wash. Coming?"

"May as well," Beaumont returned, and together they walked into the ornate washroom.

While English washed his hands, Beaumont lit a cigar and stood near him, scowling.

"You shouldn't have called off the Committee meeting, Nick," he said. lThe Commission doesn't like being treated like that. Rees was particularly angry."

"You know why I called it off?" English said, drying his hands.

"Well, I can guess. This stink about Roy. But you shouldn't have just called it off the way you did."

English tossed the towel into the basket, and took the Senator's arm.

"Have a highball and relax," he said, leading the Senator into the bar. "Rees may be Chairman of the Commission but he'll stand for everything I dish out, and you know it."

"He won't. He said it was time someone clamped down on you, and he's going to do it."

English passed a highball to Beaumont, and ordered a martini for himself.

"And how does he intend to clamp down on me?" he asked, smiling.

"He didn't say, but I've heard he's pressing the D.A. to investigate the blackmail rumors about Roy."

English shrugged.

"There's nothing to investigate. Let him go ahead if he wants to, but if he starts anything he can't prove, I'm going to sue the coat off his back!"

Beaumont nodded.

"I told him so," he said, a satisfied expression coming into his eyes. "He didn't like it. All the same, Nick, if there's any truth in it, you've got to be damned careful."

"Don't talk crap!" English said roughly. "There's nothing for me to do—nothing at all. He's got to prove that Roy was a blackmailer, and he can't do it."

"How about that girl? Roy's secretary?"

"She's been taken care of. The press didn't hook her to Roy, nor did the D.A. You've nothing to worry about, so relax, can't you."

"It's all very well for you to talk," Beaumont said crossly, "but I've my position to think of. Those sonsofbitches are gunning for me as well as you."

"So long as I'm here, you've nothing to worry about," English said.

'Talk of the devil," Beaumont muttered. "Here's Rees now."

English glanced up.

Standing in the doorway, was a squat, hard-faced man in his late sixties, talking to a pretty vivacious-looking girl who was wearing a silver blue mutation mink in a cape stole over a black evening gown.

"I wonder if he bought her that cape or if she hired it," English said out of the corner of his mouth. "That's Lola Vegas. She used to dance at the Golden Apple before I threw her out. She went for anything in trousers—even the waiters."

"Keep your voice down, for God's sake!" Beaumont mumbled. "Rees is poison to you and me."

"Who are you kidding?" English said, and laughed.

Rees came up to the bar and sat away from English. He nodded stiffly to Beaumont and then to English.

English nodded back, waved a careless hand at Lola who glared at him before turning her back.

"When she tried to make the bellhop, I thought it was time she went," English said. "As you can see, she still nurses a grudge."

Beaumont hurriedly switched the conversation to the coming election, but Nick wasn't listening. He had seen Corinne English, standing in the doorway. She was wearing a white evening dress that had seen better days. Her hair was untidy, and her face was flushed. Already people were staring at her.

"Here's Roy's wife," English said. "This is the last time I come to this restaurant. Every crum in town seems to be patronizing it."

Beaumont looked across the room, his small, wiry frame stiffening.

"Well! She looks drunk," he said, clutching hold of the arms of his chair.

"She is drunk," English said.

He pushed back his chair and stood up as Corinne made unsteady progress across the bar towards him.

"Hello, Corinne," he said, "If you're alone, perhaps you'll join me."

"Hello, louse," she said shrilly. "I'd rather be in a snake pit than with you."

The hum of conversation in the bar petered out, and all eyes turned to English in a silence that seemed to pile up around him like a snowdrift. He continued to smile.

"If that's the way you feel, Corinne," he said quietly, "then I'm sorry I asked," and turned back to his table.

"Don't run away," Corinne said shrilly. "I've got a lot to say to you," and she grabbed hold of his arm, pulling him around.

A hard-faced man in a tuxedo appeared suddenly behind the bar. He looked quickly at English, then said something to the barman.

English made no attempt to shake free from Corinne's grip.

"Your whore's in bed with Harry Vince," Corinne said, raising her voice. "They've been lovers for months, you poor, stupid sucker! Every time you have a business date, she sneaks off to his apartment. She's in bed with him right now!"

People were leaning forward, staring and not missing a word. The hard-faced man came out from behind the bar and walked smoothly up to English.

"Shall I get her out, Mr. English?" he asked without moving his lips.

"It's all right," English said gently, his face expressionless. "I'll do it Come on, Corinne. I'll see you home. You can tell me all about it as we go."

Corinne stepped back, her face going white. She expected some reaction from English but his calmness cut the ground from under her feet.

"Don't you believe me," she screamed. "I tell you Julie Clair's in bed with your manager!"

"Well, why shouldn't she be?" English said smiling. "What business is it of yours or mine, Corinne?"

Rees half started out of his chair, thought better of it and sat down again.

Lola said in a clear, hard voice, "My God! How absolutely disgusting."

"Come on, Corinne, let's go home," English said, taking her arm.

"Don't you mind?" Corinne wailed, trying to pull away from a grip that looked gentle but that held her like a vice.

"Why, no, I don't think I do," English said soothingly as if talking to a child. "You know as well as I do, it's all nonsense. Come along. People are staring at you, my dear."

Corinne began to cry. What had seemed such a spectacular opportunity for revenge was petering out like a damp firecracker. By bis quiet, kindly behavior, she could feel that English had the crowd with him. They looked on her as some souse making a scene.

She made one more desperate attempt to save the situation.

"It's true!" she screamed, trying to break free. "And you killed your brother! You robbed me of twenty thousand dollars. Let go of me!"

A man laughed suddenly, and she knew with a sickening sense of frustration that she had muffed the whole plan.

English continued to walk her from the bar into the empty lobby. The hard-faced man who had followed them, said "Shall I give her a little tap, Mr. English?"

"Why, no, Louis," English said, "but I'd be glad if you could see her home. Get a taxi, will you?"

"Okay, Mr. English."

Corinne leaned against English and continued to cry. He put his arm around her.

"Take it easy," he said. "You get off home and have a sleep, I know how you're feeling."

"You don't," Corinne moaned. "I wanted to hurt you. I wanted to make you suffer as you have made me suffer."

"How do you know you haven't?" English said, and tilted up her chin. "Is it true?"

She couldn't meet his eyes.

"Is it true?" he repeated.

She nodded.

"Well, that's all right. Then we're quits. I shouldn't have threatened to hand Roy's letters to the press. I wouldn't have done it, of course, but I shouldn't have used such a threat against you."

Louis came up.

"The taxi's here, Mr. English. Ill take care of her."

English watched Louis lead Corinne across the sidewalk to the waiting taxi. His face was a little pale now, but still expressionless.

Beaumont joined him.

"My God, Nick! The press will get this. Why the hell didn't you stop her talking? Rees was drinking it in. He'll spread it all over the town."

"Shut up!" English said harshly. "I played it the right way. Do you think that anyone will believe that drunken little act?"

Beaumont hesitated.

"Is it true?"

English turned and looked at him. His light-blue eyes were chips of ice.

"What the hell is it to do with you or anyone else if it is true or not?"

Beaumont recognized the danger signals.

"That's right It's none of my business," he said hurriedly. "Well, maybe we'd better go into dinner."

"I'm not staying. I have something to do," English said. "I'll see you tomorrow, Senator. Excuse me now."

He walked over to the cloakroom, got his hat and coat from the check girl, and walked across the lobby to the revolving door.



IV

At ten minutes to eight o'clock, Roger Sherman turned out the lights in his bedroom and moved over to the double windows that overlooked the street.

He was dressed to go out. His brown slouch hat was pulled down low over his face, and the collar of his coat was turned up.

He lifted the shade a few inches away from the window and peered down into the street. Rain, beating against the glass, made it hard to see clearly. Sherman's eyes searched the opposite doorways. He spotted the figure of a man, standing in a porch out of the rain, the red tip of a cigarette pin pointing his face, half concealed under a pulled down hat brim.

He crossed the room, opened the door that led into the kitchen and went to the window without turning on the light. Again, he lifted the shade and looked down into the back street that ran the length of the rear of his apartment block. He finally spotted another man standing under a tree, and nodded.

It was now obvious to him that English was making certain that he would be kept informed of his movements. Since noon, Sherman had known that he was being tailed, and tailed by experts.

These two men knew their business, and they didn't seem to care if he was aware or not that they were tailing him. They were intent only on not letting him give them the slip.

They were now waiting for him to make a move, guarding the only two exits of the block, and it was essential to his plan that he wasn't followed this evening.

He returned to the living room and turned on the radio. Then he took from his pocket a pair of thin silk gloves—so thin, that when he put them on, they seemed to form a second skin on his hands.

He went over to his desk, opened a top drawer and took out a .38 Colt automatic. He released the clip, checked the bullets replaced the clip and jacked a bullet into the breech. He clicked down the safety catch and slipped the gun into his overcoat pocket.

Leaving the lights on in the living room, knowing the watcher below had a clear view of the lighted windows, Sherman walked softly to the front door, opened it a few inches, and peered into the empty passage.

Sherman stepped out, closed the front door, and walked swiftly and silently to the staircase. He went up, two steps at a time, until he reached the next landing. He paused for several seconds while he leaned over the banister rail, listening.

He went along the passage to a window, pushed it open and looked out into the dark night. Below was a sheer drop of a hundred feet or more. The window looked out onto the roofs of houses and business premises, dwarfed by the vast block. He glanced back down the passage, then put one foot up on the window sill, and holding on to the window frame, he stood up, half in and half out of the window.

He reached up and his fingers closed around a narrow horizontal pipe that ran the length of the building. Holding on with one hand, he reached in and closed the window.

Rain beat down on him as he braced himself against the face of the building. His left hand went up and caught hold of the pipe.

The pipe was wet, and felt slippery—something he hadn't bargained for, and he cursed the rain. But this was the only way he could leave the block if he was to avoid the two men waiting for him below, and he didn't hesitate.

He shifted his hands until his body was at an almost forty-five degree sideways slant, his hands on the pipe, his feet on the window sill. Then he swung his feet clear of the sill and hung in space by his hands.

With the agility of a gymnast, he swung himself along the pipe, hand over hand, until he reached a drainpipe that went down to a foot-wide ledge about twenty feet below his dangling feet.

He had one dangerous moment, as he was changing his hold from the horizontal pipe to the vertical one.

His right hand failed to get a grip and he swung outwards, hanging on only by his left hand.

He looked down into the dark depths below, his jaws moving rhythmically as he chewed, completely unafraid and unruffled. His right hand clawed out for the drainpipe, got a grip, and he pulled himself against it, digging his knees into the sides of the pipe while he slowly released his grip on the horizontal pipe with his left hand.

He remained like that, clinging on with hands, knees and toes until he had properly adjusted his balance, then he began to let himself down, inch by inch, until he reached the ledge.

He stood against the face of the building while he recovered his breath.

Thirty feet below him was a flat roof, an ugly projection that covered the kitchens of the restaurant of the apartment block.

He rested for a minute or so, then gripped the vertical pipe again and lowered himself to the flat roof. Bending low, to avoid being seen against the skyline, he walked silently to the edge of the roof to the fire escape ladder that would take him to the ground. He went down the ladder swiftly.

He found himself in a dark alley, lined with garbage cans— the tradesman's entrance to the apartment block. At the far end of the alley was the main street, and he walked quickly and silently towards it. When he reached the end of the alley, he stopped and peered cautiously into the street.

Some thirty yards to his right was the main entrance to the apartment block. He looked across the street. The watcher was still in the porch, sheltering from the rain, his eyes on the revolving doors opposite.

Sherman pulled his hat brim lower over his face, and moved out of the mouth of the alley, keeping in the alley, keeping in the shadows, his eyes fixed on the man in the porch.

He began to walk sideways away from the watcher, but the the man in the porch didn't look his way, and as Sherman turned the corner into the street, he gave a little nod of satisfaction.

He walked in the rain for some minutes until he was well clear of the block, then he signaled to a passing taxi.

'Take me to Fifth and Tenth Street," he said, got into the cab and slammed the door.



V

Julie lifted her head from the pillow and peered at the clock. The hands showed three minutes after nine o'clock.

"It's not time yet, surely?" Harry Vince said, pulling her close to him.

"No. Another half hour. Dear Harry," Julie sighed, her hand touching his bare chest. "I wish I didn't have to leave you. Time goes so quickly."

"Nick will be tied up for hours yet," Harry said. "Can't you give up the club tonight, Julie? Can't you give it up altogether? I want you all the time."

"And I want you," Julie said, a little untruthfully. She lifted her face so he could kiss her, and for some moments they gave themselves up to their love. Then Julie said suddenly, "Better not, darling. No, really, Harry, I must be going in a few minutes."

"You're going to be late, Julie," he said. "I don't care, and you're not going to care."

"I mustn't be," Julie said, her mind only half made up.

"You're going to be."

"Then quickly, darling," she said, and she kissed him so hard that he tasted salty blood on her mouth crushed against his. "Oh, darling," she said, and caught her breath. "Oh, darling, darling, darling!"

Time stood still for them. Only their quick breathing and her sharp little cry of pleasure disturbed the silence.

Then suddenly he felt her fingers stiffen into little hooks, digging into his shoulders, and her body was like a bow that has been bent by its string.

"What was that?" she said sharply, her mouth against his ear. Her hands pushed him away, and she half sat up, staring into the firelit darkness.

"What's the matter?" he asked, lying back on the pillow and frowning at her.

"I heard something," she said, and he saw how pale she had gone as the light from the fire lit up her tense face.

A cold chill snaked down his spine, and he, too, sat up to listen.

"There's someone in the other room," Julie whispered.

"Can't be," he said, feeling suddenly sick. "The door's locked. You're imagining things."

"No. Someone's there," Julie said, and her groping hand caught his. "I know there is."

Harry tried to listen, but all he could hear was the steady hammering of his heart, and the sound of blood pounding through his arteries.

"There can't be anyone," he said hoarsely. "You're scaring me out of my wits, Julie."

"Go and see," she said. "Fm sure I heard something."

He hesitated, not believing her, but wondering if English could have got in. Suppose he had? Suppose that when he opened the bedroom door, he found English out there.

"Harry!" Julie said sharply. "Go and see!"

He pushed back the sheet and swung his legs off the bed, his hand grabbing up his dressing gown.

"You're imagining it," he said. "There's no possible way for anyone to get in."

Then he stiffened into a rigid, horrified stillness, feeling the hair on the nape of his neck bristle.

Across the silence of the room came a faint, scraping noise, then very slowly the bedroom door began to open.

"Oh, Harry!" Julie breathed, her fingers digging into his arm.

Roger Sherman came in. In his right hand, he held the Colt automatic. Dark patches from the rain stained his raincoat. Water dripped from the rim of his hat. He stepped into the room, his jaws moving slowly as he chewed, his amber eyes reflecting the bright flame from the fire.

The automatic swung up and covered Harry.

"Don't move," Sherman said quietly, "either of you."

He came further into the room and shut the door.

The relief to Harry that it wasn't English, was overpowering.

"Get out of here!" he said, his voice still unsteady, his eyes on the gun.

Sherman moved over to an armchair by the fire, and sat down. His deliberate, quiet movements, horrified Julie.

"Stay where you are," he said, crossing one leg over the other. The automatic pointed between Julie and Harry. "Don't do anything stupid or I'll have to kill you."

"Who—who are you?" Harry said, suddenly realizing this man couldn't be a burglar—he was too well dressed to be that.

"My name is Roger Sherman," Sherman returned mildly. "Not that that will tell you anything." His amber eyes moved from Harry to Julie who was holding the sheet over her breasts, staring at him with wide, frightened eyes. "Hello, Julie. You don't know me, but I know you. I've been watching you two for some time. It seems to me that you're taking unnecessary risks coming here. After all, you were paying Roy English to keep his mouth shut, weren't you?"

"How do you know that?" Harry said, his face paling.

"My dear man, I was English's boss."

"So it's blackmail'. All right. How much?"

Sherman smiled.

"This time it's not money I want. I'm using you two to bait a trap."

Harry felt Julie stiffen. He half turned, taking her hand.

"What do you mean?"

"I'm waiting for Mr. English," Sherman said, and smiled. He looked at Julie. "By now, he should have heard what you two are up to. I imagine he'll come here as fast as a car can bring him."

"Now, look," Harry said, feverishly. "I don't care what it costs—I'll pay to get out of this. How much?"

"It's not a matter of money . . ." Sherman said softly.

"But this won't get English out of your way," Harry said desperately. "It'll make him all the more determined to crack down on you. All right, I admit it, it'll hit him hard, but you don't know him as I do. He'll hit back."

Sherman lifted the Colt in his silk clad hand. "This is his gun. I stole it from his apartment this afternoon. He's going to be arrested for murder—two murders, in fact."

Harry stiffened.

"What do you mean?"

"It's obvious, isn't it? When I hear his car arrive, I'm going to shoot you both. Who's going to prove he didn't do it?"

Julie caught her breath sharply.

"He's bluffing, darling," Harry said. "He wouldn't dare."

She was looking at Sherman. The expressionless eyes terrified her.

"He's going to do it," she said, through dry lips.

"Of course I am," Sherman said mildly. "You two have had your fun, and now you're going to pay for it"

"You won't be able to get away!" Harry exclaimed. "You'll be caught."

Sherman laughed.

"This window overlooks the river. I shall go that way. I'm a good swimmer."

"You can't do it!" Harry said, realizing that Sherman wasn't bluffing. "Let her go," he went on, huskily. "Don't touch her. One murder's enough."

"Sorry I can't oblige," Sherman returned. "You must see that I can't afford to let her live after I have shot you. She would give me away."

"She wouldn't," Harry said. "She'd promise not to."

"Sorry," Sherman repeated. "Besides, a double killing is much more dramatic. English might get off if he just killed you, but the jury wouldn't like him killing Julie." He moved back to the chair and sat down again.

Harry decided that he was dealing with a lunatic. Somehow, he had to divert Sherman's attention, and then close with him. He judged the distance between them. He was badly placed, as he was sitting on the side of the bed away from Sherman. Eight to nine feet separated them.

Julie said, "I'll give you all the money I have if you'll stop this. I can raise twenty thousand. If you give me time, I can get more."

Sherman shook his head.

"Save your breath," he said. "I'm not interested in money." He glanced at his watch, and Harry's hand reached behind him and gripped the pillow.

Julie saw the move. She was breathing quickly, her face white and drawn. She sensed Harry was going to do something.

"I—I think I'm going to faint," she gasped, closing her eyes, and she reached out as if to steady herself and her hand pushed over the night table which crashed to the floor.

Sherman's eyes went from Harry to the overturned table. Harry flung the pillow, threw himself off the bed as the pillow hit Sherman in the chest, smothering the gun.

Harry, white-faced, his eyes staring, sprang forward, propelling his body across the nine foot space toward Sherman.

Sherman half started up, throwing the pillow from him. Harry saw that he couldn't reach Sherman before Sherman shot him, but he kept on, his mouth dry, his heart hammering, trying to close the space between himself and Sherman.

There was a crash of gunfire. The bullet got Harry just below the knee, bringing him down. His hands caught Sherman's belt, gripped, dragging Sherman forward.

Sherman hit Harry a glancing blow with the gun barrel on his temple and kicked him away. He was completely unruffled, and his jaws moved rhythmically as he looked quickly at Julie who crouched petrified on the bed, the sheet fallen from her, her hands covering her breasts.

Harry rolled away, blood running down his leg. He began to crawl toward Sherman, his lips drawn off his teeth in a snarl.

Sherman backed away, smiling.

"You fool!" he said, softly.

Harry kept on. The pain in his shattered knee filled him with a murderous rage. He wasn't frightened any more. All he wanted to do now was to get his hands on Sherman.

Sherman raised the Colt, and aimed carefully. Harry was only a few feet from him. He looked up at the little black sight of the gun pointing at him and at the cold amber eyes squinting along the barrel.

Julie screamed wildly.

"Don't! No—don't!"

The bullet took Harry squarely between the eyes. The force of the blow threw him backwards and he rolled over on bis side, his fingers opening and closing convulsively, his muscles twitching, blood smothering his face.

"A little premature, I'm afraid," Sherman said, frowning. "Well, it can't be helped."

He heard a car door slam, and he smiled.

"Here he is," he said, and moved quickly to the window. He pulled aside the curtains, opened the window, and glanced out.

"Go to him, Julie," he said softly, pointing to the door. "Let him in."

Julie didn't move. Her eyes turned from Harry's body to Sherman. She scarcely seemed to breathe.

The front door creaked as English threw his weight against it.

Sherman raised the Colt as Julie's hand closed over the key in the lock. The sight of the gun aimed at a point in the exact center of her shoulders.

Something seemed to warn her that he was going to shoot, and she looked back over her shoulder.

Her terrified scream blended with the crash of gunfire. A small blue-black hole appeared between her shoulder blades. She was flung against the door, and her knees sagged.

Sherman shot her again. The bullet got her above her right hip. Her body arched in agony, her hands clawed at the door, then her knees hinged and she fell face down, her arms and legs sprawling.

Unruffled, Sherman tossed the gun onto the floor near where she lay, turned and went swiftly back into the bedroom, across to the window.

He stepped up on the sill as he heard the front door crash open. Still unruffled, he paused long enough to draw the curtains, then he got out on to the sill, closed the window, straightened and dived without hesitation into the dark river flowing below him.

\vspace{2\nbs}
\ChapterDeco[c1]{\decoglyph{e9665}}
\clearpage
\thispagestyle{empty}

\begin{ChapterStart}
\vspace{3\nbs}
\ChapterSubtitle[l]{Chapter ch6}
\ChapterTitle[l]{ch6}
\end{ChapterStart}
\FirstLine{\noindent Lois Marshall leaned forward and impatiently snapped off the television set. She turned on the shaded lamp, and bent to poke the fire. Rain continued to patter against the window-panes. Restlessly, she glanced at the clock on the mantel. It was ten minutes after nine.}
    
She had been thinking regretfully of English's suggestion that they should have dinner together on Saturday night. It was the first time he had asked her to go out with him, and she had been badly caught off balance. Her immediate reaction was to have accepted, then she realized that Julie would find out, and she would tell Harry Vince who would tell someone else, until it was all around the office that poor Lois had at last been taken out by the boss.

She was sure that most of the staff, including Harry, guessed she was in love with English. Blood rose to her face as she thought of the gossip that probably went on in the office about her. Well, she was in love with English! It was something she couldn't help.

She got up and fetched her work basket, and settled down before the fire again. She was essentially a domestic sort of person, and would nave preferred to run a home than work in an office; and the small pile of mending that she had saved for a rainy evening, had a soothing effect upon her.

She paused in her work to look around the sitting room, and it pleased her. It would have pleased her more if she didn't have to live in it alone. Again, she headed herself off from brooding, and to divert her thoughts, she leaned over to switch on the radio when the front door bell rang.

She frowned, her eyes going to the clock. It was now twenty-five minutes to ten. She hesitated, wondering whether to go to the door or not

The bell rang again—two sharp impatient rings.

She laid aside her mending and walked into the lobby. Quietly, she slipped on the chain, then keeping to one side, she opened the door a few inches.

"Who is it?" she asked, sharply.

"Can I come in, Lois?" English said.

She felt herself turn hot, then cold. Quickly, she controlled herself.

English stood just outside. His light gray overcoat glistened with dampness.

"I'm sorry to call so late, Lois," he said quietly. "Am I in the way?"

"Of course not. Come in," she said, a cold feeling around her heart at the sight of his white, drawn face.

He entered the living room, and stood looking around.

"What a nice room, Lois," he said. "I can see your hand in everything here."

"I—I'm glad you like it," she said, watching him. She had never felt so frightened before. She could tell by his expression that something bad had happened, and she knew that he would never have come to her apartment unless he had nowhere else to go.

"Can I have your coat, Mr. English?"

He smiled at her.

"Don't let's be formal tonight, Lois. Call me Nick, will you?" He pulled off his coat.

"I'll take it into the bathroom," she said. "Go over to the fire, Nick."

"That's better," he said.

When she returned, he was sitting before the fire, his hands out toward the blaze.

She went over to the sideboard, mixed a stiff highball and brought it to him.

He took it, and smiled up at her.

"You always know the right thing to do, don't you?"

She saw his eyes were frozen and hard.

"What's happened?" she asked, sharply, standing before him.

He gave her a sharp look, then reached out and patted her hand.

"Sorry, Lois," he said. "This is going to be a shock. Julie was murdered tonight. She and Harry. It all points to me."

"My God!" she said, then she pulled herself together. "What happened, Nick?"

"I was having a drink with Beaumont," English said, speaking rapidly. "Corinne came in. She was drunk. The bar was crowded—everyone, including Rees and Lola Vegas, heard what she said. She told me that Julie and Harry had been lovers for months—that Julie was with Harry in his apartment. I got rid of Corinne and took a taxi to Harry's place. The door was locked. I knocked and called out. Julie answered. She sounded terrified. She said she was going to be shot. She screamed for me to save her. It took me some moments to get the door open. I heard a shot; then another. I smashed the lock. Julie was lying on the floor. She was dying." He paused and took a long drink, set down the glass and rubbed his eyes. "She died hard, Lois. She didn't deserve a death like that. She said it was Sherman who shot her. That he had gone out through the bedroom window. I held her in my arms until she died." He groped in his pocket, vaguely. Lois reached out, took a cigarette from a box, lit it and gave it to him. "Thanks," he said, not looking at her. "I hope I made things a bit easier for her," he went on, half to himself. "She was frightened I'd be angry with her. She didn't seem to realize she was dying. She kept asking me to forgive her."

Lois suppressed a shudder.

"What happened, then," she asked, sharply.

He looked up and frowned.

"I went into the bedroom. Harry was on the floor. He was dead too. I pulled aside the curtain, but I couldn't see anyone in the river. It was dark and raining hard. I went to the telephone to call the police, then I saw the gun on the floor. It looked familiar. I picked it up. That was stupid of me. It was my gun. It's been in my desk for years. Then I realized what a frame he had built for me. A dozen witnesses will testify that Corinne had told me that Julie and Harry were lovers. The taxi-driver will testify he took me to Harry's apartment. The gun that killed them is my gun. They were shot a minute or so after I had arrived. The motive, the time, the weapon—what more can the D.A. want?"

"If Sherman killed them," Lois said quietly, "Leon will know about it. He was following Sherman, wasn't he?"

English stiffened, and then drove his right fist into the palm of his left hand.

"Why, damn it! I'd forgotten that. Of course, Ed wouldn't have let him out of his sight. That's it! I believe we've got him, Lois! Try and get Ed. Call my apartment first."

As Lois began to dial the number she said, "You didn't call the police?"

"No. I walked out. I wanted to get my bearings."

"You left the gun?"

"Yes."

Leon's voice came over the line.

"Hello?"

"This is Lois Marshall," Lois said. "Did you keep contact with Sherman tonight."

"He never left his apartment," Leon returned. "What's the idea? Why are you calling?"

"He says Sherman didn't leave his apartment," Lois said, turning cold as she looked at English. "Are you sure he didn't leave?" she went on to Leon.

"Of course, I'm sure! Both exits are guarded. There's no other way out. Besides, I've been along to his apartment every half hour. The radio's playing non-stop, and the lights are on."

"He's certain Sherman didn't leave his apartment," she said, turning to English.

'TeJl him to come here at once!"

"Will you come to my apartment?" she said. "It's urgent."

"I'm waiting for English," Leon said, impatiently. "What's the trouble?"

"I can't talk on the telephone," she said. "You must come right away."

"Well, all right," Leon growled, and hung up.

"Shall I get Mr. Crail?" Lois asked, as she broke the connection.

English nodded.

"Yes. But not that he can do anything."

While she was dialing Crail's home number, English began to pace slowly.

"Julie couldn't have been mistaken," he said savagely. "She described Sherman. Damn Leon I"

Lois spoke over the telephone, and then hung up.

"He's coming," she said, and went unsteadily to a chair and sat down. "You shouldn't have left the gun, Nick."

"The gun doesn't matter," English said, continuing to pace up and down. "It would ruin my case if I hid it. I've got to stick to the truth, Lois, if I'm to beat the rap. I've got to prove that Sherman stole my gun."

"How did Corinne know about Julie?" Lois asked.

English frowned.

"I don't know, unless . . ." He stopped to think. "Yes! That's it! Of course! Roy was blackmailing Julie. He must have found out what was going on between Julie and Harry. He must have told Corinne."

"Don't you think it's more likely that Sherman told Corinne?" Lois said. "Don't you think they're working together?"

"What makes you say that?" English asked, staring at her.

"How could Sherman know for certain that you would go to Harry's apartment?" Lois said. "How could he be sure you'd arrive while he was there unless the whole thing had been planned? Of course, Corinne was in on this!"

"I believe you're right," English said. "If we could get her to talk . . . ! I'll tell Ed to pick her up as soon as he gets here. If we can get her to talk, we're half way to proving Sherman did it."

"I'll get her," Lois said, jumping to her feet. "You have to talk to Leon. It'll only waste time for him to go. I'll be back by the time you have finished talking to him."

"She may not come," English said, uneasily.

"Oh, yes, she will," Lois said, her face hardening. "I promise you that." She went quickly into her bedroom to change. She came out a few minutes later, struggling into a coat. "Don't move from here, Nick," she said. "I won't be half an hour."

"I don't like your going," English said. "It's raining like hell."

Lois tried to smile.

"A little rain won't hurt me. I won't be long."

He reached out and took her hand.

"I'm damned if I know what I should do without you," he said.

She pulled her hand away and ran to the door, fighting back her tears.

"I won't be long," she repeated, huskily.



II

Roger Sherman's fingers hooked over the rungs of the iron ladder. Slowly, he hauled himself up, paused to look up and down the deserted waterfront, and then climbed on to the jetty.

Moving quickly and silently, he came to a dark hut that stood at the shore end of the jetty, pushed open the door and entered a room half full of empty crates and barrels.

He dipped into one of the crates and pulled out a suitcase he had left there the previous evening.

He stripped off his wet clothes and rubbed himself down with a towel. Then he took from the case, a complete change of clothing, dressed quickly and packed bis wet clothes in the case.

He left the hut, looked to right and left, then dropped the case into the river. It sank with scarcely a ripple. Again, he looked to right and left, and satisfied he had the waterfront to himself, he walked quickly off the jetty, up an alley until he reached Tenth Street.

He headed for the subway, and he took an uptown train and got off at 110th Street. He walked the length of the street before hailing a taxi.

"Mason Street," he said, as he climbed in.

He sat in the corner of the taxi, chewing, his eyes thoughtful, every now and then glancing through the rear window.

He left the taxi at the corner of Mason Street, turned left at Lawrence Boulevard and, still keeping in the shadows, walked quickly to the door of Corrine English's bungalow.

He pressed the bell, grimacing as he heard the chimes on the other side of the door. He waited several minutes, frowning, then he pressed the bell again.

A light sprang up in the lobby and the front door opened. Corinne stood before him, holding on to the door. She smelled of liquor.

"Who is it?" she said, peering at him.

"Have you forgotten me so soon, Corinne?" he said softly.

He saw her stiffen, and her hand went to the door handle. He put his foot against the door to stop her slamming it in his face.

"What do you want?" she said, sullenly.

His amber-colored eyes searched her face.

"I was expecting you to call me, but you didn't I think Td better come in."

"I don't want you to come in," she said, trying to close the door. "I don't want to see you any more."

He moved forward, forcing her back into the hallway.

"I'm getting wet," he said, with deceptive mildness. "Did you see English?"

She turned and went unsteadily into the living room. She lurched as she reached the fireplace. On the mantel, was a bottle and a glass half full of brandy.

He dropped his wet coat and hat on the floor of the hall, then he turned and quietly pushed the bolt on the front door.

He walked into the living room, smiling.

"You haven't answered my question. Did you see English?"

"I saw him," she said, and dropped on to the couch, holding the glass of brandy, slopping some of it as she sat down.

"You don't sound very happy," he said. "Wasn't our idea a success?"

"It was your idea, not mine," Corinne said, "and it was a lousy idea. He didn't give a damn."

Sherman went over to the liquor cabinet, selected a brandy glass and came over to the fire. He half filled the glass, sniffed at it, and cocked his head on one side.

"It was a very good idea," Sherman said. He drank some of the brandy and put down the glass. "Tell me what happened."

"I'm not going to. It was horrible!" Corinne said, and began to cry. "I wish I hadn't done it. They—they laughed at me."

"Who laughed at you?" Sherman asked, his eyes intent.

"I don't know. They all laughed at me. He was so damned smooth about it. They could see I was drunk."

"Who are they'?"

"The people in the bar, of course." Corinne's voice went shrill. "Who else do you think?"

"You told English they were lovers then?" Sherman asked, watching her.

"Of course, I did! That's what you told me to do, and he didn't give a damn. He said it wasn't my business, nor his," Corinne said, dabbing her eyes. "He sent me home with some smooth punk from the club. That's how your lousy idea worked out."

Sherman nodded. He had learned what he wanted to know —that there had been witnesses to Corinne's outburst.

"You might be interested to know," he said, "that after you had left the club, English went to Vince's apartment. He found Julie and Vince there. He shot Vince, and then Julie. The police are already on the scene, and I imagine English is under arrest by now for murder."

Corinne stared at him, her plump, baby face seemed to shrink, and her big, blue eyes looked enormous.

"He shot them?" she said, huskily.

"That's what he did," Sherman said, taking out a package of chewing gum and stripping off the paper. "Do you think my idea is so lousy now?"

"You mean—he killed them?" Corinne's voice went up a note.

"Yes, he killed them."

"I don't believe it!"

"You will when you see tomorrow's newspapers."

"How do you know? You talk as if you were there!"

"I wasn't very far away," Sherman said, smiling. "I, more or less, saw what happened."

"I didn't want them to be killed!" Corinne said, starting to her feet. "I—I only wanted to hurt him!"

"You have hurt him," Sherman said. "You've done more than that—you've ruined him! He'll go to the chair."

"But I don't want to ruin him!" Corinne wailed. "He was kind to me."

"How touching," Sherman said, with a little sneer. "In spite of the fact that he didn't hesitate to steal twenty thousand dollars from you?"

Corinne stared at him, her fists clenched.

"I don't believe Roy ever had all that money," she said. "I was a fool to have listened to you. You're responsible for this. It was your idea. You wanted to get even with him, and you used me to do it!"

"What a clever girl you've suddenly become," Sherman said, smiling. "Suppose that was so, what are you going to do about it?"

"I'm going to the police!" Corinne said. "It was a terrible thing to do."

"Don't be stupid, Corinne," Sherman said. "There's nothing you can do now except keep your mouth shut."

"We'll see about that!" Corinne said, angrily. "I'll talk to Lieutenant Morilli. I'll tell him it was your idea."

Sherman nodded as if he expected her to say that. He began to wander around the room, his hands in his pockets, his jaws moving, his eyes expressionless.

"Yes, I suppose you will," he said, pausing by the window. He reached out and took hold of a red silk curtain cord, hanging by a hook. His fingers absently tested its strength.

'This is an extraordinary thing," he said, "I've been looking for a curtain cord like this for weeks. You wouldn't believe it, but I can't find this exact shade anywhere." He took the cord,off the hook and moved over to the lamp to examine it "Do you happen to remember where you bought it?"

"You're not going to put me off like that!" Corinne snapped. "You're trying to change the subject. I'm going to telephone Lieutenant Morilli right now!"

"I'm not trying to change the subject," Sherman said, mildly. The cord hung like a red snake in his fingers. "I do wish you could remember where you bought this."

"I don't remember," Corinne said, and picked up the telephone book.

"Well, if you can't remember, you can't—a pity!" Sherman said.

Corinne was bending over the telephone book she had placed on the table. Sherman moved behind her. He arranged the cord into a loop.

The sudden sound of chimes at the front door turned him into a motionless statue.

Corinne looked up, frowning. She saw Sherman's reflection in the mirror above the mantel. He was standing close behind her, his hands raised, the loop of cord hovering above her head.

She knew at once what he was about to do, and she stumbled aside, keeping her back turned to him.

"I'll answer it," she managed to get out, and before he could stop her, she ran unsteadily to the door, opened it and went into the hall.

She tried to open the front door, her knees buckling under her. Then she saw the bolt had been pushed home and she jerked it back.

A tall, dark girl in a soaked raincoat stood on the step.

"Mrs. English?"

Corinne nodded. Her breath whistled through her open mouth and she was trembling so violently that she could hardly stand.

"I'm Lois Marshall, Mr. English's secretary," Lois said. "May I come in?"

"Oh, yes," Corinne gasped. "Yes, come in."

Lois looked at her sharply as she stepped into the hallway.

"Is anything the matter? You look frightened."

"Frightened?" Corinne said huskily. "I'm terrified. There's a man in there . . ."

Sherman came to the living room door, a .38 Police Special in his hand. He pointed it at Lois, and smiled.

"Come in, Miss Marshall," he said quietly. "Unexpected, but nevertheless welcome."

Corinne's hands fluttered to her face.

"I—I think he was going to strangle me," she said, and slid to the floor in a faint.



III

English lifted bis hands.

"Well, there you are! That's the setup. How do you like it?"

Crail took out his handkerchief, and wiped his sweating face.

"This is bad, Nick," he said in a hard, tight voice.

"A master of the understatement," Leon said from his armchair. "The man says it's bad. Brother, it's a lot worse than bad—the lid's blown right off."

English said curtly, "You haven't been much help, Ed. I told you to watch that devil."

"Take it easy," Leon said. "We were watching-him. I hired two of Black's men, and they're good. We haven't let him out of our sight since noon. There are only two exits to Crown Court, as you know. I had them both covered. I remained in your apartment, and every half hour I went along to Sherman's apartment and listened outside the door. He was in there, playing his radio."

"But he shot Julie and Harry!"

"Sure she didn't make a mistake?"

"No, she described him. It was Sherman all right."

"He couldn't have left the building."

"Is he there now?" Crail put in.

"He should be. When Miss Marshall called me, I left Burt and Horwill watching the entrance and the rear exit. I guess he's there or they'll know about it,"

English went over to the telephone, dialed Sherman's number and listened to the steady ringing. After a while, he hung up.

"He doesn't answer."

"That doesn't prove he isn't there," Leon said.

"There's only one thing you can do," Crail said. "Come down with me to headquarters and let me give the Commissioner the whole story."

English smiled sarcastically.

"How he'll love it. How Rees will love it. How the Mayor will love it. Do you think any of them will believe me? Not a chance in hell!"

"He's right," Leon said.

"He's got to give himself up!" Crail said violently. He turned to English. "You can see that, can't you? It's your only hope of beating this rap."

English shook his head.

"Take it easy, Sam. I know your advice is sound, but you're forgetting what I'm up against. I've got too many enemies. Rees is only waiting for a chance to fix me, and I've given it to him. With me in jail, the D.A. knows that Beaumont will fold up. It can't be done. No matter how smart you are, you can't beat the combination. It's too strong. There's only one way of beating this rap. We've got to find Sherman, and we've got to crack him so that he'll come clean. There's no other way."

Crail stood looking at English for a long moment, then he lifted his fat shoulders.

"All right. But don't forget I warned you. I'll do what I can when it comes to the trail, but you're tying my hands."

He picked up his hat and coat. "You know where to find me, Nick, when you want me. Good luck to you!"

English came over and shook hands.

'Take it easy, Sam. I've handled my affairs all right up to now, and I think this is the way to play it."

When Crail had gone, English poured a little whiskey into a glass and drank it His face was hard and pale.

"He's right, you know, Ed," he said, beginning to pace up and down. "If we can't find Sherman we're sunk."

"We'll find him and we'll make him talk."

English glanced at the clock on the mantel.

"I wish Lois would hurry up," he said, sitting down. "She's been gone three quarters of an hour."

Leon stretched his long legs toward the fire.

"Gone where?"

"To get Corinne. I didn't tell Sam because he would have started fussing about the legal end, but Corinne must have been working with Sherman. If I could talk to her, I might be able to get her to admit it. She could be a big help in upsetting Sherman. Once we've got Sherman in the box, Corinne's evidence might unseat him."

"Let's hope Sherman hasn't thought of that angle," Leon said lazily, reaching for a pack of cigarettes.

English stiffened and half sat up.

"What did you say?"

Leon glanced up, surprised at the sharpness of English's tone.

"I said I hoped that Sherman doesn't realize Corinne would be used as a witness against him. Might be bad for her if he did."

English got to his feet The look in his eyes brought Leon out of his chair.

"What's biting you?" Leon demanded.

"I must be out of my mind!" English said. "I let that girl go ...."

"So what? What are you worrying about?"

"Suppose Sherman's there. Suppose she walks into him. I shouldn't have let her go. That fellow is a homicidal maniac I'm going to see what's happened to her."

"Now, wait a minute," Leon said, his voice sharpening. "You're staying right here. How far do you think you'll get? I'll go. The chances are she'll be here before I get back."

"I'm going with you!"

"Then if she came back with Corinne, she'd find no one here. Use your head, Nick!"

English hesitated, then shrugged.

"I guess that's right. Well, get going, Ed. For God's sake, get there fast."

"Leave it to me," Leon said. He snatched up his coat and hat and plunged out of the room.

He reached Lawrence Boulevard after five minutes' furious driving, his eyes alert for any sign of trouble, but the long street was rainswept and deserted. He pulled up some yards from Corinne's bungalow, and got out of the car.

He stood for a moment in the rain, to look up and down the street, then he walked toward the bungalow, noting that there was a light on in the living room.

He went up the path and rang the bell. He waited several minutes, then rang again. No one answered the door, nor did he hear any sound of movement in the bungalow.

Cautiously, he turned the handle and pushed, but the door was locked. He rang again, then after waiting a minute, he stepped out of the shelter of the porch, on to the flower bed. ' to see if he could look into the living room, but the curtains were too closely drawn, and he could see nothing.

He walked across the wet grass to the path leading to the back of the bungalow. When he turned the handle of the door, he found it unlocked.

He pushed it open and stepped into a small kitchen. His feet kicked against something that clanked noisily, and he cursed under his bream. From his pocket, he took a small flashlight and turned it on.

He opened the kitchen door, glanced into the dark hall, listened, then moved forward, making no sound.

The living room was empty. An overturned bottle had emptied its content on the rug before the dying fire. A broken glass lay on the hearth.

He moved into the room, frowning, not liking the spilled whiskey, feeling that here might be a bit of violence. He moved around the room, his eyes missing nothing, not knowing what he was looking for but hoping to find something that would explain why the light was on and the room empty.

On the couch, pushed half out of sight, he saw something white, and he fished it out from under the cushion. It was a woman's handkerchief, embroidered in the corner were the initials "L.M."

He shook his head. Lois must have persuaded Corinne to leave with her, he thought, and they had forgotten to turn off the light

He looked around for the telephone, to call English, to ask him if Lois had returned, when his eyes encountered the overturned bottle again. He frowned. Had Corinne been tight, he wondered? Had Lois' ring startled her so that she had upset the bottle? It seemed unlikely, and he went out into the hallway.

Facing him was a door. He turned the handle and pushed it open. The room was in darkness, and he groped for the light switch and turned it on.

The bedroom was as untidy as the kitchen. In the middle of the floor was a rose colored silk wrap. Stockings, underclothes and a fur coat lay on the bed. The dressing table was a smother of face powder, and the mirror above it hadn't been dusted for days. A bottle of hand lotion had been knocked over, and its white, creamy contents had made a messy puddle on the floor.

Leon grimaced, shrugging, and as he was about to turn off the light, he paused, his eyes narrowing.

A door opposite him attracted his attention. It was open a few inches, and fastened to one of the dress hooks screwed to the door was a red silk cord that ran over the top of the door and disappeared down the other side.

The cord looked taut! Too taut, as if it were supporting a heavy weight.

Leon quickly crossed the room, pushed against the door which opened sluggishly. Something heavy bumped against the other side as he pushed.

He stepped into a blue and white bathroom, his heart skipping a beat.

He was half prepared for what he saw, but, even at that, his stomach gave a little heave as he looked at Corinne English's dead face.

She hung grotesquely against the door, her knees drawn up in agony, her baby face puffed and swollen, her tongue pushing out between her small, white teeth. The red silk cord had bitten deeply into her neck, and her hands were rigid claws as if she had been frantically trying to push someone away in the last moments of her life.

Leon touched one of her hands. It was still warm, and he stepped away, his face hard and white.

For a long moment he stood thinking, his eyes averted from the hanging body, then he moved around the door into the bedroom, walked quickly into the hall and into the living room.

He was thinking now of Lois. Had she come to the bungalow and found Corinne, or had she arrived before Corinne had been murdered?

Leon felt sweat beading his face. If he told English what had happened to Corinne, English would come out of cover. There'd be no controlling him, especially if he thought Lois

was in Sherman's hands.

Uneasily, Leon wiped his face with his handkerchief. It did look as if Lois were in Sherman's hands. He stood, hesitating, trying to make up his mind what to do. He decided he had to find out if Lois had returned to her apartment. This might be a false alarm. She might be there, and safe.

He went over to the telephone, thumbed through the telephone directory until he found Lois' number, and then dialed.

He waited impatiently, listening to the burr-burr-burr on the line.

There was a sudden click and a man's voice said, "Who is that?"

Leon stiffened, sure it wasn't English who was speaking.

"Is that Westside 57794?" he asked, cautiously.

"That's right. Who's calling?"

It wasn't English, Leon thought.

"I'd like to speak to Miss Marshall," he said.

"She's not here," the voice told him. "Who's that speaking?"

"Come to that," Leon said sharply, "Who are you?"

Leon felt a chill run down his spine. Morilli! Had English got away? He hurriedly dropped the receiver back on to its cradle.



IV

Nick English paced slowly up and down, his hands in his trousers pockets, his face set and anxious. He kept looking at the clock on the mantel. It was now a little more than an hour since Lois had left the apartment—a little less than a quarter of an hour since Leon had gone to look for her.

English calculated it would take Leon twenty minutes to get to Lawrence Boulevard. Even if he didn't find Lois there, it didn't necessarily mean she was in danger. She might have left the bungalow before Leon arrived.

What a thoughtless fool he had been to let her go! He should have realized that Corinne was dangerous to Sherman.

With a sense of shock he realized that Lois meant something to him. Only now that Julie was dead, was he able to judge Lois' worth. Julie had been a physical attraction—a doll to dress, to amuse, to sleep with. Whereas Lois had worked by his side for five hard years, and he knew that it had been largely due to her help and confidence in him, that he had succeeded.

Impatiently, he went to the window, pulled aside the shade and looked down into the wet street below.

He stood watching the empty street for several minutes, hoping to catch a glimpse of Lois. Then as he was about to drop the shade, he saw the headlights of a fast moving car coming down the street, and he stiffened to attention, wondering if it were Lois returning.

The car swung to the curb and pulled up outside the walk-up. English saw the red spotlight on the hood and recognized the black-and-white check pattern of the body. He quickly dropped the shade.

The police!

Did they know he was here, or were they checking on the off-chance of finding? He moved quickly across the room, snatched up his hat and coat, and went into the hall.

He had no idea if there were a rear exit to this building. Even if he found it, the chances were that he'd walk into one of them.

He hesitated for a moment, then tossed his hat and coat on to a chair and returned to the sitting room.

If he was cornered, then he was cornered. He'd be damned if he would run like some cheap punk.

He stood before the fireplace, his hands behind his back, his face hard and set, and waited.

Minutes ticked by, and just when he was beginning to think that it was a false alarm, the front door bell rang sharply.

He stepped quickly to the telephone, took up the receiver and dialed Sam Crail's home number. His call was answered almost immediately by Crail, himself.

"Sam? This is Nick," English said, speaking quietly and rapidly. "You win. They're ringing the bell now."

"Say nothing," Crail snapped. "I'll be at headquarters before you get there. Leave it to me, Nick. Just say nothing. Where's Leon?"

"He's not here. Keep in touch with him, Sam. I've got to rely on you two."

"You can rely on us," Crail said. "Just keep your mouth shut and leave everything to me."

"Very comforting advice," English said drily. He heard the front door bell ring again. 'They're getting impatient. See you at headquarters," and he hung up.

He walked across the room, into the hall, and opened the door.

Morilli stood in the passage, one hand in his coat pocket. His lean, hatchet face looked pallid in the soft light, and his eyes were wary.

"Hello, Lieutenant," English said calmly. 'This is unexpected. What do you want?"

"Can I come in, Mr. English?" Morilli said.

"You alone?"

"I have company, but he's downstairs."

English nodded, and stood aside.

"Come on in."

Morilli walked into the hall, shut the front door and waved English toward the living room. English went ahead, crossed over to the fireplace, and turned to face Morilli.

Morilli looked suspiciously around the living room as he came in.

"There's no one here but me," English said. "Miss Marshall is out."

Morilli nodded, ran a thumbnail along his black moustache.

"I don't have to tell you why I'm here, Mr. English?"

English smiled.

"I gave up making guesses long ago," he said. "Suppose you tell me."

"You're to be charged with the murder of Julie Clair and Harry Vince," Morilli said, and his small, hard eyes shifted away from English.

"I'm surprised you've taken the job on, Lieutenant," English said. "I had an idea you gave service."

"I'm still giving service," Morilli returned. "That's why I'm here. I thought it would be safer for you if I made the arrest."

English raised his eyebrows.

"What does that mean?"

"You wouldn't be the first guy who's been shot in the back while resisting arrest," Morilli said. "There are a lot of high-ups who would be happy to be rid of you, Mr. English."

"Including the Commissioner?"

Morilli lifted his shoulders.

"I don't know, but I thought I'd be doing you a favor to handle this myself. This is a bad business, Mr. English. The D.A. thinks he has a watertight case."

English didn't say anything.

"You went to Vince's apartment, didn't you?" Morilli asked, his eyes probing.

"Crail told me not to talk," English said, lightly. 'Tve paid him a lot of money in the past so I'd better take his advice, now, Lieutenant."

"I guess that's right." Morilli said, and again stroked his moustache. "This rap will want a lot of beating."

English said, "Well, I mustn't keep you. Shall we go?"

As he moved toward the door, the telephone bell began to ring. He made a movement to answer it, but Morilli got there first.

English watched him, his eyes narrowed, his face set.

"Who's that?" Morilli said sharply. He listened, then said, "That's right. Who's calling?" He listened again, said "She's not here. Who's that speaking?"

English felt a cold chill run down his spine. It must be Ed who was asking for Lois. That meant he hadn't found her at Corinne's place.

"This is Lieutenant Morilli of the Homicide Bureau," Morilli snapped. "Quit stalling! Who are you?"

He cursed softly as the connection was broken, then he rattled the telephone plunger.

"Operator! This is Lieutenant Morilli, police headquarters. Where was that call made from?" He waited, then said, "Thanks. Put me through to headquarters, will you?" Again he waited, then said, "Barker? Morilli. Get a car over to 25 Lawrence Boulevard as fast as you can. There may be trouble there. Call me back as soon as you've had a report. I'm at Westside 57794."

English said, "That's my sister-in-law's place. What makes you think she's in trouble?"

Morilli gave him a cold, searching stare.

"Why didn't she answer the phone?" he demanded. "What was Leon doing there?"

"Leon?" English frowned. "Was he there?"

"I recognized his voice. I'm not all that dumb. Your sister-in-law is an important witness against you. The Comissioner wouldn't want anything to happen to her."

"Why should anything happen to her? Do we go or do we wait?"

"We wait," Morilli said, curtly, and began to move about the room, his eyes shifting to English continuously.

English sat down. His mouth was dry, and his heart beat unevenly. At least, now, he would know if there were something wrong at Corinne's place.

They waited while the hands of the clock crawled forward. Then the telephone rang, and Morilli scooped up the receiver.

"Yeah, Morilli speaking," he said. "What's that? Well, for crying out loud! Did they pick up Leon? Then send a call out for him. He was there not more then ten minutes ago. I want that guy. Yeah, I'll get over as soon as I can. Let Jamie-son handle it. Okay, be seeing you," and he slammed down the receiver.

English braced himself. He could tell by Morilli's expression that something bad had happened.

"Your sister-in-law was found hanged," Morilli said, his face white with fury. "How do you like that? You wouldn't have sent Leon down to shut her mouth, would you?"

"Dead?" English said, getting to his feet.

"Murdered! Hanged like Mary Savitt was hanged, only this time I'm not covering up for you," Morilli snarled..

Where was Lois? English thought, cold fear gripping at his heart. At all costs, he must find her.

"Would ten thousand buy me anything, Lieutenant?" he said, quietly, his eyes on Morilli's face.

"Quit kidding yourself," Morilli said viciously. "Your spending days are over. By tomorrow morning, the banks won't touch your checks. The Commissioner didn't forget that money is your power. All that's been taken care of. You re washed up. Don't try to wave your dough in my face. You haven't any. Come on, let's get out of here."

"I have money in the office," English said. "Don't be a fool. No one knows I'm here. Give me an out and make yourself six thousand."

Morilli showed his teeth in a grin.

"There's an officer sitting by your safe right at this moment. The Commissioner has thought of all the angles. You haven't any money. Come on!"

English shifted his shoulders. He was determined now he wasn't going to be locked in a cell while Lois was in danger. Casually, he moved toward Morilli, but something about his attitude warned Morilli who jerked out a gun.

"Take it easy," he said, evenly.

English smiled.

"Don't be dramatic, Lieutenant. Even if I did get away, where would I go?"

"Get going, and watch your step," Morilli said.

They went out of the apartment and down the four flights of stairs to the lobby.

At the bottom of the stairs, a thickset, red-faced detective leaned against the wall, chewing on a toothpick. He eyed English over, then glanced at Morilli.

"Let's get going," Morilli said, impatiently. "We've got a murder on our hands after we've turned this guy in."

The red-faced detective went down the steps to the waiting car.

English followed him, with Morilli at his heels. As English paused by the car and set himself, Morilli rammed his gun into his side.

"Start something, and I'll spread your guts on the sidewalk!" he said, viciously.

"For a blackmailer, you show very little respect for your benefactor," English said and smiled.

"Get in!" Morilli snapped. "And watch it!"

English climbed into the car, and Morilli followed him.

"Okay, Nankin," Morilli said to the detective. "Let's have some speed."

The car shot away from the curb and headed downtown, keeping to the side streets.

English sat motionless, feeling Morilli's gun against his side, and inwardly seething. He realized his chances of escaping were slight, and his hopes would now have to rest on Ed.

As they swept over the Blackstone Bridge, English said sharply, "This isn't the way to headquarters. What's the idea?"

Morilli smiled.

"I have a call to make first Relax. You're in no hurry to get anywhere."

"But he'll get there just the same," Nankin said, and laughed.

English relaxed back into the corner of the seat. He should have guessed that Morilli wouldn't have risked bringing him in alive. He knew too much for Morilli's safety. There was the five thousand dollars he had given Morilli. Maybe there was no proof that Morilli had received the money, but an accusation like that would lead to an investigation, and Morilli's bank manager might have a story to tell.

Besides, Morilli wouldn't only be covering himself, he would also be doing a service to a number of high-ups by getting rid of English. It would be a neat way of closing an embarrassing case.

English's eyes went to Morilli's gun. It was pointing at him, and Morilli's finger was on the trigger. He decided that it would be useless to start anything in the car. He would have to make his break when they got out.

The waterfront was deserted. A good place to kill someone, English thought. A shot, and then the river.

Morilli said sharply, "Okay, Nankin." His voice sounded tight and metallic.

Nankin slowed down, steered the car into the shadows of a warehouse, and pulled up. "Get out," Morilli said to English.

English looked at him.

"What's this—an unofficial execution?"

Morilli rammed the gun into his side.

"Get out. I don't want you to bleed in this car."

As English opened the door, Nankin got out hurriedly and ran around the front of the car, pulling a gun as he did so. He covered English until Morilli got out

"Unwise to have a witness, Lieutenant," English said, calmly. "He'll blackmail you if you kill me."

Nankin laughed.

"Me and the Lieutenant work together, pal," he said. "Don't you bother your brains about us."

Morilli swung up his gun and pointed it at English.

"This is yours, English," he said. "I'm not taking a chance on you talking. Back up against that wall."

English braced himself. He was too far from the river to jump for it—too far from Morilli, to close with him. He knew he was within a heartbeat of death. He was surprised that he felt no fear—only an angry frustration that he now wouldn't be able to even things up with Sherman.

He stepped back.

"Drop those rods!" A voice barked from behind the car. "Quick, or I'll blast both of you to blazes!"

Nankin hurriedly dropped his gun. Morilli half turned, his lips coming off his teeth in a furious snarl.

A gun crashed, and he staggered, dropping his automatic and gripping his wrist, cursing.

Chuck Eagan came out from behind the car.

"Thought I'd better come along for the ride, boss," he said, cheerfully. "I never did trust this goddam flat-foot"

English stepped forward and picked up Morilli's gun. He kicked Nankin's gun across the waterfront into the river.

"Phew! You timed it a little close, Chuck," he said, with a wry smile.

"Better late than never," Chuck returned, grinning. "What do we do with these lice?"

"I want them out of the way for a few hours, Chuck," English said. "What do you suggest?"

"Easy," Chuck said, and stepping up to Nankin, he slammed him over the head with his gun butt.

Morilli backed away as Nankin fell face down.

English said, "Don't move. I'm tempted to make a hole in your hide!"

Morilli snarled at him.

"You'll be sorry for this."

Chuck hit him on the back of his skull, driving him to his knees. Then he hit him again, and Morilli spread out on the rain-soaked concrete.

"Stick with them, Chuck. Put them somewhere out of the way. I want a couple of hours to myself."

"Don't rush off alone," Chuck said uneasily.

"Stick with them," English said, curtly. "That's an order."

He walked over to the police car and slid under the wheel.

As he started the engine, he leaned out of the window.

"Thanks, Chuck, I'll remember this."

He reversed the car, and sent it shooting along the waterfront, heading uptown.



V

Lois opened her eyes, and blinked painfully up at an amber lamp that was screwed flush to the ceiling. The light sent sharp, stabbing pains through her head, and she shut her eyes, biting her hp to stop from crying out.

She lay still for several minutes. Her mind slowly came out of the fog of unconsciousness. Where was she? she wondered. She remembered seeing Corinne fall to the floor in a faint. She remembered bending over her, hearing a sudden swishing sound, feeling a shocking blow—then nothing.

She opened her eyes again, not looking at the light, and after a moment or so, the hot pricking in her eyes went away.

She was in what must be the cabin of a ship. It was a luxury cabin, paneled in walnut and furnished expensively, with taste. She was lying on a bed, and she looked hastily to see if she was still dressed. Someone had taken off her coat, hat and shoes, but otherwise she was still in the clothes in which she had left her apartment.

She slowly lifted her head, grimacing as a stab of pain drove into her temples.

"So you're all ready to join the party," a man's voice said near her, making her start. She looked quickly to her left.

A big man, with a thin white scar, running from his right ear to his mouth and with a cast in his left eye, sat in an armchair that was set against the cabin door. A cigarette hung from his thin lips, and he nursed a heavily bandaged wrist.

"That must have been quite a smack you walked into," he said, his eyes running over her. "You've been out for over an hour."

Her hand went automatically to her skirt and pulled it down as far as it would go, as she saw the expression in his eyes.

"Don't excite yourself," the man with the scar said, taking out a pack of cigarettes. "That's not the first pair of gams I've seen, and they won't be the last." He stuck a cigarette on his lower lip, flicked a match alight and set fire to the cigarette.

"Where am I?" Lois asked, her voice unsteady.

"On Sherman's yacht," the man with the scar told her. "He'll be along in a little while. He wants to talk to you."

"Who are you?" Lois asked, half sitting up.

"My name's Penn," he returned, and grinned. "I take care of Sherman's business. That's why I'm taking care of you. Anything more you want to know?"

"Why has he brought me here?"

"He wants to talk to you. Between you and me and the bedpost, sister, I don't think you're going to live much longer," Penn said, and winked. "He's knocking them off so fast, I've given up counting the bodies. He knocked off Corinne tonight. A waste of a good body—but he's like that. Did you know he stretched her neck?"

Lois' heart skipped a beat and she felt suddenly sick. "Maybe if you're nice to me," Penn went on, "I might talk him out of it. Think you would be nice to me?"

"If you come near me, I'll scream!" Lois said, fiercely.

Penn nodded, and flicked ash onto the floor.

"When Sherman's off the boat, you can scream your lungs out," he said. "There's no one within six miles of us. Well, okay, if you want it the hard way, I don't care. I like a little opposition."

Lois didn't say anything. She looked quickly around the cabin for a way of escape, but the only way out was through the door against which Penn had placed his chair.

Penn cocked his head on one side, then got to his feet. "He's coming now," he said. "Watch your step, sister; he gets mean if he's crossed."

As he moved the chair from the door, the door opened and Sherman looked into the cabin. He stood in the doorway, his jaws moving, his amber-colored eyes on Lois, his hands in his pockets.

"Get out!" he said to Penn.

The big man went past him without a word, and closed the door after him.

Sherman pulled up the chair, and sat down.

"Sorry I had to hit you, Miss Marshall," he said, mildly. "But you came at an inconvenient moment. Why did you come?"

"Why have you brought me here," Lois demanded, swinging her legs off the bed and sitting up.

"You will answer my question," Sherman said, a sudden rasp in his voice. "If you're going to be truculent, I'll call Penn, and he'll deal with you. Why did you come to Corinne English's house?"

Lois hesitated. The cold, expressionless eyes scared her, but she had no intention of telling Sherman that she had hoped to persuade Corinne to give evidence against him.

"I heard about the scene she had made at the Silver Tower," she said, quietly. "I wanted to find out if Mr. English had seen her home."

Sherman studied her, not sure if she were lying or not.

"You don't know where English is?"

She shook her head.

"Are you sure?"

Again she shook her head.

"You know, of course, he killed Julie Clair and her lover tonight, and the police are hunting for him?"

"I heard they had been murdered, but I'm sure Mr. English had nothing to do with it."

Sherman smiled.

"Of course—you're in love with him. I should have thought of that before."

Lois didn't say anything.

,cYou are in love with him, aren't you?"

"Is it any of your business?"

"It could be," Sherman said, staring at her thoughtfully. "The police haven't picked him up yet, and when a man like English is running around loose, he's dangerous. I want him picked up quickly or I'll have to do something about him myself."

"You'd better let me go," Lois said firmly. "Kidnaping is a serious offense."

Sherman smiled.

"So is murder. But I don't care to kill you just yet. I shall wait until tomorrow morning. Then if English hasn't been arrested, I must find him myself, and that's where you come in. I don't think it'll be difficult if he gets to know I'm holding you. I have an idea he'll come to terms, then, of course, he will commit suicide like his brother. They'll find him shot, with a gun in his hand. They'll find you some time later conveniently drowned, and they'll assume you died like Mary Savitt died —because you were unable to go on living without your lover. It is a convenient method, and I see no reason why I shouldn't repeat it."

"I think you must be mad," Lois said, steadily. "No one sane could talk as you do."

Sherman shrugged.

"What if I am mad? What's wrong with being mad, anyway. Why have people such a horror of being thought mad? I haven't. I'm perfectly satisfied with the way my mind works. After all, madness is just a matter of viewpoint. You say you're sane. Well, look at you. I'm not in your position. A man in what they call his right mind would shrink from murder, and as it happens murder is my only way out. I don't shrink from it—therefore, I must be mad according to you." He crossed one leg over the other. "Murder is an odd thing. It's like a snowball running down a hill. One murder leads to another. I wouldn't be in this jam if that cheap little chiseler hadn't tried to gyp me. I was a fool to have picked on him to work for me. Before he came, I had a good business. Now, if I'm not very careful, the bottom could drop out of it. It's worth a quarter of a million to me, and I'm not giving that up without a fight. I killed Roy English in a moment of anger. It would have been simpler to have kicked him out and got someone else to do the work, but I was angry when I found out he was cheating me, and I shot him. Then the snowball started running down hill. Mary Savitt had to go. She knew as much about me as English did; and when she heard he was dead, she would talk. So she had to go. Then that old fool Hennessey got garrulous, and he had to go. But by that time, your clever Mr. English was on to me. He was unwise to threaten me. At first, I thought I would kill him, but it seemed simpler and more amusing to let him ruin himself in his own way. I arranged that he should hear about his mistress and Harry Vince. I couldn't be sure he would kill them, so I did it for him. Then you had to come along and I realized Corinne English could be dangerous, so she had to go. You see, I'm being frank with you. Murder is an interesting subject—it grows and grows. Soon I shall kill you, then English. It might stop there, but there's Leon to think about. He knows too much. I shall probably have to silence him. Then someone else will have to be silenced. One murder starts a chain of others. Interesting, isn't it?"

Lois didn't say anything. She stared at Sherman, horror in her eyes.

"English worries me," Sherman went on, half to himself. "He's dangerous. He's like a bull—he'll charge against any odds, and he might make things different for me unless he's arrested very soon."

"He will make things difficult for you," Lois said. "But don't think he'll care what happens to me. He won't. He's ruthless like that. I mean nothing to him, so don't imagine you can use me to trap him, because it won't work. He'll come after you in bis own way and in his own time. You can be sure of that!"

Sherman laughed.

"You don't believe that," he said, and got to his feet. "Whatever else he is, English is the chivalrous type. You and he have worked together for some time. Even if you don't mean anything to him, he'll come charging along like a mad bull when he hears you are in danger. That type always does. But it may not be necessary. I'll wait until tomorrow morning, then if the cops haven't picked him up, I'll set my trap. He'll walk into it"

He opened the door, and motioned Penn back into the room.

"Watch her," he said, curtly. "I'll come on board again by ten o'clock tomorrow."

Penn smiled.

"She'll be right here when you get back," he said.

"She'd better be," Sherman returned, and went away along the narrow corridor to the companion hatch.

Penn lolled against the doorway, his face smirking. He stood there for several minutes, not moving, his head cocked on one side. Then they both heard the roar of a motorboat engine as it started up. Still Perm remained leaning against the doorway. Lois watched him, her cold hands clenched in her lap.

They stared at each other until the sound of the motor died away, then Penn came into the cabin and closed the door. He turned the key, took it from the lock and put it in his pocket.

\vspace{2\nbs}
\ChapterDeco[c1]{\decoglyph{e9665}}
\clearpage
\thispagestyle{empty}

\begin{ChapterStart}
\vspace{3\nbs}
\ChapterSubtitle[l]{Chapter ch7}
\ChapterTitle[l]{ch7}
\end{ChapterStart}
\FirstLine{\noindent Ed Leon drove slowly past Lois' walk-up, his eyes alert for the first sign of trouble, but there was no police car outside the building nor did a light show. He pulled up at the corner of the street and walked back to look up at the window.}
    
Had English given Morilli the slip? Sam Crail should know, he decided, and he returned to the car.

If English had been arrested, then it was up to him to find Lois, Leon told himself as he slid under the driving wheel. But where to look for her? Sherman wouldn't take her to his apartment. He probably had some other place where he could duck her out of sight—but where?

In the next street, Leon spotted an all-night drugstore. He swung the car to the curb and went in, crossing to a pay booth. He shut himself in, and dialed Crail's number. He dropped the receiver back on to the cradle when he heard the busy signal, and fumbled for a cigarette. Then he remembered Gloria Windsor. Maybe she knew if Sherman had a hide-out He dialed Crail's number again.

"Hello!" Crail's voice snapped in his ear. "That you, Leon?"

"Yeah—so they got Nick?"

"He phoned a couple of minutes ago. The cops were at the door while he was talking to me. I'm on my way to headquarters now. Goddam it! He should have given himself up as I said."

"Don't blow your stack," Leon said, shortly. "Lois is missing; looks like Sherman's got her. Corinne English has been murdered."

"What are you talking about?" Crail demanded, his voice shooting up.

"Lois went over to Corinne's place. Nick reckoned that Corinne and Sherman were working together. Lois was going to bring her back so Nick could talk to her. Lois didn't come back, so I went to see what had happened. I found Corinne strangled, and Lois missing. I found her handkerchief. I've got to find her, Crail. Tell Nick I'm going to put pressure on this Windsor girl. She's our only chance. Tell him not to worry. I'll find Lois if it kills me."

"Who's the Windsor girl?" Crail asked, blankly.

"Never mind. Tell him. He knows who she is. I've got to get moving."

"Keep in touch with me," Crail said, urgently.

"Sure. I'll call you back after I've talked to this girl," Leon said, and hung up.

He left the pay booth, and went back to his car. Ten minutes' fast driving brought him to Seventh Street, and he pulled up outside the building that housed the Alert Agency, got into the elevator and slammed the grill.

The elevator creaked upwards. It finally came to rest on the top floor, and Leon stepped out into the passage. The clatter of the teletypers from the news agency covered the sound of the grill opening. There was a light shining through the transom above Gloria Windsor's door.

He walked along the passage, lifted the brass knocker and rapped twice.

After a delay, a bolt shot back.

A tall, redheaded girl in a green high neck sweater and a pair of fawn-colored slacks looked at him. She was around twenty-eight or \-nine. Her face had an alert beauty, marred by a hard mouth and an overaggressive chin. Leon thought she had the most provocative shape he had ever seen on a woman.

"Miss Windsor?" he asked, tipping his hat.

Gray eyes looked into his. Scarlet lips twisted into half a smile.

"Sure. What do you want?"

"I'm Ed Leon," Leon told her. "I'm a detective. I want to talk to you."

She continued to smile, but her eyes grew suddenly wary.

"Don't kid me," she said, scornfully. "If you're a flatfoot, then I'm Sophie Tucker."

Leon flashed his badge.

"Does that convince you?"

"Oh, a shamus," she said, with withering contempt. "Run along, boy scout, I can't be bothered with amateurs."

She began to close the door, but Leon's foot was in the way. He moved forward.

"I said I wanted to talk to you," he told her.

She gave ground, her gray eyes angry.

"You're going to walk into a load of grief, shamus," she said, "if you try to force yourself on me."

"It's a risk I'll gladly take," Leon said, inside the room now. He closed the door and leaned against it. "It's not often I have the opportunity of forcing myself upon a redhead as well stacked as you."

A hint of a smile came into the gray eyes.

"A smooth guy!" she said, in mock despair. "I meet them twenty-four hours a day. Well, now you're in, say your piece and beat it."

"We're not in yet," Leon said, and stepped past her. He pushed open a door and walked into a large, airy room. "Well, you know how to make yourself comfortable," he went on, looking around. "My, my! You must be doing pretty well with your silhouette."

"Put that in the plural, or I'll take a poke at your left eye," she said, languidly, and walked over to a deep armchair and sank into it.

"Or maybe it's the blackmail racket that's paying off," Leon said, watching her.

She looked at him out of the corners of her eyes, and her mouth tightened.

"What are you talking about?" she demanded, frostily.

"You're in trouble, baby," Leon said, moving over to the fireplace and standing before the bright fire. "This is the wind-up for you. How do you like the idea of spending the next ten years in a nice, cosy jail?"

She looked up at him, her eyes jeering.

"What makes you think I'm going to jail, shamus?"

"Facts and figures—not your figure, mathematical ones," Leon said, taking out a pack of cigarettes. "Smoke?"

She shook her head.

"What facts and figures?"

Leon lit up, and flicked the match into the fire.

"Sherman's racket has blown up in his face. We've got all we want on him. While we're waiting for him to show, we're picking up the small fry like you."

She raised her eyebrows.

"Who's Sherman? What are you talking about?"

Leon smiled.

"Don't give me that stuff. You know what I'm talking about. You fingered Roy English. You're Sherman's sounding board. Everything that went on in English's office was heard by you and passed on to Sherman."

"Aw, you're crazy!" she exclaimed, angrily. "Get out of here before I call the cops."

"Go ahead and call them. It'll save me the trouble of dragging you down to headquarters."

She got out of the chair and walked over to the telephone.

"The cops in this town know how to deal with a louse like you," she said. "Take a tip and beat it!"

"Go ahead and call them," he said, leaning against the wall. "I've got enough on you to put you away for ten years. Blackmail comes high these days."

"You can't prove a thing," she said.

"I can tie you in with Sherman. Within the last few days, he's knocked off six people—Roy English, Mary Savitt, Joe Hennessey, Julie Clair, Harry Vince, and, an hour ago, Corinne English," Leon said, watching her. "You're tied in to Roy's killing. I can prove that. If you're not careful, they'll put this nice body of yours in the chair."

She half turned as she lifted the receiver, then she slammed it down, jerked open a drawer and whipped out a .25 automatic. She spun around and pointed the gun at Leon.

"Don't move, shamus," she said, her face hard and her eyes glittering. "I'm tempted to put a slug in you, and tell the cops you broke in here."

"What—with that toy? It wouldn't even make me bleed," Leon said, not feeling as confident as he sounded.

"You make a move, and we'll see if it would make you bleed!"

"Where's this going to get you?" he asked. "Why don't you use your head and do the sensible thing?"

"And what's that?" she demanded, resting her hips against the table, the gun sight centered on his chest.

"I want Sherman," he said. "I could afford to let you go. He's ducked out of sight"

She studied him.

"Suppose I talk, what then, shamus?"

"I'd give you twelve hours to pull out of town. After twelve hours, I'd have to tell the cops you were working with Sherman, but a girl with your assets can get a long way in twelve hours."

"I don't know anything about Sherman," she laughed. "Why, you're crazy! I never heard of the guy until you walked in here. Now, get out!"

"Okay, if that's the way you want to play it, don't blame me if you land in the chair."

"Get out!"

"A one track mind," Leon remarked, and moved over to the door. "I forgot to mention that there'd be a getaway stake thrown in with my offer of a twelve hour start. I wouldn't expect a girl like you to take a powder without a little folding money to keep her warm."

When he saw her stiffen, he knew that he'd struck the right note.

"Keep going," she said, but she didn't sound quite so convincing this time.

As he reached the door, she said, "How much?"

"A couple of grand. That's not a bad proposition, sister— two grand and twelve hours start."

"Not interested," she said, curtly. "That's chicken feed. Get out of here!"

"Suppose you make a suggestion."

She hesitated.

"Ten."

Leon laughed.

"That's funny. Ten grand for something the cops could beat out of you. But I'll go to five because redheads soothe my ulcer."

"Seven," she said promptly.

Leon realized he was wasting time.

"Do you know where Sherman is?" he asked.

She nodded.

"Well, okay. What have I got to lose? It isn't my dough. I'll close at seven. Where is he?"

"Do I look all that damp behind the ears?" she said, scornfully. "I want the dough first."

"Where is he?" Leon barked, suddenly losing his nonchalant air. "You'll get the money but you'll talk first!"

"I want the money first," she returned, obstinately.

He grabbed her by her arm.

"Listen. Sherman has kidnapped English's secretary! He's taken her somewhere. If I don't find her fast, he'll knock her off; and if he does, I'll damn well see that you're tied in with him. Where is he?"

She hesitated.

"You'll give me the money and twelve hours start, if I tell you?"

"Yes. Where is he?"

"Where's the money coming from?"

"Sam Crail, the attorney, will give it to you."

She hesitated, and then said "He's got a yacht anchored off Bay Creek. That's where he spends his weekends. If he's anywhere, that's where he'll be. You can't miss it—it's the only yacht anchored there."

"Is this on the level?" Leon demanded.

"Of course it is! Now\*, how do I collect the dough?"

Leon went over to the desk by the window, pulled a sheet of note paper from a pigeonhole and scribbled a note. He handed it to her.

"Give that to Crail. Tell him what you've told me, and he'll pay you."

"If he doesn't . . . !"

"He'll do it," said Leon, making for the door. "Maybe not tonight, but first thing in the morning. You'll still have twelve hours start. I promise you that."

When he had gone, she stood, thinking, her eyes worried, then she went swiftly into her bedroom, pulled out two suitcases from under her bed, and began to pack hurriedly.

Without bothering to change out of sweater and slacks, she pulled on a fur coat, picked up her two suitcases and went swiftly to the front door. She jerked it open, and then came to an abrupt stop, her heart skipping a beat.

Sherman was standing in the passage, his hands in his pockets, water dripping from his hat brim, his jaws moving slowly, his eyes expressionless.

"Hello, Gloria," he said, quietly.

She didn't say anything.

"Running away?" he went on, his eyes going to the two suitcases.

"What do you mean?" she managed to get out. "I'm only going away for the weekend."

"But not coming back?" he said. "Got cold feet, Gloria?"

"Why should I have cold feet?" she said, struggling to keep her voice steady. "What's the matter with you? Can't I go away for a weekend without your imagining things?"

Sherman smiled.

"Can I come in a moment?"

"I—I don't want to miss my train . . ."

He moved toward her, and she gave ground. He entered the living room. Slowly, as if hypnotized, she put the two suitcases on the floor and leaned against the wall, watching him.

"You don't have to run away, Gloria," he said, moving about the room. "I've got English where I want him. He can't cause trouble now. The cops are looking for him. He shot his girl friend."

She didn't say anything. Her eyes followed him as he moved over to the window.

"It looked at first as if he could stop me," Sherman went on, "but it's all right now. How are you fixed for dough, Gloria? I think I owe you something, don't I?"

"I'm all right," she said huskily. "I—I don't need anything at the moment."

He smiled at her.

"First time I've ever known you to say that. Perhaps you're scared of taking my money now, Gloria? You don't have to be."

"If you've got it, I'll have it," she said, "but I'm not hard up."

"No, I don't suppose you are." He had stopped by the window, and was examining the curtain cord. "Now, this is an odd coincidence. I've been looking for a cord like this for weeks. You may not believe it but I can't find this exact shade anywhere." He took the cord off the hook and appeared to examine it closely. "Do you remember where you got it?"

"From Sackville's," Gloria said, watching him uneasily.

"Are you sure?" he asked, moving casually toward her. "I think I tried there."

She looked at the cord, seeing it now hanging in a loop between his fingers, and she tried to screw herself into the wall, her eyes opening wide with terror.

"Keep away from me!" she said in a tight, strangled voice.

He was within a few feet of her, now. She suddenly threw herself blindly across the room to the door. He went after her with quick, silent steps, and as she reached the door, he dropped the loop over her head.

Her frantic scream of terror was throttled back into her throat as he crossed his hands and tightened the cord.

As Sam Crail got out of his car, a man came out of the darkness.

"Sam?"

"Why, Nick!" Crail looked uneasily to right and left, scared anyone might be watching. "What the hell are you doing here?"

"Let's get inside," English said, his voice tense.

Crail snapped off the car's headlights, and then led the way up the dark path to his house. He opened the door, and English followed him into the lobby.

Helen Crail came out of the sitting room. She was a tall, willowy girl with light brown hair and shrewd friendly eyes.

"Come in by the fire, Nick," she said, smiling at him. "I'll get you a drink."

"No, please don't, Helen," English said. "I'm all right."

Helen looked swiftly at Crail, who shook his head.

"Heard from Ed yet?" English asked.

"I've heard from him," Crail returned, following English into the big, brightly lit sitting room. He took off his coat and dropped into a chair. "Take your coat off. You're sopping wet." As English took off his coat, Crail went on; "What happened to you? I went down to headquarters and waited. Captain Swinney hadn't any information. He said there was a call out for you. I didn't tell him you'd been found. Did you give him the slip?"

English smiled grimly.

"Eventually. Morilli staged a private arrest for his own benefit. What's happened to Lois?"

"I don't know. Ed's looking for her. He said he was calling me back in an hour. He should come through at any minute now."

Helen took English's coat and hung it in the hall.

"Did he say what he found when he arrived at Corinne's place?" English asked.

Crail nodded.

"Yes. Sherman had been there. He strangled Corinne. Lois had been there, too. Ed found her handkerchief, but we don't know if Sherman has her or not."

English clenched his fists, his pale face hardening.

"He's got to be stopped, Sam! This can't go on. I've got to find him."

"Now, look, you're in a bad spot yourself," Crail said anxiously. "You should have given yourself up when Morilli came for you. Running away from him . . ."

"I didn't run away from him. I let him arrest me," English said as Helen came back into the sitting room. "He took me for a ride. If Chuck hadn't spotted us leaving and got himself a ride on the rear bumper, I'd be in the morgue by now."

Crail stared at him.

"You aren't serious?"

"You bet I'm serious. Morilli made no bones about it. He was scared I'd talk. He was about to shoot me when Chuck showed up. And that's the deal I'd get if I gave myself up. They'll frame me into the chair if I give them half a chance."

Crail wiped his face with his handkerchief.

"I'll go to the Commissioner right now and tell him," he said. "He'll have to listen to me. Where did you say you've left Morilli?"

"Hampton Wharf," English told him. "Chuck is with him. Take a newspaper man with you, Sam. It's a good idea. Maybe Morilli will give himself away."

"Leave it to me," Crail said, putting on his coat again. "In the meantime, you stay here, Nick, and keep out of sight. I'll fix that rat, Morilli!"

"You're harboring a suspect," English pointed out. "Maybe I'd better move on, Sam."

"You stay here! See that he does, Helen," Crail said. "They won't think to look here for you. I'll be back as soon as I can."

When he had gone out to the garage, Helen said, "You're worrying about Lois, aren't you, Nick?"

He nodded.

"If that devil's killed her . . ."

"You mustn't think like that," she said, soothingly. "Sit down and rest. Ed will find her."

"But the police are looking for him, now, and he doesn't know it. Morilli put out a call for him."

'Trust him to keep out of trouble," she returned.

English flopped down in an armchair.

"If I only knew where Sherman is," he said, angrily. "I can't go out looking all over the town. I'd be picked up within minutes."

"Ed said he was going to talk to some girl—Windsor, I think he said her name was. He thought she might know where Sherman is."

English's face brightened.

"I'd forgotten her. Ed thinks she's working with Sherman. I wonder if he got anywhere with her."

"He'll call in a little while," Helen said.

"He may be with her now," English said, jumping to his feet. "I might get him on the phone."

He went over to the telephone and riffled through the pages of the directory until he found Gloria Windsor's number. He dialed and waited. The line was busy.

"No answer. Maybe she's out, and he hasn't talked to her yet." He looked at his watch. "When I think of Lois . . ." He drove his fist into his palm. "Goddam it! I must do something! I can't just sit and wait!"

"Take it easy, Nick," Helen said. "You've got to rely on Ed."

"It's all very well . . ." He broke off and smiled crookedly at her. "You know, I've been a mug about Lois, Helen. I didn't realize what she meant to me until I'd lost her."

"Aren't we all mugs sometimes?" she returned, gently. "I'm glad, Nick. She's been good to you."

"I know. Well, if she's alive, I'll make up for it."

"Listen!" Helen said, sharply, holding up her hand.

They heard the sound of a fast-moving car, coming down the street. A moment later, it pulled up outside the house with a squeal of tortured tires.

As English moved to the window, Helen pushed him aside.

"You must keep out of sight, Nick. It may be the police," she said, sharply. "Let me see."

She lifted the shade, then turned swiftly, her face alight with excitement

"It's Ed!" she exclaimed, and ran across the room to the front door.

Leon was soaked with rain, and there was an anxious, harassed look in his eyes.

"Sam in?" he asked.

"Come in," Helen said. "Nick's here."

"What a break," Leon said to English. "I'd given you up as lost."

"Where's Lois?" Nick demanded.

"I'm not sure yet. I came here for some money. I've got to hire a boat Sherman has a yacht six miles off Bay Creek. It's my bet Lois is on board. Have you got a hundred bucks?"

"Of course, I have," English said. "I'm coming with you."

"Better not. The cops are still looking for you."

"They're looking for you, too," English said. "Morula's put a call out for you. He's going to pin Corinne's murder on you. Come on, let's get going!"

He struggled into his overcoat

"How far is Bay Creek?" he asked.

"About three miles from here," Leon said, opening the front door.

'Tell Sam where I've gone," English said to Helen. "And thanks for putting up with me."

"Good luck, Nick," Helen said, her eyes anxious. "And be careful."

Leon sent the car shooting down the deserted street

"I got the Windsor girl to talk," he told English, "but it's going to cost you seven grand, and it may come to nothing. All the same, I imagine Sherman would take Lois to the yacht if he took her anywhere."

"Where's this Bay Creek, Ed?" English asked.

"You know the golf club? A mile further on is Bay Creek. There's a boathouse there. I've seen the yacht. It's anchored about six miles out in the bay. Someone's on board. Lights are showing, but the guy who owns the motorboat wouldn't play unless I paid him the hundred. I nearly went crazy trying to talk him into it, but the louse wouldn't budge."

English glanced over his shoulder.

"There's a car after us, Ed!" he said, his voice sharpening.

Leon promptly shoved his foot down hard on the gas pedal.

"Cops?"

"Could be. Maybe they spotted your number. I told you they were on the lookout for you."

"I can't hope to shake a prowl car in this old heap," Leon said. "What are we going to do?"

"Can't we lose them?"

"Not in this district." He looked in the driving mirror. "Hell! They're coming up fast!"

"You stall them, Ed. I'm going after Lois. Get around the next corner, slow down and let me drop off. I'll take my chance of giving them the slip."

"They're right behind us," Leon said. He shoved the gas pedal to the boards, and the car fell back a little. "Hang on tight. I'm going to take the next corner."

Twenty yards from the corner, Leon slammed on his brakes. The back of the car swung around in a violent skid. He heard the screaming tires as the other car braked frantically. Beams from the other car's headlamps lit up Leon's car as he wrestled with the wheel, steering into the skid. He released the brake, and trod on the gas pedal. The car shot into the side street. The pursuing car went on, braked violently as Leon slowed down.

"Good luck!" he exclaimed, as English opened the door.

English jumped out, took two staggering steps forward before falling heavily. He rolled over, staggered to his feet, and ran blindly for an alley facing him.

The police car had reversed and was swinging into the street as he reached the mouth of the alley. A voice yelled at him, but he didn't look around.

There was a flash and a crash of gunfire. Something zipped perilously close to his head, then he dashed into the darkness of the alley.

For some seconds he ran blindly. The alley led to the river, and he came out on the waterfront He heard the sound of pounding feet coming after him, and he looked to right and left for cover. A few yards from him, was a vast pile of empty wooden crates. He darted over to them and dodged behind them.

A moment later, a cop came out of the alley, gun in hand. He looked up and down the deserted waterfront, then stood listening. He came on and began to circle the pile of crates. Moving without a sound, English followed him, keeping just out of sight, until the two of them had made a complete circle of the crates.

With a grunt of disgust, the cop went off along the waterfront, using a powerful flashlight, his gun thrust forward.

English didn't move until the cop was out of sight, then he went off in the opposite direction, walking fast, his head bent against the driving rain.

He was about a mile from the golf club, and time was running out. He decided to risk a taxi. He made his way back toward town. As he walked along in the pouring rain, he wondered what had happened to Leon, and he wished he had a gun.

After walking for some minutes, he saw a taxi coming toward him, and he waved.

The taxi pulled up.

"Know the golf club?" he asked, keeping his head bent so the driver couldn't see his face clearly.

"Sure," the driver returned.

"A mile further on, there's a boathouse. That's where I want to go."

"I know it. Tom Kerr's place."

English got into the cab.

"Twenty bucks if you get me there in ten minutes."

"Can't be done, but I'll get you there in fifteen."

"Get going!"

Twelve minutes later, the driver said, "That's Kerr's joint right ahead."

English leaned forward to peer through the rain soaked windshield. He could see a big wooden shed by the river bank. Lights came through the windows.

He fumbled in his billfold and took out a twenty dollar bill.

"Want to wait?" he said. "I'll be coming back, but I may be some time. It rates another twenty."

"I'll wait all night for that kind of dough," the driver said, eagerly.

He swung down a steep slope that led directly to the shed, and pulled up.

"You'll find Kerr in that cabin down by the jetty," he told English, as he took the twenty dollar bill.

English walked quietly down the path to the cabin, and rapped on the door. A fat man in a turtle-neck sweater and thick rubber boots looked at him enquiringly.

"You Tom Kerr?" English asked.

"That's right, mister. Come in."

English stepped into a warm, pleasant room.

"I want a motorboat in a hurry," he said to Kerr. "How soon can you get one ready?"

Kerr looked sharply at him.

"What's the trouble, Mr. English?" he asked.

English smiled crookedly.

"I wish my face wasn't so familiar," he said. "I want to get to a yacht moored in Bay Creek."

"I'll take you there," Kerr said.

English wiped the rain off his face.

"Do you know the police are looking for me?" he asked. "I don't want to get you into trouble."

"I mind my own business. I'm glad to do something for you."

English nodded.

"I've more friends than I thought," he said.

They went out into the rain to a powerful speedboat that bobbed up and down on the heavy swell. Kerr helped him aboard, cast off, pushed forward the throttle and sent the boat shooting toward the bay.

"We didn't talk terms," English said, standing close to Kerr. "Would a hundred settle it?"

Kerr nodded.

"Anything you say, Mr. English."

"There may be some trouble on the yacht," English went on. "A girl I know has been kidnapped, and I think she's on board. Ill tackle it. You stay with the boat. I'll want you to take us back if she's there."

"If there's going to be any rough stuff, count me in," Kerr said, his face lighting up. "I used to be the Midwestern heavyweight champion before I married, and I haven't had any action in years."

"I guess I can use you if there are more than one of them."

They could see the lights of the yacht in the distance.

"Push her along," English said, impatiently.

Kerr pulled the throttle. The speedboat raced over the swells, throwing a foaming wash behind as it cleaved through the water.

English peered through the blinding spray, his eyes on the yacht. Out of the shelter of the bay, the wind whistled and the sea thundered. English thought it was unlikely that anyone on board would hear the approaching speedboat.

"Slow down," he said to Kerr, "and drift up to her. I don't want them to know we're coming."

The boat moving on its own impetus, ran on towards the yacht, and in a few minutes, Kerr brought it alongside.

English caught hold of the glittering brass rail, and steadied the boat while Kerr made it fast.

Then they both swung aboard.

"I'll go first," English said, under his breath. "You keep out of sight. If there's trouble, take them in the rear."

He moved softly to the companion hatch and paused to listen at the head of the stairway. Hearing nothing, he cautiously began to descend, and as he reached the bottom step, a cabin door toward the end of the passage, abruptly opened.

He crouched down, waiting, knowing he couldn't get along the passage before he was seen; nor had he time to get up the stairway and out of sight. If whoever it was coming out of the cabin, had a gun, he would be shot down before he could make a move.

Then he saw Lois.

She came out of the cabin, her face white, her eyes scared. Her white nylon blouse was torn off her shoulder, and her skirt was ripped at the waist.

"Lois!" English said, softly.

"Oh, Nick!" she said, and ran towards him.



III

Halfway down the stairway, Kerr stopped and gaped. He was expecting to run into real trouble, and the sight of English holding a girl in his arms, stopped him short.

But English was oblivious to Kerr's astonishment. He held Lois close to him, thankful to find her alive.

"Are you all right?" he asked, anxiously. "You're not hurt?"

"I'm all right. I thought it was Sherman coming back. Oh, Fm so glad to see you, Lois said, pushing away from him, embarrassed.

"My dear . . ." English began, then realized this was no time for idle talk. "Is there anyone else on board?" .

Lois shivered.

"There's Perm. He's in there." She pointed to another cabin. "I've been scared to go in there again. I hit him."

"You hit him?" Enghsh said blankly. "What happened?"

"He attacked me. I got away from him, and hit him over the head with a bottle. I—I think I may have killed him."

He could see she was struggling not to cry, and he put his arm around her.

"It's all right," he said. "I'm going to get you out of here." He looked over his shoulder at Kerr. "Take a look in there and see what's happened."

Kerr pushed past them, opened the cabin door, and went in. He came out after a minute of so, grinning.

"Well, you certainly did hit him, miss," he said, admiringly, but he's all right. He'll probably have a cracked skull, but he's not going to croak."

Lois leaned against English.

"I was so frightened he would die," she said, "but he was a brute."

"Come on," English said, "you're going home."

"No, wait," she said, catching hold of his arm. "This is important, Nick. There's something in the next cabin we must take with us."

"All right. Just a moment." English turned to Kerr. "Think you can get that thug into the boat? I want him."

"Sure," Kerr said. "Leave him to me."

English followed Lois into the cabin.

"I found this, Nick," she said, pointing to a square leather suitcase. "It's a tape recorder. I think Perm intended to blackmail Sherman. The tape contains all kinds of conversations between Sherman and Perm, and something that clears you. Sherman talked to me. Penn must have set the machine going. Listen to this."

She opened the case, and flicked down the switch. The two reels began to revolve.

*"Murder is an odd thing," Sherman's voice said clearly out of the machine. "Its like a snowball rolling down a hill. One murder leads to another. I wouldn't be in this jam if that cheap little chiseler hadn't tried to gyp me. I was a fool to have picked on him to work for me. Before he came, I had a good business. Now, if I'm not very careful, the bottom could drop out of it. Ifs worth a quarter of a million a year to me, and I'm not giving that up without a fight. . . .*"

They stood side by side, listening to the flat, metallic voice, and when it said, "*I arranged he should hear about his mistress and Harry Vince. I couldn't be sure he would kill them, so I did it for him. . . .*" English put his arm around Lois, and hugged her.

"That's it! That let's me out!" he said. "Now we've got him where we want him!"

"Let's go now, Nick," Lois said, switching off the machine. "I can't wait until we've given this to the police."

English was looking past her, a sudden, puzzled expression in his eyes.

"I don't remember shutting the door, do you, Lois?" he said, and walked over to the door and turned the handle. He pulled, shook the door, and then stepped back.

"That's odd. It's locked."

"Oh, Nick!" Lois said, her eyes frightened. "You don't think he's herer

"Of course not," English said, and rattled the door handle. "Hey, Kerr! Open the door. We're locked in!"

"Nick, put your hand on the wall." He could feel a faint vibration, and he nodded.

"You're right. Maybe Kerr's decided to take the yacht in."

"It isn't Kerr. It's Sherman," Lois said. "I know it is."

English went swiftly to the porthole, and looked out. He was in time to see the speedboat drifting away into the darkness. Even as he caught a glimpse of it, it vanished from sight, wallowing in the heavy swell.

"He's cut the boat adrift," he said, turning to face her. "You're right. Sherman is on board."

He went over to the door and rattled the handle again.

The vibration was stronger now, as if the engines were mounting to full speed; and when Lois looked through the porthole, she could see the water foaming against the snip's side as it forged ahead.

"He's heading out to sea. What are we going to do, Nick?"

English was examining the door.

"The damn thing opens inwards. There's not much hope of smashing the lock. This table! I could use it as a battering ram."

"That's an idea. Let's try."

Together they wrenched the table from its fastenings, and carried it over to the door.

English picked up the table and slammed it against the door, drew back, and slammed it again. One of the door panels split.

"Once again," English said. "I think it's going to work." He drew back, and then ran at the door. The corner of the table smashed through the panel, making a gaping hole. He kicked out the rest of the panel, leaned through the opening and found the key in the lock. He pushed open the door.

"Now, look, Lois, you stay here," he said, "or better still, go into the next cabin and lock yourself in. Take the recorder with you. Whatever happens, we're not going to lose that. I'm going to see what's happening."

"No, don't, Nick. Don't leave me. If anything happened to you . . ."

"I'll be careful. Now get into the other cabin and wait for me." He picked up the recorder and pushed her into the passage. "I'll be all right."

Before she could argue further, he handed her the recorder, and then went along the passage to the companionway.

He went to the stairway slowly, his ears cocked for the slightest sound, but all he could hear was the noise of the engines and the heavy thud of the sea against the yacht as she drove through the water.

When he was almost at the top of the companionway, he stopped. He listened, then went on, and very cautiously looked along the dark deck. He saw nothing. The deck was deserted, and he looked toward the bridge, but that too was deserted.

He guessed Sherman must have lashed the wheel, and was hiding somewhere, waiting for him to show himself.

Then he saw a movement in the shadows ahead of him, and he quickly ducked down so that he was no longer outlined against the white hatchway.

"Hello, English," Sherman said, from out of the shadows. "I can see you and I'm covering you with a gun."

English looked in the direction of the voice. He decided Sherman was too far away for a quick rush. He moved down a step so Sherman couldn't pick him off, and waited.

"I thought you'd walk into my trap, sooner or later," Sherman went on. "She wouldn't believe you'd come after her. I told her you would. I said that you had the mentality of a Grade B movie star."

"Where do you think you're going?" English asked. "Every Coast Guard boat on the coast is on the lookout for you."

"That, of course, is a stupid lie," Sherman returned. "In a few hours, when Kerr recovers from the clout on the head I gave him, they might look for us but, by that time, it will be too late."

"Don't be too sure," English said. "You don't imagine you can get away in this yacht, do you?"

Sherman laughed.

"No, but it'll be at the bottom of the sea by the time they come after us," he said, and came out of the shadows. He held an automatic in his hand, and it covered the companionway. 'That's where we're going, English—you and the girl and II To the bottom of the sea."

"Is that necessary?" English asked. "Surely, you don't want to join us."

"I'm going to end it," Sherman said. "I'm sick of killing people. I shouldn't have killed Gloria. The janitor saw me leave. Of course, I would have killed him, but I can't go on and on killing people. I'm sick of it! There seems no end to it. Well, I'm going to end it, and end you, too."

"And how do you propose to end it?" English asked, seeking information. He knew it was hopeless to attempt to close with Sherman. The distance between them was too great.

"I've set fire to the yacht," Sherman said. "There should be a pretty good blaze before long. You'll have the opportunity of either burning or drowning. We're about twelve miles off shore now, and we're still going. Personally, I prefer to drown."

English had heard all he wanted to know now. He slid down the stairs and landed heavily in the passage.

Lois had come along the passage and had heard what had been said. She looked at English, her face pale but her eyes unafraid.

"He's cracked," English said. "He says he's set fire to the yacht. Maybe he's lying, but if he isn't, we may have to swim for it. Can you swim, Lois?"

She smiled.

"Yes. You don't have to worry about me."

"But I do worry about you." He put his hand on her arm and looked down at her. "This is the wrong time and place, my dear, but I'd better tell you now. I'm in love with you. I guess I've been in love with you for years. It was only when I thought I was going to lose you, I realized it. Sorry, Lois, but there it is. Better late than never, I. suppose. Having got that off my chest, let's get busy. There must be some life belts somewhere down here."

She gave him a quick, searching look before going into the cabin. They found three life belts and two oilskins.

"We'll wrap the recorder in the oilskins, and then put a life belt around it," English said. "I'm not losing it unless I have to."

"I can smell smoke," Lois said.

English stepped into the passage. Smoke was drifting up through the floor boards. He returned to the cabin to help Lois tie the life belt around the recorder.

"We can't get off the boat without going up on deck," he said, "and he's guarding the head of the stairs. You wait here."

"Be careful, Nick."

He put his fingers under her chin, and kissed her.

"You bet. But we've got to get out of here."

A sudden gust of smoke swirled into the cabin, making them cough; and when he went into the narrow passage, he found it full of smoke and the heat intense.

"Come on, Lois, you can't stay here."

She joined him, and they ran along the passage to the companionway.

English hadn't yet put on his life belt. He didn't want Sherman to know they had life belts, and he put his belt on the stairs before he looked along the deck.

Cautiously, he went up the stairs and onto the deck. Still, he could see no sign of Sherman.

"Lois!" he called softly.

She joined him, and he motioned her to keep down.

"I can't see him. Let's get out of here. Give me the recorder."

"Your life belt," she said, thrusting the belt into his hands.

As he started to take it, he saw Sherman coming through the smoke. He dropped the belt, grabbed Lois by the arm and rushed her across the deck.

"In you go," he said, and lifting her, dropped her into the sea.

He ran back for the recorder and as he snatched it up, Sherman saw him.

"Don't move!" he shouted.

English dodged to the right, reached the rail and tossed the recorder into the sea. As he put his hand on the rail to vault over. Sherman shot him.

English felt something hit him violently in his side, sending a scorching pain through his body. He fell face down on the hot deck.

The deck was so hot, his soaking clothes sizzled; and as he tried to push himself to his-feet, his hands began to blister. He rolled over, frantically trying to get under the rail and into the sea.

Sherman ran over to him, caught hold of one of his ankles and dragged him back.

"You're not going to get away!" he cried, wildly. "You'll roast here with me. How do you like it, English? How do you like your first taste of hell?"

English kicked out. The heel of his shoes crashed against Sherman's knee cap, bringing him down. Sherman's gun went off, and a slug ploughed a furrow in the deck near English's head.

English rolled on Sherman, pinning him flat on the deck. Snarling with pain and fury, Sherman tried to get his gun hand up, but English caught his wrist in both hands and pressed Sherman's hand down on the metal guard that ran the length of the yacht.

Sherman screamed as the hot metal burned into his flesh. Exerting all his great strength, English kept Sherman's hand down against the metal.

Sherman slammed his free fist into English's face, but English held on until Sherman's fingers opened in agony, and the gun dropped into the sea.

He let go of Sherman's wrist, tried to get to his feet, but the pain in his side was now so intense he blacked out for a moment.

He came out of the faint, the hot deck scorching his back. Sherman was kneeling on him, his fingers digging into his throat. English caught hold of Sherman's thumbs and wrenched them back, breaking Sherman's hold. As Sherman groped for his throat again, English smashed his fist into Sherman's face, sending him sprawling on his back.

English grabbed hold of the rail and dragged himself to his feet. Before Sherman could reach him, English overbalanced and fell headfirst into the sea.

The shock of the cold water revived him, and when he broke surface, he shook the water out of his eyes and turned on his back.

The yacht was now blazing like a torch, lighting up the sea. English kicked out to send himself away from the yacht and the intense heat.

"Nick!"

A hand closed over his shoulder. He turned his head. Lois was beside him, her other hand holding on to the recorder.

"Oh, darling, are you hurt?"

"It's all right," English gasped. "It's nothing much. What happened to him?"

"I think he's still on the yacht."

English reached out and put his hand over the recorder. With its help, he kept his head above water. His legs hung like leaden weights; and if it hadn't been for the buoyancy of the recorder, he would have sunk.

"Keep near me, Lois," he said. "I'm bleeding a little."

"Get on your back," she urged. "I can hold you. Keep a grip on that case."

As he turned on his back, he saw Sherman, swimming strongly towards them. Sherman's eyes were gleaming, and his teeth showed in a vicious snarl.

"Look out!" English-panted, and pushed Lois away from him.

Sherman's hand caught hold of English's shoulder.

"We'll go down together!" he cried, shrilly. "This is the pay-off, English!"

English struck out at him, but his strength was failing. He felt Sherman's fingers shift from his shoulder to his throat.

They went down together, Sherman locking his legs around English's body, his fingers digging into English's throat.

Lois saw them go down, and she dived after them, but the buoyancy of her life belt immediately returned her to the surface.

Frantically, she wrestled with the strings to get it off, but the knots had hardened in the water.

"Nick!" she screamed, and again tried to go down, but again the buoyancy of the belt brought her to the surface.

Then suddenly there was a commotion under the water. She caught a glimpse of the two men, still locked together, as they came to the surface. She saw English's hand grope for Sherman's face, and his thumbs sink into Sherman's eyes as they went down again, the water closing over them.

She waited, her heart pounding, sick with fear for English, watching the bubbles of air as the two men fought under the water. They broke water a second time. Sherman seemed no longer to be struggling. His arms and legs were locked around English's body while English was fighting desperately to throw him off.

She swam towards them, trying to reach them before they sank again, but she was too late. They went down again as English was within a few inches of her hand.

Then, after a pause, a body came to the surface, rolled over and floated half submerged, near her. She reached it, turned it and saw with a sob of relief, English's white unconscious face.

She held him up, pushing him toward the floating recorder and propping him over it.

She was still holding him above the water when Kerr found them, fifteen minutes later, as he brought the speedboat towards the flaming wreck.

THE END

\vspace{2\nbs}
\ChapterDeco[c1]{\decoglyph{e9665}}
\clearpage
\thispagestyle{empty}


\scenebreak
\scenebreak
{\centering\textsc{the end}\par}

\clearpage

\null

\centering\textsc{www.TalesofMurder.com}\par

\vspace*{10\nbs}

%\centering\InlineImage[0, 3em]{/home/darkstar/dox/working-files/LaTeX/atticus.jpg}

TALES OF MURDER PRESS, LLC

\null

\scshape{675 TOWN CENTER BLVD
BLDG 1A STE 200 PMB 530
GARLAND, TEXAS 75040}

\null

\textit{atticus@talesofmurder.com}
\vfill


\end{document}
