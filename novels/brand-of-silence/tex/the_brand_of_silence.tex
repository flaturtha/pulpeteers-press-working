% !TeX TS-program = LuaLaTeX
% !TeX encoding = UTF-8
\documentclass{novel}
%%% METADATA (FILE DATA):
\SetTitle{The Brand of Silence}
\SetAuthor{Harrington Strong}
\SetPDFX{X-1a:2001}
\SetTrimSize{4.25in}{6.875in}
\SetMediaSize{4.5in}{7.12in}
\SetMargins{0.5in}{0.5in}{0.5in}{0.7in}
\SetParentFont{Libertinus Serif}
\SetFontSize{9.5pt}
\SetHeadFootStyle{5}
\SetHeadJump{1.5}
\SetFootJump{1.5}
\SetLooseHead{50}
\SetEmblems{}{} % Default blanks.
\SetHeadFont[\parentfontfeatures,Letters=SmallCaps,Scale=0.92]{\parentfontname}
\SetPageNumberStyle{\thepage}
\SetVersoHeadText{\theAuthor}
\SetRectoHeadText{\theTitle}
%%% CHAPTERS:
\SetChapterStartStyle{footer} % Equivalent to empty, when style has no footer.
\SetChapterStartHeight{10}
\SetChapterFont[Numbers=Lining,Scale=1.6]{\parentfontname}
\SetSubchFont[Numbers=Lining,Scale=1.2]{\parentfontname}
\SetScenebreakIndent{false}
%%% BEGIN DOCUMENT:
\begin{document}
\frontmatter
\thispagestyle{empty}
% Half-Title Page.
\begin{parascale}[2]
\vspace*{3\nbs}
\centering\charscale[0.75]{The Brand of Silence }\par
\centering\charscale[0.75]{The Brand of Silence LINE 2}\par
\centering{The Brand of Silence LINE 3}\par
\end{parascale}
\clearpage
\thispagestyle{empty}
\null % Necessary for blank page.
% Alternatively, List of Books.
\clearpage
\thispagestyle{empty}
% Title Page.
\begin{parascale}[4]
\centering\charscale[0.75]{The Brand of Silence }\par
\centering\charscale[0.75]{The Brand of Silence LINE 2}\par
\centering{The Brand of Silence LINE 3}\par
\end{parascale}
\vspace*{2\nbs}

\begin{parascale}[1]
\centering\textit{SERIES}\par
\vspace*{3\nbs}
\charscale[2]{Harrington Strong}\par
\end{parascale}
\vfill
\begin{parascale}[1]
% \centering\InlineImage[0, 3em]{/home/darkstar/dox/working-files/LaTeX/atticus.jpg}

A Tales of Murder Press, LLC\par
\textit{Noir} Novel\par
\end{parascale}
\clearpage
\thispagestyle{empty}
% Copyright Page.
\null\vfill
\allsmcp{First edition} Not Available by Tales of Murder Press, LLC\par
\null\null
\allsmcp{ISBN}\par
\null\null
\vfill
\begin{adjustwidth}{3em}{3em}
\textit{This novel is in the public domain.} Certain \mbox{elements} in this edition are Copyright © 2024 Tales of Murder Press, LLC
\end{adjustwidth}
\clearpage
\thispagestyle{empty}
\clearpage % because ToC must start recto
\thispagestyle{empty}
\begin{toc}[0.5]{0em}
{\centering\charscale[1.25]{Contents}\par}
\null

\tocitem*[1]{1}{1}
\tocitem*[.]{.}{.}
\tocitem*[ ]{ }{ }
\tocitem*[I]{I}{I}
\tocitem*[n]{n}{n}
\tocitem*[ ]{ }{ }
\tocitem*[T]{T}{T}
\tocitem*[h]{h}{h}
\tocitem*[e]{e}{e}
\tocitem*[ ]{ }{ }
\tocitem*[H]{H}{H}
\tocitem*[a]{a}{a}
\tocitem*[r]{r}{r}
\tocitem*[b]{b}{b}
\tocitem*[o]{o}{o}
\tocitem*[r]{r}{r}
\tocitem*[ ]{ }{ }
\tocitem*[2]{2}{2}
\tocitem*[.]{.}{.}
\tocitem*[ ]{ }{ }
\tocitem*[T]{T}{T}
\tocitem*[h]{h}{h}
\tocitem*[e]{e}{e}
\tocitem*[ ]{ }{ }
\tocitem*[G]{G}{G}
\tocitem*[i]{i}{i}
\tocitem*[r]{r}{r}
\tocitem*[l]{l}{l}
\tocitem*[ ]{ }{ }
\tocitem*[O]{O}{O}
\tocitem*[n]{n}{n}
\tocitem*[ ]{ }{ }
\tocitem*[T]{T}{T}
\tocitem*[h]{h}{h}
\tocitem*[e]{e}{e}
\tocitem*[ ]{ }{ }
\tocitem*[S]{S}{S}
\tocitem*[h]{h}{h}
\tocitem*[i]{i}{i}
\tocitem*[p]{p}{p}
\tocitem*[ ]{ }{ }
\tocitem*[3]{3}{3}
\tocitem*[.]{.}{.}
\tocitem*[ ]{ }{ }
\tocitem*[S]{S}{S}
\tocitem*[o]{o}{o}
\tocitem*[m]{m}{m}
\tocitem*[e]{e}{e}
\tocitem*[ ]{ }{ }
\tocitem*[D]{D}{D}
\tocitem*[i]{i}{i}
\tocitem*[s]{s}{s}
\tocitem*[c]{c}{c}
\tocitem*[o]{o}{o}
\tocitem*[u]{u}{u}
\tocitem*[r]{r}{r}
\tocitem*[t]{t}{t}
\tocitem*[e]{e}{e}
\tocitem*[s]{s}{s}
\tocitem*[i]{i}{i}
\tocitem*[e]{e}{e}
\tocitem*[s]{s}{s}
\tocitem*[ ]{ }{ }
\tocitem*[4]{4}{4}
\tocitem*[.]{.}{.}
\tocitem*[ ]{ }{ }
\tocitem*[A]{A}{A}
\tocitem*[ ]{ }{ }
\tocitem*[F]{F}{F}
\tocitem*[o]{o}{o}
\tocitem*[e]{e}{e}
\tocitem*[ ]{ }{ }
\tocitem*[A]{A}{A}
\tocitem*[n]{n}{n}
\tocitem*[d]{d}{d}
\tocitem*[ ]{ }{ }
\tocitem*[A]{A}{A}
\tocitem*[ ]{ }{ }
\tocitem*[F]{F}{F}
\tocitem*[r]{r}{r}
\tocitem*[i]{i}{i}
\tocitem*[e]{e}{e}
\tocitem*[n]{n}{n}
\tocitem*[d]{d}{d}
\tocitem*[ ]{ }{ }
\tocitem*[5]{5}{5}
\tocitem*[.]{.}{.}
\tocitem*[ ]{ }{ }
\tocitem*[T]{T}{T}
\tocitem*[h]{h}{h}
\tocitem*[e]{e}{e}
\tocitem*[ ]{ }{ }
\tocitem*[C]{C}{C}
\tocitem*[o]{o}{o}
\tocitem*[u]{u}{u}
\tocitem*[s]{s}{s}
\tocitem*[i]{i}{i}
\tocitem*[n]{n}{n}
\tocitem*[ ]{ }{ }
\tocitem*[6]{6}{6}
\tocitem*[.]{.}{.}
\tocitem*[ ]{ }{ }
\tocitem*[M]{M}{M}
\tocitem*[u]{u}{u}
\tocitem*[r]{r}{r}
\tocitem*[k]{k}{k}
\tocitem*[—]{—}{—}
\tocitem*[A]{A}{A}
\tocitem*[n]{n}{n}
\tocitem*[d]{d}{d}
\tocitem*[ ]{ }{ }
\tocitem*[M]{M}{M}
\tocitem*[u]{u}{u}
\tocitem*[r]{r}{r}
\tocitem*[d]{d}{d}
\tocitem*[e]{e}{e}
\tocitem*[r]{r}{r}
\tocitem*[ ]{ }{ }
\tocitem*[7]{7}{7}
\tocitem*[.]{.}{.}
\tocitem*[ ]{ }{ }
\tocitem*[E]{E}{E}
\tocitem*[v]{v}{v}
\tocitem*[i]{i}{i}
\tocitem*[d]{d}{d}
\tocitem*[e]{e}{e}
\tocitem*[n]{n}{n}
\tocitem*[c]{c}{c}
\tocitem*[e]{e}{e}
\tocitem*[ ]{ }{ }
\tocitem*[8]{8}{8}
\tocitem*[.]{.}{.}
\tocitem*[ ]{ }{ }
\tocitem*[L]{L}{L}
\tocitem*[i]{i}{i}
\tocitem*[e]{e}{e}
\tocitem*[s]{s}{s}
\tocitem*[ ]{ }{ }
\tocitem*[A]{A}{A}
\tocitem*[n]{n}{n}
\tocitem*[d]{d}{d}
\tocitem*[ ]{ }{ }
\tocitem*[L]{L}{L}
\tocitem*[i]{i}{i}
\tocitem*[a]{a}{a}
\tocitem*[r]{r}{r}
\tocitem*[s]{s}{s}
\tocitem*[ ]{ }{ }
\tocitem*[9]{9}{9}
\tocitem*[.]{.}{.}
\tocitem*[ ]{ }{ }
\tocitem*[P]{P}{P}
\tocitem*[u]{u}{u}
\tocitem*[z]{z}{z}
\tocitem*[z]{z}{z}
\tocitem*[l]{l}{l}
\tocitem*[e]{e}{e}
\tocitem*[d]{d}{d}
\tocitem*[ ]{ }{ }
\tocitem*[1]{1}{1}
\tocitem*[0]{0}{0}
\tocitem*[.]{.}{.}
\tocitem*[ ]{ }{ }
\tocitem*[O]{O}{O}
\tocitem*[n]{n}{n}
\tocitem*[ ]{ }{ }
\tocitem*[T]{T}{T}
\tocitem*[h]{h}{h}
\tocitem*[e]{e}{e}
\tocitem*[ ]{ }{ }
\tocitem*[T]{T}{T}
\tocitem*[r]{r}{r}
\tocitem*[a]{a}{a}
\tocitem*[i]{i}{i}
\tocitem*[l]{l}{l}
\tocitem*[ ]{ }{ }
\tocitem*[1]{1}{1}
\tocitem*[1]{1}{1}
\tocitem*[.]{.}{.}
\tocitem*[ ]{ }{ }
\tocitem*[C]{C}{C}
\tocitem*[o]{o}{o}
\tocitem*[n]{n}{n}
\tocitem*[c]{c}{c}
\tocitem*[e]{e}{e}
\tocitem*[r]{r}{r}
\tocitem*[n]{n}{n}
\tocitem*[i]{i}{i}
\tocitem*[n]{n}{n}
\tocitem*[g]{g}{g}
\tocitem*[ ]{ }{ }
\tocitem*[K]{K}{K}
\tocitem*[a]{a}{a}
\tocitem*[t]{t}{t}
\tocitem*[e]{e}{e}
\tocitem*[ ]{ }{ }
\tocitem*[G]{G}{G}
\tocitem*[i]{i}{i}
\tocitem*[l]{l}{l}
\tocitem*[b]{b}{b}
\tocitem*[e]{e}{e}
\tocitem*[r]{r}{r}
\tocitem*[t]{t}{t}
\tocitem*[ ]{ }{ }
\tocitem*[1]{1}{1}
\tocitem*[2]{2}{2}
\tocitem*[.]{.}{.}
\tocitem*[ ]{ }{ }
\tocitem*[B]{B}{B}
\tocitem*[a]{a}{a}
\tocitem*[t]{t}{t}
\tocitem*[t]{t}{t}
\tocitem*[e]{e}{e}
\tocitem*[r]{r}{r}
\tocitem*[e]{e}{e}
\tocitem*[d]{d}{d}
\tocitem*[ ]{ }{ }
\tocitem*[K]{K}{K}
\tocitem*[e]{e}{e}
\tocitem*[y]{y}{y}
\tocitem*[s]{s}{s}
\tocitem*[ ]{ }{ }
\tocitem*[1]{1}{1}
\tocitem*[3]{3}{3}
\tocitem*[.]{.}{.}
\tocitem*[ ]{ }{ }
\tocitem*[A]{A}{A}
\tocitem*[ ]{ }{ }
\tocitem*[P]{P}{P}
\tocitem*[l]{l}{l}
\tocitem*[a]{a}{a}
\tocitem*[n]{n}{n}
\tocitem*[ ]{ }{ }
\tocitem*[O]{O}{O}
\tocitem*[f]{f}{f}
\tocitem*[ ]{ }{ }
\tocitem*[C]{C}{C}
\tocitem*[a]{a}{a}
\tocitem*[m]{m}{m}
\tocitem*[p]{p}{p}
\tocitem*[a]{a}{a}
\tocitem*[i]{i}{i}
\tocitem*[g]{g}{g}
\tocitem*[n]{n}{n}
\tocitem*[ ]{ }{ }
\tocitem*[1]{1}{1}
\tocitem*[4]{4}{4}
\tocitem*[.]{.}{.}
\tocitem*[ ]{ }{ }
\tocitem*[M]{M}{M}
\tocitem*[o]{o}{o}
\tocitem*[r]{r}{r}
\tocitem*[e]{e}{e}
\tocitem*[ ]{ }{ }
\tocitem*[M]{M}{M}
\tocitem*[y]{y}{y}
\tocitem*[s]{s}{s}
\tocitem*[t]{t}{t}
\tocitem*[e]{e}{e}
\tocitem*[r]{r}{r}
\tocitem*[y]{y}{y}
\tocitem*[ ]{ }{ }
\tocitem*[1]{1}{1}
\tocitem*[5]{5}{5}
\tocitem*[.]{.}{.}
\tocitem*[ ]{ }{ }
\tocitem*[A]{A}{A}
\tocitem*[ ]{ }{ }
\tocitem*[M]{M}{M}
\tocitem*[o]{o}{o}
\tocitem*[m]{m}{m}
\tocitem*[e]{e}{e}
\tocitem*[n]{n}{n}
\tocitem*[t]{t}{t}
\tocitem*[ ]{ }{ }
\tocitem*[O]{O}{O}
\tocitem*[f]{f}{f}
\tocitem*[ ]{ }{ }
\tocitem*[V]{V}{V}
\tocitem*[i]{i}{i}
\tocitem*[o]{o}{o}
\tocitem*[l]{l}{l}
\tocitem*[e]{e}{e}
\tocitem*[n]{n}{n}
\tocitem*[c]{c}{c}
\tocitem*[e]{e}{e}
\tocitem*[ ]{ }{ }
\tocitem*[1]{1}{1}
\tocitem*[6]{6}{6}
\tocitem*[.]{.}{.}
\tocitem*[ ]{ }{ }
\tocitem*[M]{M}{M}
\tocitem*[u]{u}{u}
\tocitem*[r]{r}{r}
\tocitem*[k]{k}{k}
\tocitem*[ ]{ }{ }
\tocitem*[R]{R}{R}
\tocitem*[e]{e}{e}
\tocitem*[c]{c}{c}
\tocitem*[e]{e}{e}
\tocitem*[i]{i}{i}
\tocitem*[v]{v}{v}
\tocitem*[e]{e}{e}
\tocitem*[s]{s}{s}
\tocitem*[ ]{ }{ }
\tocitem*[A]{A}{A}
\tocitem*[ ]{ }{ }
\tocitem*[B]{B}{B}
\tocitem*[l]{l}{l}
\tocitem*[o]{o}{o}
\tocitem*[w]{w}{w}
\tocitem*[ ]{ }{ }
\tocitem*[1]{1}{1}
\tocitem*[7]{7}{7}
\tocitem*[.]{.}{.}
\tocitem*[ ]{ }{ }
\tocitem*[M]{M}{M}
\tocitem*[u]{u}{u}
\tocitem*[r]{r}{r}
\tocitem*[k]{k}{k}
\tocitem*[ ]{ }{ }
\tocitem*[I]{I}{I}
\tocitem*[s]{s}{s}
\tocitem*[ ]{ }{ }
\tocitem*[T]{T}{T}
\tocitem*[e]{e}{e}
\tocitem*[m]{m}{m}
\tocitem*[p]{p}{p}
\tocitem*[t]{t}{t}
\tocitem*[e]{e}{e}
\tocitem*[d]{d}{d}
\tocitem*[ ]{ }{ }
\tocitem*[1]{1}{1}
\tocitem*[8]{8}{8}
\tocitem*[.]{.}{.}
\tocitem*[ ]{ }{ }
\tocitem*[A]{A}{A}
\tocitem*[ ]{ }{ }
\tocitem*[W]{W}{W}
\tocitem*[o]{o}{o}
\tocitem*[m]{m}{m}
\tocitem*[a]{a}{a}
\tocitem*[n]{n}{n}
\tocitem*[’]{’}{’}
\tocitem*[s]{s}{s}
\tocitem*[ ]{ }{ }
\tocitem*[W]{W}{W}
\tocitem*[a]{a}{a}
\tocitem*[y]{y}{y}
\tocitem*[ ]{ }{ }
\tocitem*[1]{1}{1}
\tocitem*[9]{9}{9}
\tocitem*[.]{.}{.}
\tocitem*[ ]{ }{ }
\tocitem*[C]{C}{C}
\tocitem*[o]{o}{o}
\tocitem*[a]{a}{a}
\tocitem*[d]{d}{d}
\tocitem*[l]{l}{l}
\tocitem*[e]{e}{e}
\tocitem*[y]{y}{y}
\tocitem*[ ]{ }{ }
\tocitem*[Q]{Q}{Q}
\tocitem*[u]{u}{u}
\tocitem*[i]{i}{i}
\tocitem*[t]{t}{t}
\tocitem*[s]{s}{s}
\tocitem*[ ]{ }{ }
\tocitem*[2]{2}{2}
\tocitem*[0]{0}{0}
\tocitem*[.]{.}{.}
\tocitem*[ ]{ }{ }
\tocitem*[U]{U}{U}
\tocitem*[p]{p}{p}
\tocitem*[ ]{ }{ }
\tocitem*[T]{T}{T}
\tocitem*[h]{h}{h}
\tocitem*[e]{e}{e}
\tocitem*[ ]{ }{ }
\tocitem*[R]{R}{R}
\tocitem*[i]{i}{i}
\tocitem*[v]{v}{v}
\tocitem*[e]{e}{e}
\tocitem*[r]{r}{r}
\tocitem*[ ]{ }{ }
\tocitem*[2]{2}{2}
\tocitem*[1]{1}{1}
\tocitem*[.]{.}{.}
\tocitem*[ ]{ }{ }
\tocitem*[R]{R}{R}
\tocitem*[e]{e}{e}
\tocitem*[c]{c}{c}
\tocitem*[o]{o}{o}
\tocitem*[g]{g}{g}
\tocitem*[n]{n}{n}
\tocitem*[i]{i}{i}
\tocitem*[t]{t}{t}
\tocitem*[i]{i}{i}
\tocitem*[o]{o}{o}
\tocitem*[n]{n}{n}
\tocitem*[ ]{ }{ }
\tocitem*[2]{2}{2}
\tocitem*[2]{2}{2}
\tocitem*[.]{.}{.}
\tocitem*[ ]{ }{ }
\tocitem*[A]{A}{A}
\tocitem*[n]{n}{n}
\tocitem*[ ]{ }{ }
\tocitem*[U]{U}{U}
\tocitem*[n]{n}{n}
\tocitem*[e]{e}{e}
\tocitem*[x]{x}{x}
\tocitem*[p]{p}{p}
\tocitem*[e]{e}{e}
\tocitem*[c]{c}{c}
\tocitem*[t]{t}{t}
\tocitem*[e]{e}{e}
\tocitem*[d]{d}{d}
\tocitem*[ ]{ }{ }
\tocitem*[V]{V}{V}
\tocitem*[i]{i}{i}
\tocitem*[s]{s}{s}
\tocitem*[i]{i}{i}
\tocitem*[t]{t}{t}
\tocitem*[o]{o}{o}
\tocitem*[r]{r}{r}
\tocitem*[ ]{ }{ }
\tocitem*[2]{2}{2}
\tocitem*[3]{3}{3}
\tocitem*[.]{.}{.}
\tocitem*[ ]{ }{ }
\tocitem*[A]{A}{A}
\tocitem*[ ]{ }{ }
\tocitem*[S]{S}{S}
\tocitem*[t]{t}{t}
\tocitem*[a]{a}{a}
\tocitem*[r]{r}{r}
\tocitem*[t]{t}{t}
\tocitem*[l]{l}{l}
\tocitem*[i]{i}{i}
\tocitem*[n]{n}{n}
\tocitem*[g]{g}{g}
\tocitem*[ ]{ }{ }
\tocitem*[S]{S}{S}
\tocitem*[t]{t}{t}
\tocitem*[o]{o}{o}
\tocitem*[r]{r}{r}
\tocitem*[y]{y}{y}
\tocitem*[ ]{ }{ }
\tocitem*[2]{2}{2}
\tocitem*[4]{4}{4}
\tocitem*[.]{.}{.}
\tocitem*[ ]{ }{ }
\tocitem*[H]{H}{H}
\tocitem*[i]{i}{i}
\tocitem*[g]{g}{g}
\tocitem*[h]{h}{h}
\tocitem*[-]{-}{-}
\tocitem*[H]{H}{H}
\tocitem*[a]{a}{a}
\tocitem*[n]{n}{n}
\tocitem*[d]{d}{d}
\tocitem*[e]{e}{e}
\tocitem*[d]{d}{d}
\tocitem*[ ]{ }{ }
\tocitem*[M]{M}{M}
\tocitem*[e]{e}{e}
\tocitem*[t]{t}{t}
\tocitem*[h]{h}{h}
\tocitem*[o]{o}{o}
\tocitem*[d]{d}{d}
\tocitem*[s]{s}{s}
\tocitem*[ ]{ }{ }
\tocitem*[2]{2}{2}
\tocitem*[5]{5}{5}
\tocitem*[.]{.}{.}
\tocitem*[ ]{ }{ }
\tocitem*[A]{A}{A}
\tocitem*[n]{n}{n}
\tocitem*[ ]{ }{ }
\tocitem*[A]{A}{A}
\tocitem*[c]{c}{c}
\tocitem*[c]{c}{c}
\tocitem*[u]{u}{u}
\tocitem*[s]{s}{s}
\tocitem*[a]{a}{a}
\tocitem*[t]{t}{t}
\tocitem*[i]{i}{i}
\tocitem*[o]{o}{o}
\tocitem*[n]{n}{n}
\tocitem*[ ]{ }{ }
\tocitem*[2]{2}{2}
\tocitem*[6]{6}{6}
\tocitem*[.]{.}{.}
\tocitem*[ ]{ }{ }
\tocitem*[T]{T}{T}
\tocitem*[h]{h}{h}
\tocitem*[e]{e}{e}
\tocitem*[ ]{ }{ }
\tocitem*[T]{T}{T}
\tocitem*[r]{r}{r}
\tocitem*[u]{u}{u}
\tocitem*[t]{t}{t}
\tocitem*[h]{h}{h}
\tocitem*[ ]{ }{ }
\tocitem*[C]{C}{C}
\tocitem*[o]{o}{o}
\tocitem*[m]{m}{m}
\tocitem*[e]{e}{e}
\tocitem*[s]{s}{s}
\tocitem*[ ]{ }{ }
\tocitem*[O]{O}{O}
\tocitem*[u]{u}{u}
\tocitem*[t]{t}{t}


\end{toc}
\clearpage

\mainmatter
\cleartorecto
\thispagestyle{empty}


\begin{ChapterStart}
\vspace{3\nbs}
\ChapterSubtitle[l]{Chapter ch1}
\ChapterTitle[l]{ch1}
\end{ChapterStart}
\FirstLine{\noindent ## In The Harbor}
    
Now the fog was clearing and the mist was lifting, and the bright sunshine was struggling to penetrate the billows of damp vapor and touch with its glory the things of the world beneath. In the lower harbor there still was a chorus of sirens and foghorns, as craft of almost every description made way toward the metropolis or out toward the open sea.

The Manatee, tramp steamer with rusty plates and rattling engines and a lurch like that of a drunken man, wallowed her way in from the turbulent ocean she had fought for three days, her skipper standing on the bridge and inaudibly giving thanks that he was nearing the end of the voyage without the necessity for abandoning his craft for an open boat, or remaining to go down with the ship after the manner of skippers of the old school.

Here and there showed a rift in the rolling fog, and those who braved the weather and lined the damp rail could see other craft in passing.

A giant liner made her way past majestically, bound for Europe, or a seagoing tug clugged by as if turning up her nose at the old, battered Manatee.

Standing at the rail, and well forward, Sidney Prale strained his eyes and looked ahead, watching where the fog lifted, an eager light in his face, his lips curved in a smile, a general expression of anticipation about him.

Sidney Prale himself was not bad to look at. 30-eight he was, tall and broad of shoulder, with hair that was touched with gray at the temples, with a face that had been browned by the weather. Sidney Prale had the appearance of wearing clothes that had been molded to his form. He had a chin that expressed decision and determination, lips that could form in a thin, straight line if occasion required, eyes that could be kind or stern, according to the needs of the moment. A man of the world would have said that Sidney Prale was a gentleman of broad experience, a man who had presence of mind in the face of danger, a man who could think quickly and act quickly when such things were necessary.

He was not alone at the rail---and yet he was alone in a sense, for he gave no one the slightest attention. He bent over and looked ahead eagerly, waving a hand now and then at the men on passing craft, like a schoolboy on an excursion trip. He listened to the bellowing sirens and foghorns, drank in the raucous cries of the ship's officers, strained his ears for the land sounds that rolled now and then across the waters.

"It's great---great!" Sidney Prale said, half aloud.

He bent over the rail again. A hand descended upon his shoulder, and a voice answered him.

"You bet it's great, Prale!"

Sidney Prale's smile weakened a bit as he turned around, but there was nothing of discourtesy in his manner.

"You like it, Mr. Shepley?" he asked.

"Do I like it? Does Rufus Shepley, forced to run here and there around the old world in the name of business, like it when he gets the chance to return to New York? Ask me!"

"I have my answer," Prale said, laughing a bit. "And judge, then, how I like it---when I have not seen it for ten years."

"Haven't seen New York for ten years?" Rufus Shepley gasped.

"A whole decade," Prale admitted.

"Been down in Honduras all that time?"

"Yes, sir."

"And you live to tell it? You are my idea of a real man!" Rufus Shepley said.

Shepley took a cigar from his vest pocket, bit off the end, lighted it, and puffed a cloud of fragrant smoke into the air. Rufus Shepley was a man of fifty, and looked his age. If human being ever gave the appearance of being the regulation man of big business affairs, Rufus Shepley did.

Sidney Prale had held some conversation with him on board ship, but they had not become very well acquainted, though they seemed to like each other. Each man seemed to be holding back, waiting, trying to discover in the other more qualities to like or dislike.

"10 years," Sidney Prale went on thoughtfully. "It seems a long time, but the years have passed swiftly."

"I always had an idea," Rufus Shepley said, "that a genuine white man who went to one of those Central American countries turned bad after the first year and went to the devil generally. But you don't look it."

"The idea is correct, at that, in some instances," Prale admitted. "Some of them do turn bad."

"They get to drifting, eh? The climate gets into their blood. Do you know what I think? I think that, in seven cases out of eight, it's a case of a man wanting an excuse for loafing. I knew a chap once who went down to that part of the world. Got to drinking too much, threw up his job, used to loaf all the time, married some sort of a half-black woman who had a bit of coin, and went to the dogs generally."

"Oh, there are many such," Sidney Prale admitted. "But the majority of them are men who made some grave mistake somewhere else and got the idea that life was merely existence afterward. A man must have an incentive in any climate to make anything of himself---and down there the incentive has to be stronger."

"I assume that you---er---had the proper incentive," Rufus Shepley said, grinning.

"I don't know how some persons would look at the propriety of it. I wanted to make a million dollars."

"Great Scott! Your ambition was a modest one, I must say. And you managed to win out? Oh, I beg your pardon! It isn't any of my business, of course!"

"That's all right," Prale answered good-naturedly. "I don't mind. I'm so happy this morning that I'm willing to overlook almost anything. And I don't mind telling you that I've won out."

"A million in ten years," Shepley gasped.

"Yes; and with an initial capital of ten thousand dollars," Sidney Prale replied. "I'm rather proud of it, of course. I suppose this sounds like boasting------"

"My boy, you have the right to boast! A million dollars in ten years! Great Scott! Say, would you consider being general manager of one of my companies? We need a few men like you."

Sidney Prale laughed again. "Sorry---but I'm afraid that I can't take the job," he replied. "I am going to have my little holiday now---going to play. A million isn't much in some quarters, but it is enough for me. I don't care for money to a great extent. I just wanted to prove to myself that I could make a million---prove it to myself and others. And, ready to take my vacation, I naturally decided to take it in New York---home!"

"Ah! Home's in New York, eh? Old friends waiting at the dock, and all that!"

Sidney Prale's face clouded. "I am afraid that there will be no reception committee," he said. "I didn't let anybody know that I was coming---for the simple reason that I didn't know whom to inform."

"My boy!"

"I have a few old friends scattered around some place, I suppose. I have no relatives in the world except a male cousin about my own age, and I never communicated with him after going to Honduras. There was a girl once------"

"There always is a girl," Shepley said softly, as Prale ceased speaking.

"But that ended ten years ago," Prale continued. "I stand alone---with my million."

"You advertise that fact, my boy, and there'll be girls by the regiment looking up your telephone number."

"And the right one wouldn't be in the crowd," Prale said, the smile leaving his face again.

"Well, you are in for a fine time, at least," Rufus Shepley told him. "There have been quite a few changes in New York in the past ten years. Yes, quite a few changes! There are a few new boarding houses scattered around, and a new general store or two, and the street cars run out farther than they used to."

"Oh, I've kept up to date after a fashion," Sidney Prale said, laughing once more. "I'm ready to appreciate the changes, but I suppose I will be surprised. The New York papers get down to Honduras now and then, you know."

"I've always understood," Shepley said, "that there are certain gentlemen in that part of the world who watch the New York papers very closely."

"Meaning the men who are fugitives from justice, I see," said Prale.

"I didn't mean anything personal, of course."

"It does look bad, doesn't it?" said Prale. "I went straight to Honduras when I left New York ten years ago, like a man running away from the law, and I have remained there all the time until this trip. And I have been gone ten years---thereby satisfying certain statutes of limitation------"

"My boy, I never meant to insinuate that------"

"I know that you didn't," Prale interrupted. "My conscience is clear, Mr. Shepley. When I land, I'll not be afraid of some officer of the law clutching me by the shoulder and hauling me away to a police station."

"Even if one did, a cool million will buy lots of bail," Rufus Shepley said.

The fog was lifting rapidly now. Here and there through the billows of mist could be seen the roofs of skyscrapers glistening in the sun. Sidney Prale almost forgot the man at his side as he bent over the rail to watch.

"Getting home---getting home!" he said. "I suppose no man ever gets quite over the home idea, no matter how long he remains away. 10 years ought to make a change, but I find that it doesn't. I'll be glad to feel the pavements beneath my shoes again."

"Sure!" said Rufus Shepley.

"Confound the fog! Ah, there's a building I know! And there are a few I never saw before. We're beginning to get in, aren't we? Ought to dock before noon, don't you think?"

"Sure thing!"

"A hotel, a bath, fresh clothes---and then for hour after hour of walking around and taking in the sights!" Prale said.

"Better engage a taxi if you expect to take 'em all in before night, my boy," Shepley said.

"I forgot! We haven't any too many taxis in Honduras. I had a car of my own, but sold it before I came away."

"You let the busy auto agents know that, and you'll have a regiment of them------"

"And there!" Sidney Prale cried. "Now I know that I am home! There is the Old Girl in the Harbor!"

Prale removed his cap, and a mist came into his eyes that did not come from the foggy billows through which the ship was plowing. The sun was shining through the murk at last, and it touched the Statue of Liberty. The great figure seemed like a live thing for a moment; the mist made it appear that her garments were waving in the breeze.

"Now I know that I am home!" Sidney Prale repeated.

"She sure is a great old girl!" Rufus Shepley agreed. "Always glad to see her!"

"Well, I've got to get ready to land; I'm not going to waste any time," Prale said. "I'm glad that I met you---and perhaps we'll meet again in the city."

"Hope we do!" said Shepley, grasping Prale's hand. "Our factories are out in Ohio, but the company headquarters are in New York, of course. Here's my business card, my boy. And I generally put up at the Graymore."

Sidney Prale took the card, thanked Rufus Shepley, and hurried down the deck toward his stateroom, one of the best on the ship. Rufus Shepley looked after him sharply.

"Went straight to Honduras and stayed there for ten years, eh?" Rufus Shepley said to himself. "Um! Looks bad! I never put much stock in those Honduras chaps---but this one seems to be all right. Never can tell, though!"

Sidney Prale, still smiling, and humming a Spanish love song, reached his stateroom and threw open the door; and just inside, he came to a stop, astonished.

Somebody had been in that stateroom and had been going through his things. The contents of his suit case were spilled on the floor. A bag was wide open; he had left it closed and in a corner less than an hour before.

Prale went down on his knees and made a quick inspection. There did not seem to be anything missing. A package of papers---business documents for the greater part---had been examined, he could tell at a glance, but none had been taken.

"Peculiar!" Prale told himself. "Some sneak thief, I suppose. No sense in complaining to the ship's officers at this late hour, especially since nothing has been stolen. Makes a man angry, though!"

He put the suit case on the table and began repacking the things that had been scattered on the floor. Then he gathered up his toilet articles, bits of clothing he had left out until the last minute, a few souvenirs of Honduras he had been showing a tourist the evening before. He turned toward the berth to pick up his light overcoat.

There was a sheet of paper pinned to the pillow, paper that might have been taken from an ordinary writing tablet. Sidney Prale took it up and glanced at it. A few words of handwriting were upon the paper, words that looked as if they had been scrawled hurriedly with a pencil that needed sharpening badly.

"Retribution is inevitable and comes when you least expect it."

The smile fled from Sidney Prale's lips, and the Spanish love song he had been humming died in his throat. He frowned, and read the message again.

"Now what the deuce does this mean?" he gasped.

\vspace{2\nbs}
\ChapterDeco[c1]{\decoglyph{e9665}}
\clearpage
\thispagestyle{empty}

\begin{ChapterStart}
\vspace{3\nbs}
\ChapterSubtitle[l]{Chapter ch10}
\ChapterTitle[l]{ch10}
\end{ChapterStart}
\FirstLine{\noindent ## On The Trail}
    
Farland engaged a taxicab, bade Murk get into it, got in himself, and they started downtown. The detective leaned back against the cushions and regarded Murk closely. He knew that Sidney Prale had guessed correctly, that Murk was the sort of man who would prove loyal to a friend.

"This is a bad business," Farland said.

"It's tough," said Murk.

"If it was anybody but Sid Prale, I'd say he was guilty. It sure looks bad. And there is that fountain pen!"

"Somebody's tryin' to do him dirt," Murk said.

"There's no question about that, Murk, old boy. Well, we are going to get him out of it, aren't we?"

"I'll do anything I can."

"Like him, do you?"

"Met him less than twenty-four hours ago, but I wish I'd met him or somebody like him ten years ago," Murk replied. "If it hadn't been for Mr. Prale, I'd be a stiff up in the morgue this minute."

"Strong for him, are you?"

"Yes, sir, I am!"

"Um!" said Jim Farland. "We're going to get along fine together. I was strong for Sid Prale ten years ago, before he went away. And I'll bet that, when we get to the bottom of this, we'll find something mighty interesting."

The taxicab stopped at a corner, and Farland and Murk got out. Farland paid the chauffeur and watched him drive away, and then he led Murk around the corner.

"Know where you are?" he asked.

"Sure. Right over there is the little shop where Mr. Prale bought me my new clothes," Murk said.

"Fine! That goes to show that Prale told the truth. Well, Murk, you stand right here by the curb and watch the front door of that shop. And when you see me beckon to you, you come running."

"Yes, sir."

Jim Farland hurried across the street, opened the door of the little shop, and entered. The proprietor came from the rear room when he heard the door slammed.

He knew Jim Farland and had known him for years. There were few old-timers in that section of the city who did not know Jim Farland. The man who faced the detective now was small, stoop-shouldered, a sort of a rat of a man who had considerably more money to his credit than his appearance indicated, and who was not eager to have the world in general know how he had acquired some of it.

"Evenin', Mr. Farland," he said. "Anything I can do for you, sir?"

"Maybe you can and maybe you can't," Farland told him. "You been behaving yourself lately?"

"What do you mean, Mr. Farland? I've been trying to get along, but business ain't been any too good the last year."

"Save that song for somebody who doesn't know better!" Farland advised him. "Change the record when you play me a tune."

"Yes, sir. Is there anything I can do for you, Mr. Farland?"

"Remember a little deal a couple of years ago?" Farland demanded suddenly.

"I---I------"

"I see that you do. 1 little word from me in the proper quarter, old man, and you'll be doing time. You've sailed pretty close to the edge of the law a lot of times, and once, I know, you slipped over the edge a bit."

"I---I hope, sir------"

"You'd better hope that you can keep on the good side of me," Jim Farland told him.

"If there is anything I can do, Mr. Farland------"

"Do you suppose you could tell the truth?"

"Yes, sir."

"I'm going to give you a chance. If you tell the truth, I may forget something I know, for the time being. But, if you shouldn't tell the truth---well, my memory is excellent when I want to exercise it."

Farland stepped to the door and beckoned, and Murk hurried across the street and entered the shop.

"Ever see this man before?" Farland demanded.

The storekeeper licked his lips, and a sudden gleam came into his eyes.

"I---he seems to look familiar, but I can't say."

"You'd better say!" Farland exclaimed. "I want the truth out of you, or something will drop. And when it drops, it is liable to hit you on the toes. Get me?"

"I---I don't know what to do," wailed the merchant.

"Tell the truth!"

"But---there is something peculiar about------"

"Out with it! Know this man?"

"I've seen him before," the merchant replied.

"When?"

"La-last night, sir."

"Now we are getting at it!" Jim Farland exclaimed. "When did you see him last night, and where, and what happened?"

"He was in the store, Mr. Farland, about half past ten or a quarter of eleven o'clock. He---he bought those clothes he's got on."

"Pay for them?"

"Yes, sir."

"Who paid for them?" Farland demanded.

"A gentleman who was with him," said the merchant.

"Ah! Know the gentleman?"

"I saw him to-day---at police headquarters."

"And you said that you never had seen him before---that he was not here last night with this man. Why did you lie?"

Jim Farland roared the question and smashed a fist down upon the counter. The little merchant flinched.

"Out with it!" Farland cried. "Tell the truth, you little crook! I want to know why you lied, who told you to lie. I want to know all about it, and mighty quick!"

"I---I don't understand this," the merchant whimpered. "I was afraid of making a mistake."

"You'll make a mistake right now if you don't tell the truth!" Jim Farland told him.

"I---I got a letter, sir, by messenger. I got it early this morning, sir."

"Well, what about it?"

"The letter was typewritten, sir, and was not signed. There was a thousand dollars in bills in the letter, sir, and it said that a Mr. Prale had just been arrested for murder, and that he probably would try to make an alibi by saying that he was here last night and bought some clothes for another man. The letter said that I was to take the money and ask no questions, and that, if I was called to police headquarters, I was to say the man had not been here and that I never had seen him in my life before."

"And you fell for it? You wanted that thousand, I suppose."

"I'll show you the letter, Mr. Farland. There was no signature at all, and the paper was just common paper. I---I thought it was politics, sir."

"You did, eh?"

"Thought it had something to do with politics, sir. I thought the letter and money might have come from political headquarters. I was afraid to tell the truth at the police station."

"You mean you have been so crooked for years that you're afraid of everybody who has a little influence," Farland told him.

"I thought it was orders, sir, from somebody who had better be obeyed."

"Oh, I understand, all right. Well, I scarcely think it was politics. You've been played, that's all. Get me that letter!"

"Yes, sir."

The merchant got it and handed it over, together with the envelope. He had told the truth. The letter was typewritten on an ordinary piece of paper, and the envelope was of the sort anybody could purchase at a corner drug store. Farland put the letter in his pocket.

"Here between ten thirty and a quarter of eleven, was he?"

"Yes, sir," said the merchant.

"All right! You remember that, and don't change your mind again, if you know what is good for you. You'll hear from me in the morning. That's all!"

Jim Farland went from the store with a grinning Murk at his heels, leaving a badly frightened small merchant behind him.

"I know that bird," he told Murk. "He's a fence, or I miss my guess. It's no job at all to run a bluff on a small-time crook like that. And now we'll run down and see that barber."

They engaged another taxicab and made a trip. Once more Murk remained outside, and Jim Farland entered and beckoned the barber to him.

"Step outside the door where nobody will overhear," he said. "I want to ask you something."

The barber stepped outside, wondering what was coming. This man knew Jim Farland, too, and he knew that a call from him might mean trouble.

"Trying to see how far you can go and keep out of jail?" Farland demanded.

"I---I don't know what you mean, sir."

"Trying to run a bluff on me? On me?" Farland gasped. "You'd better talk straight. Do you expect to run a barber shop by day and a gambling joint by night all your life?"

"Why, I------"

"Don't lie!" Farland interrupted. "I know all about that little back room. Maybe I'm not on the city police force now, but you know me! I've got a bunch of friends on the force, and if I told a certain sergeant about your little game and said that I wanted to have you run in he wouldn't hesitate a minute."

"But what have I done, Mr. Farland?" the barber gasped. "I've always been friendly to you."

"I know it. But are you going to keep right on being friendly?"

"Of course, sir."

"Willing to help me out in a little matter if I forget about that gambling?"

"I'll do the best I can, Mr. Farland."

"Then answer a few questions. Did you get a typewritten letter this morning, with a wad of money in it?"

The barber's face turned white.

"Answer me!" Farland commanded.

"Yes, I---I got such a letter and I don't know what to make of it," the barber said. "I've got the letter and money in my desk right now. There wasn't any signature, and I didn't know where the letter came from, or what it meant."

"Then why did you do what the letter told you to do?" Farland asked.

"I---I don't understand."

Farland motioned, and Murk now stepped around the corner.

"Know this man?" Farland demanded.

"I---I've seen him before."

"That letter told you to go to police headquarters, if requested to do so, and deny you knew this man, didn't it? It told you not to help a man named Sidney Prale, arrested for murder, to make his alibi by telling that he was here with this man last night about eleven o'clock, didn't it?"

"Y-yes, sir."

"And you did just what the letter told you?"

"I was afraid not to do it, sir. I didn't know where that letter came from, you see."

"Had an idea it came from some boss, didn't you?"

"I didn't know and I didn't dare take a chance, Mr. Farland. You know how it is?"

"I know how it is with a man who has busted a few laws and knows he ought to be pinched!"

"Did I make some sort of a mistake, sir? What should I do now?"

"Something you don't do very often---tell the truth," Jim Farland replied. "How about this man?"

"He came here with the other gentleman last night about eleven o'clock, sir. He got a hair cut and a shave, and the other gentleman paid the bill."

"Thanks. Sure about the time?"

"I know that it was almost a quarter after eleven when they left the shop."

"Well, I'm glad you can speak the truth. Get on your hat and coat!"

"I---what do you mean, sir? Am I arrested?"

"No. Get that letter and come with me. I want you to tell the truth to somebody else, that's all."

The frightened barber got his hat and coat and the letter, and followed Jim Farland and Murk to the corner. There Farland engaged another taxicab, and ordered the chauffeur to drive back to the little clothing store.

"Running up a nice expense bill for Prale, but he won't care," Jim Farland said to Murk.

He compelled the merchant to shut up his shop and get into the cab, and then the chauffeur drove to police headquarters. Farland had telephoned from the clothing store, and the captain of detectives was waiting for him. He ushered the merchant and the barber into the office, looked down at the captain, and grinned.

"What's all this?" the captain demanded.

"It's Sid Prale's alibi," Jim Farland said. "These two gents want to tell you how they lied to-day, and why they lied. It is an interesting story."

The captain sat up straight in his chair, while Jim Farland removed his hat, sat down, motioned for Murk to do the same, and made himself comfortable.

"About that alibi," Farland said. "I know that George Lerton lied about meeting Sid Prale on Fifth Avenue, but you don't, and so we'll let that pass for the time being and get to it later. I just want to show you now that Prale's story about meeting this man Murk was a true tale. This clothing merchant is ready to say now that Prale and Murk were in his place last night about half past ten, and that Murk got his clothes there. And this barber is ready to swear that Prale and Murk arrived at his shop about a quarter of eleven or eleven, and did not leave until a quarter after eleven. Prale and Murk got to the hotel, as you know, at midnight. Prale couldn't have gone to that other hotel, murdered Rufus Shepley, and got to his suite by twelve o'clock, not if he left that barber shop far downtown at a quarter after eleven, could he?"

"Scarcely," said the captain.

"Very well. Ask these two gents some questions."

The captain did. He read the two typewritten letters and he understood how the fear of a political power might have been in the hearts of the two men. He rebuked them and allowed them to go.

"Well, it looks a little better for Mr. Prale," the captain said, "but this isn't the end, by any means. Remember that fountain pen of his that was found beside the body of Rufus Shepley!"

"I didn't say that it was the end," Jim Farland declared. "I don't want it given out that any evidence has been found that is in Prale's favor. I just want you to whisper in the ear of the court that the alibi looks good, and let it go at that. There's something behind this case, and we want to find out what it is. Prale is out on bail---and let it go at that, as far as the public is concerned."

"I grasp you," said the captain. "You want these enemies of his to think he is in deep water, so they'll be off guard and you can do your work."

"Exactly," said Jim Farland.

"Good enough. I'll do my part."

"Know anything about a woman calling herself Kate Gilbert?"

"Never heard of her."

Farland explained what Prale had told him. The captain fingered his mustache.

"Several thousand women in this town answer that general description," he said. "I'm afraid I can't help you, unless you can pick her up."

"That's what I'll do as soon as I can," Farland replied. "If I can get my eyes on her once, I'll trail her and find out a few things. She may have nothing to do with this, and she may have a great deal to do with it. What do you know about George Lerton?"

"Shady broker," the captain replied. "Never done anything outside the law, as far as I know, but he's come pretty close to it. I'd hate to have him handling my money."

"Well, he lied about meeting Prale. He did his best to get Prale to run away from town. That was a couple of hours before the murder, of course, so it probably had nothing to do with that. But why should he try to get Prale out of town? And, being a man of that sort, why did he say that he wouldn't handle Prale's funds? You'd think a man of his sort would like nothing better than to get his fingers tangled up in that million."

"I'll have a man take a look at George Lerton."

"Don't strain yourself," said Jim Farland. "I'm going to take a look at him myself, the first thing to-morrow morning."

He left headquarters with Murk, and this time he did not engage a taxicab. He walked up the street, Murk at his side, and puffed at a cigar furiously.

"Well, Murk, we've made a good start," Farland said, after a time.

"Yes, sir."

"How do you like working with a detective now?"

"Aw, you ain't a regular detective," Murk said.

"What's that?"

"I mean you ain't an ordinary dick. You got some sense."

"Thanks for the compliment. I know men who would dispute the statement," Farland told him.

They walked and walked, and after a time were on Fifth Avenue and going toward the hotel where Prale had his suite. Suddenly, just ahead of them, they saw Sidney Prale and the man from headquarters. They hurried to catch up with them.

"What's the idea?" Farland asked.

"Needed a walk," Prale replied. "Didn't feel like going to bed, and a walk would do me good, I knew."

"I'll have some things to tell you in the morning," Farland said. "But I'm not going to tell you to-night, except to say that it is good news, and I'm issuing orders to Murk not to tell you, either. I want you to forget the thing and get some rest."

"All right," Prale said, laughing; and then he stopped still and gasped.

"What is it?" Farland asked.

"Kate Gilbert!"

"Where?"

"There---just getting into that limousine. See her? The girl with the red hat!"

"I see her," Farland replied, signaling the chauffeur of a passing taxicab. "This is what I was hoping for, Sid. Go on to the hotel with Murk and guard. I'm going to find out a few things about Miss Kate Gilbert!"

He gave the chauffeur of the taxicab whispered directions, and then sprang into the machine.

\vspace{2\nbs}
\ChapterDeco[c1]{\decoglyph{e9665}}
\clearpage
\thispagestyle{empty}

\begin{ChapterStart}
\vspace{3\nbs}
\ChapterSubtitle[l]{Chapter ch11}
\ChapterTitle[l]{ch11}
\end{ChapterStart}
\FirstLine{\noindent ## Concerning Kate Gilbert}
    
Given a definite trail to follow, Jim Farland was one of the best trackers in the business. He liked to know his quarry by sight, and conduct the hunt in a proper manner. And so he rejoiced, that now he was following a person he believed to be interested in some way in the Shepley case.

The limousine went up Fifth Avenue toward Central Park, and the taxicab with Jim Farland inside followed half a block behind. Farland did nothing except look ahead continually and make sure that his chauffeur did not lose the other machine. He wanted to discover, first, where Miss Kate Gilbert was going, and after that he wanted to acquire all the information he could concerning her.

There was little traffic on the Avenue at this hour, and the limousine made good progress. It curved around the Circle and went up Central Park West. In the Eighties it turned off into a side street, and finally drew up to the curb and stopped. The taxicab came to a halt a hundred feet behind it. "Wait," Jim Farland instructed the chauffeur, showing his shield. "Wait until I come back, even if I don't come back until morning. You will get good pay, all right."

The chauffeur settled back behind his wheel, and Farland stepped to one side in the darkness and watched. He saw an elderly gentleman emerge from the limousine and turn to help Kate Gilbert out. Then the elderly gentleman got into the car again and was driven away, and Kate Gilbert went into the apartment house before which the limousine had stopped.

Detective Jim Farland hurried forward, but when he came opposite the apartment house he slowed down and walked slowly, glancing in. It was not an apartment house of the better sort. The lobby was small, there was an automatic elevator, and no hall boy was on duty, that Farland could see. There was a row of mail boxes against a wall, with name plates over them.

Farland went up the steps, opened the door, and stepped inside the lobby. He walked across to the mail boxes and began looking at the names. He found some one named Gilbert had an apartment on the third floor, front.

The stairs were before him, and Farland was about to start up them when a door leading to the basement was opened, and a janitor appeared. He was an old man, bent and withered, and he looked at Farland with sudden suspicion.

"You want to see somebody in the house?" he asked, in a voice that quavered.

"I want to see you," Jim Farland answered.

"What about, sir?"

Farland exhibited his shield, and the old janitor recoiled, fright depicted in his face.

"I ain't done anything wrong, mister," he said hoarsely. "I obey all the regulations about ashes and garbage and everything like that."

"Don't be afraid of me," Farland said. "I'm not accusing you of doing anything wrong, am I? I can see that you're a law-abiding man. You haven't nerve enough to be anything else. Suppose you step outside with me for a few minutes. I just want to ask you a few questions about something."

"All right, sir, if that's it," the old janitor said.

He opened the front door and led the way outside, and Farland forced him to walk a short distance down the street, and there they stopped in a doorway to talk.

"I'm going to ask you a few questions, and you are going to answer them, and then you are going to forget that you ever saw me or that I ever asked you a thing," Farland said.

"I understand, sir. I won't give away any police business," the old janitor replied. "I know all about such things. I had a nephew once who was a policeman."

"There's a party living in your place who goes by the name of Gilbert, isn't there?"

"Yes, sir."

"How many are there in the family, and who are they, and what do you know about them?"

"There is an old man, sir," the janitor answered. "He's a sort of cripple, I guess. He always sits in one of them invalid chairs, and when he goes out somebody has to wheel him. If he ain't exactly a cripple, then he's mighty sick and weak."

"Who else is in the family?"

"He's got a daughter, whose name is Miss Kate," the janitor said. "She's a mighty fine-lookin' girl, too. She's a nice woman, I reckon. 'Pears to be, anyway."

"Do you know anything in particular about her?" Jim Farland asked him.

"Well, she's been away for about three months, and she just got back," the janitor replied. "I don't know where she was---didn't hear. While she was gone, there was a man nurse 'tended to her father---cooked the meals and kept the apartment clean and took him out in his wheel chair. Miss Kate has a maid they call Marie---a big, ugly woman. She takes care of things generally when she is here, but she was away with Miss Kate."

"How long have they lived here?" Farland asked.

"About three years, sir. But I don't know much about them. They ain't the kind of folks a man can find out a lot about. They act peculiar sometimes."

"Are they rich?"

"My gracious, no!" said the old janitor. "They pay their rent on time, and they always seem to have plenty to eat, and I guess they can afford to keep that maid and hire a nurse once in a while, but they ain't what you'd call rich. But Miss Kate comes home in a big automobile now and then, and she seems to have a lot of clothes. There's something funny about it, at that."

"Think she isn't a decent woman?" Farland asked.

"Oh, I don't think she's a bad sort, sir, if that is what you mean. She doesn't seem to be, at all. I guess she gets her swell clothes honest enough. I think that she works for somebody and has to dress that way."

"Do they get much mail and have many visitors?"

"They get a few letters, and some newspapers and magazines," the janitor replied. "And they don't seem to have many visitors. I've seen a man come here once or twice to see them, and once he brought Miss Kate home in an auto. He looks like a rich man."

"Is he old or young?" Farland asked.

"Oh, he has gray hair, sir, and looks like a distinguished gentleman, like a lawyer or something. I guess he's rich. I think maybe he is an old friend of Mr. Gilbert's, or something like that."

"They live on the third floor, don't they?"

"Yes, sir."

"Any vacant apartments up there?"

"Why, the apartment adjoining theirs happens to be vacant just now, sir."

"You take me up to that vacant apartment," Jim Farland directed. "Let me in without making any noise, and then forget all about me until I speak to you again. Here is a nice little bill, and there will be more if you attend to business. I'm an officer, so you'll not get in trouble with the landlord."

The old janitor accepted the bill gladly, and led the way back to the house. Jim Farland refused to use the elevator; he insisted on walking up the stairs, and on going up noiselessly. When they reached the third floor, he was doubly alert.

The old janitor pointed out the door of the vacant apartment, and handed Farland a key. Then he pattered back down the stairs. Farland slipped along the hall, unlocked the door of the vacant apartment, darted inside, and locked the door again, putting the key in his pocket. And then he moved noiselessly through the apartment until he had reached the front.

He could hear voices in the apartment adjoining, and could make out the conversation. A woman was speaking---Farland decided that she was Kate Gilbert---and the weak voice of a sick man was answering her now and then.

"Let's not talk about it any more to-night, father," the girl was saying. "You'll not sleep well, if you get to thinking about it. You must go to bed now, and we'll have a real talk about things when I have something of importance to tell you. Get a good sleep, and in the morning Marie can take you out in the Park."

Jim Farland could hear the old man mutter some reply, and then there reached his ears the squeaking of a wheel chair being rolled across the floor. He remained for a time standing against the wall, listening. He decided that those in the Gilbert apartment were preparing to retire. Half an hour later, Farland slipped from the room and went to the basement to find the janitor.

"Here's your key," he said. "I'll be back here in the morning, and I'll want to see you. And remember---you're not to say a word about all this."

"Not a single word, sir."

Farland went back to the taxicab and drove to his own modest home, where he tumbled into bed and slept the sleep of the just. When Jim Farland slept, he slept---and when he worked, he worked. Farland did not mix labor and rest.

He arose early, hurried through his breakfast, got another taxicab and went up into the Eighties again. The old janitor was sweeping off the walk in front of the apartment house. The curtains at the windows of the Gilbert apartment were still down.

"Give me that key again and give me a pass key, too," Farland told the old janitor. "If the maid takes Mr. Gilbert out, and Miss Gilbert is gone at the same time, I want to get into their apartment and take a look around. Understand? And I'll want you to watch, so I'll not be caught in there."

"I understand, sir. Here are the keys."

Farland reached the vacant apartment without being seen. The Gilberts were up now and eating breakfast. He could hear Kate Gilbert trying to cheer her father, but not a word she said had anything to do with Sidney Prale, or Rufus Shepley, or anybody connected in any way with the Shepley murder case.

"Now you must let Marie take you to the Park, father," he heard the girl say. "It is a splendid day, and you must get a lot of fresh air. You can go down and watch the animals. I'm going out now, but I'll be back some time during the afternoon, and then we'll talk about things."

Jim Farland waited in the vacant apartment until he heard Kate Gilbert depart. A quarter of an hour later, he opened the front door a crack and saw the gigantic Marie wheel out the chair with Mr. Gilbert in it. They went down in the elevator.

Farland waited for another quarter of an hour, until the old janitor came up and told that he had watched the maid wheel Mr. Gilbert into the Park.

"I'll just leave the elevator up here until somebody rings," the old janitor said, "and I'll watch the floor below from the top of the stairs. Then, if any of them come back, I'll tell you so you can get out."

He took his station at the head of the stairs, leaving the elevator door open so that the contrivance could not be operated from below. Jim Farland unlocked the door of the Gilbert apartment and stepped inside.

The first glance told him that it was an ordinary apartment furnished in quite an ordinary manner. It certainly did not look like a home of wealth, and Sidney Prale had said that it had been understood in Honduras that Kate Gilbert was of a rich family and traveling for her health.

Many tourists claim to have money when they are away from home, of course, but the part about traveling for her health seemed to Jim Farland to be going a bit too far. Would such a woman be traveling for her health and leave behind her at home an old father who was an invalid?

"There's something behind that little trip of hers," Farland told himself. "It looks to me as if she had gone down to Honduras to look up Sid Prale for some reason. And Honduras isn't exactly on the health-trip list, either."

He began a close inspection of the apartment, leaving no trace of his search behind him, disarranging nothing that he did not replace. Jim Farland was an expert at such things.

He ransacked a small desk that stood in one corner of the living room and found a tablet of writing paper similar to that upon which had been written the anonymous messages Sidney Prale had received. He found scraps of writing in the wastebasket, too, and inspected them carefully.

"Somebody in this apartment wrote those notes, all right," Farland mused. "But why? That's the question I want answered, and I'll have to be careful how I start in to find out. You can't bluff that girl; one look is enough to tell me that. If I jump her about those notes, she'll probably get wise and cover her tracks, and then I'll be strictly up against it."

He found nothing else of importance in the apartment. There were some letters, but they seemed to be from relatives scattered throughout the country, ordinary letters dealing with family affairs of no particular consequence, and they told Jim Farland nothing that he wished to know.

But Kate Gilbert was only one angle of the case, he reminded himself, and so he decided that he was done for the present as far as she was concerned. It would be only a waste of valuable time, he thought, to remain longer in the Gilbert apartment; and there were plenty of other things for him to be doing.

Farland went all over the apartment once more, making sure that he was leaving everything in its proper place, that there would be nothing to show that anybody had been making an investigation there. Then he hurried out and locked the door, returned the keys to the old janitor, gave him another bill and instructed him to forget the visit, lighted a black cigar, and started walking rapidly southward.

When the proper time arrived, Jim Farland would tell Miss Kate Gilbert that he knew she had written the anonymous notes to Sidney Prale---or that her maid had---and he would ask her why.

He reached Columbus Circle, made his way over to Fifth Avenue, and continued his walk down that broad thoroughfare. Farland had decided to go to the hotel and have a talk with Sidney Prale and Murk. He told himself that he was going to like Murk, the human hulk who suddenly had become of some use in the world.

But he did not get a chance to go to the hotel just then. He came to a busy corner, and stopped to wait for a chance to cross the street congested with traffic. Suddenly, a few feet to his right, he saw Kate Gilbert, who had left her apartment only a short time before.

There was nothing startling in that fact alone, for this was a district where there were fashionable shops and beauty parlors, and well-dressed women were on every side.

What interested Detective Jim Farland the most was that Kate Gilbert was standing before the show window of a fashionable shop in intimate conversation with George Lerton, Sidney Prale's cousin!

\vspace{2\nbs}
\ChapterDeco[c1]{\decoglyph{e9665}}
\clearpage
\thispagestyle{empty}

\begin{ChapterStart}
\vspace{3\nbs}
\ChapterSubtitle[l]{Chapter ch12}
\ChapterTitle[l]{ch12}
\end{ChapterStart}
\FirstLine{\noindent ## Battered Keys}
    
Farland started moving slowly toward them, making his way through the crowd in such fashion that he did not attract too much attention to himself. He was feeling a sudden interest in this case. There were great possibilities in the fact that two persons connected with it from different angles were in conversation.

As he made his way toward the show window, he remembered how this George Lerton had tried to induce Sidney Prale to leave the city and remain away, and how, afterward, he had denied that he had seen Prale on Fifth Avenue and had spoken to him.

"He's connected with this thing in some way," Farland told himself. "It's my job to discover exactly how."

But he was doomed to be disappointed. Before he could get near enough to make an attempt to overhear what they were saying, they suddenly parted. Kate Gilbert went into the shop, and George Lerton crossed the street and hurried down the Avenue.

It was no use wasting time on Kate Gilbert. Farland knew where to find her if he wanted her, and he knew there would be no use in shadowing her now, since she probably had gone into the shop to purchase a hat. But George Lerton was quite another matter.

The detective did not hesitate. He swung off down Fifth Avenue in the wake of George Lerton.

Farland was a rough and ready man, and he had little liking for male humans of the George Lerton type. Lerton always dressed in the acme of fashion, running considerably to fads in clothes, appearing almost effeminate at times. And yet it was said in financial circles that Lerton was far from being effeminate when it came to a business deal. There had been whispers about his dark methods, and it was well known that a business foe got small sympathy or consideration from him. He was a fashionable cut-throat without any of the milk of human kindness in his system.

It was a surprise to Jim Farland to see Lerton walking. He was the sort of man who likes to advertise his success, and he had a couple of imposing motor cars that he generally used. But he was walking this morning, and the fact gave Farland food for thought.

Lerton continued down the Avenue, and Jim Farland followed him closely. He expected to see Lerton meet some one else and engage in another whispered conversation, but Lerton did not.

"That boy is worried," Farland told himself. "He's one of those birds who like to walk when they want to think something out. If I could only know what was going on in that mind of his------"

Lerton had reached Madison Square, and there he did something foreign to his nature. He crossed the Square, proceeded to Fourth Avenue, and descended into the subway.

Farland was a few feet behind him, and got into the same car when Lerton caught a downtown train. He followed when Lerton got off and went up to the street level again, and now the broker made his way through the throngs and along the narrow streets until he finally came to the financial district. After a time he turned into the entrance of an office building---the building where his own offices were located.

The detective watched him go up in the elevator, and then he turned back to the cigar stand in the lobby and purchased more of the black cigars he loved. For a time he stood out at the curb, puffing and thinking. He watched the building entrance closely, but George Lerton did not come down again.

As a matter of fact, Farland scarcely had expected that he would. He believed that Lerton had kept an appointment with Kate Gilbert, and then had continued to his office to take up the work of the day. Farland decided that he would give Lerton a chance to attend to the morning mail and pressing matters of business, before seeking an interview.

Finally, Farland threw the stub of the cigar away, turned into the entrance of the building once more, and walked briskly to the elevator. He shot up to the tenth floor, went down the hall, and entered the reception room of the Lerton offices. An imp of an office boy took in his card.

"Mr. Lerton will see you in ten minutes, sir," the returning boy announced.

Farland touched match to another cigar. He was a little surprised that Lerton had sent out that message. Lerton knew Farland, as Sidney Prale had known him in the old days. He knew Farland's business, and he knew that the detective and Prale were firm friends. He could guess that Prale had engaged Jim Farland to work on this case and clear him of the charge of having murdered Rufus Shepley.

After a time the boy ushered him into the private office. George Lerton was sitting behind a gigantic mahogany desk, looking very much the prosperous man of business.

"Well, Farland, this is a pleasure!" Lerton exclaimed. "Haven't seen you for ages. How's business?"

"It could be better," Jim Farland replied, "and it could be a lot worse. I'm making a good living, and so have no kick coming."

"If I ever need a man in your line, I'll call you in," George Lerton said. "And the pay will be all right, too."

"Don't doubt it," Farland replied.

"Want to see me about something special this morning?"

"Yes, if you can give me a few minutes."

"All the time you like," Lerton replied.

That was not like the man, Jim Farland knew. Lerton was the sort to try to make himself important, the always-busy man who had no time for anybody less than a millionaire.

Farland smiled and sat down in a chair at one end of the desk. He twisted his hat in his hands, looked across at George Lerton, cleared his throat, and spoke.

"You know about Sidney Prale being in a bit of trouble, of course?"

"Yes. Can't understand it," Lerton replied, frowning. "Sidney always had a temper, of course, but I never thought he would resort to murder during a fit of it. You know, I never got along with him any too well. He had a quarrel with his sweetheart in the old days and left for Honduras twenty-four hours later and remained there for ten years."

"I know all about that, of course," Farland said. "You perhaps have guessed that he sent for me---engaged me to get him out of this little scrape."

"Murder, a little scrape?" Lerton gasped. "I should call it a very serious matter."

"Let us hope that it will not be a serious matter for Sid," Farland said with feeling. "I believe that the boy is innocent, and I hope to be able to clear him. Will you help me?"

"I never had any particular love for Sidney, and neither did he for me," George Lerton said. "However, he is my cousin, and I hate to see him in trouble. But how can I help you? I don't know anything about the affair."

"An alibi is an important thing in a case like this," Farland said. "We want to prove an alibi, if we can, of course. Sidney says that you met him on Fifth Avenue------"

"And I cannot understand that," Lerton interrupted. "Why should he say such a thing?"

"You didn't meet him?"

"I certainly did not! I cannot lie about such a thing, even to save my cousin. Why, it would make me a sort of accessory, wouldn't it? I cannot afford to be mixed up in anything of the sort. You must understand that!"

"And you didn't urge him to leave New York and remain away for the rest of his life?"

"I didn't see him at all," George Lerton persisted. "Why on earth should I care whether he remains in New York or takes his million dollars elsewhere?"

"I don't know, I'm sure," Farland said. "But it seems peculiar to me that Sid would tell a rotten falsehood like that. Doesn't it look peculiar to you?"

"I must confess that it does not," George Lerton replied. "I suppose it was the first thing that came into his head. He was trying to establish an alibi, of course, and he probably thought he would get a chance to telephone to me and ask me to stand by the story he had told, thinking that I would do it because of our relationship."

"I was hoping that you would tell me you had met him on Fifth Avenue," Farland said. "It would have made his alibi stronger, of course, and every little bit helps."

"Stronger? You mean to say that he has any sort of an alibi at all?"

"A dandy!" Farland exclaimed. "In fact, we have an alibi that tells us that Sid was quite a distance from Rufus Shepley's suite when Shepley was slain."

"Why, how is that?"

"Sid picked up a bum and tried to make a man of him. He bought the fellow some clothes and took him to a barber shop. The clothing merchant and the barber furnish the alibi."

An expression of consternation was in George Lerton's face, and Jim Farland was quick to notice it.

"Of course, I am glad for Sidney's sake," Lerton said. "But I had really believed that he had killed Shepley. It caused me a bit of trouble, too."

"How do you mean?" Farland asked.

"Shepley was a sort of client of mine," Lerton said. "I handled a deal for him now and then. He has been traveling on business for some time, as you perhaps know. I had hopes that he would give me a certain large commission and that I would make a handsome profit. He was about convinced, I am sure, that I was the man to handle it for him. His small deals with me had always been to his profit and my credit."

"Oh, I understand!"

"And a possible good customer is removed," Lerton went on. "So you have an alibi for Sidney, have you? In that case---if he did not kill Rufus Shepley---he must have told that story about meeting me when he was in a panic immediately following his arrest. Sid always was panicky, you know."

"I didn't know that a panicky man could pick up a million dollars in ten years."

"Oh, I suppose Sidney was fortunate. There are wonderful opportunities at times in Central America, and I suppose he happened to just strike one of them right. He was very fortunate, indeed. Not every man can have good luck like that."

"Well, I'm sorry that I troubled you," Farland said. "And now, I'll get out---if you'll do me a small favor."

"Anything, Farland."

"I see you have a typewriter in the corner, and I'd like to write a short note to leave uptown."

"Just step outside and dictate it to one of my stenographers," said George Lerton.

"That'd be too much trouble," Farland replied. "It's only a few lines, and I can pound a typewriter pretty good. Besides, this is a little confidential report that I would not care to have your stenographer know anything about."

"Oh, I see! Help yourself!"

Farland got up and hurried over to the typewriter. He put a sheet of paper in the machine, wrote a few lines, folded the sheet and put it into his coat pocket.

"Well, I'm much obliged," he said. "I think we'll have Sid out of trouble before long."

"Let us hope so!" George Lerton said.

There was something in the tone of his voice, however, that belied the words he spoke. Farland gave him a single, rapid glance, but the expression of Lerton's face told him nothing. Lerton was a broker and used to big business deals. He was a master of the art of the blank countenance, and Jim Farland knew it well.

Farland had said nothing concerning Kate Gilbert, for he was not ready to let George Lerton know that he suspected any connection of Miss Gilbert with the Rufus Shepley case. Farland was not certain himself what that connection would be, and he knew it would be foolish to say anything that would put Lerton on guard and make the mystery more difficult of solution.

He thanked Lerton once more and departed. Out in the corridor and some distance from the Lerton office, he took from his pocket the note he had written on Lerton's private typewriter and glanced at it quickly. Farland was merely verifying what he had noticed as he had typed the note.

"That was a lucky hunch about that typewriter," he told himself. "This case is going to be interesting, all right---and for several persons."

Farland had noticed particularly the typewritten notes that had been received by the clothing merchant and the barber. There were two certain keys that were battered in a peculiar manner, and another key that was out of alignment.

He knew now, by glancing at the lines he had written himself, that those other notes had been typed on the same machine. He guessed that it had been George Lerton, the broker, who had sent those notes and the money to the barber and the merchant.

Why had George Lerton been so eager to destroy his cousin's alibi?

Why was George Lerton trying to have Sidney Prale sent to the electric chair for murder?

\vspace{2\nbs}
\ChapterDeco[c1]{\decoglyph{e9665}}
\clearpage
\thispagestyle{empty}

\begin{ChapterStart}
\vspace{3\nbs}
\ChapterSubtitle[l]{Chapter ch13}
\ChapterTitle[l]{ch13}
\end{ChapterStart}
\FirstLine{\noindent ## A Plan Of Campaign}
    
Naturally, a man facing prosecution on a murder charge is liable to be nervous, whether he is innocent or not. If an attempt is being made to gather evidence that will clear him, he wishes for frequent reports, always hoping that there will be some ray of hope. And so it was with Sidney Prale this morning, as he paced the floor in the living room of his suite in the hotel.

Murk had done everything possible to make Sidney Prale comfortable. Now he merely stood to one side and watched the man who had saved him from a self-inflicted death, and tried to think of something that he could say or do to make Prale easier in his mind.

They had not seen or heard from Jim Farland since the evening before, when he had engaged the taxicab and had started in pursuit of the limousine Kate Gilbert had entered. Prale wondered what Farland had been doing, whether he had discovered anything concerning Kate Gilbert, whether he had found a clew that would lead to an unraveling of the mystery.

"Are you sure about that Farland man, Mr. Prale?" Murk asked, after a time.

"What do you mean by that, Murk?"

"Well, he's a kind of cop, and I never had much faith in cops," said Murk.

"Farland is an old friend of mine, Murk, and he is on the square---if that is what you mean."

"He sure started out like a house afire, sir, but he seems to be fallin' down now," Murk declared. "He sure did handle that barber and the clothin' merchant, but he ain't showed us any speed since he left us last night."

"He is busy somewhere---you may be sure of that," Sidney Prale declared.

"Well, boss, I ain't got any education, and I ain't an expert in any particular line, but I've often been accused of havin' common sense, and I'm strong for you!"

"Meaning what, Murk?"

"Nothin', boss, except that I'd like to be busy gettin' you out of this mess. Seems to me I know just as much about it as you do, and if we'd talk matters over, maybe I'd get some sort of an idea, or somethin' like that."

Prale sat down before the window, lighted a cigar, and looked up at Murk.

"Go ahead," he said. "It won't hurt anything, and it will serve to kill time until we hear from Jim Farland. What do you want to talk about first?"

"It seems to me," said Murk, clearing his throat and attempting to speak in an impressive manner, "that this is a double-barreled affair."

"What do you mean?" Prale asked.

"Well, there's the murder thing, and then there's this thing about you havin' some powerful and secret enemies that are tryin' to do you dirt without even comin' out in the open about it. Maybe them two things are mixed together, and maybe again they ain't. If they ain't, we've got two jobs on our hands."

"And, if they are?" Prale asked.

"Then it looks to me, boss, like the gang that's after you is tryin' to hang this murder on you after havin' had somebody croak that Shepley guy."

"I've thought of that, Murk. But it doesn't look possible," Prale said. "If my enemies merely wanted to hang a murder charge on me, as you have suggested, I think they would have planned better and would have made the evidence against me more conclusive. It would mean that there would be a lot of persons in the secret; the men who plan murder do not like to take the entire town into confidence about it."

"Well, that sounds reasonable," Murk admitted.

"And why Rufus Shepley?"

"Because you had that spat with him in the lobby of the hotel, and it could be shown that you had a reason for knifin' him," Murk said, with evident satisfaction.

"Nobody could have known I was going to have that quarrel with Shepley, because I had no idea of it myself when I entered the hotel lobby," Prale said. "After I left the hotel, I met Farland and then walked down to the river and met you---and you know the rest. How could they have contemplated hanging that crime on me when they did not know but that I had a perfect alibi? I think we're on the wrong track, Murk."

"Well, boss, how about your fountain pen?" Murk asked. "How come it was found beside the body?"

"That is one of the biggest puzzles in the whole thing, Murk. I cannot remember exactly when I had the pen last. I cannot imagine how it got into Shepley's room and on the floor beside his body. That fountain pen of mine is an important factor in this case, Murk, and it has me worried."

"It seems to me," Murk said, "that if I had any powerful enemies after my scalp, I'd know the birds and be watchin' out for them all the time, to see that they didn't start anything when I was lookin' in the other direction."

"But, Murk, I haven't the slightest idea who they are," Sidney Prale declared. "I don't know why I should have enemies that amount to anything, and that is what makes it so puzzling. How can I work this thing out when I don't even know where to start? I wish Jim Farland would come."

Jim Farland did, at that moment. Murk let him in, and the detective tossed his hat on a chair, sat down in another, lighted one of his own black cigars, and looked at Sidney Prale through narrowed eyes.

"Well, Jim?" Prale asked.

"I talk when I've really got something to say, but I'm not going to make general conversation and muddle your brains with a lot of scattered junk," Jim Farland replied. "I'll say this much---things are looking much better for you."

"That sounds good, Jim. Can't you tell me anything?" Prale asked, sitting forward on his chair.

"The barber and the clothing merchant have fixed up a part of your alibi, Sid, as perhaps Murk has told you. That is the first point. It makes it look impossible for you to have slain Rufus Shepley, and I think Lawyer Coadley could get the charge against you dismissed on that alone."

"But I want to be entirely cleared."

"Exactly. You don't want to leave the slightest doubt in the mind of a single person. There is but one way to clear you absolutely, Sid. We've got to show conclusively that you could not have killed Shepley, and the best way to do that is to find the person who did."

"I understand, Jim."

"There seems to be some sort of a mysterious alliance against you, Sid. You say that you can't understand why you should have enemies that hate you so, and I know you're telling the truth. Whether that business has anything to do with the murder, or not, I am not prepared to say now. But we want to find out about this enemy business, too, don't we?"

"Certainly," Prale said.

"I followed Kate Gilbert. I know where she lives. She does not belong to a rich family and does not live in splendor. But she wears expensive gowns and has plenty of spending money, and has mysterious dealings with a distinguished-looking man. Her father is mixed up in it in some way, too. I went through their apartment, Sid. Somebody in that apartment wrote the anonymous notes you received."

"What?" Prale gasped.

"I found a tablet of the same sort of paper, and scraps of writing in the wastebasket that were in the same hand. Think, Sid! On the ship------"

"By George!" Prale exclaimed. "She could have slipped into my stateroom and pinned that note to my pillow, and she could have stuck the second one on my suit case as I walked past her on the deck."

"And could have sent the others," Farland added.

"But, why?" Prale demanded. "I never saw the woman until I met her at a social affair in Honduras. What could she or any of her people have against me?"

"Perhaps it was the maid," Farland said.

"She could have done it, of course, the same as Kate Gilbert," Prale said. "But the same difficulty holds good---why? Kate Gilbert did seem to avoid me, and I caught her big maid glaring at me once or twice as if she hated the sight of me. But why on earth------"

Farland cleared his throat. "Here is another thought for you to digest," he said. "This Kate Gilbert knows your cousin, George Lerton."

Sidney Prale suddenly sat up straight in his chair again, his eyes blinking rapidly.

"Doesn't that open up possibilities?" Jim Farland asked him. "The woman seems to be working against you for some reason, and we know that George Lerton lied about meeting you on Fifth Avenue that night. It appears that he is working against you, too, for some mysterious motive."

A dangerous gleam came into Sidney Prale's eyes. "That simplifies matters," he said. "I'll watch for Kate Gilbert, and when I see her I'll ask why she sent me those notes. Then I'll get George Lerton alone and choke out of him why he lied about meeting me on the Avenue. I've trimmed worse men than George Lerton."

"You'll be a good little boy and do nothing of the sort," Farland told him. "We are playing a double game, remember---trying to solve this enemy business, and at the same time trying to clear you of a murder charge. If any of those persons get the idea that we are unduly interested in them, we may not have such an easy time of it."

"I understand that, of course."

"Let me tell you a few more things, Sid. I saw Lerton talking to Miss Gilbert on the street. They were speaking in very low tones. When they parted, I followed Lerton to his office, and went in and talked to him. I did it just to size him up. He still declares that he never met you on Fifth Avenue. He acts like a man afraid of something; and I discovered an interesting thing, Sid. He has a typewriter in his private office, one for his personal use. I managed to type a short note on it."

"What of that?"

"That typewriter has a few bad keys, Sid. And I discovered this---that the notes sent to the barber and merchant, that caused them to lie and try to smash your alibi, were written on the typewriter in George Lerton's office!"

Prale sprang to his feet. "Then Lerton has something to do with this!" he cried. "He tried to get me to leave town, and he tried to break down my alibi. How did he know I was going to make an alibi like that?"

"My guess is that your cousin has been having you watched since you got off the ship."

"But, why?" Prale cried. "It is true that he married the girl who had jilted me a few years before, but I do not hold that against him. I know of no reason why he should work against me so."

"Know anything about him that might cause him serious trouble if you talked?"

"No," Prale replied. "As much as I dislike him, as much as I suspect that he is crooked in business, all that I really could say would be that he had a mean disposition and was not to be trusted too far."

"I thought maybe you had something on him, and he was trying to get you out of the way so you'd not talk," Farland said. "That would explain a lot, of course."

"It can't be that."

"Then we are up in the air again."

"Why not ask him?" Prale demanded. "Believe me, I'll wait for him to come from his office---and he'll answer me, and tell the truth!"

"Put that hot head of yours under the nearest cold-water faucet!" Farland commanded. "You make a move that I don't sanction, and I'll quit the case! You'll spoil things, Sid, if you're not careful. Just digest what I have told you."

"You're in command, Jim!"

"Very well. You leave George Lerton to me, Sid. There are many angles to this case, and I can't attend to all of them at once. I don't want to call in other detectives, because they may be in the pay of these mysterious enemies of yours, and I haven't an assistant with an ounce of brains. Sid, you've got to turn detective yourself---you and Murk."

"I was just wonderin' if I was goin' to get a chance to do anything," Murk said.

"Plenty of chances," Farland replied. "Sid, you pick up this Kate Gilbert, if you can. Act as if you did not suspect a thing. Try to talk to her---you were introduced to her in Honduras, and all that. Don't let her get nervous about you, but watch her as much as you can, and let me know everything you see and hear. Take a look at that big maid, Marie, when you get a chance. If you can do so, and think it advisable, put Murk on Marie's trail. I'll want to use Murk later myself."

Sidney Prale was quick to agree. And thus, without being aware of it, he started on a short career of adventure and romance.

Had Murk been a crystal gazer or something of the sort, and could he have looked into the future in that manner, he would have said that the crystal lied.

\vspace{2\nbs}
\ChapterDeco[c1]{\decoglyph{e9665}}
\clearpage
\thispagestyle{empty}

\begin{ChapterStart}
\vspace{3\nbs}
\ChapterSubtitle[l]{Chapter ch14}
\ChapterTitle[l]{ch14}
\end{ChapterStart}
\FirstLine{\noindent ## More Mystery}
    
Jim Farland went from the hotel to Coadley's office, to ascertain whether the attorney's private investigators, who were working independently of him, had unearthed anything of importance in connection with the case.

Sidney Prale stated that he would go for a walk, and the police detective, now thoroughly convinced that he would not try to run away, raised no objection. It was Prale's intention to make an attempt to meet Kate Gilbert. Murk hurried around getting his coat and hat and gloves and stick.

"Fool idea!" Prale told himself. "Kate Gilbert has given me the cold shoulder already, and she certainly will do it now, since I stand accused of murder. Not a chance in the world of getting better acquainted with her now."

"What do you want me to do, boss?" Murk asked. "I don't seem to be amountin' to much in this game. I'd like to be in action, I would! Can't I take a hand?"

"As soon as possible," Prale told him. "Remember, Farland said he wanted you to help him later."

"I'd rather help you or work alone," Murk said. "I reckon he is pretty decent for a detective, but I don't put much stock in any of 'em."

Prale laughed as he finished dressing, put on his hat and gloves, and reached for his stick.

"Suppose you just shadow me this fine day," he told Murk. "Get a little practice in that line. Don't bother me, but just follow and watch."

"I getcha, boss. You want me to be within hailin' distance in case you need help?"

"Exactly, Murk. We never can tell what is going to happen, you know. I may need you in a hurry."

"I'll be on hand," Murk promised.

Sidney Prale went down in the elevator, Murk going down in the same car. Prale lounged about the lobby for a time, and Murk made himself as inconspicuous as possible in a corner. Prale believed, as Farland had intimated, that he was being followed and watched, possibly by the orders of George Lerton, his cousin. He did not know why Lerton should have done it, but it angered him, and he wanted to discover the man following him.

He saw nobody in the lobby who appeared at all conspicuous, and after a short time he left and started walking briskly down the Avenue, like any gentleman taking a constitutional. The midday throngs were on the streets. Prale was forced to walk slower, and now and then he stopped to look in at a shop window. Once in a while he stepped to the curb and glanced behind. But if there was a "shadow" Prale did not see him.

He did see Murk, however, and he smiled at Murk's methods. Murk remained a short distance behind him, moving up closer whenever Prale was forced to cross the street, so he would not lose him in the throng. Murk was ordinary-looking and had a happy faculty of effacing himself in a crowd. He was on the job every minute, watching Sidney Prale, glancing at every man or woman who approached Prale or as much as looked at him.

Prale reached 40-second Street, crossed it, and came opposite the library. He glanced aside---and saw Miss Kate Gilbert walking down the wide steps.

It was a ticklish moment for Sidney Prale, but he remembered that he was fighting to protect himself. If Kate Gilbert ignored him, he could not help it. At least, he would give her the chance.

She could not avoid seeing him, for they met face to face at the bottom of the steps. Prale lifted his hat.

"Good morning, Miss Gilbert," he said.

She turned and met his eyes squarely, and he could see that she hesitated for a moment. Then her face brightened, and she stepped toward him.

"Good morning," she replied. "Although it is a little after noon, I am afraid."

Her words might have been for the benefit of any who heard. They were light enough and cordial enough, but she did not offer him her hand, and the expression on her face was scarcely one of welcome.

"I am glad to see you again," Prale said.

"You are settled and feeling at home?"

"In a measure," he said.

She had not mentioned the crime of which he was accused, and he did not wish to be the first to speak of it. She stepped still closer.

"I want to talk to you, Mr. Prale," she said. "Kindly get a taxi and have the chauffeur drive us through the Park."

Prale scarcely could believe his good fortune. He had doubted whether he would have a chance to talk to her, and here she was asking him to engage a taxicab so that they could enjoy a conversation.

He hailed a passing taxi, put her in, gave the chauffeur his directions, and sprang in himself. The machine turned at the first corner and started back up the Avenue in the heavy traffic.

"You wished to speak to me about something in particular?" Prale asked.

"Yes. I have read of the crime of which you are accused. I am sure that you are not guilty."

"Thank you, Miss Gilbert. I assure you that I am not. It is an unfortunate affair, which we hope to have cleared up within a short time."

"I hope that you will be free soon," she said. "And then you will be able to enjoy yourself, I suppose."

"I hope to have my vacation yet," Prale said.

"You are going to remain in New York?"

"Certainly; it is my home."

"Sometimes a man does better away from home."

"But I have been away from home for ten years. I have made my pile, as the saying is, and have come home to show off and lord it over my neighbors," Prale replied, laughing.

They had reached the lower end of Central Park now, and the taxi turned into a driveway, and made its way around the curves toward the upper end. The chauffeur was busy nodding to others of his craft and paying no attention to his fares. Sweethearts, he supposed, talking silly nothings as they were driven through the Park. The chauffeur was used to such; he hauled many of them.

Kate Gilbert leaned a bit closer to Prale, and when she spoke it was in a low, tense voice.

"Go away from New York, Mr. Prale!"

"Why should I do that?" he asked.

"It would be better for you, I feel sure."

"Because of the absurd charge against me? I intend to have my innocence proved, and I'd hate to run away and let people think that perhaps I was guilty after all."

"You have the right to prove your innocence of such a charge to all the world," she said. "But, after you have done it conclusively, you should go away."

"Why?" he asked, again.

"Because---you have enemies, Mr. Prale!"

"I have discovered that; but I do not know why I should have enemies."

"Perhaps you did something, some time, to create them."

"But I haven't," Prale declared.

"Retribution comes when we least expect it, Mr. Prale."

"Yes. I believe that you wrote that in one of your notes."

He had said it! And Jim Farland had told him not to let her suspect that they knew. Well, he couldn't help it now.

Kate Gilbert gasped and sat back from him.

"In my note?" she said.

"The notes interested me greatly, Miss Gilbert. I have saved them. But why should you send them to me?"

"You can ask me that!" she exclaimed. "So you know that I wrote them, do you? In that case, Mr. Prale, you know why I spoke of retribution, you probably know my identity and intentions, and you know why you have enemies!"

"But I do not!" he protested.

"Please do not attempt to tell a falsehood, Mr. Prale. You know I wrote the notes, do you? Then you know everything else. So you are going to fight."

"I fail to understand all this."

"Another falsehood!" she cried. "I have asked you to leave New York and------"

"And I fail to see why I should."

"Then remain---and receive the retribution!" she said. "You will deserve all you get, Sidney Prale! When I think of what you have done------"

She ceased speaking, and turned to glance through the window.

"You were kind enough to say that you believed me innocent of the murder charge------"

"I do. I hate to have you facing a thing like that when you are innocent. But this other thing is------"

"Can't you explain? I give you my word of honor that I do not understand this."

"Your word of honor!" she sneered, facing him again. "You speak of honor---you? That is the best jest of all!"

Sidney Prale's face flushed.

"I had hoped that I was a man of honor," he said. "I always have tried to be honorable in my dealings with men and women, all my life. Please understand that, Miss Gilbert."

"If you have tried, you have failed miserably. Why do you persist in telling falsehoods, Mr. Prale. Do you think that I am a weak, silly woman ready to be hoodwinked by lies?"

"But I assure you------"

"I do not care for any of your assurances," she interrupted. "I wish it understood that we are strangers hereafter. You are going to fight, are you? Fight, Sidney Prale---and lose! What I said was correct---you cannot dodge retribution. It will take more than a million dollars to be able to do that."

"My dear young lady------"

"I am done, Mr. Prale. I have said all that I intend saying to you."

"Then it is my turn to talk!" Prale said. "This thing is getting to be so serious that I demand an explanation. Why should you, and others, be so eager to run me out of New York?"

"Others?"

"Yes---particularly one man we both know."

"His name, please?"

"Why ask, Miss Gilbert?"

"Very well."

"Why do you want me to run away?"

"I did not know that others were trying to get you to leave," she said. "I suggested it because---well, because I am a woman, I suppose. You deserve the worst that can happen to you. But a woman, has a kind thought now and then. I hate to see any man ground down and down, no matter how much he deserves it---and that is what is to happen to you if you do not go away. If you leave, your enemies will not use such harsh measures, perhaps. But when you are here before their very eyes, they will lift their hands against you!"

"Who are these enemies, and why are they after my scalp?"

"You know, Sidney Prale, as well as I. I can see that it is useless to talk to you. I am sorry that I had a moment's compassion and made the attempt. Please stop the cab and let me out here."

"But I demand to know------"

"Do as I say, or I shall make a scene!"

Prale gave the signal, and the taxi stopped. He helped her out, and she started briskly down the nearest path. Sidney Prale paid the chauffeur, and started to follow.

He glanced back, and saw Murk getting out of another taxicab. He had forgotten Murk in his interest in the conversation with Kate Gilbert. But Murk had not forgotten. Murk had his orders, and he was carrying them out; he was keeping in sight, to be on hand if he was needed.

Murk had a little money Prale had given him, enough to pay the taxi chauffeur. Prale motioned for him to approach.

"Here's a roll of bills," he said. "Keep up the game, Murk. Don't get too far away."

"I'll be right at your heels, boss."

"And keep your eyes open."

"Yes, sir."

"That woman was Kate Gilbert."

"Then I'll know her whenever I see her again, sir."

Prale hurried on down the path. Murk kept pace with him, a short distance behind.

Kate Gilbert had been walking swiftly. She had reached the street, and, as Prale watched, she crossed it. Prale followed.

The girl did not look behind. She came to the middle of the block and ran up the steps of an apartment house. Prale passed the entrance, glanced at the number, and continued down the street. At the corner he allowed Murk to catch up with him.

"She turned in at the address Jim Farland gave us," Prale said. "She has gone home, Murk. I fancy that we are done with her for to-day!"

A lot he knew about it!

\vspace{2\nbs}
\ChapterDeco[c1]{\decoglyph{e9665}}
\clearpage
\thispagestyle{empty}

\begin{ChapterStart}
\vspace{3\nbs}
\ChapterSubtitle[l]{Chapter ch15}
\ChapterTitle[l]{ch15}
\end{ChapterStart}
\FirstLine{\noindent ## A Moment Of Violence}
    
Sidney Prale turned around and walked back along the street to the Park, Murk once more following at a short distance, as he had been ordered to do.

Because he wanted to think of his predicament, Prale crossed into the Park and began following one of the paths toward the south, making his way along it slowly, paying little attention to the persons he passed now and then.

He crossed a drive and followed another path; and now he came to a secluded spot where the path was hidden from passers-by on the other walks and drives. Here the way ran through a tiny gulch, the sides of which were banked with bushes. Squirrels scampered and birds chattered at him, but Prale saw none of them.

He was trying to explain to himself why Kate Gilbert had warned him to leave New York, why she had interested herself in his affairs at all, asking himself for the thousandth time what species of net it was in which he suddenly had found himself enmeshed without knowing the reason for it.

He had demanded information and it had not been given him. She had said nothing at all that gave him an inkling as to the nature of what seemed to be a plot against him. He had been as firm as he dared, he told himself. A man could not threaten a woman, could not use violence in an attempt to make her speak and reveal secrets.

"We'll have to work from another corner," Sidney Prale told himself. "I can't threaten a woman, but I can pummel a man; and if I meet George Lerton again, I am liable to forget what Jim Farland told me and use my own methods."

He walked on through the tiny ravine. He came to a cross path, and a man lurched down it and against him.

"Beg pardon!" Prale murmured.

"Wonder you wouldn't look where you're going!" the other exclaimed. "Got an idea you own the whole Park, or something like that? Men like you shouldn't be running around loose!"

"You ran into me, not I into you," Prale reminded him.

As he spoke, he looked at the other closely. He saw a gigantic man who had the general appearance of a thug, whose chin was thrust forward aggressively, and whose hands were opening and closing as if he wished they were around Sidney Prale's throat.

"I've a notion to smash you one!" the fellow said, advancing toward Prale a bit.

Prale's temper flamed at once. His own chin was shot forward, and his own hands closed.

"If that is the way you feel about it, start in!" Prale said. "Perhaps I can teach you to act decently and keep a civil tongue in your head!"

The man before him made no comment---he simply launched himself forward like a thunderbolt. Sidney Prale darted quickly to one side, and tossed his hat and stick on the ground. He did not have time to get off his coat; he could not even remove his gloves.

The other, missing him in that first rush, turned and came back, swinging his fists. Prale did not dart aside now. He put himself on guard, braced himself against the side of the little gulch, and waited for the attack.

They clashed, and Prale knew that he had a real fight on his hands, for the man who had attacked him was no mean antagonist. But, after the first real clash, Prale had no fear of the outcome. The man was brutal, but he had no skill. He delivered blows that would have felled any one---but they did not reach their objective.

Then a second man crashed down through the brush and joined in the attack. Sidney Prale realized in that moment that the attack had been premeditated and the fight forced upon him purposely. It fed fuel to the flames of his wrath. He did not know whether this was the work of some of his unknown enemies or whether these thugs were mere robbers intent upon getting his wallet and watch. It made little difference to him which they were.

With his back against the side of the gulch, he fought with what skill he could, trying to stand off both of them. The attack had come with a rush, and all this had occupied but a few seconds.

Presently a human whirlwind appeared and took part in the battle. There was an angry roar from a human throat, a raucous curse, a rushing body, the thuds of swift, hard blows. Mr. Murk had reached the scene!

The battle immediately became two-fold. Murk fought as these thugs fought, disregarding the finer rules of combat, seeking only to put his opponent out, no matter by what means. Murk was not unaccustomed to fighting of that character, and he was doubly formidable now, for he was angry at the attack on Sidney Prale. Murk had been too far away to hear what had been said when the trouble started, but he had seen, and he guessed immediately that some of Sidney Prale's enemies were engaged in the attempt.

Murk went after his opponent with determination if not with skill. He fought him down the path, and there the fellow rallied from the surprise and rushed back. But Murk was not the sort to give ground. In a fight, a man should stand up to another until one of them was whipped, Murk thought.

He knew how to give blows, but not how to guard against them. He was marked, and marked well, before the battle was a minute old, but he had the satisfaction of seeing blood on the face of his antagonist. Foot to foot they stood and hammered each other, and gradually Murk began wearing the other man down.

As for Sidney Prale, now that he had but the one thug against him, he fought with skill and cunning, knowing that the other was a bit the stronger, but realizing that he would be victor if he used reasonable care.

His flare of anger had passed, and now he was fighting like a clever pugilist. He warded off the other's powerful blows, and now and then he slipped beneath a guard, or smashed his way through one, and sent home a blow of his own.

At the end of three minutes, the thugs were getting much the worst of it. Gradually they were being fought back toward the nearest driveway. Back and back they went, but did not turn and run. Sidney Prale sensed that they were fighting for money, that they were being paid for this attack, and he realized that, but for the presence of Murk, he would have had no chance whatever, and probably would be a senseless, bleeding thing now.

None of them knew that the fight had attracted attention, but it had. 2 women, coming around a curve in the path, had seen it, and had run back toward the nearest driveway, screeching. 2 mounted policemen hurried toward them, heard the story, and charged down the path.

The two thugs made no effort to escape. They stopped fighting, and Prale and Murk ceased also, though the latter was eager to continue until a decision had been rendered. Murk had fought often where there was no interference and he disliked to be bothered now, but he desisted at Prale's command.

"Well, what's all this about?" one of the officers demanded. He did not address any of them particularly. "I was walking along the path, and these men attacked me," Sidney Prale said. "My valet was a short distance behind and he came to my assistance. I never saw these fellows before."

"Nothin' like it!" one of the thugs snarled. "Me and my pal were walkin' along this path and met these men, and the one with the stick ordered us out of the way as if we were dogs. When we didn't move quick enough, they jumped into us."

"That's a lie------" Murk began.

"You can settle this at the station," the officer replied. "All of you come along with us!"

Prale picked up his hat and stick, took off his torn gloves and threw them away, and motioned for Murk to walk at his side and to keep quiet. They went to the driveway and along it, the policemen watching the four of them closely, the thugs growling to each other and remarking that it was a fine day when honest workingmen could not stroll in Central Park without a dude and his valet trying to beat them up.

There was a short wait when the station was reached, and then, at the lieutenant's command, one of the thugs poured forth his story. He gave his name and address, as did the other, and both made the statement that they were out of work at present.

Prale stepped forward and gave his name. The lieutenant stared at him in surprise.

"Why, it's the guy who croaked that man Shepley!" one of the thugs cried. "There ought to be a way of stoppin' him runnin' around and assaultin' and killin' folks. If it hadn't been for the cops------"

"Shut up!" Sidney Prale commanded loudly, ignoring the presence of the officers. "You fellows made a deliberate attack on me and you know it. And I want to know who paid you to do it---understand?"

"You're crazy!" said one of the thugs.

Prale turned to the lieutenant. "I'd like to have Jim Farland sent for," he said. "He has been handling things for me. I want him to investigate these men. I have an idea that the names and addresses they gave are fictitious. Recently enemies of mine have caused me considerable trouble, and I feel sure that these men were hired to attack me. Fortunately, my valet was walking a short distance behind me, and rushed up and helped me hold them off."

"I'm ready to put up bail, and so is my pal!" said one of the thugs angrily.

"In that case, I'll have to let you go for the present," the lieutenant said. "The charge is fighting and disorderly conduct, and bail will be one hundred dollars in each case. You may use the telephone if you wish, Mr. Prale."

Prale hurried to the telephone, called Jim Farland's office, and was informed that Farland had not been there, and that the girl in charge did not know where he was, or what he was doing, or when he would return. Prale left instructions for Farland and went back to the desk.

"This is a serious business, though it may not look like it on the face," he said. "I'd like to have these men held until we can make sure they have given correct names and addresses."

"No use holding them if they have given bail," the lieutenant replied. "I think it's nothing but a regular scrap. You can talk to the judge later, all of you."

Prale took a roll of bills from his pocket and put up cash bail for both Murk and himself. 1 of the thugs followed suit and pulling out a roll of bills, stripped off two hundred dollars, and arranged for the release of himself and his partner.

"You seem to have a lot of money for men who are out of work," Prale said.

"Been savin' it, and it's none of your business anyway," growled the other.

They started toward the door, and Prale and Murk followed them, watched them until they started away, and then turned back to bathe their faces and hands. Then Prale got a taxicab, and drove to the office of a physician, who did his best to make the countenances of Prale and Murk presentable.

It was an hour later when Jim Farland called Prale by telephone at the hotel.

"I've investigated that little matter, Sid," he reported. "Those fellows gave fictitious addresses, as you supposed they had done, and it is an even bet that the names they gave were fictitious, too. No doubt about it, Sid---they were hired to get you. You'd better be on guard and a bit careful."

\vspace{2\nbs}
\ChapterDeco[c1]{\decoglyph{e9665}}
\clearpage
\thispagestyle{empty}

\begin{ChapterStart}
\vspace{3\nbs}
\ChapterSubtitle[l]{Chapter ch16}
\ChapterTitle[l]{ch16}
\end{ChapterStart}
\FirstLine{\noindent ## Murk Receives A Blow}
    
An hour before dinner, Detective Jim Farland suddenly appeared in Sidney Prale's suite at the hotel.

"They are working on me now, Sid," he said. "I got a telephone message when I was in the office, and the gent at the other end of the line informed me that it would be beneficial to my health if I immediately ceased having anything to do with the Rufus Shepley murder case and stopped working for you."

"Any idea where the message came from?" Prale asked.

"It came from a public pay station in the subway. I had the call traced immediately, of course. No chance of finding out who sent it, naturally. I doubt whether I'd recognize the voice if I heard it again---could tell by the way the fellow talked that he was trying to disguise his tones. I told him to go to blazes, and he informed me that I was up against something too big for a man to face, or something like that."

"Jim, if there is any danger, I don't want you to work for me," Sidney Prale said. "You're married and a father and------"

"And that will be about all from you, Sid!" Farland interrupted. "Think I'm going to let some man who doesn't tell me his name throw a scare into me?"

"But, if there is danger------"

"I thrive on danger," said Jim Farland. "Think I'm going to desert you at this stage of the game? That is what they want, of course. If I did, you'd probably hire another detective, and it might be one of their own men---whoever they are. I'm in this game to stay, Sid, first because you are an old friend of mine and I think you are being made the victim of some sort of a dirty deal, and also because I'm not the kind of man to be bluffed out of a job. We are going right ahead. I got a note at the office, too."

"A note!" Prale gasped.

"Typewritten, but not on George Lerton's battered typewriter this time. It remarked that unless I gave up this case, somebody would make things hard for me, or words to that effect. Old stuff! If they are so scared that they send threatening letters, they're whipped right now---and they know it!"

"I had an interesting experience this afternoon," said Prale.

"The fight?"

"I don't mean that. I met Kate Gilbert in front of the library. She asked me to get a taxicab and drive her through the Park. I did it. She begged me to leave New York and remain away, and said that my enemies might not be so harsh if I did. I tried to get her to explain, and she insisted that I knew all there was to know. She left the taxicab and walked to her home."

"I'll have to investigate that girl more thoroughly," Farland said.

"She is on guard now, as far as I am concerned."

"Does she know Murk by sight?"

"I think not."

"Then here is where Murk gets a steady job for a time," Jim Farland declared. "Murk, you go up to Kate Gilbert's home and watch a bit. Give him plenty of money, Sid, for expenses. Just see if she leaves the place, Murk, and if so, where she goes, and to whom she talks. Get any general information you can. Try to keep her from knowing that you are watching her, but if she finds it out drop the chase and get back here, and we'll put another shadow on the job. When you are sure that she has decided to remain in her apartment for the night, report back here to Mr. Prale."

"You watch me," Murk said. "I never expected to be caught doin' detective work and I reckon it's somethin' like a disgrace, but this is a sort of special occasion."

Prale gave Murk more money, in case he would have to engage taxicabs or follow Kate Gilbert where money would be necessary for tips and bribes.

"Your face looks pretty good, but you want to remember that there are some marks on it," Prale told him.

"It's looked worse, boss," Murk replied, grinning. "I'll try to do this thing right."

Murk hurried down in the elevator and went from the hotel. He got a cab immediately, and promised that dire things would happen to the chauffeur if he did not get to a certain corner up beside the Park in record time. Jim Farland had given him a badge to be used if he was questioned by a police officer, and he was to say that he was an operative attached to Farland's office.

Murk discharged the taxi at the proper corner, touched match to cigarette, and walked slowly down the street toward the apartment house where Kate Gilbert lived with her father and her maid.

Jim Farland had told him the location of the Gilbert apartment, and Murk saw that the lights in it were burning. It was about time for dinner, he knew.

He went to a drug store on the nearest corner and hurried into a telephone booth. He called the apartment house and asked to be connected with the Gilberts. A woman's hoarse voice answered his call, and he guessed that it was the maid speaking.

"Miss Kate Gilbert there?" Murk asked.

"Who is calling, please?"

"Tell her it is about that Prale affair," Murk replied.

"1 moment. I'll call her."

Kate Gilbert's voice came to him over the wire almost immediately.

"Miss Gilbert?" Murk asked. "I was to tell you that------"

And then Murk jerked down the receiver hook, and grinned as he put the receiver on it. Kate Gilbert would believe that a careless central girl had cut them off and put an end to the conversation.

He had learned what he had wished to learn---that Kate Gilbert was at home. He walked back up the street. All he had to do now was to watch, and if Kate Gilbert left the place follow her. If she did not, Murk would wait half an hour or so after the lights in the apartment were turned out, to be sure that she had retired, and then would hurry back to the hotel.

Murk watched from a distance at first, and then went slowly forward, for he did not wish to attract attention by remaining in one position too long. There were few persons on the block; and now and then some automobile or taxicab would discharge a passenger and go on. Murk made his way slowly to the end of the block, always watching the entrance of the apartment house, crossed the street, and started back on the other side.

He came in front of a dark passageway between two buildings, and went on. And out of the mouth of that dark passageway came a blow that caused Murk to groan once and topple forward. Hands gripped his unconscious body and drew him back into the darkness.

\vspace{2\nbs}
\ChapterDeco[c1]{\decoglyph{e9665}}
\clearpage
\thispagestyle{empty}

\begin{ChapterStart}
\vspace{3\nbs}
\ChapterSubtitle[l]{Chapter ch17}
\ChapterTitle[l]{ch17}
\end{ChapterStart}
\FirstLine{\noindent ## Murk Is Tempted}
    
The next thing that impressed itself upon Murk's consciousness was the fact that he had a terrific pain in the back of his head. Many times during his career Murk had experienced similar pains. And he knew that the best thing to do was to remain quiet for a short time, keep his eyes closed, and gradually pull himself together.

So he pretended that he had not regained consciousness. He knew that he had been stretched upon a bed or couch of some sort, and that his wrists were lashed together, and his ankles. He was not gagged, however.

Gradually the pain ceased, Murk's senses cleared and he became aware of what was going on around him. He could hear whispered voices, but could not distinguish words and sentences; neither could he tell whether the voices were those of men or women.

Finally Murk opened his eyes.

He found that he was in a small room furnished in quite an ordinary manner. He was stretched on an old-fashioned sofa. There were a few chairs scattered about, and a cupboard in one corner. In the middle of the room was an ordinary table covered with a red cloth. Upon the table a kerosene lamp was burning.

Murk groaned and made an attempt to sit up, but fell back again because of a fit of dizziness. It became evident that his groan had been heard in the room adjoining, for the door, which had been ajar, now was thrown open wide, and two men entered.

Murk knew them instantly; they were the men who had attacked Sidney Prale in the Park.

"Back to earth, are you?" one of them snarled. "If I had my way, you'd have been cracked on the head for good."

Murk snarled in reply, despite the fact that he was bound and at the mercy of these men.

"Sore because I smashed your face!" Murk said.

"That'll be about all out of you! I may take a smash at you yet!"

"You've got a good chance while my hands and feet are tied," Murk replied. "It's the only time you could get away with it, all right! Turn me loose and I can clean up the two of you!"

"You're not doin' any cleanin' for the present," he was told.

Murk began wondering at the object of the assault upon him. He could feel the roll of bills Prale had given him bulging his vest pocket, so he guessed robbery was not the motive. He managed to sit up on the sofa now, and he glared at the two thugs before him with right good will.

1 of the men went back into the adjoining room, and the other remained standing before Murk, sneering at him, his hands opening and closing as if he would take Murk's throat in them and choke the life out of Sidney Prale's valet and comrade in arms.

Then the man who had left the room returned, and there was another with him. Murk looked at this stranger with sudden interest. He was well dressed, Murk could see, but he wore an ulster that had the wide collar turned up around his neck, and he had a mask on his face---a home-made mask that was nothing more than a handkerchief with eye slits cut in it.

"Afraid to show yourself, are you?" Murk sneered. "Who are you---the chief thug?"

The masked man pulled a chair up before the sofa and sat down. His eyes glittered at Murk through the slits in the handkerchief.

"You are not going to be harmed, my man---if you are reasonable," he said.

"Reasonable about what?" Murk demanded.

"We want some information and we think you can give it to us; that is all."

"I don't know much," said Murk.

"Tell us why you were prowling around that house near the Park."

"Maybe I was takin' a walk," Murk answered.

"And maybe you were spying, as I happen to know you were. We assume that Sidney Prale sent you to watch the comings and goings of a certain young woman and her friends."

"Go right ahead assumin'."

"It will avail you nothing, my man, to adopt this attitude," Murk was told. "And it might help you a great deal if you are willing to listen to reason."

"I'm listenin'," Murk replied.

"You haven't been working for Sidney Prale very long, have you?"

"Only a few days---since you seem to know all about it, anyway. Why ask foolish questions?"

"Very well. We understand that Prale kept you from committing suicide and then gave you a job. There is no reason why you should feel an overwhelming gratitude for Prale. He merely got a valet cheap."

"What about it?" Murk growled.

"Sidney Prale has a million dollars, but you'll never see much of it. He isn't the sort of man to toss his money away. And there are others, not particularly Prale's friends, who have many millions between them."

"Well, that ain't doin' me much good."

"But it may do you a lot of good. We want information and we stand ready to pay for it."

"I guess you'll have to do a little explainin'," Murk told him. "I never was any good at guessin' riddles. Life's too short to be spent workin' out silly puzzles."

"Very well," the masked man said. "As you perhaps are aware, Prale has certain enemies. That is enough for you to know, if he has not told you more. If you can give me information concerning Sidney Prale's plans, and tell us how much he knows, we will pay you handsomely."

"I getcha," Murk said.

"And if you can manage to continue working for Prale, and let us know everything as it comes up, there'll be considerably more in it for you."

"Want me to do the spy act, do you?"

"Call it whatever you like. There is a chance for you to earn some good money."

"How much?" Murk demanded.

"That depends upon the services you render us. But let me assure you that you will be richly rewarded. We will not fool you or defraud you."

"What do you want to know?"

"What is Jim Farland, the detective, doing? What has he reported to Prale?"

"He ain't reported much of anything," said Murk.

"We want to know what Prale thinks about the situation. Tell us all you know concerning the Rufus Shepley murder case. Has Sidney Prale said anything you have been able to hear about the enemies who are bothering him? You understand what we want to know---everything possible about Prale's plans. And we want you to watch henceforth, and keep us informed in a way I shall explain to you."

"Well, explain it!" said Murk.

"Scarcely, until we know that you are our man. Try to think of things now, and tell us. Be sure you let us have everything. What you deem unimportant may be really important to us."

"I'd feel a lot more friendly to you gents if you'd untie me," said Murk. "I can't talk business when I'm treated like a prisoner, or somethin' like that."

"You'll be untied as soon as we feel sure of you, and not before," Murk was told. "We are not taking chances with you. Are you going to work for us?"

"I'm not sure that the proposition looks good to me," Murk said. "I make a deal with a man whose face I can't see, and do the dirty work---and then maybe you turn me down cold and don't give me a cent, and I lose my job with Mr. Prale and get in a nice fix. Don't you suppose I got some common sense?"

"Make the deal with us, and you shall have five hundred dollars in cash before you leave this room," the masked man promised. "And, take my word for it, you'll be rewarded richly if you serve us well."

"Well, I don't know much about this business," Murk said. "You know I ain't been with Mr. Prale very long. All I know is that he's got some enemies who are tryin' to get the best of him. He says he ain't guilty of that murder charge, and I happen to know he ain't, because he was with me when Shepley was killed."

"Maybe you both had a hand in the killing," the masked man said. "And if you don't come to terms with us, you may find yourself in jail charged with being an accessory."

"You can't bluff me, and you can't threaten me and get away with it!" Murk cried.

"Softly---softly!" said the masked man. "I was merely showing you where you stand."

"Well, don't start talkin' to me that way, if you want to do business with me. If I'm goin' to work for you, I've got to know what's what. Who's got it in for Mr. Prale, and why? That's what I want to know. And what is it you're tryin' to do to him? How can I help if I ain't wise?"

"Some of the wealthiest and most influential men in the city are against Sidney Prale. They are determined to run him away from this, his old home town. They are going to strip him of his fortune if they can. They are going to grind him down until he is nothing better than a tramp."

"Well, why are they goin' to do all this?"

"It is not necessary for you to know at present. Perhaps you will learn that from Sidney Prale, if you keep your ears and eyes open. All we want you to do is to watch and listen and make frequent reports to us. You'll have to be loyal to us, of course. If you are not, we shall punish you."

"But what did Mr. Prale ever do to get such a bunch down on him?" Murk demanded.

"You'll find that out in time---maybe."

"I guess I'd better know right now."

"It is not necessary. Besides, we are not sure of you yet, please remember."

"How could you ever be sure of me?" Murk cried. "If I threw down Mr. Prale, wouldn't I be liable to throw you down, if somebody happened along and raised the price? Why, you simp, I wouldn't turn against Mr. Prale for a million dollars! He's treated me decent, and he was the first man who ever did that! I was just stringin' you, you fool! Mr. Prale himself don't know why your gang is causin' him trouble, and I was tryin' to pump you and find out!"

"So he has told you that he doesn't know why he has enemies?"

"He has---and he told the truth. There's something phony about that murder case; somebody's tryin' to frame him. And when Jim Farland gets through, somebody is goin' to jail!"

"So you will not work for us?"

"You're right; I won't. Maybe I don't amount to much, but I'm mighty square compared to some people I know about."

"And what do you suppose is going to become of you, if you refuse to do as I say?"

"I guess I'll manage to struggle along," Murk said.

"We'll see about that!" the masked man replied, getting up from the chair. "Perhaps a night spent in your present position, without food or water, will cause you to change your mind. If it does not, there are other methods that can be used."

"Goin' to pull rough stuff, are you?" Murk sneered. "Go as far as you like! You can manhandle me, but you can't make me turn against Sidney Prale. That's a golden little thought for to-day, as the preacher says."

\vspace{2\nbs}
\ChapterDeco[c1]{\decoglyph{e9665}}
\clearpage
\thispagestyle{empty}

\begin{ChapterStart}
\vspace{3\nbs}
\ChapterSubtitle[l]{Chapter ch18}
\ChapterTitle[l]{ch18}
\end{ChapterStart}
\FirstLine{\noindent ## A Woman's Way}
    
The masked man stepped forward, snarling behind his mask, his hands closing, and the two thugs stepped forward also, as if to use Murk roughly if the other gave the command.

But there was an interruption. Kate Gilbert came in from the adjoining room.

The masked man whirled to meet her.

"You should not---" he began.

"It makes no difference," Kate Gilbert said. "This man knows me, or he would not have been set to spying on me. Sidney Prale knows that I am associated with his enemies, since I was talking to him to-day. It is not necessary for me to mask my face!"

"It really was not necessary for you to come," said the masked man. "This fellow refuses to have anything to do with us."

"I cannot blame him. You used violence to get him here. I am afraid that I should refuse to have business relations with a man who knocked me on the head."

"It was the only way. We couldn't approach him on the street very well. We have him here now and perhaps may be able to force him to see the light."

"I shall not countenance more violence!" Kate Gilbert said. "I told you in the beginning that force was not to be used. This man is not to be blamed in any way. He merely is an employee of the man we are fighting."

"I think it justifiable to use any method that will get results," the masked man told her. "You seem to forget------"

"I do not forget!" Kate Gilbert cried. "Who has a better right to hope to see Sidney Prale punished? Who has suffered more than I and mine? But I do not wish to see violence used. This man may be made to help us, but I fear you have taken the wrong method. And what do you intend doing now?"

"Perhaps it will be as well for you to return home and allow us to handle this part of the affair," the masked man told her. "No woman likes violence, of course, but at times it is necessary. We are going to leave him here to-night to think things over. He will be stiff and sore and hungry in the morning."

"But------" Kate Gilbert protested.

"It is the better way, I assure you---and quite necessary. This thing is so big that it must be handled with firmness and decision. You have aided us greatly, but I think it will be a mistake to let you take command of the situation."

Kate Gilbert's eyes flashed angrily, and her face flushed.

"Very well, sir," she said. "But let me talk to this man alone. Perhaps common sense and kindness will prevail where violence did not. I sincerely hope so."

"I am willing to let you talk to him, but you are to be guarded in your speech. Tell him nothing about the real affair; we want to be sure of him before we take him fully into our confidence. All we wish him to do is to keep us informed about Prale and Jim Farland, and any others who may be helping Prale."

"I understand, and I am not quite a fool!" Kate Gilbert told him, still angry.

The masked man motioned the two thugs out of the room, and then followed them, closing the door behind him. Kate Gilbert sat down in the chair before the sofa, and looked at Murk.

"First, I want you to know that I had nothing to do with the blow you received," she said. "That was going a bit too far. I knew nothing of it until I received a telephone message saying that you were spying on the place where I live, and that you had been captured and brought here."

"I understand that, lady," Murk replied.

"I know that you have been with Mr. Prale only a few days. If he were in your place now, I might be inclined to turn my back and let those men handle him. But you are not to be blamed for the faults of your employer."

"No, ma'am," said Murk.

"I am going to tell you only this much: Sidney Prale committed a great wrong against several persons. Those persons have banded together to have vengeance. Sidney Prale deserves everything that can happen to him."

"I think you've got him wrong, ma'am," said Murk. "He's even accused of murder, and I know he ain't guilty."

"Neither do I believe that he is guilty of that crime, but that has nothing to do with this other affair. The persons who are banded together against Sidney Prale have nothing to do with the murder charge, I am sure."

"I reckon he'll be glad to know that. But you've got him wrong in this other thing, lady. Mr. Prale is worried almost to death because he don't know who his enemies are, or why they are causin' him a lot of trouble."

"He has led you to believe that?" she asked.

"I know he's tellin' the truth, ma'am. He's got a detective workin' tryin' to find out what it all means."

"Then he is fooling you, and the detective also. Sidney Prale knows who his enemies are, and why they are troubling him. He tried to tell me that he did not know, and almost in the same breath he told me something that convinced me he did know. You have received an offer to help us. Are you willing?"

"I don't intend to turn against Mr. Prale!" Murk declared. "I ain't a man like that! These gents can keep me here and starve me and beat me up, and that's all the good it'll do 'em. I know a man when I see one, and Mr. Prale's a man, and a square man, and I'm goin' to stand by him!"

"He has fooled you! You do not know him for the scoundrel that he is."

"Maybe it's you that's bein' fooled, lady."

"No. If you knew all, you would understand."

"Well, why don't you tell me, then? If you prove to me that Mr. Prale is a crook or somethin', and that you people ain't, maybe I'll change my mind about some things."

"I can tell you nothing now, except that I am right and that Sidney Prale is fooling you," Kate Gilbert said.

"Then I'll stay right here and take my beatin' at the hands of them thugs."

"You will do nothing of the kind," she said. "I will not see them use violence toward you."

"I don't see how you're goin' to help it, ma'am."

"I am going to have you released. You may return to Sidney Prale and tell him that we intend to punish him, but that I, for one, will not resort to violence. He may fight unfairly, but we do not." She lowered her voice and bent toward him. "I'll attract their attention, and send my maid to release you," she said. "Remain where you are."

"Yes'm."

Without another word, Kate Gilbert got up and left the room, closing the door behind her. In the other room were the masked man, the two thugs, and Marie, the maid.

"I have talked to him, and I have a plan," Kate Gilbert told the others. "Marie, I wish you to do something for me. Take the taxicab and go on the errand, and after I am done here I will go home in another car."

She stepped across to the maid and gave her whispered instructions, while the men waited. Marie left the room, walked through the hall, and left the house. Kate Gilbert sat down at the table and called the others to her.

"That man is loyal to Prale," she explained. "Prale has fooled him. He honestly believes that Prale does not know his enemies or why he is being bothered, and he is grateful to Prale for what Prale has done for him. So, naturally, he refuses to turn against his employer."

"If you will leave the matter in my hands------" the masked man suggested.

"I may do so after we have had this little talk. Come closer, so I can speak in a low tone and he will not hear."

They pulled their chairs up to the table.

"This man is stubborn," she said. "You could starve him or beat him, and it would do you not the slightest good. It would only make him the more determined to be faithful to Prale. We would gain nothing. We've got to convince him that we are in the right."

"I object to telling him the whole truth," said the masked man.

"He could do nothing except tell it to Prale---and Prale knows it already, doesn't he?" Kate Gilbert asked.

"You want to let the fellow go?" the masked man cried. "Why, we can use him as a sort of hostage!"

"As if Sidney Prale would care if he never saw his valet again!"

"He is more than a valet; he is one of Prale's spies! If we can hold this man prisoner, and attend to Jim Farland, that detective, Prale would stand alone. There are not many men he would trust to help him. And, if he stands alone, it will be easier for us to torment him, cause him trouble, drive him away!"

"Sometimes I regret that we started this thing," Kate Gilbert said. "What will it avail us to make Prale's life miserable?"

"You seem to forget---"

"I forget nothing! I know how I have suffered, how my father and others have suffered. But I am not sure that retribution will not visit Sidney Prale even if we keep our hands off."

"You're a woman; that is why!" the masked man accused. "You have a soft heart, as is right and proper in a woman. But when you remember your father------"

"I am not quitting!" she declared. "I will continue the game. But I will not permit violence toward anybody, least of all to a poor fellow who has nothing to do with the affair except that he is working for Sidney Prale. We can accomplish our aims without becoming thugs and breaking laws ourselves. I understood that we always were to keep inside the law."

"Well, what have you to suggest?" the masked man asked.

"Let Prale's valet go, for he can do us no harm. Prale knows that I am against him, but he can make no move unless we break the law and his detective has us apprehended. We play into Sidney Prale's hands if we do that. Can't you see it? We do not want to give him an advantage, do we? If we use violence or break a law, we do just that. We must break him down cleverly."

"I see that point, all right."

"I am astonished that you did not see it before. You appear to be very vindictive lately, yet you did not suffer as some others suffered."

"I have my reasons. I always have hated Sidney Prale."

"Then you are making this fight for personal reasons?"

"Do not forget that some very good friends of mine suffered because of Prale. But, about the valet------"

"Let him go, I say. What harm can he do?"

"We slugged him to get him here. He can report it to the police, and have you arrested, and these two men."

"And what evidence would he have?" she asked. "Who would testify that he was telling the truth? These two men can keep out of sight for the present. He has not seen your face because of your mask. And to charge me with slugging him would be ridiculous."

"This house------"

"Is vacant, so far as the neighbors know; it is owned by a man whose wife died, and who has been gone for more than a year. The agent who rented it to us furnished, is one of us. We can simply close it up and not come here again. If he complained, and the police investigated, they would find the house closed, and the nearest neighbors would declare that it had been closed since the owner went away. The furniture is not even dusted."

"That part is all right."

"And that attack on Prale in the Park during the afternoon!" she went on. "That was a mistake. Suppose Detective Farland managed to connect that with us. I tell you we must not break a law, or Sidney Prale may get the advantage!"

"We can't handle an affair like this with kid gloves!" the masked man declared.

"We do as I say, or I shall go to Sidney Prale and tell him everything and rob you of your vengeance!"

"You would do that!" the masked man cried, springing from his chair.

"I'll do it if there is any more violence!" she declared. "It was understood that no rough tactics were to be used, and I demand that we carry out the original plan!"

"We'll see about this!" the masked man cried. "I'll talk to some of the others------"

"And I'll leave the game if there is any more violence---do not forget that!" Kate Gilbert cried.

She continued to talk and plan, for she was fighting for time. She had known that, at the last moment, this man would refuse to release Murk.

Marie, the big maid, had hurried from the house, which sat far back from the street and was surrounded by trees. But she had returned after watching for a few minutes.

Murk, sitting on the sofa, heard somebody at one of the windows. He watched the sash being raised slowly and cautiously, and after a time saw the head of Marie. She motioned him for silence, listened a moment, and then crawled inside.

Marie hurried across to Murk and fumbled with the cords that bound his wrists together behind his back. The bonds slipped away, and Murk made quick work of the one around his ankles. He hurried across the room, got through the window, and helped the big maid through. Marie led him toward the street.

"Come right along with me!" she commanded, when they were some distance from the house.

"Thanks for helpin' me out, but I guess I'll hang around," Murk replied. "I'm right eager to get a look at the face of the man who was wearing the mask."

"I supposed you'd want to do that," the big maid told him. "And that's what I've got orders to keep you from doing. You come along with me!"

Murk got a surprise. Marie gripped his shoulder with her left hand---and it was no gentle grip. Then he saw that she was holding an automatic pistol in her right hand.

"There is a taxi at the corner," she informed Murk. "We are going to get into it and drive back to the city. You may be able to find this house afterward, but I doubt it."

"Suppose I take a notion not to go?" Murk asked.

"I'm not afraid to shoot," Marie informed him.

"Aw, let me go!" he exclaimed. "You're in wrong in this deal; see? I tell you that Mr. Prale, my boss, is an all-right man, and you people are makin' some kind of a mistake."

"I like to see a man stick up for his boss," replied the gigantic Marie. "And I'm stickin' up for mine right this minute, and she told me to see that you went to town. Why don't you quit that man Prale and get a real job with a gentleman? You're not a bad-looking man at all."

Murk felt himself blushing at this unexpected announcement. Praise from the lips of a woman was something new in his life. He glanced at the amazon beside him.

"And you're sure some woman!" he said. "And that ain't just nice talk---I sure mean it! But you ain't got this from the right angle. I've got to work for Mr. Prale. I'd be a dead one this minute if it wasn't for him. If I didn't stick by him now, I'd never be able to look at myself in a shavin' mirror again. You don't want me to be an ungrateful pup, do you? You see------"

Having directed her attention to another topic for a moment, Murk put his plan into action. He made a quick lunge forward as he spoke, springing a bit to one side as he did so, and trying to seize the automatic and tear it from her grasp.

But the gigantic Marie had been anticipating something like that, despite Murk's speech and his manner that said he was a willing captive. She lurched forward and hurled Murk back, sprang after him, crashed the butt of the weapon against the side of his head, and then, while he was a trifle groggy from the blow, she grasped him with her powerful hands and piloted him toward the street with strength and determination.

"Never try to play them child's tricks on me!" she announced.

Murk regarded her with mingled admiration and chagrin, and spoke with enthusiasm.

"Some woman!" he commented.

\vspace{2\nbs}
\ChapterDeco[c1]{\decoglyph{e9665}}
\clearpage
\thispagestyle{empty}

\begin{ChapterStart}
\vspace{3\nbs}
\ChapterSubtitle[l]{Chapter ch19}
\ChapterTitle[l]{ch19}
\end{ChapterStart}
\FirstLine{\noindent ## Coadley Quits}
    
Murk, compelled to ride back to the city in the taxicab with Marie, spent the time in ordinary conversation with the amazon, and told himself repeatedly that she was a great woman, a dangerous state of mind for a bachelor.

The only reason Murk wanted to remain in the vicinity of the cottage was to catch a sight of the countenance of the man who had worn the mask. As far as the cottage itself was concerned, he had noticed a signboard on a street corner not far from it, and he would be able to locate it again if Sidney Prale or Jim Farland thought it necessary.

Marie stopped the taxicab near the Park, and Murk got out and gallantly offered to pay the bill for his enemy, but Marie would not allow it.

"Hope to see you often and get to know you better when this little scrap is over," Murk made bold to say, and then, chuckling at her retort, he started walking down the street.

He did not care to ride, for it was not so very many blocks to the hotel, and Murk wanted time to formulate in his mind the report he intended to make to his employer.

Prale was waiting for him, and Murk told his story in detail and without embellishment.

"So Kate Gilbert had you freed, did she?" Prale said. "And she told the others that she would quit them if they used any more violence? Murk, old boy, when our foes begin fighting in their own camp it is time for us to begin to hope. A house divided against itself cannot stand, as you probably have heard."

"She certainly panned the man who wore the handkerchief over his face," Murk said. "I think I'd know him again, boss. He talked a good deal, remember, and he got careless toward the last and used his regular voice. And I watched his hands---boob didn't have sense enough to wear gloves. Anybody but a boob would know that a hand can be recognized as easy as a face."

"Let us hope that they make a lot of mistakes like that, Murk," Prale replied. "I'll be glad if we ever solve this confounded mystery. It's getting on my nerves."

They remained up until one o'clock in the morning, but Jim Farland neither visited the hotel again nor called them up, and so they went to bed.

They did not rise early, but had breakfast in the suite and took their time about eating it. After that, they waited for Farland to arrive or telephone and give orders and tell news. Farland did not come, but Attorney Coadley did.

Murk admitted him, and the distinguished criminal lawyer sat in the window beside Prale, a grave expression on his face, his manner that of a disconcerted man.

"I gather you do not bring good news, judging from your countenance," Prale said.

"At least, I have not come to say that the case against you is any stronger," Coadley replied. "I'd like to speak to you alone, Mr. Prale."

"Certainly. You may go into the other room, Murk, and remain until I call."

Murk obeyed, and Sidney Prale bent forward in his chair and looked at the attorney again, wondering what this visit meant, what was coming, half fearing that the news would be ill after all.

"Mr. Prale," Coadley said, "I have come here to your apartment to tell you that I wish you to get another attorney."

"I beg your pardon!" Prale gasped.

"I wish to withdraw from the case, Mr. Prale---that is all. An attorney does that frequently, you know."

"But I want you to handle my case," Prale said. "I have been given to understand that you are one of the foremost criminal lawyers in the city. And you have done so much already------"

"I insist that I withdraw, Mr. Prale. I shall be ethical. I shall give the man you name in my place all the knowledge at my command regarding this case, and I shall see that the change does not embarrass you or place you in jeopardy. The court will grant extensions if they are necessary."

"Farland has given me to understand that my alibi now is of such a nature that the case against me may be dismissed. I had hoped that you had come here this morning to tell me so."

"I fancy that any good attorney can get the charge dismissed," Coadley said.

"But I do not want to be freed under a cloud. I want the public to be sure I did not kill Rufus Shepley---I want to have the public know the identity of the man who did."

"That is what I thought, and that will take considerable time, perhaps," Coadley said. "And so I wish to withdraw------"

"If it is a question of fee------"

"Nothing of the sort, Mr. Prale. I am sure you would pay me any reasonable fee I asked. There is no question regarding your financial ability."

"May I ask, then, why you desire to leave the case?" Sidney Prale asked.

"I'd rather not state my reasons, Mr. Prale. Just let me withdraw, and make arrangements with the court, after you have named the man to take my place. The bail arrangement will stand, of course."

"So you do not care to tell your reasons!" Prale said. "Mr. Coadley, a banker refused to handle my funds. A hotel manager ordered me out, you might say, for no good reason whatever. I understand that I have some powerful enemies who are working in the dark, and who cause these annoyances. Do you wish me to understand, Mr. Coadley, that they have been to see you? Do you wish me to think that you are under the thumbs of these persons, whoever they may be?"

The attorney's face flushed, and he looked angry for an instant, but quickly controlled himself.

"I do not care to go into details, Mr. Prale," he said.

"Then it is the truth!" Prale said. "The big criminal lawyer is not so big but that others can force him to do as they please."

"Let us say as I please, Mr. Prale."

"Then you think that you have a good reason for withdrawing?"

"I do."

"In other words, something has been told you that convinced you I am not a fit client. Is that it? And, instead of telling me what it is, and giving me a chance to refute the charge or explain, you simply take the easiest course and believe my enemies. Do you call that an example of the square deal?"

"Let us not talk about it further, Mr. Prale," Coadley replied. "I feel quite sure that you have a complete understanding of the situation."

"But I have not! I seem to be able to understand nothing in regard to this affair of which I am the central figure. I would give half my fortune, I believe, to have an explanation and be able to set things right."

"No doubt you would be willing to give half your fortune to set things right!" Coadley said. "It is your privilege, of course, to say that you do not understand. Mr. Prale, you must see that this interview is painful to me, and it must be painful to you. Why prolong it?"

"As far as I am concerned, this interview may be terminated at once, sir!" Sidney Prale exclaimed. "I'll send you a check for your services as soon as you submit your bill; and please do not neglect to do so at once. I'll inform you as soon as possible of the name of the man I select to fill your legal shoes in this matter. That is satisfactory? Very well. Murk!"

Murk hurried in from the adjoining room when he heard Sidney Prale's call.

"Show Mr. Coadley to the hall door, Murk!" Sidney Prale said. "And while you are about it, please close that ventilator in the corner of the room. It creates a draft, I am sure, and Mr. Coadley already has cold feet!"

The attorney glared at Prale, and then got up and walked quickly across to the door, which the grinning Murk held open to let him pass out.

\vspace{2\nbs}
\ChapterDeco[c1]{\decoglyph{e9665}}
\clearpage
\thispagestyle{empty}

\begin{ChapterStart}
\vspace{3\nbs}
\ChapterSubtitle[l]{Chapter ch2}
\ChapterTitle[l]{ch2}
\end{ChapterStart}
\FirstLine{\noindent ## The Girl On The Ship}
    
Sidney Prale folded the piece of paper carefully and slipped it into his wallet. Winning a fortune in ten years in a foreign country had taught Prale many things, notably that everything has its cause and effect, and that things that seem trifles may turn out to be of great importance later.

He finished his packing, locked the suit case, put on coat and hat and went out upon the deck. The Manatee was docking. A throng was on the wharf. Prale glanced at the buildings in the distance and forgot for the time being the scrap of paper, because of his happiness at being home again and his eagerness to land. Returning to New York after an absence of so many years was in the nature of an adventure. There would be exploring trips to make, things to find, surprises at every turn and on every side.

The passengers were crowding forward now, preparing to go ashore. Sidney Prale picked up his suit case and started through the jostling crowd. Already those on board were calling greetings to relatives and friends on the wharf, and Prale's face grew solemn for a moment because there was nobody to welcome him.

"Not a friend in the world," he had said to Rufus Shepley that morning.

"A man with a million dollars has a million friends," Shepley had replied. "The only trouble is, you can't enjoy that sort of friends except by getting rid of them, unless you happen to be a miser."

Well, that was something, Sidney Prale told himself now. He had ample funds, at least, and perhaps he could enjoy himself after ten years of battling with financial sharks, of inspecting and working mines, of cutting through dense forests and locating growths that could be turned into wealth.

Prale put his suit case against the rail to wait until he could move forward again. He looked down at the throng on the wharf, and up and down the rail at his fellow passengers. Then he saw the girl again!

He had seen her before. The first time had been at Tegucigalpa, at a ball given by some society people for charity. He had known her at once for an American, and finally had obtained an introduction. Her name was Kate Gilbert, and she lived in New York. It was understood that she was of a wealthy family and traveling for her health. She was accompanied only by a middle-aged maid, a giant of a woman who seemed to be maid and chaperon and general protector in one.

That night at Tegucigalpa, Prale had talked to her and had danced with her twice. He judged her to be about twenty-eight, some ten years younger than himself. She was small and charming, not one of the helpless butterfly sort, but a woman who gave indication that she could care for herself if necessary.

Prale had been surprised to find her aboard the Manatee, but she had told him that she was going home, that her health had been much benefited, and that she felt she could not remain away longer. It had seemed to Prale that she avoided him purposely, and that puzzled him a bit. He could not understand why any woman should absolutely dislike him. His record in Honduras was a clean one; it was known that he did not care much for women, and surely she had learned that he was a man of means, and did not think he might be a fortune hunter wishing to marry a prominent heiress.

He had not spoken to her half a dozen times during the voyage. She made the acquaintance of others aboard and, for the first few days, had been busy in their company. The last three days had been stormy ones, and Kate Gilbert had not been much in evidence. Prale judged that she was a poor sailor.

Now she stopped beside him, the middle-aged maid standing just behind her.

"Well, we're home, Mr. Prale!" she said.

"I suppose that you are glad to get home?"

"Surely!" she replied. "And I'll be angry if there are not half a dozen to meet me when I land. I've been trying to spot some friends in that crowd, but it is a hopeless task."

"I hope you'll not be disappointed," Prale said.

As he spoke, he glanced past her at the middle-aged maid, and surprised a peculiar expression on the face of the woman. She had been looking straight at him, and her lips were almost curled into a sneer, while her eyes were flashing with something akin to anger.

Prale did not understand that. Why should the dragon be incensed with him? He was making no attempt to lay siege to the heart of Miss Kate Gilbert. He was no fortune hunter after an heiress. The expression on the face of the maid amused Prale even while he wondered what it could mean.

"Picked your hotel?" Kate Gilbert was asking.

"Not yet, but I hope to get in somewhere," Prale told her. "May I be of assistance to you when we land?"

"Marie will help me, thanks---and there will be others on the wharf," she answered.

A cold look had come into her face again, and she turned half away from him and looked down at the crowd on the wharf. Sidney Prale looked straight at her, despite the glare of the middle-aged maid. Kate Gilbert was a woman who would appeal to a majority of men, but there seemed to be something peculiar about her, Prale told himself. He knew that she had avoided him purposely during the voyage, and that she had spoken to him purposely now, yet had asked nothing except whether he had chosen a hotel.

Why should Kate Gilbert wish to know where he was going to stop? Perhaps it had been only an idle question, he explained to himself. In her happiness at getting home, she had merely wished to speak to somebody, and none of her shipboard friends happened to be near.

He turned from her and glanced at the maid again. She was not the sort to be named Marie, Prale told himself. Marie called up a vision of a petite, trim woman from sunny France, and this Marie was nothing of the sort. She appeared more to be a peasant used to hard labor, Prale decided.

And he could not understand the expression on the woman's face as she looked at him. It was almost one of loathing.

"Got me mixed up with somebody else, or somebody has been giving me a bad reputation," Prale mused. "Enough to make a man shiver---that look of hers."

Kate Gilbert, apparently, did not intend to have anything more to do with him. Smiling a little at her manner, Prale lifted his hat, picked up the suit case, and turned away. Once more he tried to force a passage through the jostling crowd. He had not taken three steps when Kate Gilbert touched him on the arm.

"Pardon me, Mr. Prale, but there is something sticking on the end of your suit case," she said.

Prale glanced down. On one end of the suit case was a bit of paper. It had been stuck there by a drop of mucilage, and the mucilage was still wet.

He thanked Kate Gilbert and picked the paper off, but he did not throw it over the rail into the water. He crumpled it in his hand and, when he was some distance away, he smoothed it out.

There was a single word written on it, in the same handwriting as that of the note he had found pinned to the pillow in the stateroom---"Retribution."

Sidney Prale glanced around quickly. Nobody seemed to be paying particular attention to him. Kate Gilbert and her maid had passed him and were preparing to land. Prale put the piece of paper into his coat pocket and picked up his suit case again. That bit of paper, he knew well, had not been on the suit case when he had left the stateroom. It had been put there as he had made his way through the crowd of passengers along the rail. Who could have stuck it there---and why?

Now the passengers were streaming ashore, and Sidney Prale stepped to one side and watched them. Perhaps he had some business enemy on board, he told himself, some man he had not noticed, and who was trying to frighten him after a childish fashion. He searched the faces of the landing passengers, but saw nobody he had known in Central America, nobody who looked at all suspicious.

"Either a joke---or a mistake," Prale told himself again.

He started ashore. He saw Kate Gilbert just ahead of him, the bulky maid at her heels. An elderly man met her, but did not greet her as a father would have been expected to do. Prale saw them hold a whispered conversation, and it seemed to him that the elderly man gave him a searching glance.

"I must look like a swindler!" Prale mused.

Finally, as he went out upon the street to engage a taxicab and start for a hotel, he saw Kate Gilbert and her maid and the elderly man again, getting into a limousine. The girl held a piece of paper in her hand, and was reading something from it to the elderly man. As she got into the car, she dropped the piece of paper to the curb.

The limousine was gone before Prale reached the curb. He put his suit case down and picked up the piece of paper. There was nothing on it except a couple of names that meant nothing to Sidney Prale. But his eyes bulged, nevertheless, as he read them.

For the paper was similar to that upon which had been written the note that he had found on the pillow in the stateroom---and the coarse handwriting was the same!

"What the deuce------" Prale caught himself saying.

Had Kate Gilbert written that message about retribution and had her maid leave it in the stateroom? Had Kate Gilbert written that single word and had her maid paste it on his suit case as he passed, or pasted it there herself?

Why had Kate Gilbert---whom he never had seen and of whom he never had heard until she appeared at the ball in Tegucigalpa---avoided him in such a peculiar manner? And why had the misnamed Marie glared at him, and expressed loathing and anger when her eyes met his?

"What the deuce------" Prale asked himself again.

Then a taxicab drew up at the curb, and he got in.

\vspace{2\nbs}
\ChapterDeco[c1]{\decoglyph{e9665}}
\clearpage
\thispagestyle{empty}

\begin{ChapterStart}
\vspace{3\nbs}
\ChapterSubtitle[l]{Chapter ch20}
\ChapterTitle[l]{ch20}
\end{ChapterStart}
\FirstLine{\noindent ## Up The River}
    
Coadley had not gone for more than an hour when Detective Jim Farland arrived at the hotel and made his way immediately to Sidney Prale's suite.

He found Prale pacing the floor angrily, and Murk sitting in a corner and watching him. The police detective, after doing duty for a few days, had been withdrawn, as it seemed evident that Prale had no intention of jumping his bail or eluding trial in any other way.

"What's the trouble now?" Farland asked.

"Coadley has just been here," Prale replied. "He has quit us. Our friends the enemy have reached him."

"You couldn't get any sort of an explanation out of him?" Farland asked.

"Nothing at all. He simply informed me that he was done, and that I had to get another lawyer."

"I'll try to find an honest one for you," Farland declared. "I happen to know a clever young chap who probably will take the case, especially if I explain the thing to him, for he loves a fight. There is no special hurry, but I'll try to attend to it some time to-day."

"Anything new?" Prale asked.

"That is what I am waiting to hear. What did you do last night, Murk?"

Murk related his adventure at length, while Jim Farland listened gravely, nodding his head now and then, and looking puzzled at times.

"I'd like to know the identity of that masked man," the detective said, when Murk had finished. "The main trouble in this case is that we do not know the people we are fighting. We know that Kate Gilbert is one of them, and have reason to suspect that George Lerton is another. But there is somebody bigger behind, and that's a fact."

"What are you going to do next?" Prale asked.

"I'm going to pay a little attention to the Rufus Shepley murder case. I'm going to find out, if I can, who killed Shepley, and why. I am of the opinion that the murder is distinct from this other trouble, Sid. Perhaps a clew to the murder, however, will give us a clew to the whole thing, for it is certain that somebody has attempted to hang that crime on you."

"How about George Lerton?" Prale asked.

"We know that he tried to help smash your alibi by telling a falsehood, and by sending those notes to the barber and the merchant. But we do not know his motive, unless it is simply a hatred of you, Sid, and envy of the million dollars you got in Honduras. I'm going to get out of here now, and get busy."

"Anything for us to do?" Prale asked.

"Keep out of trouble---that is the principal thing. It appears that every time either of you goes out, you get knocked on the head. I'll report again as soon as I can."

Jim Farland left them and hurried from the hotel. He went to the hostelry where Rufus Shepley had met his death, was admitted to the suite, and made an exhaustive investigation, which revealed nothing of importance.

He visited the New York offices of the company in which Shepley had been interested, and questioned officials and clerks, but got no inkling of a state of affairs that might have led to a murder. He was told that the company's business was in proper shape, and that Rufus Shepley had had no financial trouble of any sort so far as his associates knew.

Farland left the office and continued his investigations. In the evening he went to his home for a meal, and admitted to himself that he did not know any more than when he had started out that morning.

"It gets my goat!" he said to his reflection in the bathroom mirror. "I'll have to begin working from some other starting point. I've made a mistake somewhere, or overlooked something that I should have seen. Makes me sore!"

The telephone bell rang, and Farland went to the instrument to hear the voice of a man he did not know.

"I understand that you are interested in the Shepley murder case," his caller said.

"I am working on it, yes. Who is talking?" Farland demanded.

"I'm not ready to mention any names. If you want to hang up, go ahead and you'll miss something important. Or if you want to listen for a minute------"

"I'll listen!" Farland said.

"I know a lot about that Shepley case, but I am in a position where I have to be careful. If you'll do as I say, you can learn something you'd like to know."

"What do you want me to do?" Farland asked.

"Meet me in some place where nobody will see us talking, and I'll tell you a few things. But I must have your promise that you'll not reveal the source of the information."

"I'll protect you, unless you are mixed up in it to such an extent that I'd dare not do so," Farland said. "I'm not guaranteeing to shield any murderer or accessory."

"I had nothing to do with the murder, if that is what you mean," came the reply.

"Then where do you want me to meet you---and when? Can you make it this evening?"

"Yes; and suppose that you set the meeting place, one that you know will be all right for both of us."

Farland was glad to listen to that sentence. He had half believed that this was nothing more than a trap, that some of Sidney Prale's mysterious enemies were attempting to lure him to some out-of-the-way place and get him in their power. But if he was to be allowed to name the meeting place, it seemed to indicate that everything was all right in that regard.

Farland though a moment, and then suggested a certain famous restaurant on Broadway and a table in a corner of the main room, where a man could lose himself in the crowd. But that did not meet with the approval of the man at the other end of the telephone wire.

"Nothing doing in that place," he said. "1 of the men interested in this thing hangs out there almost every evening. He'd be sure to see us, he knows how much I know about it, and he'd suspect things in a second if he saw me talking to you. Then it'd be made hot for me. I've got to protect myself, of course."

"Suggest a place yourself," Farland said.

"Make it outside somewhere. How about some place in Riverside Park?"

"Suits me," Farland replied.

The man at the other end of the wire gave the directions after much seeming speculation and many changes. Jim Farland was to go to Grant's Tomb, and from there to a certain place near the river. The other man would be in the neighborhood watching, he said, would recognize Farland as he passed the Tomb, and then would follow and speak to him when nobody else was near.

Farland agreed, and made the engagement for an hour and a half later, saying that he could not get there before that time. It would not be the first time that Jim Farland had obtained an important clew because somebody interested had grown disgruntled and had turned against his pals; and he supposed this to be a case of that sort.

Before leaving home, Farland made sure that his automatic was in excellent condition, and that he had his handcuffs and electric torch and other paraphernalia of his trade. He made his way to Columbus Circle, having decided to walk to the rendezvous. Farland was in no hurry. He observed all who passed him, and he frequently made experiments to ascertain whether he was being followed. He decided, after a time, that if he was being shadowed the person doing it was too clever for him.

He came to Riverside Drive through a cross street, and approached the famous Tomb as cautiously as possible, keeping in the shadows, alert to discover anybody who might be acting at all suspiciously. Farland felt sure that this was no trap, but he was not taking chances. He always had been known to his friends as a cautious man.

He reached the Tomb finally, and glanced around. Half a dozen persons were passing, some men and some women, some alone and others in couples, but none were of suspicious appearance.

Farland glanced at his watch to be sure that it was the appointed time. He strolled around the Tomb and waited ten minutes longer, for he did not care to find later that he had left the appointed spot too early and that the other man had not seen and followed him.

At the end of the extra ten minutes, Farland lighted one of his big, black cigars and started walking toward the river, following the route the other man had designated over the telephone. He walked slowly and not for an instant did he throw caution aside.

Here and there were dark spots where Farland expected to hear his name spoken, spots where an attack might be made if one was contemplated by foes.

It was as he was passing one of these that a whisper came from the darkness:

"Mr. Farland!"

The detective whirled toward the sound, one hand diving into a coat pocket and clutching his automatic.

"Well?"

"Be as silent as possible. Do not flash your torch yet; you may do so presently, so you can see who is talking. I am the man who called you up by telephone."

"Come out where I can get a glimpse of you," Farland commanded, ready for trouble.

He could see a shadow detach itself from the patch of gloom in front of him and approach.

"That is close enough for the present!" Farland said. "I'm not taking chances on you until I know who's talking to me."

"I don't blame you, Mr. Farland, under the circumstances. If you are sure there is nobody approaching, I'll come out into the light so you can see my face."

Farland glanced up and down the walk quickly. As he did so, he heard a step behind him. He whirled, the automatic came from his pocket ready for use---and a man crashed into him.

The one who had been talking from the patch of shadow rushed forward at the same instant. Farland managed to fire once, but the shot went wild. Then a third man rushed from the darkness, and the detective had the automatic torn away, and found that he had a battle on his hands.

1 man was upon his back, throttling him so that he could not utter a cry. The others were trying to throw him to the ground. Farland wondered whether that single shot had been heard, whether assistance would reach him, for he knew that here was a battle he could not win by force.

Finally they got him down. Something was thrust into his mouth and bandaged there, effectually gagging him. He was turned over on his face, and his wrists were lashed behind him. Then his ankles were fastened, and two of the men, at the whispered instruction of the third, picked him up like a sack of meal and carried him into the deep shadows.

They did not stop there, but continued toward the river, holding a conversation in whispers at times, and stopping now and then for a moment to rest and listen. Farland had been quiet, gathering his strength, and suddenly he began to struggle.

It was nothing worse than annoyance for his opponents. He was unable to make an outcry that would attract attention, and he was unable to put up an effective fight. They threw him upon the ground again and held him there.

"Another little trick like that, and we'll give you something to keep you quiet," one of the men whispered into his ear. "We've got you, and you'd better let it go at that!"

Once more they picked him up and went toward the river. They reached it, and one of the men hurried away while the other two guarded Farland. 5 minutes passed, and then a powerful motor boat slipped toward the shore. An instant later Farland was aboard it, a prisoner, and the boat was rushing through the great river toward the north.

Farland made an attempt to watch the lights along the shore, but one of the men threw a sack over his face, so that he could not see. And so he merely listened to the beating of the boat's engine, and tried to estimate with what speed they were running and how much mileage the craft was covering.

The sack was heavy, and Jim Farland felt himself half smothered, the perspiration pouring from his face and neck. He had grown angry for a moment, angry at himself for walking into the trap even while suspecting that one might exist, angry at these three men who had captured him so close to Riverside Drive.

Then his rage passed. He was experienced enough to know that an angry man is at a disadvantage in a game of wits, and that wits and nothing else could get him out of the present predicament.

Finally, he felt the boat turning, the speed was cut off, and it drifted against something. Farland was lifted out of the motor boat, but one of the men held the sack over his head, and he was unable to see. Once more he was carried, this time away from the river, and he could tell nothing except that the men who carried him were struggling up a sharp slope.

Farland made no attempt to fight or struggle now, knowing that it would avail him nothing to attempt to throw off these three men. He had decided to conserve his strength, and to trust to his usual good fortune to get a chance later to even things by turning the tables on his captors.

Suddenly the sack was taken from his head, and he was able to breathe better. He found that he was beside a road in which stood an automobile. 2 of the men lifted him, tossed him inside the machine, and then got in themselves. The driver started the engine, threw in the clutch, and soon the car was being driven at a furious pace along the winding road.

"Look around all you want to!" one of Farland's captors growled at him. "You won't even know where you are when you get there!"

\vspace{2\nbs}
\ChapterDeco[c1]{\decoglyph{e9665}}
\clearpage
\thispagestyle{empty}

\begin{ChapterStart}
\vspace{3\nbs}
\ChapterSubtitle[l]{Chapter ch21}
\ChapterTitle[l]{ch21}
\end{ChapterStart}
\FirstLine{\noindent ## Recognition}
    
Through a maze of crossing and winding roads the car made its way, now over highways as smooth as a city pavement, and now over rough mileage that jolted the occupants and threatened the springs with destruction.

Jim Farland did not recognize this particular district. He did not even know upon which side of the river he was being hauled along as a prisoner. In the city proper, his abductors would have found it very difficult to take him to a section where he could not have recognized some sort of a landmark, but here they had him at a serious disadvantage.

The night was dark, too, and a fine drizzle was falling. Farland tugged at his bonds when he could, and finally convinced himself that they would not give. He tried to work one end of the gag from the corner of his mouth and found that he could not do that. He was utterly helpless for the time being, at the mercy of the three men who had kidnaped him, and the chauffeur, and whoever might be where they were going.

For half an hour longer the car made its way across the country, and then Farland noticed that it left the principal thoroughfare and turned into a rough, narrow lane that was bordered with big trees. At the end of a quarter of a mile of this lane, the chauffeur brought the car to a stop. Farland could see a building that had the appearance of being an abandoned farmhouse.

He was lifted from the car and carried to the door. 1 of the men threw it open, and Farland was carried inside. They took him through a hall, turned into a room, and tossed him upon a couch in a corner there. 1 of them struck a match, lighted a lamp, and then they turned to survey him.

Farland glared at them, waited for them to speak. They were making no attempt to hide their features. Typical thugs they were, the three of them, and Farland supposed that the chauffeur, who had not come into the house with the others, belonged to the same class.

1 of them stepped forward and removed Farland's gag, while another went into another room and presently returned with a dipper of water, which he held to Farland's lips. He drank greedily, for the gag had parched his mouth and throat.

"Bein' as how you are a copper, I'd slip a knife between your ribs and call it a good job," one of the men told him, "but we are supposed to treat you nice and keep you in condition for a little talk with the boss. So you needn't tremble with fear any."

"It'd take more than three bums like you to make me afraid!" Farland told him.

"Nasty, ain't you? Maybe we'll get a little chance to beat you up later, especially if your little talk with the boss ain't what they call productive of results. You've got some reputation as a dick, but I reckon it's all a fake. We didn't have much trouble gettin' you and bringin' you here."

"Isn't that enough to make you worry a bit?" Farland asked.

"How do you mean?"

"Did you ever stop to think that maybe I wanted to be captured and hauled here? Have you any idea how many men watched and trailed us? You've led me to where I wanted to come, to a place I wanted to find, perhaps."

"That bluff won't work," came the reply. "We had a couple of men watchin' for that very thing, and they'd have given us a high sign if we had been followed. You're here all by your lonesome, and so you'd better be good."

2 of the men left the room, and the third sat down by the table to act as guard. 15 minutes passed, during which Jim Farland and the man by the table exchanged pleasant remarks concerning each other, neither getting much the best of the argument.

Then the hall door was opened again, and a masked man entered the room!

Remembering what Murk had related to him concerning his experience of the night before, Jim Farland looked up at this newcomer with sudden interest.

This man, undoubtedly, was a sort of leader, one who had hired others to help him in his work and who knew the identities of Sidney Prale's mysterious enemies, and why they were working against him; perhaps, also, the man who could tell a good deal about the murder of Rufus Shepley.

Farland did not betray too much interest, though, for he sensed that he was opposed to a person of brains and cunning, a different type from the thugs he hired to work for him. So the detective merely blinked his eyes rapidly as he looked up at the other and waited for him to speak.

"You are Jim Farland, a detective?"

The voice was low and harsh, a monotone, a disguised voice in fact. Jim Farland knew that at once.

"That's my name, and some people are kind enough to say that I am a detective," Farland replied. "What's the idea of treating me rough like this?"

"I regret that violence was necessary to get you here, Mr. Farland," the masked man replied, "but it seemed to be the only way in which I could get a chance to talk to you freely without subjecting myself to danger."

"Why regret?" Farland asked.

"Because I want you for my friend instead of my enemy, Mr. Farland, and I fancy that we may be able to come to terms. I shall send this man of mine from the room and submit a proposition to you. I hope you see fit to accept it."

He motioned for the other man to leave, which he did immediately, closing the hall door behind him. Then the masked man sat down in the chair by the table.

Farland was watching him closely now. The collar of his coat and the handkerchief mask effectually shielded his face and head. But, as Murk had told, this man did not have the common sense to cover his hands, and Farland looked at them when he could, careful not to let the other suspect his object.

"I am the man who talked to Mr. Prale's valet last night," Farland heard the other say. "In some manner, the valet escaped, and so we were obliged to have you brought here instead of to the place where we had him, and which was considerably nearer the city. I regret it if the long ride annoyed you, but you will appreciate that it was necessary for my men to bind and gag you."

"It certainly was if they expected to get me here!" Jim Farland declared.

He heard the masked man chuckle.

"I understand that you have been engaged by Sidney Prale to clear him of the charge of murdering Rufus Shepley."

"I don't mind admitting that, since the whole city knows it," said Farland.

"And also to aid Sidney Prale in outwitting certain persons who are trying to punish him for something he did."

"I don't know anything about that. I do know that some people are trying to make things hot for Sid Prale, and he doesn't deserve it, and------"

"Pardon me, if I interrupt!" the masked man said. "You say that he does not deserve it. Do you believe that influential persons would persecute him if he did not deserve it?"

"Sid Prale doesn't know what it is all about!"

"That is what he told the valet, too. But believe me when I say that he does know what it is all about, and is deceiving you when he says otherwise."

"What has all this to do with me?" Jim Farland demanded. "Did you have me brought here to argue the case with me?"

"I had you brought here because I want you to cease working for Sidney Prale. I want you to go back to him and tell him that you are done."

"As Coadley, the attorney, did?"

"Exactly!"

"Your people must be men of influence if they can buy off Coadley like that!"

"Perhaps Coadley was shown that it would wreck his future if he continued working for Prale."

"Well, you can't wreck my future, because I haven't any," Farland told him.

"Do not be too sure of that, Mr. Farland. Agree to my proposition and you may have a great future. You may find business thrown your way. You may find yourself able to spread out, have a protective service, become a wealthy man. If you give up the Prale case, we'll see that you are paid cash immediately, of course, in lieu of the fee you would receive from Prale---and considerably more than he would pay you."

"I suppose that would appeal to a lot of men," Jim Farland said, "but it isn't the right bait to use if you are eager to catch me. I have all the business I want. I can make a living for myself and my small family, and we do not hanker after riches. A larger business would make me a human machine, and I'd rather just drift along and be an ordinary good husband and father. I'd rather be running a little, third-rate detective agency as I am, making just enough to get along, and have a lot of friends. I wouldn't throw down a friend for a million dollars! I suppose I'm the only man in town that thinks this way, but I'm a sort of peculiar duck!"

"You mean to tell me that you are not anxious to better yourself, to get along in the world?"

"Oh, I manage to get along!" Jim Farland replied. "I even eat meat now and then. I haven't seen the face of the famous wolf outside my door for some time. What is money?"

"Everything!" the masked man replied.

"That's what you think. It gives me an inkling as to what sort of man you are. I happen to know a fellow to whom money is everything---and I have reason to suspect that he is considerably interested in the case of Sidney Prale. Be careful you do not betray your identity to me!"

Farland had the satisfaction of hearing the masked man gasp, and he chuckled.

"Well, what is the proposition?" Farland inquired. "You seem to waste a lot of time."

"We want you merely to tell Sidney Prale that you will not work on the case any more---that you are done. Then go about your regular business. We'll have you watched, and as soon as we are satisfied that you are keeping faith with us, we'll send you ten thousand dollars in cash. If you make the agreement with me, I'll give you a thousand cash to-night before you leave this place, as a sort of retainer and expression of our sincerity. Then, following the fee of ten thousand dollars, you'll find that much business is flowing your way. All you have to do to get all this is to withdraw from the Prale case at once."

"You must be afraid that I am finding out some things," Jim Farland suggested.

"That is scarcely the reason," the masked man answered. "We want Sidney Prale to stand alone, to be without help of any sort---that is all."

"But I am more than Sidney Prale's employee. I am his friend!" Farland protested.

"You were his friend ten years ago, sir, but a man may change a great deal in ten years. Are you quite sure that the Sidney Prale of to-day is the boyish, friendly Sidney Prale of ten years ago?"

"I am quite sure; and that is why I am trying to help him," Jim Farland declared.

"I fear that he is fooling you---as he is deceiving others. He is not worthy of such friendship as you are giving him."

"How do I know that?" Farland asked. "If I could have some sort of an explanation------"

He awaited the other's reply. If he could get some inkling as to why Prale had powerful enemies, it might help a lot.

"I can tell you this much: Sidney Prale did something that wrecked and ruined several lives. Certain prominent persons have decided to punish him. He is to have his life made miserable, he is to have his fortune taken away from him, he is to be subjected to petty annoyances and hard blows alike, driven from this, his home town, forced to realize that a man cannot do what he did and escape retribution."

"Sounds like he murdered a nation!" Jim Farland commented. "Did he wreck the national treasury or turn traitor to the flag?"

"I am not jesting, Mr. Farland."

"Neither am I. My eyes have got to be opened, sir. You've got to come clean with me. Prale's enemies may strike at him from the dark, but Jim Farland never works in the dark! I want to see where I'm stepping. I never like to trip over anything."

"I have told you all that I can at present."

"Why?"

"Because I do not care to give you information if you are still to work for Prale."

"You say that Prale knows his enemies and why they are fighting him. If he does, he never has told me. Tell me that much---since you say Sid Prale knows it already. It couldn't hurt your side at all."

"We might tell you later."

"You've got some very good reason for not telling me!" Farland accused. "It is the truth, isn't it, that Prale does not know a single thing about it. You are afraid to tell me because I may inform him of what you say, and we may straighten out the tangle? I can see through you, sir, as easily as through a newly cleaned window."

"I see that you have faith in Sidney Prale," the masked man said. "But I assure you that your faith is misplaced. Is there any way in which I can get you to stop your work for him?"

"Meaning against his influential enemies, or on the Rufus Shepley murder case?" Farland asked.

"We simply want you to stop working for him. If he stands alone, we can punish him the sooner."

"I understand about that, of course. But how about the murder case? Do you think Sid Prale is guilty of that crime?" Farland asked.

"I do not know, I am sure. I understand that the evidence against him is damaging. But we are not awaiting the outcome of that. He may manage to have the charge against him dismissed, and we are going ahead with our plans for punishment."

"Then you want me to quit Prale so I won't be helping him work against his enemies, and not because you are afraid that, in clearing him of the murder charge, I may find something detrimental to other persons?"

"That is the idea," the masked man replied. "The murder case can take care of itself, I suppose."

"Suppose I refuse to make this deal with you?"

"In that event, we may feel called upon to detain you---and perhaps to use further violence."

"Then you might as well start!" Jim Farland cried. "For you are lying to me like blazes! It's the murder case that's worrying you, and you know it! And I know you! I've been trying to place those hands of yours and I have succeeded. Besides, you have said one or two things that have convinced me------"

The masked man gave a shriek and started toward the couch, his hands reaching out, clutching. 2 of the thugs ran in from the hall.

\vspace{2\nbs}
\ChapterDeco[c1]{\decoglyph{e9665}}
\clearpage
\thispagestyle{empty}

\begin{ChapterStart}
\vspace{3\nbs}
\ChapterSubtitle[l]{Chapter ch22}
\ChapterTitle[l]{ch22}
\end{ChapterStart}
\FirstLine{\noindent ## An Unexpected Visitor}
    
Waiting in anticipation of hearing good news, Sidney Prale paced the floor of the living room of his hotel suite until noon the following day, expecting Jim Farland to put in an appearance at any time and make his report.

Murk, having done all the work that there was to do, spent the most of his time looking from the window at the busy, fashionable avenue, and glancing now and then at Prale as if wishing to anticipate his wishes and save him the trouble of voicing them.

Prale had luncheon served in the suite, and then he stepped to the telephone and called Jim Farland's office. Farland's stenographer informed him that the detective had not been there during the morning, though there was some business that needed his attention.

Then Prale got Farland's residence on the telephone, and the detective's wife answered the call. Prale gave his name, and asked where Jim could be found.

"That is more than I can tell, Mr. Prale," Mrs. Farland said. "He got a telephone call last evening, and from what I overheard I think he went some place to meet a man. He left soon after he received the call, and I have not heard from him since. That is peculiar, too. When he is obliged to remain away, he generally finds time to telephone and let me know."

This conversation bothered Sidney Prale, but he tried to tell himself that Farland was following a hot trail, and that perhaps it had led him some distance away, or that he was in a locality where he did not care to telephone.

He did not want to miss Farland if he did call, and so he remained at the hotel during the afternoon and kept Murk there also.

"I have a hunch that something is going to happen soon," Prale said to his valet.

"A little action wouldn't make me mad any!" Murk declared. "I'm spoilin' to mix with the enemy, Mr. Prale. Most of all, I'd like to meet up with them two thugs that got gay with us. You're sure about that Jim Farland, boss?"

"I've told you a hundred times, Murk, that Jim Farland is my friend and as square a man as you can find anywhere. He has not deserted us, if that is the thought in your head."

"I'm beginnin' to like him a bit myself," said Murk. "Ain't you got any idea, boss, who's engineerin' this deal against you?"

"Once more, Murk, old boy, allow me to state that I haven't the faintest idea who my enemies are, or why they are trying so hard to make life miserable for me. If I knew where to start to round them up, I wouldn't be standing in this room talking to you---I'd be out rounding them up!"

"Well, if you ask me, I think it's about time that Farland settled that murder case," Murk said. "If he don't get busy pretty quick, I'll tackle it myself. I've got an idea------"

The ringing of the telephone bell cut his sentence off. Sidney Prale was near the instrument, and he answered the call.

"Mr. Prale?" asked a man's voice.

"Talking."

"I just wanted to inform you that you needn't depend on Detective Jim Farland any more. We've got him---and we'll get anybody else you engage. And we'll get you, too, Mr. Prale, before very long. Don't think we'll not!"

The man at the other end of the wire hung up his receiver. Prale paced the floor and told Murk of the conversation.

"They've got Farland!" Prale exclaimed. "They probably got him last night, decoyed him in some way. Well, Murk, if that is the truth, and I imagine that it is, we'll have to do our sleuthing ourselves."

"Suits me!" Murk said. "I'm ready to start out right now and sleuth until it's settled. Let's get in action, boss!"

"We are in the same old quandary, Murk. We don't know where to start," Sidney Prale said. "If our foes would come out in the open, instead of fighting from the dark, we might have a chance. This is some city, Murk, and there are several million persons in it and around it. Starting right in such a maze isn't the easiest thing in the world, you know."

For the second time that afternoon, Murk was interrupted by the ringing of the telephone bell, and once more Sidney Prale happened to be near and answered the call.

"Send them up at once!" Murk heard him say.

And then Sidney Prale hung up the receiver and whirled around with a puzzled expression on his face.

"Murk," he said, "Miss Kate Gilbert is coming up here with that big maid of hers---coming to see me. What she wants is more than I can guess, remembering what happened the last time I talked with her. It may be good news, Murk!"

They waited impatiently for the ring at the door. Murk opened it and ushered them in.

He grinned at the gigantic Marie, but she did not return the compliment. There was a serious expression in her face, and Murk looked past her at Kate Gilbert, who was being greeted by Sidney Prale.

Something important had happened, Murk told himself immediately. Kate Gilbert did not look frightened exactly or sorrowful or triumphant. There was a peculiar expression about her mouth, and her face seemed pale.

"I felt that I had to come, Mr. Prale, and have this talk with you," Kate Gilbert said, when she was seated near the window. "I wanted to speak to you here instead of in some public place, and so I brought Marie and came to your suite."

"You are welcome, Miss Gilbert, I am sure," Prale said. "If you wish to speak in private, Marie and Murk can step into the adjoining room."

"Please," she said softly.

Murk opened the door, and the maid stepped in. Then he followed and closed the door again. Prale sat down near Kate Gilbert and turned toward her.

"Now, Miss Gilbert," he prompted.

She met his eyes squarely as she spoke, but her lips trembled at times as if she were undergoing an ordeal.

"Mr. Prale," she said, "as you know, I have been associated with others in an attempt to bring retribution home to you. When I became associated with them, it was understood between us that there was to be no violence, nothing outside the law. We were simply to attack you from every angle, cause you trouble and annoyance, take away your money if we could, break you in every way."

"Pardon me, but------"

"Please say nothing until I am finished, Mr. Prale. We began at once to gather all the information we could about you and your affairs. We began to plan for your downfall. We found that we could do nothing that amounted to anything while you were in Honduras, where you were a powerful man. But we were about to try, even there, when we learned that you were selling out your properties and preparing to return to New York.

"You may know how that struck us. You had gone away and made your fortune, and you were coming home, possibly with the hope that the past had been forgotten. We intended showing you that it had not been forgotten, that you could not return and enjoy the fortune whose foundation was------But enough of that!

"I had been in Honduras spying upon you. I was sent because you did not know me, and would not be on guard, as you might have been, had some man gone down there. We did not care to send an ordinary detective, of course. I kept the people here informed of all your movements. I began the punishment by leaving that note in your stateroom and pasting the other on your suit case, began it by reminding you that the past lived in the minds of some persons.

"You know the rest. We began our work. We caused you annoyance from the first, with the banker, the hotel manager, and all that. Before we could do any more, you were accused of murder. That pleased us, of course. We did not believe you guilty, but we were glad to see that you were being caused some trouble, that your name was being stained. Some of us even began to think that the law of retribution was at work itself, without our poor help.

"We went ahead with our plans, however. You engaged a prominent attorney, and finally we induced him to leave you. But some who were handling the affair went too far. You were assaulted in Central Park. Your valet was knocked on the head and kidnaped, and an attempt made to get him to take payment and spy upon you. At that time I told a certain man who had the handling of the affair that there could be no more violence.

"We should not break a law to undo you, I declared. If we did that, we were as bad as you. I said that, if there was any more violence, I should cease having anything to do with the affair, and would come to you and tell you so. An hour ago, I found out that Detective Farland, a man in your employ, had been seized and treated with violence and was being held prisoner because he insisted upon remaining loyal to you. So I am here!"

"This is amazing, Miss Gilbert!" Sidney Prale told her. "The whole thing has been amazing. Somebody has tried to connect me with that murder. Somebody tried to smash my alibi. The little annoyances were bad enough, and the knowledge that I had unknown foes who fought in the dark; but the murder charge was the worst of all, for it placed me in a position where I had to clear myself absolutely or remain forever suspected by many persons."

"I understand that," Kate Gilbert said.

"And now you have come to me to say that you are no longer associated with my enemies?"

"For what you did, there can be no forgiveness, Mr. Prale. I want to see you punished. But I will not be a party to violence. It seems to me that the man who has been managing this affair has gone beyond proper bounds. For some reason, he is particularly vindictive, though he did not suffer at all, as did some of the others. I cannot forgive you for what you did, Sidney Prale. But I can wash my hands of the entire affair and try to forget you entirely and hope that there is a law of retribution that will take vengeance for me. That is all, Mr. Prale. Only please remember that, from this hour, I am not concerned with the others in this affair."

She started to rise, but Prale motioned for her to retain her seat. He bent forward and looked at her searchingly.

"I am very glad that you have come here and spoken to me in this way, Miss Gilbert," he said. "I scarcely know how to express what I feel that I must tell you. I have listened to you patiently, without interruption. Will you be kind enough to listen to me for a moment now?"

"I'll listen, though it will be useless," she said.

"When I left Honduras, Miss Gilbert, I was a happy man. I had made my pile and was coming home. I had left ten years before because a selfish woman, whom I imagined I loved, jilted me for a wealthier man. That wound had healed, and when I left Honduras, I did not think that I had an enemy in the world, unless it was some poor devil of a disgruntled native workman I had been forced to discharge, or somebody like that.

"I believed those notes on the ship to be in the nature of a jest, or else that somebody was making a mistake. Then troubles began, and I was at a loss to understand them. Next came the murder charge! We will put that aside for the moment, for it seems to be the result of circumstantial evidence and probably has nothing to do with the other affair---merely a coincidence.

"Miss Gilbert, look at me! I want you to believe what I am going to say. You must believe it! In the name of everything I hold sacred, I swear to you that I do not know these foes of mine, or the reason for their enmity!"

"How can I believe that?" she cried. "Why should you ask me to believe such a statement?"

"Because I want some light on this subject, Miss Gilbert, and I am determined to get it. There is some terrible mistake. I am being punished for the fault of some other person."

"Can you not remember back ten years?" she asked.

"Easily. I can live over again the last day I spent in New York ten years ago."

"And the few days before that time?"

"Certainly, Miss Gilbert."

"And yet you ask why others should seek to punish you? Perhaps you are one of those men whose natures are so dishonorable that you think you did nothing wrong at that time."

"So it was then that I was supposed to have done this terrible thing---whatever it was?"

"As you know, Mr. Prale."

"But I do not know, Miss Gilbert. To the best of my recollection I left New York without having done anything in the least dishonorable; and certainly I did nothing to merit a band of enemies working against me."

"What is it that you wish me to do?" she asked.

"Be fair with me, Miss Gilbert. I tell you that there is some terrible mistake! If I am supposed to know all about this, what harm can there be in your repeating the details to me? Tell me what crime I am supposed to have committed to merit this attack. Give me a chance to prove my innocence! The common thug gets that chance in a court of law, you know."

"But this is ridiculous!" she exclaimed. "There can be no question of it! The whole thing came out at the time."

"Then you do not wish to be fair?" Prale asked.

"I cannot allow you to say that. I will tell the story to you, Mr. Prale, tell exactly what you did---as you know very well---if that will be any satisfaction to you. But it will do you no good to deny it!"

"Tell me!" Sidney Prale said.

\vspace{2\nbs}
\ChapterDeco[c1]{\decoglyph{e9665}}
\clearpage
\thispagestyle{empty}

\begin{ChapterStart}
\vspace{3\nbs}
\ChapterSubtitle[l]{Chapter ch23}
\ChapterTitle[l]{ch23}
\end{ChapterStart}
\FirstLine{\noindent ## A Startling Story}
    
"This is a painful subject for me, as you must be aware," Kate Gilbert said. "I shall tell the story in as few words as possible, and if you are a gentleman, you will not interrupt or cause me more suffering by protesting your innocence."

"I promise not to interrupt," Sidney Prale replied. "I want justice and nothing more, Miss Gilbert."

"10 years ago you were a clerk in the office of Griffin, the big broker, were you not?"

"Yes."

"Mr. Griffin took a fancy to you, after your father died and left you alone in the world without any money. He gave you odd jobs to do around his residence, fed and clothed you and arranged it so that you could go to school. Your uncle, the father of George Lerton, your cousin, would do nothing for you because there had been a family quarrel several years before.

"Had it not been for Mr. Griffin you might have been an ordinary street Arab. He sent you to a business college after you had finished the public schools, and then he took you into his office and started you on a business career.

"You showed great promise, and Mr. Griffin was delighted and advanced you rapidly. You seemed to know the meaning of gratitude and worked hard. You were ambitious, too---always said that some day you would be worth a million dollars.

"Step by step, you went up the ladder. Then it happened that your cousin, George Lerton, obtained a position in the same office after his father's death. He had had the advantage of a college education and knew how to handle himself in the presence of other men, and yet you, after your early struggle and with an inferior education and inferior opportunities, easily outdistanced him.

"Other men began talking about you as a coming man---bankers and brokers, business men and financiers. Mr. Griffin finally gave you the post of chief clerk and adviser. You worked hard and seemed to be loyal and faithful. You got profits for your employer where other men would have caused losses. So he let you more and more into his confidence.

"You got to know the secrets of big deals, the inside facts of the country's finance. You spoke in millions, but got only a nice salary. Your ambition to be worth a million dollars seemed to be not susceptible of gratification. Yet you saved money, and took advantage of small, solid investments now and then.

"After a while you met a girl and fell in love with her. She was the sort who wished wealth above all, and you soon found that out. You became engaged to her, however. Then a rival appeared in the field, a wealthier man. You realized that the girl was shallow in that she favored the man with more money, but you were so infatuated that you overlooked that. You wanted the girl and, to get her, you had to have more money.

"Then you began to feel dissatisfied. You didn't want to grow gradually, as other men did. You wanted the foundation for a fortune---enough to use in a plunge in the market. You wanted to be rich as soon as possible.

"You began to think, perhaps, that you were not getting ahead. You worked in an atmosphere of wealth, you heard men speak in terms of millions, while you had less than ten thousand dollars in the bank. You began to think that Mr. Griffin should do more for you, that he had not done enough. You forgot that he had picked you up and made you what you were, that you had so much more than other men who had not been equally fortunate in finding a sponsor."

She ceased speaking for a moment, but Sidney Prale never took his eyes from her face. Be ungrateful to Griffin? He never had dreamed of that! He always had worshiped Griffin for what the broker had done for him; he realized what he might have been only for Griffin. But he had promised not to interrupt, and so he said nothing, merely waited for Kate Gilbert to continue her recital.

"You made certain plans," she went on. "Certain big business deals were in the wind, and, as Mr. Griffin's confidential and chief clerk, you knew all about them. There were millions of dollars involved, the control of several large companies, and more than that; for Mr. Griffin and his associates were fighting a group of financial thieves who were trying to wreck excellent properties for the sake of making a gain. It was a fight for more than money---it was a fight to keep big business honest, to drive off the wolves and make finance solid. It was a tremendous thing!

"And you, a boy picked up and educated by a broker, who had risen through his kindness, knew as much of the big deal contemplated as some of the wealthiest and most influential men of the country. There were men in the other group who would have given a million gladly to know what you, a clerk, knew.

"You were approached by one of that band of financial wolves. You were willing to listen. You wanted money because the girl with whom you were infatuated demanded it before she would marry you. You believed that Griffin had not done enough for you and you agreed to sell him out---him and his associates."

Sidney Prale gasped, sat up straight in his chair, opened his mouth as if to speak, but did not when he saw the expression in her face. He decided to keep his word.

"The agreement was made," she went on. "And you, who could have demanded half a million easily for the information you had, sold out your benefactor and his friends and the decent element on the Street for a paltry hundred thousand! You sold your honor and your manhood for that.

"At this juncture, the woman in the case informed you that she wished to break the engagement, because a man of money---your rival---had asked her to marry him, and she wanted his wealth. Instead of seeing what sort of woman she was---instead of coming to your senses then and stopping your deal with the other side---you took the opposite course. You would take the money, betray your benefactor and his friends, and leave the country! With that money as a foundation, you would build up a fortune. And that is what you did, Sidney Prale!

"You arranged everything nicely. You gave those men the information and received your hundred thousand and then you quit your job and sailed away to Honduras.

"The battle began on the Street, and because of the information you had sold them, the financial wolves got the better of the honest element. It was a battle that lasted for two weeks. The wolves met every move, because they knew everything that had been planned. Fortunes were lost overnight. A score of big, decent men were ruined in their attempt to defeat the wolves and keep finance clean.

"Mr. Griffin, the man who had done everything for you, went down in the crash---because you had sold him out! It was only five years ago that he got new backing and fought his way up again. Others went down with him, and some never regained their footing---because of what you had done, because you had played traitor! They knew there had been a leak, and there was an investigation. You had sailed away the day before the fight began, and that looked suspicious, for you had made up your mind suddenly. Finally it was discovered that you were the traitor in the camp!

"My father was one of Mr. Griffin's associates, Mr. Prale. He lost his fortune, of course. We could have endured that, but the blow cost him his health. He was a giant of a man at that time, the best father in the world. You should see him now, Mr. Prale---see what your treason made of him. He is an invalid who sits all day in his wheel chair. At times his mind wanders and he fights that battle over again and calls curses down upon the head of the man who played traitor! My big, handsome, rich father is a broken, thin-faced man whose voice is a whisper and whose hands tremble---because of what you did. You beast!"

She began sobbing softly as she glanced through the window, and Sidney Prale started to get out of his chair. But she faced him again quickly and motioned for him to remain silent.

"You wanted to hear it, and so I shall tell it all!" she declared. "You had been clever; you had done this thing in such a manner than the law could not touch you. Yet you must have been afraid of it, for you fled the country. It was some time before things were adjusted, and then those men you had betrayed got together and determined to make you pay!

"They told the story to others, and they began gathering information about you. You were making your million, all right, on the foundation that had wrecked a score of fortunes and lives---on treason instead of superior financial ability---and they swore that you should pay.

"They knew my father's story, of course, and knew that we had very little money. So they provided for him, and gave me funds and sent me to Honduras to spy upon you. Marie, my maid since girlhood, who worshiped my father and knew all the circumstances, went with me. Soon after I reached Honduras, I found that you were selling out with the intention of returning to New York and enjoying your million.

"I communicated with the others and told them all I knew of your plans, whereupon they made some plans of their own. They won the sympathy of the most influential men in the city. They determined to make you pay!

"That is why the big trust company would not accept your account. A whisper in the ear of the hotel manager by the president of the company that owned the hotel, and you were as good as ordered out. Can you understand now, Sidney Prale? Coadley, the lawyer, was told that he will be made a nobody by the influential men of the town unless he ceased to work for you, and he dropped your case.

"But there was to be no violence, and because they have descended to that, I have ceased to be interested in the affair. I know nothing about the Shepley murder case or any trouble it may have caused you. That is quite another matter. Now that I have told my story, I hope that you are satisfied. It has shown you, I trust, that I know all, and that any falsehood you may utter will have no effect on me."

"I do not intend uttering a falsehood, Miss Gilbert," Sidney Prale assured her. "What you have said has amazed and shocked me. So that is why I was treated so badly upon returning to my home?"

"Exactly," she said.

"Now listen to me one moment, I beg of you. There is some mystery here, and though it is ten years old, I shall solve it. Miss Gilbert---whether you believe me or not---I am not guilty of such treachery. I had no dealings with the financial wolves. When I left the United States I took with me the ten thousand dollars I had saved---nothing more. And I left nothing behind."

"You made a million in ten years with a capital of ten thousand?" she asked, with a slight sneer.

"I did, Miss Gilbert! I can prove every transaction, show you or anybody else exactly how I did it. Disbelieve me or not, it is the truth that I am innocent. If my people were sold out at that time, somebody else got the selling price. I was chagrined because my love affair had gone wrong. I shook the dust of New York from my feet. I did not even look at a New York newspaper for more than a year. Somebody else got the money, and I got a nasty name. And Mr. Griffin, who was as a father to me, thinks that I was an ungrateful cur!

"This thing is hard to believe, Miss Gilbert. But I never can thank you enough for telling me. I am going to clear myself before I am done."

"I cannot believe you, Mr. Prale! The proof was there!"

"And who furnished it?" he demanded. "Who is handling this campaign of vengeance against me now?"

"You scarcely can expect me to tell you that," she said. "I am done---have nothing more to do with the affair---but I am not going to be a traitor, as you were!"

"If you ever are convinced, Miss Gilbert, that I am entirely innocent, that somebody has put this stain upon me for their own reasons, can I count upon your friendship?"

"Convince me that injustice has been done you, Mr. Prale, and I'll do everything in my power to make amends---and so will all the others!"

"Thanks for that assurance," Prale said. "I am going to clear myself in your eyes, and in the eyes of the others. I remember the details of that big deal perfectly and I shall know how to start to work."

"I cannot understand this," she said. "You speak as if you were indeed innocent, but I cannot believe it!"

"I am innocent!"

"If so, who is guilty?"

"That is what I intend finding out."

"But you were in their confidence---you knew all the details of their financial plans," Kate Gilbert said. "You were the only one who could have betrayed them. You scarcely expect me to believe that they betrayed themselves."

"Any spying clerk in the Griffin offices could have told the enemy enough to betray the plans," Prale replied. "By the way, who is this man who goes too far and insists upon using violence? Who is the man who seems to be so extraordinary vindictive toward me in this affair?"

"I can tell you nothing more," she declared. "It would not be fair to them."

"But they have Jim Farland, and Heaven knows what they are doing to him, simply because he will not turn against me. Is it fair to Jim Farland's wife and child?"

"I---I am being kept informed," she assured him. "If they treat Mr. Farland badly, or detain him much longer, I shall speak. But until then, I have nothing to say. You see, Mr. Prale, I cannot believe that you are innocent and have been misjudged. The evidence against you is so conclusive, and I have learned to hate you as the man who betrayed his benefactor and friends and wrecked my father's health. But, if you are innocent, I hope that you will forgive me."

"I'll forgive you gladly," said Sidney Prale. "I realize what you must have suffered, and what your father must have suffered, too. I am going to prove my innocence; and then I hope to claim you as one of my friends."

"I am sorry that I cannot believe you," she said again, "although I would like to. I would prefer to think that no man could be so ungrateful as to do such a thing. I'd like to have my faith in human nature restored. If you prove your innocence, I shall be very glad indeed!"

Then she called for Marie, and when the maid came from the adjoining room, Sidney Prale ushered the two women to the door and watched as they went down the hall toward the elevator. But Kate Gilbert did not glance back.

\vspace{2\nbs}
\ChapterDeco[c1]{\decoglyph{e9665}}
\clearpage
\thispagestyle{empty}

\begin{ChapterStart}
\vspace{3\nbs}
\ChapterSubtitle[l]{Chapter ch24}
\ChapterTitle[l]{ch24}
\end{ChapterStart}
\FirstLine{\noindent ## High-Handed Methods}
    
Sidney Prale closed the door and turned around to face a grinning Murk.

"Some pair of chickens!" Murk said. "That Marie girl may be a bear for size and strength, but she's got a lot of good common sense. I'm strong for her!"

"Sit down!" Prale commanded.

And then, walking up and down across the room, he told Murk what Kate Gilbert had revealed to him, simply because he felt that he had to tell it to somebody.

"How is that for a dirty deal, Murk?" he asked when he had finished. "Doesn't that make ordinary dirty work look rather pale?"

"Who did it, boss? Name the gent, and I'll get his address out of the city directory and pay him a visit!" Murk said. "I'll have some things to say to him---and some things to do, maybe."

"I'm a sort of husky individual myself, Murk, and, if I knew him, I think I'd beat you to it," Prale replied. "Now we must get busy!"

"Just say the word, Mr. Prale. What is it to be?"

"I haven't quite decided yet, Murk. How far will you go?"

"I'll croak him, if it's necessary!"

"That'd be a bit too far, Murk, and might lead to the electric chair and a far country. Let's take a walk and think it over. We will confine ourselves to the Avenue, and you may trail me as before. I scarcely think they'll assault us on the Avenue."

10 minutes later, Sidney Prale was walking down the street, and the faithful Murk was trailing in his wake, watching carefully. That walk lasted for an hour. Then they returned to the hotel and Prale ordered an early dinner. He did not say what he had decided to do, despite Murk's hints that he should state his plans.

But Murk had noticed that Prale had stopped in at a printing office during the walk, and shortly after they finished dinner, a bell boy brought a small package to the suite. Prale unwrapped it, and some cards spilled out.

"Nice cards, Murk," he said. "I had them printed this afternoon. They bear the name of Horace Greenman, whoever he may be, and state that he is connected with the General Utilities Company---whatever that is."

"What's the big idea, Mr. Prale?" Murk asked wonderingly.

"I wish to get into a certain place, Murk, and I'd never do it if I send in my own card. What time is it?"

"A few minutes of eight, sir."

"Then we'll be going. Let us hope that we find our man at home. If this happens to be his opera or theater evening, we are going to be delayed."

Murk followed him down in the elevator and to the street, where Prale engaged a taxicab. The machine took them up past the Park and to an exclusive residence section, where it stopped on a corner. Prale and Murk got out, and Prale instructed the chauffeur to wait. Then he led the way to the middle of the block.

"Murk, you remain just outside this gate," he instructed. "If I have good luck, I'll come out with a man, and I may want to take him with us. Be ready to help in case I get in wrong."

"Sure thing, sir," Murk said.

Prale passed through the gate, went up the walk, and lifted the knocker on the front door. A moment, and a servant appeared and looked at him searchingly.

"I wish to see Mr. Griffin at once on important business," Prale said. "Kindly take my card to him."

Then Prale waited with his heart in his mouth. Was Griffin at home? The servant instantly assured him of that, and carried the card away. Prale had written "Important Business" on it.

The servant returned soon and announced that Mr. Griffin would see the visitor. Prale followed him down the hall to the library. He was glad that Griffin had chosen to receive him there, for there was less likelihood of an interruption. The servant opened the door, and Sidney Prale stepped inside.

Griffin was sitting beside the long table, and he arose immediately and turned.

"You!" he gasped.

"Pardon the deception------"

"James! James!" Griffin thundered.

The servant was in the room instantly.

"Show this fellow the door!" Griffin commanded. "Look at him well, and never admit him again!"

James took a step forward and indicated the door. But Sidney Prale reached into the pocket of his coat, drew out an automatic pistol, and held it menacingly.

"Close the door, James---softly!" he commanded in a stern voice. "Now advance to the table and stand where I can watch you. Don't you make a move, Mr. Griffin! I used to handle men down in Honduras, and I feel confident that I can take care of this situation."

"You thug!" Griffin cried. "I'll have you sent up for this, Prale, if it's the last thing I do!"

"I know that it is against the law to be carrying a gun without a permit, but this situation demands a show of force," Prale said. "I merely want you to listen to me for a moment, Mr. Griffin."

"I don't want to hear anything you may have to say to me, Sidney Prale!" the financier said.

"You are going to hear it, nevertheless! Mr. Griffin, I did not know until this afternoon why I had secret enemies and why they were trying to cause me endless trouble. Miss Kate Gilbert was kind enough to enlighten me."

"Well, sir?"

"I am sorry that you believe me guilty of such base ingratitude to you and of such dishonorable conduct, for I am not guilty, Mr. Griffin! You were like a father to me---which was enough to compel my loyalty---and, aside from that, you had taught me several things regarding honor in business deals. I went away on the spur of the moment because a woman had jilted me. But before I went, I did not betray you and your associates."

"A likely story!"

"But a true one, Mr. Griffin! I did not sell you out for a hundred thousand dollars or any other sum. My conscience is clear, and I came back to New York expecting to greet old friends and have a pleasant time. You know what I found instead of that happy state of affairs. I am not here to talk at length. I demand a chance to prove my innocence!"

"How can you do the impossible, sir?"

"It is not the impossible, Mr. Griffin! I intend to prove to you that I was not disloyal, and then I shall prove that I had nothing to do with the murder of Rufus Shepley. I have an idea, sir, what is behind all this."

"We are wasting time------"

"I think not, sir! Time is not wasted in which a man shows that he is not a scoundrel! I think you owe it to me to give me a chance. You have condemned me unheard."

"I would give almost anything to have you prove your innocence," Griffin said. "You don't know how it hurt me. But the case against you was so strong---and is so strong------"

"Let us waste no more time," Prale said. "I remember the details of the big deal that was under way when I left New York ten years ago. If you recall, sir, I helped plan the campaign. If I can look at papers in your office, I think I can show that I am not guilty."

"I'd like to believe you, but this is preposterous!" Griffin cried. "I tell you the evidence------"

"It probably was strong, because the guilty man wanted to make it so. Mr. Griffin, were I guilty I should not be here. Please give me a few minutes, and let us talk this over. Then, if you wish, we can go to your office and continue the investigation."

Griffin sat down and motioned for Sidney Prale to do the same. Prale returned the automatic to his pocket, much to the relief of the servant.

Murk, standing outside by the gate, paced back and forth and wondered whether he should attempt to take the house by storm and rescue his employer. The chauffeur, waiting at the corner, wondered whether his fare had slipped down the next street without paying the bill. Murk relieved him on that point and threatened to beat him up because he intimated that Prale might do such a thing.

It was more than two hours later when Prale left the house and went out to the street. He paid the chauffeur and dismissed him, and told Murk to return to the hotel. Then he went back into the house and joined Mr. Griffin again, and after Griffin had telephoned several persons, he ordered his car, got into it with Prale, and started downtown.

An astonished watchman took them up in an elevator in an office building in the financial district, and a little later he took up several other gentlemen.

"Them financiers make me sick!" the watchman told himself. "Why can't they lay their schemes in the daytime?"

It was almost dawn when they left the building and scattered. They had spent hours investigating books and papers. Sidney Prale had even sent a messenger to the hotel with an order to Murk for certain books and papers of his own, and these had been investigated, too.

"And there we are, gentlemen," Prale had said, at the last. "I have shown you, I think, that I did not do this thing. I do not want you to believe me fully until I have proved my innocence by revealing the man who is guilty. I merely ask you to give me a fair chance to prove my case. I have told you my suspicions. Now it is up to me to demonstrate whether they are just or worthless."

Griffin had little to say as they rode back uptown. But when he dropped Prale at the hotel just before daylight, he gripped him by the hand.

"I want to believe you, Sidney!" he said. "I hope that you have told me the truth. If you have, I hope you'll be able to clear yourself. If you only can show me that the boy I was glad to help was not ungrateful, after all------"

"I'll do it, sir!"

"And then I'll never forgive myself, Sidney!"

"You'll show your forgiveness by handling my affairs for me, sir, in that event, and by treating me as your son again!" Prale said.

He hurried up to the suite. Murk had been sleeping in a chair in the living room, as if expecting a call at any moment. He was somewhat startled to hear Sidney Prale whistling merrily at four o'clock in the morning.

\vspace{2\nbs}
\ChapterDeco[c1]{\decoglyph{e9665}}
\clearpage
\thispagestyle{empty}

\begin{ChapterStart}
\vspace{3\nbs}
\ChapterSubtitle[l]{Chapter ch25}
\ChapterTitle[l]{ch25}
\end{ChapterStart}
\FirstLine{\noindent ## An Accusation}
    
Springing toward him, the masked man stopped two feet from the bound Jim Farland.

"So you think you know me, do you?" he snarled.

"I have a pretty good idea," Farland said. "There are only a few men in the city, to my knowledge, who could be hired to do work like this, and it occurs to me that I have seen those hands of yours before. I think your face is in the rogues' gallery, too, if you want to know!"

The masked man retreated for a few feet, evidently relieved.

"So you'll not make terms with me," he said. "You'd rather work for Sidney Prale, would you? Perhaps we can change your mind."

"I doubt that like blazes!"

"You are going to be kept here as a prisoner until I decide what is to be done with you."

He crossed over to the door, opened it, and called to his men, two of whom responded.

"I want this man guarded well," he said. "I want you to understand that I am holding you responsible for him. I'll be back to-morrow evening and have another talk with him. Give him something to eat now and then, and fix him so he can sleep, but watch him all the time!"

"I was figurin' on goin' to the city this mornin', boss," one of the men spoke up.

"You'll do as I say!" the masked man cried.

"But------"

"Don't argue with me, you dog!"

Farland saw the man's eyes flash fire for a moment. And then the masked man faced toward him again, his eyes glittering through his mask.

"Sometimes it isn't healthy to know whose picture is in the rogues' gallery!" he said.

He went from the room. After a short argument one of the men remained to guard Farland, and the other went away. Farland spent a night of agony. His guards fixed the bonds so that he could be a bit more comfortable, and yet he got little sleep.

Jim Farland was considering a big idea now. He had thrown the masked man off guard by intimating that he might be a crook with a record, when, as a matter of fact, the detective did not believe him to be anything of the sort. Now Farland knew where to begin working, but he had to win his freedom first.

Night passed, morning came, and the long day of agony began. Farland had his hands untied and was given some food. Then his wrists were lashed again and his ankles loosened, and he was allowed to walk around the room for an hour or so, two of the men watching him closely. The one to whom the masked man had applied the epithet, "dog," appeared surly.

After they had bound him again and stretched him upon the couch, they guarded him one at a time, evidently secure in the belief that he could not escape. Jim Farland thought a day never had seemed so long. All the time he was busy with his thoughts. He had a plan of campaign outlined now; he wanted to be at work.

Once more the evening came. Farland, who had been sleeping for a few minutes, awoke and turned over to find that his guard had been changed again. The man who had been called a dog was on duty.

"How long are you going to keep me tied up like this?" Jim Farland asked.

"Don't ask me. Ask the high and mighty boss," was the sneering reply.

"You don't seem to stand very high with him."

"Aw, he makes me sick sometimes."

"It'd make me sick, too, if anybody called me a dog," Farland declared.

The man before him did not reply to that, but Farland could see the anger burning in his face.

"Come closer," Farland whispered.

The man obeyed instantly.

"Can anybody overhear what I say to you?"

"No. Everybody's gone---but they'll be back soon."

"Why are you working for these people?"

"Coin, of course---and precious little of it I've seen so far," was the reply.

"Then you haven't any other interest in this business? Maybe we can make a deal."

"What sort of a deal?"

"The man I work for is worth a million," Farland said. "Help me escape, and I'll give you five hundred dollars."

"Got it with you?"

"The biggest part of it," Farland replied.

He told the truth, too, for he always carried plenty of money while working on a case.

"Suppose I simply take it away from you," the guard said.

"In the first place, I don't think you are that kind of a man. And you want to get square with the man who called you a dog, don't you?"

"What's your scheme?"

"Simply let me go, right now. It is dusk outside already. Tell me how to get to town the quickest way. I'll give you almost all I have on me; I'll need a little to use to get back to the city. To-morrow I'll meet you some place and give you the rest. In addition I'll give you a chance to get out without being arrested for your part in abducting me and holding me here."

The man spent a few minutes in thought.

"I'll fix you so you can slip your bonds," he said, "and I'll hand your automatic back to you. It is there in the cupboard. But I don't want you to make a get-away while I'm guarding you---see? I don't exactly love the man who'll guard you next. I'll fix it so you can handle him. Wait for five minutes after he comes and I have gone. I will be away for an hour or so, and the escape can happen while I'm not here."

"That suits me," Farland said.

"What about the money?"

"You'll get it just as soon as I get my hands loose."

The guard walked to the hall door and opened it, peered out into the hall and listened. Then he hurried back to the couch and cut Jim Farland's bonds. Farland took the money from one of his inside pockets and handed it over. The guard got the weapon from the cupboard and gave it to Farland.

The detective stretched himself down on the couch again, and the guard adjusted the ropes on his ankles and wrists so that they would appear to be all right. Farland slipped the automatic beneath the small of his back, where he could reach it quickly.

It was half an hour later before the guard was changed and Farland's friend hurried away, warning him with a glance that he should not make a move too soon. He had declined to meet the detective the following day and get the few dollars still due him; he would rather use what he already had in getting out of town, he had said.

Farland made no attempt to talk with the new guard. He pretended to be tired, almost exhausted and sleepy. The guard sat beside the table, smoking and glancing at a newspaper now and then, apparently of the opinion that Farland was safely a prisoner.

After waiting for about half an hour, the detective began moving his ankles and wrists gently. Gradually the ropes fell away. He reached one hand beneath his back and grasped the automatic. Then he sat up quickly on the couch and covered the guard.

"Put 'em up!" he commanded.

The guard whirled from the table and sprang to his feet, surprise written on his countenance. Farland had arisen now, and advancing toward him.

"Walk past me to the couch!" the detective commanded.

The guard started to obey. He was holding his hands above his head and seemed to be afraid that his captor would shoot. But as he came opposite Farland, he lurched to one side and made an attempt to grapple with him.

The detective did not fire. He sprang aside himself, swung the automatic, and crashed it against the other man's temple. The guard groaned once and dropped to the floor.

"Thought you might try something like that!" Jim Farland growled. "Couldn't have pleased me better---won't have to waste time tying you up now. You'll be dead to the world for a few minutes at least!"

Farland darted to the door, opened it, went into the hall and closed the door again. He passed through the house noiselessly. He could hear two men in conversation in a rear room, and he knew that he would have to be cautious until he was at some distance from the old dwelling, unless he wanted a battle on his hands.

He got out of the place without being discovered, and reached the edge of a grove not far away. There he found the lane, and near the end of it was a powerful roadster, its engine dead and its lights extinguished.

Farland listened a moment, then went forward and examined the machine. He knew the model, and he was an excellent driver. Once more he stopped to listen. Then he sprang behind the wheel and operated the starter.

He drove slowly down the lane, the engine almost silent, the car traveling slowly. He proceeded in that manner until he had reached the highway. There he switched on the lights, put on speed, and sent the powerful car roaring along the winding road toward the river.

Jim Farland, being a modest man, never did tell the entire story of that night. He drove like a fiend, narrowly escaping collision a score of times. He made his way along the roads running alongside the broad river, and finally came opposite the city. He crossed over a bridge, drove through the streets with what speed he dared, left the car at a public garage with certain instructions, and hurried to a telephone.

He was unable to get either Sidney Prale or Murk, for at that hour they were on their way to the Griffin residence. Farland telephoned to his wife to say that he was all right, but would not be home until some time during the day. Then he engaged a taxicab and began his work.

He knew where to start now. An idea had come to him in that old house far up the river, a suspicion, a feeling of certainty that he was on the right track. Jim Farland was no respecter of persons that night.

When morning came he stopped only for a cup of coffee, and then worked on. He dashed from one place to another, running up a taxicab bill that made the chauffeur smile. He interviewed important gentlemen, threatening some and cajoling others, but always getting the information that he desired.

At two o'clock the following afternoon he stood on a certain corner near Madison Square, his suspicion almost proved, his investigation at an end.

"Now for the big bluff!" Jim Farland said to himself.

He fortified himself with another cup of coffee, got into the taxicab again, and started downtown. He was smoking one of his big, black cigars, puffing at it as if in deep contentment, not looking at all like a man who had been kept a prisoner a night and a day, and had been busy since that experience.

The taxicab stopped before an office building, as Jim Farland had ordered. The detective pulled out his last money and paid the chauffeur.

"You're got more coming, son, but this is all I have with me," Farland said. "Drop in at my office any time after ten to-morrow morning and get it."

"Yes, Mr. Farland---and thanks!"

"You're a good boy, but keep your mouth shut!" Farland told him.

Then he hurried into the office building, went to the elevator nearest the entrance, and ascended to the floor where George Lerton had his suite of offices.

The office boy stepped to the railing.

"Mr. Lerton busy?" Farland asked.

"He is alone in his private office, sir," said the boy, who regarded the detective with admiration and awe. After Farland's other visit, the youth had decided to be a detective when he grew up.

"I am to go right in---important business," Farland said. "Never mind announcing me."

The willing boy opened the gate, and Farland hurried across to the door of the private office. He paused there a moment and seemed to pull himself together, as if making sure before entering the room of questions he wanted to ask and information he wanted to gather. Then he threw the door open, stepped quickly inside, closed the door, and turned the key.

Lerton was sitting at his desk with his back to the door. He made no move until he heard the key turned. Then he whirled around in his desk chair.

"I---Great Scott, Farland, how you startled me!" he exclaimed. "I thought it was my secretary."

"Pardon me for butting in this way, but I am in a deuce of a hurry and told the boy it was all right," Farland said.

"You'll smash my office discipline doing things like this. But, sit down, man! What is it now? Has that cousin of mine been acting up again, or are you going to pester me with a lot of fool questions about things I don't know anything about?"

Farland had seated himself in the chair at the end of the desk, within four feet of George Lerton. He had tossed his hat to a table and twisted the cigar into one corner of his mouth. Now he stared Lerton straight in the eyes.

"You look like a madman!" Lerton said. "Why on earth are you looking at me like that? You look as if you were ill------"

The expression in Farland's face made him stop, and he appeared to be a bit disconcerted.

"Why did you kill Rufus Shepley?" Jim Farland demanded suddenly in a voice that seemed to sting.

Lerton's face went white for an instant. His jaw dropped and his eyes bulged.

"Are---are you insane?" he gasped. "What on earth do you mean by this? I'll call a clerk and------"

"The door is locked," Farland said, taking the automatic from his pocket. "You raise your voice, touch a button or make any move that I do not like, and I'll plug you and say afterward that I had placed you under arrest and had to shoot when you tried to escape. Answer my question, Lerton! You are at the end of your rope! Why did you kill Rufus Shepley and then try to hang the crime on your cousin, Sidney Prale?"

"This is preposterous!" Lerton exclaimed.

"Oh, I've got the goods on you, Lerton! I wouldn't be here talking like this if I didn't! You're going to the electric chair!"

Lerton laughed rather nervously. "I always thought that you were a good detective, Jim, but I am beginning to have doubts now," he said. "What has put such an idea into your head?"

"Facts gathered and welded together," Farland told him. "Don't try to carry out the bluff any longer, Lerton. And don't call me Jim. I never allow murderers to get familiar with me!"

"This has gone far enough!" the broker exclaimed. "I'll have to ask you to leave my office, sir!"

"I expect to do that little thing before long, and you are going with me," Farland said.

There was a knock at the door.

\vspace{2\nbs}
\ChapterDeco[c1]{\decoglyph{e9665}}
\clearpage
\thispagestyle{empty}

\begin{ChapterStart}
\vspace{3\nbs}
\ChapterSubtitle[l]{Chapter ch26}
\ChapterTitle[l]{ch26}
\end{ChapterStart}
\FirstLine{\noindent ## The Truth Comes Out}
    
Farland did not take his eyes off George Lerton.

"If you have touched a button and called some fool clerk, I'll manhandle you!" he promised. "Kindly consider yourself a prisoner!"

The knock was repeated, and Farland, still keeping his eyes on the man at the desk, backed to the door and turned the key. Then he took up a position where he could continue watching George Lerton and keep an eye on the door at the same time.

"Come in!" he called.

The door was hurled open. At the same instant, the office boy who had opened it was thrust aside. Sidney Prale sprang into the private office and stood glaring at his cousin. Behind him was Murk, and behind Murk were Kate Gilbert and her maid.

"Quite a gathering!" Farland said, grinning. "I'm glad that you are here. Kindly close and lock the door, Murk, with that young office gentleman on the outside!"

Murk obeyed. George Lerton sprang to his feet.

"What is the meaning of this intrusion?" he demanded. "Has my office been turned into a rendezvous for maniacs?"

"Sit down!" Sidney Prale cried. He had not taken his eyes off Lerton, had not even turned to speak to Jim Farland, had not even wondered how Farland had escaped and come here.

Lerton dropped back into his chair, wetting his thin lips, his eyes furtive now.

"You miserable cur!" Sidney Prale went on, advancing toward his cousin. "I should handle this affair myself. I should have you in Honduras, and fasten you to a tree and beat you until you are senseless."

"These insults------"

"Are deserved, you beast!" Prale cried. "So, when I went away ten years ago, you sold out Mr. Griffin and put the blame for it on me, did you? You wrecked that good man's faith in me, turned influential men against me, had me persecuted when I returned."

Jim Farland gave a shout of delight. "That right, Sid?" he cried, "Then I have the connecting link! So George Lerton has been causing you all this trouble, has he? I understand a lot more now. Lerton killed Rufus Shepley, also!"

"It's a lie! You are trying to save Prale by accusing me!" Lerton cried.

"Why, we've got you, you weak fool!" said Farland. "I knew you in that old farmhouse despite your mask. Your hands gave you away---I recognized them."

"And he's the man who tried to bribe me!" Murk cried. "I can tell it by his hands, too!"

"You tried to smash Prale's alibi," Jim Farland continued. "You had him followed that night and you sent those notes to the barber and the clothing merchant, with money in them."

"And you betrayed yourself when you began using violence," Prale put in. "You were too vindictive. You showed that you had some good reason of your own for wanting to drive me away from New York quickly!"

"Oh, we've got you!" Farland repeated. "You are as good as in the electric chair now!"

George Lerton looked as if he might have been in it. He was breathing in gasps, and his face was white. His eyes held an expression of terror.

"I guess---you've got me!" he said. "But I'll never---go to the chair!"

Farland stepped across to him. "Get it off your chest!" he suggested.

"I---I'll talk about it---yes!" George Lerton said. "I---I sold out Griffin. I wanted money, and I hated Griffin because he had put Sidney Prale over me. Then Sid had his trouble with the girl and ran away. I fixed things so it looked as if he had been the guilty one.

"I pretended to hate Sid for what he was supposed to have done. I suggested the scheme of vengeance, and worked to get the influential men together. Then he came back---with his million. I hated him all the more because of that. I was afraid that, if he remained in New York, he would find out the truth and I'd be exposed. I knew what that would mean, and I was beginning to get rich.

"So I had him followed and watched. I trailed him myself and met him on Fifth Avenue, and tried to get him to go away, and afterward denied that I had seen him at all, for he was accused of the murder of Rufus Shepley."

"Which was your deed!" Farland put in. "Go ahead---tell it all. Let us see whether you were clever or merely an amateur at crime."

"Oh, I was clever enough!" Lerton boasted. "I---I killed Shepley because he was about to have me arrested for embezzlement. I had been handling a vast sum for him, aside from his regular business. While he was traveling, I speculated with the money---and lost. He knew it. I could not repay.

"I had an engagement with him that night at the hotel. The detective I had working for me had reported that Sid had had a quarrel with Shepley, and where he had gone afterward and what he had done. There I saw my chance.

"I did not have myself announced at Shepley's hotel. I knew where his suite was, so I slipped up to it without anybody seeing me, and knocked at the door. He admitted me. I begged him to give me a little time to repay the money, but he would not. He called me a thief, and said that I must go to prison, that he would not have a hand in letting me remain at liberty to rob other men.

"There was a steel letter opener on the table. I---I stabbed him with it, and then I got away by the fire escape. Nobody saw me. I left him there dead. I was almost frantic when I reached home. Then I saw how I could have Sidney Prale accused and remove the menace of his presence also. I would be safe if Prale were convicted of the murder. I would not have to repay the Shepley money, and Prale never could reveal that I had betrayed Mr. Griffin and the others instead of him.

"So I sent the notes and money to the barber and clothing merchant, and they denied that Prale had visited them, thus smashing his alibi. I denied that I had met him on the Avenue. I thought that I was safe. But the barber and merchant told Farland the truth, and the police began to think that Sid was not guilty.

"I grew almost frantic then. My one hope was in running Sid out of town as quickly as possible, and so I did everything I could think of to bring about that end."

"How about that fountain pen found beside the body?" Farland asked.

"When I was talking to Sid that night on the Avenue, his coat was open and I saw the pen. Something seemed to tell me to take it, that it might be used against him some time. As I clutched his lapel, begging him to leave town, I took the pen from his pocket."

"Nothing but a plain dip, after all!" Farland sneered.

"I dropped it beside the body after I had killed Shepley. It was a part of my plan. And---and I guess that is all!"

"I guess it is!" Sidney Prale said. "Mr. Griffin and I, and some other men, made a little investigation last night and continued it this morning. We found that you were the traitor who caused that financial smash ten years ago. It may please you to know that Mr. Griffin is my friend again, and that others are being informed of my innocence. Even Coadley has come to me and asked to take my case again. But I was clearing myself of the charge of business treason, and nothing more. I did not connect you with the murder of Shepley."

"Well, I did connect him with it," Farland put in. "But when I sprung it on him here this afternoon, I was running a bluff. I had some evidence, but not enough to convict. You might have got away with it, Lerton, if you had had any nerve. But you happen to be a rank coward---and a guilty man!"

"You---you------" George Lerton gasped.

He had been holding two fingers in a pocket of his waistcoat. Now he withdrew them and, before Farland could reach him, he had swallowed something.

"You'll never------" he began, and then his head fell forward to the desk. "Get the ladies outside, Murk!" Farland commanded suddenly. "And tell that secretary out there to send in a call for a physician and the police. Lerton was right---he'll never go to the electric chair!"

\*\*\*

10 minutes later, Sidney Prale and Murk were waiting for the elevator with Kate Gilbert and Marie, but each couple was standing at some distance from the other.

"I have proved my innocence, and now I ask you to remember your promise and grant me your friendship," Prale was telling Kate Gilbert.

"I shall remember," she said. "You have my address, haven't you? If you haven't, ask Murk. He knows it. You sent him to spy on me, remember."

"Jim Farland did that," Prale protested.

Murk was talking to the gigantic Marie at that moment.

"You're mighty nice!" he was saying. "Say, I'd like to see you some more. I've got an idea my boss will be calling on your mistress, and when he does I might come up to the corner, and you might slip out and meet me, and we might take a walk in the Park. You wouldn't want to stay in the apartment and bother them, would you?"

"It would be a shame!" said Marie. "Which corner, Murk?"

\vspace{2\nbs}
\ChapterDeco[c1]{\decoglyph{e9665}}
\clearpage
\thispagestyle{empty}

\begin{ChapterStart}
\vspace{3\nbs}
\ChapterSubtitle[l]{Chapter ch3}
\ChapterTitle[l]{ch3}
\end{ChapterStart}
\FirstLine{\noindent ## Some Discourtesies}
    
Sidney Prale obtained accommodations in a prominent hostelry on Fifth Avenue, bathed, dressed, ate luncheon, and then went out upon the streets, walking briskly and swinging his stick, going about New York like a stranger who never had seen it before.

As a matter of fact, he never had seen this New York before. He had expected a multitude of changes, but nothing compared to what he found. He watched the crowds on the Avenue, cut over to Broadway and investigated the electric signs by daylight, observed the congestion of vehicles and the efforts of traffic policemen to straighten it out. He darted into the subway and rode far downtown and back again just for the sport of it. After that he got on an omnibus and rode up to Central Park, and acted as if every tree and twig were an old friend.

He made himself acquainted with the animals in the zoo there, and promised himself to go to the other zoo in the Bronx before the end of the week. He stood back at the curb and lifted his head to look at new buildings after the manner of the comic supplement farmer with a straw between his teeth.

"Great---great!" said Sidney Prale.

Then he hurried back to the hotel, dressed for dinner, and went down to the dining room, stopping on the way to obtain a ticket for a musical revue that was the talk of the town at the moment.

Prale ordered a dinner that made the waiter open his eyes. He made it a point to select things that were not on the menus of the hotels in Honduras. Then he sat back in his chair and listened to the orchestra, and watched well-dressed men and women come in and get their places at the tables.

But the dinner was a disappointment to Prale after all. It seemed to him that the waiter was a long time giving him service. He remonstrated, and the man asked pardon and said that he would do better, but he did not.

Prale found that his soup was lukewarm, his salad dressing prepared imperfectly, the salad itself a mere mess of vegetables. The fish and fowl he had ordered were not served properly, the dessert was without flavor, the cheese was stale. He sent for the head waiter.

"I'm disgusted with the food and the service," he complained. "I rarely find fault, but I am compelled to do so this time. The man who has been serving me seems to be a rank amateur, and twice he was almost insolent. This hotel has a reputation which it scarcely is maintaining this evening."

"I'll see about it, sir," the head waiter said.

Prale saw him stop the waiter and speak to him, and the waiter glared at him when he brought the demi-tasse. Prale did not care. He glared back at the man, drank the coffee, and touched the match to a cigar. Then he signed the check and went from the dining room, an angry and disgusted man.

"Another thing like that, and I look for the manager," he told himself.

He supposed that he was a victim of circumstances---that the waiter was a new man and that it happened that the portions he served were poor portions. His happiness at being home again prevented Sidney Prale from feeling anger for any length of time. He got his hat and coat and went out upon the street again.

He had an hour before time to go to the theater. He walked over to Broadway and went toward the north, looking at the bright lights and the crowds. He passed through two or three hotel lobbies, satisfied for the time merely to be in the midst of the throngs.

At the proper time, he hurried to the theater and claimed his seat. The performance was a mediocre one, but it pleased Sidney Prale. He had seen a better show in Honduras a month before, had seen better dancing and heard better singing and comedy, but this was New York!

The show at an end, Prale claimed his hat and coat at the check room and walked down the street toward a cabaret restaurant. He reached into his overcoat pocket for his gloves, and his hand encountered a slip of paper. He took it out.

There was the same rough handwriting on the same kind of paper, and evidently with the same blunt pencil.

"Remember---retribution is sure!"

"This thing ceases to be a joke!" Prale told himself.

His face flushed with anger, and he turned back toward the theater. But he had been among the last to leave, and already the lights of the playhouse were being turned out. The boy in charge of the check room would be gone, Prale knew.

He thought of Kate Gilbert again, and the bit of paper she had dropped as she got into the limousine down on the water front. Surely she could have no hand in this, he thought. What interest could Kate Gilbert, a casual acquaintance and reputed daughter of a wealthy house, have in him and his affairs?

"Somebody is making a mistake," he declared to himself, "or else it is some sort of a new advertising dodge. If I ever catch the jokesmith who is responsible for these dainty little messages, I'll tell him a thing or two."

Prale turned into the restaurant and found a seat at a little table at one side of the room. The after-theater crowd was filling the place. The orchestra was playing furiously, and the cabaret performance was beginning. Sidney Prale leaned back in his chair and watched the show. The waiter came to his side, and he ordered something to eat and drink.

Then he saw Kate Gilbert again, at a table not very far away from his. She was dressed in an evening gown, as if she had just come from the theater or opera. She was in the company of the elderly man who had met her at the wharf, and a young man and an older woman were at the same table.

Prale's eyes met hers for an instant, and he inclined his head a bit in a respectful manner. But Kate Gilbert looked through him as if he had not been present, and then turned her head and began talking to the elderly man.

Prale's face flushed. He hadn't done anything wrong, he told himself. He merely had bowed to her, as he would have bowed to any woman to whom he had been properly introduced. She had seen fit to cut him. Well, he could exist without Kate Gilbert, he told himself, but he wondered at her peculiar manner.

He left the place within the hour and went back to the hotel and to bed. In the morning he walked up the Avenue as far as the Circle, dropped into a restaurant for a good breakfast, and then engaged a taxicab and drove downtown to the financial district. He had remembered that he was a man with a million, and that he had to pay some attention to business.

He went into the establishment of a famous trust company and sent his card in to the president. An attendant ushered him into the president's private office immediately.

"Sit down, Mr. Prale," said the financier. "I am glad that you came to see me this morning. I was just about to have somebody look you up."

"Anything the matter?" Prale asked.

"Your funds were transferred to us by our Honduras correspondent," the financier said. "Since you were leaving Honduras almost immediately, we decided to care for the funds until you arrived and we could talk to you."

"I shall want some good investments, of course," Prale said. "I have disposed of all my holdings in Honduras, and I don't want the money to be idle."

"Idleness is as bad for dollars as for men," said the financier, clearing his throat.

"Can you suggest some investments? I have engaged no broker as yet, of course."

"I---er---I am afraid that we have nothing at the present moment," the financier said.

"The market must be good," Prale observed. "I never knew a time when investments were lacking."

"I would not offer you a poor one, and good ones are scarce with us at present," said the banker. "Sorry that we cannot attend to the business for you. Perhaps some other trust company------"

"Well, I can wait for something to turn up," Prale said. "There is no hurry, of course. Probably you'll have something in a few weeks that will take care of at least a part of the money."

The banker cleared his throat again, and looked a trifle embarrassed as he spoke. "The fact of the matter is, Mr. Prale," he said, "that we do not care for the account."

"I beg your pardon!" Prale exclaimed. "You mean you don't want me to leave my money in your bank?"

"Just that, Mr. Prale."

"But in Heaven's name, why? I should think that any financial institution would be glad to get a new account of that size."

"I---er---I cannot go into details, sir," the banker said. "But I must tell you that we'd be glad if you'd make arrangements to move the deposit to some other bank."

"I suppose you don't like to be bothered with small accounts," said Prale, with the suspicion of a sneer in his voice. "Very well, sir! I'll see that the deposit is transferred before night. Perhaps I can find banks that will be glad to take the money and treat me with respect. And I shall remember this, sir!"

"I---er---have no choice in the matter," the banker said.

"Can't you explain what it means?"

"I have nothing to say---nothing at all to say," stammered the financier. "We took the money because of our Honduras correspondent, but we'll appreciate it very much if you do business with some other institution."

"You can bet I'll do that little thing!" Prale exclaimed.

He left the office angrily and stalked from the building. Were the big financiers of New York insane? A man with a million in cold cash has the right to expect that he will be treated decently in a bank. Prale walked down the street and grew angrier with every step he took.

Before going to Honduras he had worked for a firm of brokers. He hurried toward their office now. He would send in his card to his old employer, Griffin, he decided, and ask his advice about banking his funds, and incidentally whether the financier he had just left was an imbecile.

He found the Griffin concern in the same building, though the offices were twice as large now, and there were evidences of prosperity on every side.

"Got an appointment?" an office boy demanded.

"No, but I fancy that Mr. Griffin will see me," said Prale. "I used to work for him years ago."

Then he sat down to wait. Griffin would be glad to see him, he thought. Griffin was a man who always liked to see younger men get along. He would want to know how Sidney Prale got his million. He would want to take him to luncheon and exhibit him to his friends---tell how one of his young men had forged ahead in the world.

The boy came back with his card. "Mr. Griffin can't see you," he announced.

"Oh, he's busy, eh? Did he make an appointment?"

"No, he ain't busy," said the boy. "He's got his feet set up on the desk and he's readin' about yesterday's ball game. He said to say that he didn't have time to see you this mornin', and that he wouldn't ever have time to see you."

"Don't be discourteous, you young imp!" Prale said, his face flushing. "You're sure you handed Mr. Griffin my card?"

"Oh, I handed it to him---and don't you try to run any bluff on me!" the boy answered. "From the way the boss acted, I guess you don't stand very high with him!"

The boy went back to his chair, and Sidney Prale went from the office, a puzzled and angry man. There probably was some mistake, he told himself. He'd meet Griffin during the day and tell him about the adventure.

He was anxious to meet some of the men with whom he had worked ten years before, but he did not know where to find them. He'd have to wait and ask Griffin what had become of them. Then, too, he wanted to transfer his funds.

Prale got another taxicab and started making the rounds of the banks he knew to be solid institutions. Within a few hours he had made arrangements to transfer the account, using four financial institutions. He said nothing, except that the money had been transferred to the trust company from Honduras, because the company had a correspondent there.

His funds secure, Prale went back uptown and to the hotel. The clerk handed him a note with his key. Prale tore it open after he stepped into the elevator. This time it was a sheet of paper upon which a message had been typewritten.

"You can't dodge the law of compensation. For what you have done, you must pay."

Sidney Prale gasped when he read that message, and went back to the ground floor.

"Who left this note for me?" he demanded of the clerk.

"Messenger boy."

"You don't know where he came from?"

"No, sir."

Prale turned away and started for the elevator again. A bell hop stopped him.

"Manager would like to see you in his office, sir," the boy said. "This way, sir."

Prale followed the boy, wondering what was coming now. He found the manager to be a sort of austere individual who seemed impressed with his own importance.

"Mr. Prale," he said, "I regret to have to say this, but I find that it cannot be avoided. When you arrived yesterday, the clerk assigned you to a suite on the fifth floor. He made a mistake. We had a telegraphic reservation for that suite from an old guest of ours, and it should have been kept for him. You appreciate the situation, I feel sure."

"No objection to being moved," Prale said. "I have unpacked scarcely any of my things."

"But---again I regret it---there isn't a vacant suite in the house, Mr. Prale."

"A room, then, until you have one."

"We haven't a room. We haven't as much as a cot, Mr. Prale. We cannot take care of you, I'm afraid. So many regular guests, you understand, and out-of-town visitors."

"Then I'll have to move, I suppose. You may have the suite within two hours."

"Thank you, Mr. Prale."

Prale was angry again when he left the office of the manager. It seemed that everything was conspiring against his comfort. He got a cab, drove to another hotel, inspected a suite and reserved it, paying a month in advance, and then went back to the big hotel on Fifth Avenue to get his baggage. He paid his bill at the cashier's window, and overheard the room clerk speaking to a woman.

"Certainly, madam," the clerk was saying. "We will have an excellent suite on the fifth floor within half an hour. The party is just vacating it. Plenty of suites on the third floor, of course, but, if you want to be up higher in the building------"

Sidney Prale felt the blood pounding in his temples, felt rage welling up within him. He felt as he had once in a Honduras forest when he became aware that a dishonest foreman was betraying business secrets. He hurried to the office of the manager, but the stenographer said the manager was busy and could not be seen.

Prale whirled away, going through the lobby toward the entrance. He met Kate Gilbert face to face. She did not seem to see him, though he was forced to step aside to let her pass.

\vspace{2\nbs}
\ChapterDeco[c1]{\decoglyph{e9665}}
\clearpage
\thispagestyle{empty}

\begin{ChapterStart}
\vspace{3\nbs}
\ChapterSubtitle[l]{Chapter ch4}
\ChapterTitle[l]{ch4}
\end{ChapterStart}
\FirstLine{\noindent ## A Foe And A Friend}
    
After settling himself in the other hotel, Prale ate a belated luncheon. For the first time that day, he looked at the newspapers. He had remembered that a New Yorker reads the papers religiously to keep up to the minute; whereas, in Honduras, it was the custom for busy men to let the papers accumulate and then read a week's supply at a sitting.

Aside from his name in the list of arrivals, Prale found no word concerning himself, though there was mention of other men who had come on the Manatee, and who had no special claim to prominence.

"I don't amount to much, I guess," said Prale to himself. "Don't care for publicity, anyway, but they might let the world know a fellow has come home."

He went for another walk that afternoon, returned to the hotel for dinner, and decided that, instead of going to a show that evening, he would prowl around the town.

He walked up to the Park, went over to Broadway, and started down it, looking at the bright lights again, making his way through the happy, theater-going throngs toward Times Square. In the enjoyment of the crowds he forgot, in part, the discourtesies of the day, but he could not forget them entirely.

Why had the banker acted in such a peculiar fashion? It was not like a financial institution to refuse a deposit of a round million. Why had Griffin refused to see him? Why had he as good as been ordered out of the hotel?

"Coincidence," he told himself. "No reason on earth why such things should happen unless I am being taken for somebody else---and that wouldn't be true in the case of Griffin."

He came to a prominent hotel and went into the lobby, looking in vain for some friend of the old days with whom he could spend an hour or so. Down in Honduras he had had his million and friends, too; and here, in his old home, he had nothing but his money. At this hour, down in Honduras, the band would be playing in the plaza, and society would be out in force. There would be a soft breeze sweeping down from the hills, bringing a thousand odors that could not be detected in New York. Here and there guitars would be tinkling, and men and maidens would be meeting in the moonlight.

There would be a happy crowd at a certain club he knew, at which he always had been made welcome. A man could sit out on the veranda and look over the tumbling sea, and hear the ship's bells strike. Sidney Prale found himself just a bit homesick for Honduras.

"Got to get over it," he told himself. "No sense in feeling this way. I'll have a hundred friends before I've been in town a month!"

He went out upon the street, made his way down it, and dropped in at another hotel. There he saw Rufus Shepley sitting in an easy-chair, smoking and looking at an evening paper.

Well, he knew Shepley, at least. Shepley was only a steamship acquaintance, but he was a human being and could talk. Prale was just a bit tired of confining his conversation to waiters and cigar-store clerks.

He stopped before Shepley and cleared his throat.

"Well, we meet again, Mr. Shepley!" he said.

Rufus Shepley looked up, and then sprang to his feet, but his face did not light and he did not extend a hand in greeting. Instead, his countenance grew crimson, and he seemed to be shaking with anger.

"You presume too much on a chance acquaintance, sir!" Rufus Shepley thundered. "I do not wish you to address me again---do you understand, sir? Never again---either in public or private!"

"Why------" Prale stammered.

"I don't want anything to do with a man of your stamp!" Rufus Shepley went on. "10 years in Honduras, were you? We all know why men go to Honduras and spend years there."

Shepley had raised his voice, and all in the lobby could hear. Men began moving toward them, and women began walking away, fearing a scene and a quarrel.

Sidney Prale's face had flushed, too, and he felt his anger rising again.

"I am sure I do not wish to continue the acquaintance if you do not, sir," he said. "I can be courteous, at least."

"Some men are not entitled to courtesy," Shepley roared.

"What do you mean by that?" Prale demanded.

"I mean that I don't want anything to do with you, that's all! I don't want you to speak to me again! I don't want anybody to know that you even know me by sight!"

"See here!" Prale cried. "You can't talk to me like that without giving me some explanation! You can't defame me before other men------"

"Defame you?" Shepley cried. "You can't make a tar brush black, sir?"

Rage was seething in Prale now. There was quite a crowd around them, and others were making their way forward.

"I don't pretend to know what is the matter with you, and I don't much care!" he told Shepley. "If your hair wasn't gray, I'd take you out on the sidewalk and smash your face in! Please understand that!"

"Threaten me, will you?"

"I'm not threatening you. I don't fight a man with one foot in the grave."

"Why you------"

"And I don't care to have you address me in public again, either," Sidney Prale went on. "It probably would be an insult."

"Confound you, sir!" Shepley cried.

He reached forward and grasped Prale by the arm. Sidney Prale put up a hand, tore the grasp loose, and tossed Rufus Shepley to one side.

"Keep your paws off me!" he exclaimed. "I think that you're insane, if you ask me!"

The hotel detective came hurrying up.

"You'll have to cut that out!" he said. "What's the row here, anyway?"

"The place is harboring a maniac!" Prale said.

"It's harboring a crook!" Shepley cried.

Prale lurched forward and grasped him by both arms, and shook him until Rufus Shepley's teeth chattered.

"Another word out of you, and I'll forget that your hair is gray!" Prale exclaimed, and then he tossed Shepley to one side again.

"Either of you guests here?" the house detective demanded. "No? Then maybe you'd both better get out until you can cool off. If you want to stage a scrap, go down and rent Madison Square Garden and advertise in the newspapers. I wouldn't mind seeing a good fight myself. But this lobby isn't any prize ring. Get me?"

Sidney Prale, his face still flaming, whirled around and started for the entrance, the crowd parting to let him through. Rufus Shepley, fuming and fussing, followed him slowly. The house detective accompanied him to the door.

Prale was waiting at the curb, a Prale whose face was white now because of the temper he was fighting to control. He stepped close to Shepley's side.

"I don't know why you insulted me, but don't do it again!" Prale said. "I ought to settle with you for what you've said already."

The house detective, who had heard, stepped forward again, but Sidney Prale swung across the street and went on his way.

He walked rapidly for a dozen blocks or more, paying no attention to where he was going, until his anger began to subside.

"Why, the raving maniac!" he gasped, once or twice.

He didn't pretend to guess what it meant. Shepley had seemed to be friendly enough when they had separated aboard ship. What could have happened to make the man change his mind and attitude?

"Must be some mistake!" Prale told himself. "If there is any more of this, I'll have to get to the bottom of it!"

He reached Madison Square, and sat down on a bench to smoke and regain his composure. He knew that he had a terrible temper, and that it had to be controlled. A temper that flashed was all right at times in the jungles of Honduras, but it was not the proper thing to exhibit in the heart of New York City. It might get him into serious trouble with somebody.

He finished his cigar, listened to the striking chimes, and lighted another smoke. A pedestrian stopped beside him.

"Old Sid Prale, or I'm a liar!" he cried.

Prale looked up, and then sprang to his feet.

"Jim Farland, the sleuth!" he cried in answer. "Old Jim, the holy terror to evildoers. Now I am glad that I'm home!"

"When did you get in?"

"Yesterday. Sit down. Have a cigar. You're the first old friend I've met!"

Detective Jim Farland sat down and lighted the cigar. "You've been gone some time," he said.

"10 years, Jim."

"Went away rather sudden, didn't you?"

"I did. I made my decision one night and sailed the night following," said Prale.

"I always wondered why you went, and what became of you. Had a good job with old Griffin, didn't you?"

"The job was all right, Jim. But there was a girl------"

"Ah, ha!"

"And she threw me over for a fellow who had some money. That made me huffy, of course. I swore I'd shake the dust of New York from my shoes, go to some foreign country, take with me the ten thousand dollars I had saved, and turn it into a million."

"And came back broke!" Farland said.

"Nothing of the sort, Jim. I came back with a million."

"Great Scott! I suppose I'd better be on my way then. I ain't in the habit of having millionaires let me associate with 'em."

"You sit where you are, or I'll use violence!" Prale told him. "I suppose you are still on the force? Still fussing around down in the financial district watching for swindlers?"

"I left the force three years ago," Jim Farland replied. "Couldn't seem to get ahead. Too honest, maybe---or too ignorant. I'm in a sort of private detective business now---got an office up the street. Doing fairly well, too---lots of old friends give me work. If you have anything in my line------"

"If I have, you'll get a job," said Prale.

"Let me slip you a card," said Farland. "You never know when you may need a detective. So you came back with a million, eh?"

"And ran into a mess," Prale added.

"I can't imagine a man with a million running into much of a mess," Farland said.

"That's all you know about it. I may need your services sooner than you think. There is a sort of jinx working on me, it appears."

"Spill it!" Jim Farland said.

Sidney Prale did. He related what had happened at the bank, at the hotel, in Griffin's office, and told of the scene with Rufus Shepley.

"Funny!" Farland said, when he had finished. "I know old Rufus Shepley, and as a general thing he ain't a maniac. Something behind all this, Sid."

"Yes; but what on earth could it be?"

"That's the question. If anything else happens, and you need help, just let me know."

"I'll do that, surely," said Prale. "And I'm glad that I've got one friend left in town."

"Always have one as long as I'm here," Jim Farland assured him. "And it ain't because of your million, either. It's true about the million?"

"Absolutely!"

"Gee! That's more than old Griffin himself has in cash, anyway," Farland declared. "Maybe it's a good thing that girl turned you down. You'd probably be a clerk at a few thousand a year, if she hadn't. How'd you make the coin?"

"Mines and fruit and water power and logs," said Prale.

"Sounds simple enough. When the detective business goes on the blink, I may take a turn at it myself."

"If you ever need money, Jim, call on me. If you want to engage bigger offices, hire operatives, branch out------"

"Stop it!" Farland cried. "I want nothing of the kind. I'm a peculiar sort of duck---don't care about being rich at all. I just want to be sure I'll have a good living for myself and the wife and kids, and have a few friends, and be able to look every man in town straight in the eye. I'd rather work for a friend for nothing than do work I don't like for ten thousand an hour."

"I believe you!" Prale said.

\vspace{2\nbs}
\ChapterDeco[c1]{\decoglyph{e9665}}
\clearpage
\thispagestyle{empty}

\begin{ChapterStart}
\vspace{3\nbs}
\ChapterSubtitle[l]{Chapter ch5}
\ChapterTitle[l]{ch5}
\end{ChapterStart}
\FirstLine{\noindent ## The Cousin}
    
An hour later, having parted with Detective Jim Farland, Sidney Prale walked slowly up Fifth Avenue, determined to go to his hotel suite and rest for the remainder of the evening. His conversation and short visit with Farland had put him in a better humor. There was no mistaking the quality of Farland's friendship. He and Prale had been firm friends ten years before, when Farland was on duty in the financial district, and they had made it a point at that time to eat luncheon together when Farland's duties permitted.

New York seemed a better place, even with one friend among several million persons. So Prale swung his stick jauntily, and hummed the Spanish love song again, and told himself that Rufus Shepley and Kate Gilbert, old Griffin and the hotel manager and the rest of the motley crew that had made the day miserable for him amounted to nothing in the broader scheme of things, and were not to be taken seriously.

He came to a block where there were few pedestrians, where the great shops had their lights out and their night curtains up. He heard steps behind him, and presently a soft voice.

"Sid! Sid!"

Sidney Prale whirled around, alert and on guard, for he did not recognize the voice. A medium-sized man stood before him, a man of about his own age, who had a furtive manner and wore a beard.

"Don't you know me, Sid?"

"Can't say that I do!"

"Why, I'm your cousin, George Lerton. I'm the only relative you've got in the world, unless you got married while you were away."

Prale stepped aside so that the nearest light flashed on the face of the man before him.

"Well, if it isn't!" he said. "Didn't recognize you at first. How long have you been wearing the alfalfa on your face?"

"2 or three years," George Lerton told him, grinning a bit. "I saw your name in the passenger list, Sid, and wanted to see you. I found out where you are stopping------"

"Why didn't you come to the hotel, then, or leave a note?" Prale asked. "Come on up now."

"I---I wanted to talk to you------"

"And I want to talk to you. What are you doing for yourself, George? Still working in a broker's office?"

"Oh, I've got an office of my own now."

"Getting along all right?"

"Fairly well," Lerton said. "Business has been pretty good the last year."

"Maybe you can dig up a few good investments for me, then," Prale said. "I've got some coin now."

"I understand that you're worth a million, Sid."

"Yes, I've made my pile, and came back to New York to enjoy it. But come along to the hotel."

"I'd---I'd rather not."

"Why not? We've got to talk over old times and find out about each other. We're cousins, you know."

The truth of the matter was that Sidney Prale never had thought very much of his cousin. 10 years before they had worked side by side for Griffin, the broker. There was something furtive and shifty about George Lerton, but he never had presumed on his relationship, at least. He and Sidney Prale had been courteous to each other, but never had been warm friends.

They came from different branches of the family. Lerton had some traits of character that Prale did not admire, but he always told himself that perhaps he was prejudiced. They had seen a deal of each other in a social way in the old days.

"Let us just talk as we walk along," Lerton now said.

"All right, if you have an engagement," Prale replied. "We can get together later, I suppose. How have the years been using you? Married?"

"I was---I am a widower."

"Sorry," said Prale. "Children?"

"No---not any children. I---I married Mary Slade."

"What?" Prale cried.

He stopped, aghast. Mary Slade had been the girl who had turned him down for a man with money---and that man had not been George Lerton, who did not have as much as five thousand at that time.

"It---it's a peculiar story," Lerton said. "You went away so quick---after you quarreled with her. And that other man---she threw him over, soon. She couldn't endure him, even with all his money. She regretted her quarrel with you. I'm quite sure she wanted you for a time. I got to taking her about. You didn't write, and she was too proud to look you up, and so---after a time------"

"You married her," said Prale.

"About three years after you went away, Sid. She died after we had been married a year."

"But she always wanted money, and I had as much as you."

"I made a strike soon after you left, Sid. I plunged with my five thousand, and turned it into a hundred thousand inside four months. I kept on, and got more. I was worth almost half a million when we were married."

"I see. Well, there are no hard feelings, George. She was a good woman, in a way, and I'm sorry you lost her. I suppose we'll have to get together, for old time's sake."

"Are you going to stay here long, Sid?"

"Long? I've sold out all my Honduras holdings, and I'm here to spend the rest of my days. I've come home for good, George. The United States is plenty good enough for me. I'm going to be a civilized gentleman from now on."

"You---you're not going back?"

"Why should I? I brought that million with me. I left nothing in Honduras except a few friends. I suppose I'll run down there some day and see them, but this is going to be home, you can bet."

"Don't do it, Sid!" Lerton exclaimed.

"Don't do what?"

"Don't stay here, Sid. Get out as quick as you can! Go back to Honduras---anywhere---but don't stay in New York."

"Why shouldn't I? What on earth is the matter with you? Are you insane?"

"I---I can't tell you, Sid. But you are in danger if you don't leave New York. I can tell you that much. That's why I didn't call at the hotel; I'm afraid. Sid, I'm afraid to have anybody see me talking to you. If you came to my office, I'd refuse to see you------"

"Why?" demanded Sidney Prale, in a stern voice.

"I---I can't explain, Sid."

"I've endured a lot of nonsense to-day, and I'm not going to endure any more!" Prale said. "You're going to open your mouth and tell me what you mean, if I have to manhandle you."

"You can beat me until I'm unconscious, Sid, but you can't make me talk!" Lerton told him.

"But what does it all mean?"

"You'd better go away, Sid; you'd better get out of the country and stay out!"

"No reason why I should. I never gave up my citizenship; I haven't done anything wrong. I'm back in my old home, and I fail to see why I shouldn't remain here if that is my wish."

"But you're in danger!"

"In danger from what?" Sidney Prale cried.

"You have powerful enemies, Sid."

"Why?"

"I---I don't know, exactly. But you have powerful enemies. Some of my best customers have informed me that they are through doing business with me if I have anything to do with you. They told me that before you had been back three hours."

"Powerful enemies? Why? Business enemies?"

"I---I don't know."

"Um! So that is why the bank refused my deposit, why I was turned out of a hotel, and why old Rufus Shepley raised such a row with me! Powerful enemies, have I? But there isn't sense in it! I haven't done anything to make powerful enemies, or any other kind. I'm about fed up with this stuff!"

"Go away, Sid. You've got money---you can live anywhere!"

"You bet I can! And I'm going to live in New York!"

"Don't try it, Sid!"

Prale whirled and faced him. "You know more than you're telling!" he accused. "You open your face and talk! I never did have any too much love for you, and you can wager that I'm not going to let you frighten me into running away from New York! Talk!"

"I haven't anything more to say, Sid!"

"If I have to choke it out of you right here------"

"You'd better not. It would give your enemies a chance!"

"Lerton, I've fought the Honduras jungles! I've fought half-savage men and treacherous employees, snakes and fever, financial sharks and common adventurers. I didn't come back to New York to back down in front of a man like you---or half a hundred like you. Maybe that is strong talk---but you have it coming! Give my enemies a chance? I'll give them all the chance they want. Maybe they'll come into the open, then, and let me see whom I'm fighting! I don't like foes that fight from the dark!"

"You'd better go away, Sid. I'm talking for your own good!"

"For my good? For yours, you mean! Afraid you'll lose a few customers and a few dollars, by standing by your cousin, are you? Why don't you be a man, tell me what you know, help me to fight! Bah! I'm disgusted with you!"

He hurled George Lerton away from him, curled his lips in scorn of the man.

"I've tried to warn you," Lerton whimpered.

"I don't understand this and I'm sure you could explain a lot, if you would. Perhaps I've got more dollars than the customers you are so afraid of losing. Suppose I hand my million to you for investment. Will you talk, then?"

"I---I wouldn't dare touch it," Lerton whimpered.

Prale looked at him closely. "It must be something pretty bad to make you toss aside the chance to handle a million in investments," he said. "I know you, George! You'd sell your soul for money! You got anything more to say to me about this?"

"I---I dare not say anything more."

"Very well. If you are afraid to be seen in my presence, kindly keep away from me hereafter and don't worry about me looking you up at your office. I'll not take the trouble!"

Sidney Prale said nothing more; he whirled around and walked rapidly up the Avenue, enraged, wondering what it all meant, determined to find out as soon as possible.

Lerton ran after him.

"Won't you go away, Sid?" he whimpered.

"No. I'll stay here, and if I have enemies I'll fight them!" Prale told him. "Why are you so eager to have me run away?"

"I don't want to see you in trouble, Sid."

"That's peculiar. In the old days you used to gloat whenever I got in trouble. You seem to have a wonderful and sudden regard for my welfare, and I can't explain it to myself."

Once more, Prale whirled around and started up the Avenue. His brain was in a tumult. What did George Lerton know that he refused to tell? Why should there be powerful enemies? He knew of no reason in the world.

"He's dead eager to get me out of town," Prale mused. "There's something behind it, all right."

\vspace{2\nbs}
\ChapterDeco[c1]{\decoglyph{e9665}}
\clearpage
\thispagestyle{empty}

\begin{ChapterStart}
\vspace{3\nbs}
\ChapterSubtitle[l]{Chapter ch6}
\ChapterTitle[l]{ch6}
\end{ChapterStart}
\FirstLine{\noindent ## Murk---And Murder}
    
Instinct, intuition, or some similar faculty caused Prale to turn off the Avenue eastward toward the river. He was not angry now. His mind was in action. He had convinced himself that there was something behind all this, and he was eager for the solution.

Those mysterious warnings had begun on board ship, he remembered. The piece of paper Kate Gilbert had dropped, and which he had picked up, had writing similar to the messages he had received. He would have to engage Jim Farland, he told himself, and learn a few things concerning Miss Kate Gilbert.

Had the journey because of ill health been a subterfuge? Had Kate Gilbert gone to Honduras to watch him? If she had, what was the reason for it?

"It's enough to make a man a maniac," Prale mused. "And that Shepley man! He was all right when we parted on the ship. Somebody said something to him about me after he landed. He treated me as if I had been a skunk."

Then he thought of George Lerton, his cousin. He couldn't quite make up his mind about Lerton. The man seemed frenzied in his eagerness to get Prale to leave New York. And Prale knew that it was not because of an overwhelming love George Lerton had for him, not anxiety lest ill fortune should come to Sidney Prale.

He would have to think it out, he told himself. At least, he knew that he had foes working against him, and could be on guard continually. Down in Honduras he had won a reputation as a fighter, and a fight was a fight in any clime, he knew; there might be a difference in the rules here and there, but the same qualities decided the winner.

He continued walking down the street toward the river. In Honduras he had become accustomed to walking up and down the beach and looking at the water whenever he wanted to think and solve some problem, and it probably was habit that sent him to the water front now.

He tossed away the butt of his cigar and did not light another at the moment. For a time he stood looking out at the black water, at the craft plying back and forth, their lights flashing. He stepped upon a little dock and started walking its length. After a time he came near the end of it without having encountered a watchman, and sat down on a box in a dark, secluded corner.

There, his back braced against the building and the building shielding him from the cold wind that came up from the distant sea, Sidney Prale sat and tried to think it out.

1 thing made a comfortable thought---he had money with which to fight. Either he was the victim of some injustice, or a grave mistake was being made. He wished that he had forced George Lerton to tell him more, and he decided that he would do so if they met again. He might even hunt him out and force him to speak. Sidney Prale thought nothing of handling a man like Lerton.

He heard steps on the dock and remained silent in the darkness, thinking that possibly some watchman was making the rounds. If he was discovered, he would say that he had been looking at the river, give the watchman his card and a tip, and leave.

The steps came nearer and Prale could make out the form of a man slipping along the dock's edge in a furtive manner. There was not light enough for Prale to see his features. He was walking bent over, a short, heavy-set man who did not wear an overcoat.

Prale watched as the man passed within six feet of him and went to the edge of the dock. There he stood, outlined against the sky, looking down at the water. Prale imagined that he heard something like a sob, and gave closer attention. Then he saw the man take off his coat and drop it behind him, remove his cap and place it on the coat, and look down at the water again.

And then Sidney Prale sprang straight forward, and grasped the body of the other as it was in mid-air.

"No, you don't!" Prale exclaimed.

He found immediately that he had a fight on his hands. The other whirled and began kicking and striking. Sidney Prale hurled him backward, rushed, caught him up again in a better hold, threw him back against the building, and held him there, breathless and panting.

"Another smash out of you, and I'll drop you into the river myself!" Prale said. "Suppose you take time to get your breath now."

"I---I thought you was a cop."

"Afraid of the cops?"

"It's against the law to---to try to commit suicide."

"So I understand," said Prale. "Well, I am not a cop. Trying to drown yourself, were you? Why?"

"Why not?" the other asked. "I'm done with livin'."

"Not just yet, but you would have been if I hadn't been sitting here."

"I've knocked all over the world---and made a few mistakes," said the derelict. "Oh, nothin' that would get me in trouble with the cops! But I just found out that I'm clutterin' up the earth and don't amount to anything. I'm sick of half starvin' to death, and workin' like a dog when I get the chance just to get enough to keep a few old clothes hung on me."

"Disgusted generally with your lot?" Prale asked.

"Yes, sir."

"Friends or relatives?"

"Not any."

"What's your name?" Prale asked.

"You mean my real name? I don't remember. It's been so long since I've used it, and I've used so many others since that I don't know. What's the difference?"

"I'll call you Murk," said Prale. "That expresses the dark river, the deed you were about to do, and the evident state of your feelings."

"It's as good as any, I suppose."

"What's your particular grievance against the world in general?"

"It ain't anything in particular," said Murk. "It's just general."

"I see. A drifter, are you?"

"I reckon I am."

"Sore at existence, eh?"

"Well, what's the use of livin'?" Murk demanded. "There ain't a man, woman or child in the world that gives a whoop what becomes of me. I'm just in the way to be kicked around."

"Maybe you haven't found your proper place in the scheme of things."

"I've sure done some travelin' lookin' for it, boss, but maybe I ain't found it, as you say. I sure ain't found any place that looks like it needed me bad."

"Hard to make a living?"

"Oh, I get along. But, what's the use?" Murk wanted to know. "I ain't got anybody---I get lonesome lots of times. If I had money, it might be different."

"I'm not so sure about that," said Prale, smiling a bit. "I've got a million dollars, and, as far as I know right this minute, I have just one friend in New York."

"If I had a million dollars I wouldn't care whether I had a friend or not," Murk said.

"You can be just as lonesome with a million dollars as you can without a cent," Prale told him. "I was sitting down here because I was lonesome, and because there are some enemies working at me, and I don't know who they are or why they want to trouble me."

"Well, let's jump in the drink together," Murk said.

"Why not fight it out?" asked Sidney Prale.

"Mister, I've been fightin' for years, and it don't get me anything. It just tires me out---that's all. The next world can't be any worse than this."

"Are you a fighter, or a quitter?"

"Nobody ever called me a quitter."

"But you were trying to be a few minutes ago. You were going to quit like a yellow dog!" Prale told him. "You were going to throw up the sponge and give the devil a laugh."

"That's between me and the devil---nobody else would care."

"If you had a friend, an influential friend, and didn't have to keep up a continual fight to hold body and soul together, could you manage to face the world a little longer?"

"I reckon I could."

"How old are you?"

"30-five," said Murk.

"Old enough to have some sense. I am three years older. I'm almost as lonesome as you are. Why not join forces, Murk?"

"Sir?"

"If I showed you a corner where you would fit in, would you be loyal? Would you stand by me, help me fight if it was necessary, and all that?"

"You just try me---that's all."

"Very well, Murk, I'm going to trust you. I told you the truth when I said I had a million dollars. I have but one friend I can depend upon, and I have enemies. I like to fight, Murk, but I like to have a good pal at my back when I do."

"That's me, too, sir; but I ain't ever had the pal."

"You've got one now, Murk. You'd be dead now, but for me. So you must be my man, understand?"

"I don't quite getcha."

"You're under my orders from now on, Murk. We'll have a nice row, standing back to back perhaps. I'll take you on as a sort of valet and bodyguard. You'll have good clothes and a home and plenty to eat and a bit of money to spend. I'll expect you to be loyal. If I find that you are not---well, Murk, I got back yesterday from Central America. I got my million down there, by fighting for it, and there were times when I had to handle men roughly. I can read men, Murk. Can you imagine what I'd do to a man who double crossed me?"

"I getcha now! You needn't be afraid I'll double cross you. I don't think this is real."

"It's real, Murk, if we strike a bargain. Do we?"

"I've got everything to win and nothin' to lose---so we do!" Murk said.

"Fair enough. Now we'll get off this dock. Pick up your cap and coat."

Murk picked them up and put them on, and then he followed at Prale's heels until they were on the street and beneath the nearest light. There they stopped and looked each other over.

Murk was short, but he was built for strength. Prale could tell at a glance that the man, even poorly nourished as he was, had muscles that could be depended on. Prale liked the look around Murk's eyes, too. Murk was a dog man, the sort that proves faithful to the end if treated right.

"Well, how do you like me?" Prale asked.

"You look good to me, sir."

"My name is Sidney Prale."

"Yes, Mr. Prale."

"You understand our little deal thoroughly?"

"Yes, sir."

"Come along, then. Here is a cigar---light up!"

Murk lighted the cigar, and Prale lighted another, and they went rapidly up the street to Fifth Avenue. Prale signaled a passing taxicab, and they got in. When the cab stopped, it was in a district where some cheap clothing stores remain open until almost midnight.

Half an hour later they emerged again. Murk was dressed in a suit which was somber in tone, and which was not at all a bad fit. He was dressed in new clothing from the skin out. Prale took him to a barber shop, and waited until the barber gave Murk a hair cut and a shave.

"Gosh!" Murk said, when he looked at himself in the glass. "This can't be me!"

"It is, however," Prale assured him. "Now, we'll go home, Murk, and get settled."

"Where is home?"

Prale named the hotel.

"I'd get thrown out on my bean if I ever stuck my nose in the kitchen door," Murk said.

"You're not going into the kitchen, Murk. You're going to be registered as my valet and bodyguard, and you're going up in the elevator with me. Kindly remember, Murk, that you are the personal servant of Mr. Sidney Prale."

"Yes, sir."

"And your boss has a million dollars and nobody knows how many secret enemies. Those things give you a standing, Murk. When we are alone, of course, you'll be a sort of pal. I never had a valet before and I couldn't stand a regular one. Instead of being a valet, when we are alone, I want you to be a regular fellow."

"I getcha, Mr. Prale."

"Off we go, then."

They arrived at the hotel, and Prale registered Murk as his valet and took him up to the suite.

"You bunk in there, Murk," Prale said, pointing to another room. "Take a bath and go to bed and get some rest. If you are inclined to throw me down, you'll find some money and jewelry in the top drawer of the dresser. Rob me and sneak out during the night, if you want to. Cut my throat, if it's necessary."

"You needn't be afraid, sir---you can trust me!"

"I do!" said Sidney Prale.

Prale slept well that night. When he awoke in the morning, Murk was dressed and sitting by the window. He drew Prale's bath without being told, and then stood around as if waiting to be of service.

"I---I found this slipped under your door, sir," he said, after a time.

"What is it, Murk?"

"A piece of paper with writing on it, sir."

"More news from the enemy, I suppose. What does it say?"

"It says as how a man's sin always finds him out."

"That's interesting, isn't it? Do you think I am a sinner of some sort, Murk?"

"I don't care if you are, sir!"

"Murk! You needn't get excited about it. Put the paper in the lower drawer of the dresser; I'm making a collection of them," Prale said. He went back into the other room and continued dressing. "Go to the telephone and order breakfast served to us here, Murk," he directed.

"What shall I order, sir?"

"Order plenty of whatever you like, and tell them to make it double," said Prale.

Murk grinned and gave a proper order. Prale was dressed by the time the breakfast was served. He and Murk made a hearty meal.

And then Prale lighted his morning cigar and began reading the newspapers. Murk went around the suite, straightening things and trying to be of service. He looked at Sidney Prale often; it was plain to be seen that Prale was Murk's kind of man.

There came a knock at the door.

"See who it is, Murk," Sidney Prale said.

He did not even look up from the paper he was reading. He supposed it was some hotel employee. Murk stalked across to the door and threw it open. 2 men stood there. Murk flinched when he saw them. He did not know either of them, but he knew them immediately for what they were. Murk was a man of experience.

"Mr. Prale in?" one of them asked.

"Yes, sir."

Without asking permission, the two men stepped inside, and one of them closed the door. Prale dropped the newspaper and turned around to face them.

"Are you Sidney Prale?" one of them asked.

"I am."

"You are under arrest, Mr. Prale."

"I beg your pardon?"

"Under arrest," I said. "You know your rights, perhaps, so you need not talk unless you wish to do so."

"You are officers?"

They showed their shields.

"Straight from headquarters," one of them replied. "We want to take a look around your room while we are here."

"Suppose," said Sidney Prale, "that you tell me, first, why I am under arrest? Of what crime am I accused?"

"You are charged with murder."

"Murder? What crazy joke is this?" Prale cried. "And what particular person am I accused of murdering?"

"You are charged with the murder of Mr. Rufus Shepley," the detective replied.

\vspace{2\nbs}
\ChapterDeco[c1]{\decoglyph{e9665}}
\clearpage
\thispagestyle{empty}

\begin{ChapterStart}
\vspace{3\nbs}
\ChapterSubtitle[l]{Chapter ch7}
\ChapterTitle[l]{ch7}
\end{ChapterStart}
\FirstLine{\noindent ## Evidence}
    
Many times in his life, Sidney Prale had been greatly surprised, astonished, shocked. But never had he experienced such a feeling as he did at this bald announcement of a police detective.

The statement was like a blow between the eyes. Prale stared at the two detectives for an instant, his face flushed, and then he began to laugh.

"It isn't a laughing matter, Mr. Prale," one of the detectives told him.

"Pardon me, but it is so utterly preposterous," Prale replied. "I fail to see how I can be accused of such a crime. I am not a cut-throat, and Rufus Shepley was a man I met on shipboard casually, and have seen him only once since."

"You can do your talking at headquarters, Mr. Prale," the officer said. "I'll have to ask you to come along with us. I'll leave my partner here to look through your rooms."

"The sooner I get to headquarters, the sooner this thing will be straightened out," Prale said. "Murk, you will remain here in the rooms until you hear from me. Let the officer look at anything he wishes to inspect."

"Yes, sir," said Murk, glaring at the two detectives.

Prale faced the detective who had been speaking to him.

"Be with you as soon as I get my hat and coat," he said. "It'll not be necessary, I hope, to put handcuffs on me."

"We can go to headquarters in a taxi, and I guess I can handle you if you try any tricks," the detective replied.

"There are going to be no tricks tried," Prale said.

"Nevertheless, I think I'll keep a close eye on you."

"Do so, by all means!" Prale retorted.

"Ain't there anything I can do, sir?" Murk asked.

"Nothing except to remain in the rooms until you hear from me," Prale told him. "If I should---er---be detained, I'll probably send for you."

"Very well, sir."

1 of the detectives left the suite with Prale and walked down the hall to the elevator. The second officer remained behind to go through Prale's things in an effort to find evidence.

Prale said nothing regarding the crime as they journeyed in the taxicab to police headquarters. His mind was busy, though. This appeared to be a culmination of the annoyances to which he had been subjected.

At headquarters he was ushered into a room where a captain of detectives awaited him.

"Don't have to talk unless you want to, Mr. Prale, but it probably will be better for you to do so, and have an end of it," the captain said. "Why did you kill Rufus Shepley?"

"That's a fool question. I didn't kill him. I had no idea he was dead until the officer arrested me for his murder. I scarcely know the man, captain. I made his acquaintance aboard a ship coming from Central America, and I met him but once after leaving the ship. He told me his business and gave me his card, and that is all. I'm ready to answer any questions you may ask. This is some terrible mistake. I want to talk about it---have an end of it, as you say."

"Very well, Prale," the captain said.

"Mr. Prale, if you please. I have not been convicted yet and am entitled to some courtesy, it seems to me."

"All right, if you're going to be nasty about it," the captain said. "But you won't gain anything by taking a high-and-mighty attitude with me."

"I simply object to being addressed in the tone you used," Prale replied. "I am no crook. Let's get down to business. Ask me any questions you like, and I'd like to ask a few myself."

"That is fair enough," the captain said, a shrewd expression coming into his face.

"Suppose you take it for granted, for a few minutes, that I am innocent, and tell me when Rufus Shepley was killed, and where, and just how."

"Very well, Mr. Prale. A hotel attendant found the body at an early hour this morning. It was in Mr. Shepley's room. The man was fully dressed. The physicians say that he was killed about eleven o'clock last night."

"I understand; go on, please."

"He had been stabbed through the heart," said the captain. "Death had been instantaneous."

"But why suspect me of the crime?" Prale asked.

"This was found beside the body," the captain replied.

From the desk before him he picked up a fountain pen. It was an elaborate pen, chased with gold, and on one side of it was a tiny gold plate, upon which Prale's name had been engraved.

"You recognize it?" the captain asked.

"Certainly; it is mine."

"Oh, you admit that, do you?"

"Naturally. But I fail to see how it came to be beside the body of Rufus Shepley."

"A man who has committed a murder generally is in a hurry to get away," said the captain. "It is easy to drop a fountain pen from a pocket, especially if a man is bending over."

"I don't even know where Shepley's rooms were located," Prale said. "I didn't know the pen was missing until this minute------"

"Possibly not," replied the captain of detectives.

"And I am quite sure I do not know how it came to be beside the body, but of one thing I am certain---I did not drop it there."

"Naturally, you would say that."

"And where is the motive?" Prale demanded. "Suppose you tell me what you have against me, and then I'll proceed to tear your shabby evidence to pieces."

"We have this particular case so well in hand that I can afford to do that," the captain said. "Attend me closely and you'll see the futility of denying your guilt."

"I am waiting to hear the evidence," Prale said.

"Very well. In the first place, you have recently spent some years in Central America."

"10 years in Honduras," said Prale.

"You made a fortune down there. We have communicated with the authorities there and have learned many things about you. We have learned that you have a hot temper and know how to handle men. You have been known to beat natives terribly------"

"Rot! I was kinder than nine out of ten men of affairs. I have punished a few natives caught stealing, for instance."

"Recently, Mr. Prale, you cashed in on all your properties down there and announced that you were about to leave the country."

"That is correct," said Prale. "I made the million I went down there to make. Honduras is all right in some ways, but a man likes to live with his own kind. My home was in New York, and so, naturally, I decided to return here."

"Did you not tell some of your friends and acquaintances, before you left, that you were returning to New York for a certain purpose."

"I suppose that I did. My purpose was no secret. I had my pile and wanted to enjoy life a bit and perhaps I wanted to show off a bit, too. That was only natural, I suppose. I am proud of my success."

"Did you not hint that the purpose was something sinister---that you were going to have revenge, or something like that?"

"Certainly not."

"Very well; let us get on," said the captain of detectives. "You say that you first met Rufus Shepley aboard the Manatee?"

"Never saw him in my life until I met him in the smoking room on the ship, and never had heard his name before."

"That is peculiar. Mr. Shepley was a man of large affairs."

"But I had been in Honduras for ten years, out of touch with men of affairs in the United States," Prale replied. "I did the most of my business with firms in South America."

"Just how did you happen to meet Mr. Shepley?"

"In the smoking room. We spoke, as passengers are liable to speak to each other on a boat or a train. We talked of ordinary things and exchanged cards."

"Did you happen to play cards?"

"1 evening, for a short time. But the game did not amount to anything, and we quit early. Are you trying to insinuate that I killed the man as the outcome of a gambling quarrel?"

"Nothing of the sort," said the captain, "Let us get on. You had no trouble with Mr. Shepley on the ship---no trouble of any sort?"

"Not the slightest. We parted good friends just before the ship docked. I went to my stateroom for my things and I suppose that he did the same."

"When did you see him next?" the captain asked.

"Last evening, in the lobby of a hotel on Broadway," said Prale.

"What happened then?"

"Ah, I see where you are trying to get the motive," Prale said. "But I think that you will agree with me, before we are done, that it is a slim thing upon which to hang a serious charge of murder. I saw Mr. Shepley sitting in the lobby and went up and spoke to him. We had been friendly on the ship, I was feeling lonesome, and was glad to find somebody with whom I could talk. Besides, he had expressed a desire to see me again."

"Well, what happened?"

"Something I am at a loss to understand. He berated me for daring to address him. He acted like a maniac. I rebuked him for his manner, and the hotel detective advised us to leave the place until we cooled off, or something like that."

"Who left first?" the captain asked.

"I did. I was angry because there was a crowd around and I hated the scene that had been caused. I went through the main entrance and stepped to the curb."

"Shepley follow you?"

"Almost immediately."

"And you went up to him and threatened him, didn't you?"

Prale thought a moment. "I told him that I didn't know why he had insulted me, but I didn't want him to do it again."

"What else?" the captain demanded.

"I believe I said that I ought to settle with him for what he had said already."

"And then------"

"And then I went on down the street. The hotel detective, I think, heard me speak to Mr. Shepley."

"Yes, I know that he did," said the captain. "And the hotel detective also says that you were white with anger, and that you went off down Broadway like a man with murder in his mind. Do you care to say anything more?"

"Of course," said Prale. "I went down to Madison Square, and there I sat down on a bench."

"Meet anybody there?"

"I did. I met an old friend, Jim Farland, who used to be on your detective force, and who now runs a private agency."

"I know Farland well, and I'll send for him."

"I talked with Jim for some time," Prale went on. "I told him, I believe, that I seemed to have enemies working in the dark. I told him about the scene with Shepley."

"Um! What did Farland have to say?"

"Nothing, except that he couldn't understand why Shepley had acted so. We talked the matter over for a while and then we separated."

"Very well. And where did you go next?"

"I walked up Fifth Avenue," said Prale. "It was after nine o'clock by that time."

"Go straight to your hotel?"

"I did not," Prale said.

"Care to tell me where you went and what you did?"

"I have no objections. I walked up the Avenue, and met my cousin, George Lerton, the broker."

"Meet him accidentally?"

"He overtook me---called to me."

"How long did you talk to him?"

"For only a few minutes," said Prale. "You must understand that, while George Lerton is my cousin, we are not exceptionally friendly, and never have been. We worked for the same firm ten years ago, and after I went to Honduras, George made some money and got into business for himself; at least he told me so last night."

"So you merely shook hands and renewed your acquaintance?" the captain asked.

"There was something peculiar about the meeting," Prale replied.

"In what way?"

"Lerton urged me to leave New York and remain away. He said that I had powerful enemies."

"What about that?"

"It is what has been puzzling me. So far as I know, I haven't a powerful enemy on earth. I suppose I have a few business foes in Central America; a man can't make a million without acquiring some enemies at the same time. But I don't know of a single influential person who is my enemy."

"Didn't Lerton explain to you?"

"He refused to do so," said Prale, "and I told him to go his way and that I'd go mine."

"Doesn't that story seem a bit weak to you, Mr. Prale?"

"It may, but it is a true story. Get Lerton and question him if you wish. I couldn't make him talk---maybe you can. I'd like to know the names of these enemies of mine, if I really have them."

"Anything else lead you to believe you might have enemies?"

"Yes. I have received several anonymous notes, some on board ship and some since landing, that say something about retribution about to be visited upon me."

"Why?"

"I don't know, captain. I never did anything in my life to merit such retribution. I am sure of that."

"What time was it when you parted from Lerton?"

"It must have been about nine thirty or a quarter to ten."

"Go to your hotel then?"

"No; I turned east and went to the river."

"Wasn't that a peculiar thing to do at that hour of the night?"

"It may seem so to you," said Prale, "and I scarcely can tell why I did it. I suppose it was because I wanted to think over what George Lerton had told me, and down in Honduras I always used to walk along the beach when I was thinking."

"Well?"

"I went out on a dock and sat down in the darkness to think."

"How long did you remain there?"

"For more than half an hour; and I had an experience. Another man came on the dock. He was going to jump into the river, but I convinced him that suicide was folly, and said I'd give him a job."

"Did you?"

"I did," said Prale. "I took him downtown and bought him some clothes, and then took him to a barber shop, and afterward to the hotel. I registered him as my valet. I call him Murk. I can prove by him that I could not have killed Rufus Shepley about eleven o'clock, because I was in Murk's company at that time."

"What time did you get back to your hotel with him?"

"It was a few minutes of midnight. We spent considerable time buying the clothes and visiting the barber shop."

"Um!" the captain said. "We'll have to question a few of these people. It seems peculiar to me that a millionaire would pick up a tramp and turn him into a trusted servant."

"Perhaps it was peculiar. I can read men, I believe, and I decided that Murk needed only a chance, and he would make good. He was broke and friendless, and I was a millionaire and almost as friendless. That's the only way I can explain it."

"I'm going to send you to another office under guard, Mr. Prale," the captain said. "I'll have these people here in a short time, and we'll question them. Just tell me where you bought the clothes for this man, and what barber shop you visited."

Sidney Prale did so, and the captain of detectives made notes regarding the addresses.

"That will be all for the present, Mr. Prale," he said. "I don't want to cause any innocent man annoyance, but I can tell you this much---things look very bad for you!"

\vspace{2\nbs}
\ChapterDeco[c1]{\decoglyph{e9665}}
\clearpage
\thispagestyle{empty}

\begin{ChapterStart}
\vspace{3\nbs}
\ChapterSubtitle[l]{Chapter ch8}
\ChapterTitle[l]{ch8}
\end{ChapterStart}
\FirstLine{\noindent ## Lies And Liars}
    
Sidney Prale waited in an adjoining office, a detective sitting in one corner of it and watching him closely. It was almost a prison room, for there were steel bars at the windows, and only the one door. Prale walked to one of the windows and looked down at the street, his arms folded across his breast, trying to think it out.

The finding of that fountain pen in the room beside Rufus Shepley's body was what puzzled and bothered him the most. How on earth could it have come there? He tried to remember when he had used it last, when he had last seen it. All that he could recall was that, the afternoon before, he had used it to write a note in a memorandum book. How and where had he lost it, and how had it come into Shepley's suite? Had he dropped it in the hotel lobby during his short quarrel with Shepley, while he was shaking the man? Had Shepley picked it up later and carried it home with him? Prale did not think Shepley would have done that under the circumstances.

Well, he'd be at liberty soon enough, he told himself. It was natural for the police to learn of his quarrel with Shepley and to make an arrest on the strength of that and of finding the fountain pen. His alibi was perfect; they soon would know that he could not have committed the crime.

It was almost an hour later when he was taken back into the other room again. Prale had spent the time standing before the window, smoking and trying to think things out. The captain of detectives was before his desk when Prale was ushered into the office.

"I've been investigating your story, Mr. Prale," the captain said, looking at him peculiarly. "It always has been a mystery to me why a man keen in business and supposed to possess brains goes to pieces when he commits a crime and tells a tale that is full of holes."

"I beg your pardon!" Prale said.

"Sit down, Mr. Prale, over there---and I'll have some of the witnesses in. I have not questioned them yet, but my men have, and have reported to me what they said. They have discovered several other things, too."

"I'm not afraid of anything they may have discovered," Prale told the captain.

"Last night, you told Jim Farland that you had had trouble with a bank, and at the hotel where you first registered after you came ashore, did you not?"

"Yes; don't those things bear out my statement about the powerful enemies?"

"We'll see presently," the captain said.

He spoke to the sergeant in attendance, who immediately left the room, and presently returned with the president of the trust company. He looked at Prale with interest, and took the chair the captain designated.

"You know this man?" the captain asked.

"I do," said the banker. "He is Sidney Prale."

"Ever have any business with him?"

"Mr. Prale transferred a fortune to our institution from Honduras," the banker said. "Yesterday he called at the bank, satisfied me as to his identity, and made arrangements concerning the money."

"Mr. Prale has said that, for some reason unknown to him, you told him you did not care to handle his business and didn't want his deposit," the captain said.

"I scarcely think that was the way of it," the banker replied. "We would have been glad to take care of the deposit, which was practically one million dollars. But Mr. Prale told me he had other plans and that he would remove the deposit during the day, which he did."

Sidney Prale sat up straight in his chair. "Didn't you tell me that you didn't want anything to do with me and my money?" he demanded.

"Certainly not," lied the banker. "You said that you wished to put your funds in other institutions."

Prale gasped at the man's statement. It was a bare-faced lie if one ever had been spoken.

"Why------" Prale began.

"I do not care to discuss the matter further," the banker interrupted. "I am a man of standing and cannot afford to be mixed up in a case of this sort."

"You'll not be mixed up in it," the captain said. "I just wanted to show Mr. Prale that there were some holes in his story. That is all, thank you!"

The banker left the room quickly, and Prale sprang to his feet, his face livid.

"That man lied!" he exclaimed. "You could read it in his face! I don't know why he lied, but he did!"

"Sit down, Mr. Prale, and let's have more witnesses in," the captain said.

Once more he spoke to the sergeant, and again the latter went out, this time to return with the manager of the first hotel at which Prale registered.

"Know this man?" the captain asked.

"He registered at my place as Sidney Prale, of Honduras."

"Well, what about it?"

"We furnished him with a suite on the fifth floor," the hotel manager said. "But he gave it up."

"Gave it up!" Prale cried. "Why, you called me into your office and told me to get out, that the suite has been reserved and that there was none vacant in the house. The bell boy can testify that he called me into the office."

"Certainly he called you into my office, and at my request," the manager said. "I wanted to know why you were leaving, whether any of the employees had treated you with discourtesy. You told me that you had been served poorly in the dining room the evening before, and that you were done with the hotel!"

Prale sprang to his feet. "That's a lie, and you know it!" he cried.

"Captain," said the hotel man, "do I have to sit here and be insulted by a man charged with a heinous crime?"

"That will be all, thank you," the captain said.

The hotel manager hurried from the room, and the captain grinned at Prale.

"So he lied, too, did he?" the captain asked.

"He did!" Prale cried.

"There seems to be an epidemic of falsehood, to hear you tell it. However, let us get on with the affair."

Once more he instructed the sergeant, and this time the man brought in the hotel detective who had witnessed the trouble between Prale and Shepley.

The hotel detective told the story much as Prale himself had told it, except that he made it appear that Prale had threatened Rufus Shepley on the walk in front of the hotel before they separated.

"Did you pick up a fountain pen of mine after I had gone?" Prale asked.

"I did not."

"See anybody else pick it up?"

"No, sir," said the hotel detective; and he went out of the room.

The sergeant next ushered in George Lerton. Prale sat up straight in his chair again. Here was where his proper alibi began, with the exception of Jim Farland. George Lerton's face was pale as he sat down at the end of the desk.

"Know this man?" the captain asked.

"He is my cousin, Sidney Prale."

"How long has he been away from New York?"

"About ten years," Lerton said. "He returned day before yesterday, I believe. I saw his name in the passenger list."

"Mr. Prale says that he met you last night on Fifth Avenue, and that you told him he had some powerful enemies seeking to cause him trouble, and advised him to leave New York and remain away."

"Why---why this is not so!" Lerton cried. "I haven't seen him until this moment. I would have looked him up, but did not know at what hotel he was stopping, and thought that he'd try to find me."

Prale was out of his chair again, his face flaming. "You mean to sit there and tell me that you didn't talk to me on Fifth Avenue last night?" he cried.

"Why, of course I never talked to you, Sid. I never saw you. What are you trying to do, Sid? Why have you done this thing? We never were close to each other, and yet we are cousins, and I hate to see you in trouble."

"Stop your hypocritical sniveling!" Prale cried. "You are lying and you know it! You saw me last night------"

"But I didn't!"

"You did---and tried to get me to run away, and wouldn't tell me your reason for it."

George Lerton licked at his lips and looked appealingly at the captain of detectives.

"I---I am a man of standing," he whimpered. "I am a broker---here is my card. This man is my cousin, but I cannot lie to shield him. I never saw him last night, and did not speak to him."

Lerton got up and started for the door, and Sidney Prale did not make a move to stop him.

"It appears that your story is full of flaws," the captain said. "A little of it is true, however; you did meet Jim Farland and talk to him in Madison Square, and remained for the length of time you said. Jim has told me that much. But he does not know where you went and what you did after leaving him. What we are interested in is what you did in the neighborhood of eleven o'clock last night. That is when Rufus Shepley was killed. And now we'll have in that new valet of yours."

There was a snarl on Murk's face as he came into the room and sat down in the chair at the end of the desk. Murk did not like policemen and detectives, and did not care whether they knew of his dislike. He flashed a glance at Sidney Prale and then faced the captain.

"Well, what is it?" he asked.

"Tell us where and how you met Mr. Prale first, what happened, and bring the story right up to date," the captain commanded.

"Well, I went down to the river to jump in," Murk said, as if stating a simple fact. "I was tired of fightin' to live and had decided to end it all. Mr. Prale grabbed me and hauled me back, and then he made me see that suicide was foolish. He offered me a job, and I agreed to take it. He was the first man who had treated me decent since I------"

"Never mind that; get down to cases."

"Well, we walked up the street and got a taxicab and drove downtown, and Mr. Prale bought me some clothes."

"What time was it when you met him?"

"I guess it was about ten o'clock. We bought the clothes, as I said, and then we went to a barber shop, and I got a hair cut and a shave. After that we went to Mr. Prale's hotel and up to his rooms. We got to bed pretty quick."

"What time did you reach the hotel?"

"About midnight."

"What happened after you went to bed?"

"Went to sleep," said Murk.

"Never mind the jokes," the captain rebuked sternly.

"Well, I stayed awake about an hour or so thinking how lucky I was, and then I went to sleep. I woke up early in the mornin' and got up and dressed. Mr. Prale got up later, and we ate breakfast in the suite. Then the cops came. 1 of them took Mr. Prale away, and he told me to stay in the rooms until sent for. The other cop rummaged around the rooms and then left."

Prale bent forward. "There is one man who can speak the truth," he told the captain. "His story corresponds with the one I told you, doesn't it? And doesn't it show that I could not have murdered Rufus Shepley at eleven o'clock last night?"

"The story is all right, and it certainly corresponds with yours," replied the captain. "Just a minute!" He faced Murk again. "Who are you and where did you come from?" he demanded.

"I ain't anybody in particular. I've been hangin' around town a couple of months doin' odd jobs. Before that I was bummin' around the country workin' whenever I got a chance."

"You felt grateful to Mr. Prale for giving you a job and a home, didn't you?"

"Sure!" said Murk. "He talked to me decent, like I was a man instead of a dog."

"Well, you don't seem to have much standing in the world," the captain said. "Your word, against that of several prominent citizens, does not carry much weight. You must see that. And there happens to be something else, too. I had the clothing merchant and the barber you mentioned look you over while you were in the other room. The clothing merchant says he sold some clothes a couple of days ago, the ones you are wearing now, but that he certainly did not sell them last night, and the barber swears that he never saw you before!"

"Why, the dirty liars!" Murk cried.

"Did they say that?" Prale demanded.

"They did," the captain replied. "And they said it in such a way that I believe them. Prale, your alibi is shot full of holes. You told the truth about meeting Jim Farland, and that much is in your favor. Aside from that, we have only the testimony of a tramp you said you picked up and gave a job. You had plenty of time to kill Rufus Shepley. You had ample time to concoct the story and get this man to learn it, so he could tell it and match yours. You are worth a million dollars, and this man probably was ready to lie a little for a wad of money."

"He tells the truth------"

"It's too thin, Prale! And don't forget the fountain pen that was found beside Shepley's body, either! As for you Murk, or whatever your right name is, you are under suspicion yourself."

"What's that?" Murk snarled.

"You are under suspicion, I said. You might have assisted at the murder, for all I know. I don't know when you met Mr. Prale, or where, but I do know that you got back to the hotel with Mr. Prale about midnight---an hour after the crime was committed."

"You can't hang anything like that on me!" Murk snarled. "All the cops in the world can't do it! I met Mr. Prale just like I said, and he bought me the clothes and took me to the barber shop, no matter what the store man and the barber say! It's a black lie they're tellin'! Mr. Prale is a gentleman------"

"That'll be enough!" the captain exclaimed. "I'm going to allow you to go, Murk, but you are to remain in Mr. Prale's rooms and take care of his things. And you can bet that you'll be watched, too."

"I don't care who watches me!"

"As for you, Mr. Prale, you'll have to go to a cell, I think. The evidence against you is such that I cannot turn you loose. You must realize that yourself."

Prale realized it. His face was white and his hands were shaking. He looked across the room at Murk.

"You go back to the hotel, Murk, and do as the captain says," he ordered. "I'll come out of this all right in time. There are a lot of things I cannot understand, but we'll solve the puzzle before we're done."

"Ain't there anything I can do, sir?" Murk asked.

"Perhaps, later. I'll engage a detective and a lawyer, and they may visit you at the hotel. I'll send you money by the lawyer. That's all now, Murk."

Murk started to speak, then thought better of it and went from the room slowly, anger flushing his face. Sidney Prale faced the captain of detectives again.

"No matter what you think, I am innocent, and know that my innocence can be proved," Prale said. "You are only doing your duty, of course. I want Jim Farland to attend to things for me. He is an old friend of mine and he is an honest man. Will you send for him?"

"He's waiting in the other room now," the captain said. "I'll let you have a conference with him before I order you into a cell!"

\vspace{2\nbs}
\ChapterDeco[c1]{\decoglyph{e9665}}
\clearpage
\thispagestyle{empty}

\begin{ChapterStart}
\vspace{3\nbs}
\ChapterSubtitle[l]{Chapter ch9}
\ChapterTitle[l]{ch9}
\end{ChapterStart}
\FirstLine{\noindent ## Puzzled}
    
Once more Prale was taken to the room in which he had first waited---the room with the barred windows. This time the watching detective was missing. When Jim Farland entered, he found Prale pacing back and forth from one corner to the other. He was trying to think out his problem, wondering what it all meant, why the witnesses had lied, and what would be the outcome.

Farland rushed into the room, grasped Prale by the hand, led him across from the door, and forced him into a chair. This done, the loyal detective sat down facing him.

"Now let us have it from beginning to end!" Farland commanded. "I don't want you to leave out a thing. I want to get to the bottom of this as soon as possible."

Sidney Prale started at the beginning and talked rapidly, setting forth all the facts, while Jim Farland sat back in his chair and watched him. Now and then he frowned as if displeased at the recital.

"Well, there is something rotten," he said, when Prale had concluded his statement. "I want you to know, Sid, that I believe you. You're not the sort of man to kill a fellow like Rufus Shepley over a little spat. I believe your story about this Murk, too. But why should everybody have it in for you?"

"I haven't the slightest idea," Prale answered. "I must, indeed, have some powerful enemies, but I cannot imagine who they are, and I know of no reason why they should be against me. I'm simply up in the air."

"You keep right on trying to figure it out," Farland advised him. "You might think of something in time that will give me a start in my work."

"Why did the banker and hotel manager lie?" Prale asked. "Why did the clothing-store man and the barber lie? Why did George Lerton declare that he did not see me and speak to me last night? And how did my fountain pen get into Shepley's room?"

"Huh! When we know a few of those things, we'll know enough to wipe this charge away from your name," Jim Farland told him. "It's my job to answer those little questions for you. And now---you want a lawyer, I suppose?"

"Yes. Can you suggest one?"

"The greatest criminal lawyer in town is named Coadley. I'll send him right up here after I explain about this case to him. Thank Heaven, you have plenty of money! A poor man in a fix like this would be on his way to the electric chair. Coadley can fix you up, if anybody can. He can make a sinner look like a saint."

"But I'm not guilty!"

"I understand that, Sid, but it doesn't hurt an innocent man to have the best attorney he can get. I'll send you Coadley. Give me a note to that fellow Murk, for I may want him to help me. Sure he's loyal to you?"

"I never saw him until last night, but I'd bank on him," said Prale. "He'll stand by us!"

"Fair enough! You write that note right now, and try to get out on bail. Tell Coadley to get busy on that right away. Get out under police supervision, under guard---any way---but get out!"

Jim Farland hurried away, and Sidney Prale was conducted through dark corridors to a cell, where he had the experience of hearing a door clang shut behind him and the bolts shot. Prale never had expected to get into jail when he was worth a million dollars, and most certainly he never had expected to face a charge of murder.

He was allowed to send out for some luncheon, and it was more than an hour before Coadley, the attorney, arrived. Prale was taken into the consultation room.

He liked Coadley, and he liked the way in which Coadley regarded him before he spoke.

"I believe that you are innocent," the lawyer said.

"The job will be to make other people think that way," Prale said, with a laugh. The attorney's words had been like a ray of hope to him. "Did Jim Farland tell you the story?"

"Yes. I'll try to get you out on bail, or get you out in some manner," Coadley said. "This appears to be a peculiar case. It is not only the charge of murder; it is the fact that several men told falsehoods about you. You haven't an idea who your enemies are?"

"Not the slightest."

"I'm glad that Jim Farland is working on this case for you, Mr. Prale. He is a good man, and I may need a lot of help. I'll get my own investigators busy right away, too, and we'll coöperate with Jim Farland. You go back to your cell and take it easy. I'll get you out before night, if I can."

Lawyer Coadley was a shrewd man, and his methods were the delight of other attorneys and jurists. He lost no time when he was confronted with a case that held unusual interest. Within an hour he was in court, acting as if fighting mad.

Had a reputable citizen any rights, he demanded? Were the police to be allowed to throw an innocent man into jail simply because there had been a crime committed and somebody had to be accused? His client did not care for an examination at this time, he said. Arraignment and a plea of not guilty were all right, however.

Sidney Prale was arraigned, and the plea of not guilty was made and entered. Then Coadley began his fight to have Prale admitted to bail.

The district attorney opposed it, of course, since that was his business. The judge listened to the statement of the captain of detectives. He heard Coadley say that his client could put up cash bail in any amount, and was willing to abide by any provisions. Finally the judge freed Prale on cash bail of fifty thousand dollars, but designated that the bail could be recalled at any time, and that he was to be in the custody of a member of the police department continually.

Coadley agreed, and left the jail with his client, a detective going with them to stand guard. The detective had explicit orders. He was not to annoy Sidney Prale. He was to withdraw out of earshot when Prale talked with his attorney or anybody else with whom he wished to converse privately. He was to allow Prale to come and go as he wished, except that Prale was not to be allowed to leave the limits of the city. If he attempted that, he was to be put under arrest immediately and taken to the nearest police station.

Prale read the newspapers as he rode to the hotel with Coadley and the detective. The story of the crime was in all of them, the tale of his quarrel with Rufus Shepley and of the finding of the fountain pen, and the inevitable statement that the police were on the track of more and better evidence.

Prale expected to be ordered out of the hotel, but he was not, the management stipulating only that he should not use the public dining room. He went up to the suite, to find Murk there, sitting in front of a window and glaring down at the street.

A cot was moved in for the use of the detective. Coadley held another conference with Prale, and then left to get busy on the case. Murk regarded the detective with scorn, until Prale explained the situation to him. After that, there was a sort of armed neutrality between them. Murk had no special liking for detectives, and he was the sort of man detectives do not like.

Presently Jim Farland arrived.

"Well, Sid, Coadley got you out of jail and home before I could get here, did he?" Farland said. "I suppose I'll not need that note of yours now. Is this Mr. Murk?"

"It is," Prale said. "Murk, meet Jim Farland. He's a detective friend of mine."

"Gosh, Mr. Prale, ain't there anybody but cops in this town?" Murk asked.

"Jim is a private cop, and he has a job now to get me out of this scrape," said Prale. "He's a friend of mine, I said."

"I guess that makes it different," was Murk's only comment.

"Oh, we'll get along all right," Farland put in. "I'm going to need you in my business, Murk. I've told the folks at police headquarters that I'd be responsible for you, so we can work together without being pestered. Understand?"

Murk grinned at him. "You just show me how to help get Mr. Prale out of this mess, and I'll sure help," he said.

Farland turned toward the police detective. "Go out into the hall and take a walk," he suggested. "Mr. Prale will give you a couple of cigars."

The detective took the cigars and went out into the hall, smiling. He had no fear of Sidney Prale slipping down a fire escape, or anything like that. Jim Farland was responsible, and Jim Farland was known to the force as a man who felt his responsibilities.

"Now we'll get busy and dig to the bottom of this mess," Farland said. "Been thinking it over, Sid? Know any reason why anybody should be out after you?"

"I can't think of a thing," Prale replied. "I suppose I made a few business enemies down in Honduras, but none powerful enough to cause me all this trouble. I can't understand it, Jim. It must be something big to cause all those men to lie as they did."

"Maybe it is, and maybe it is very simple when we get right down to it," Farland said. "I've started right in to work it out. Let me see those notes and messages you received."

Prale got them from the dresser drawer and handed them to Farland. The detective looked them over, even going as far as to use a magnifying glass.

"Don't laugh!" Farland said. "A lot of folks make fun of the fiction detective who goes around with a magnifying glass in one hand, but, believe me, a good glass shows up a lot of things. It isn't showing up anything here, though. Where do you suppose these things came from?"

"I don't know," said Prale.

"Got the first one on the ship, did you?"

"The first two. 1 was pinned to the pillow in my stateroom, and the second was pasted on the end of my suit case as I was landing. The mucilage was still wet."

"Didn't suspect anybody?"

"I didn't think much about it at first," said Prale. "I thought it was a joke, or that somebody was making a mistake."

"Sid, have you told me everything?"

Prale remembered Kate Gilbert and flushed.

"I see that you haven't," Farland said. "Out with it! Some little thing may give me the start I am looking for."

Prale told about Kate Gilbert, about the piece of paper she had dropped as she got into the limousine, about the peculiar way she acted toward him, and the attitude of Marie, the misnamed maid.

"Um!" Farland grunted. "We had one thing lacking in this case---and we have that. The woman!"

"But I only met her down there and danced with her twice."

"Don't know anything about her, I suppose?"

"Not a thing. It was understood that she belonged to a wealthy New York family and was traveling for the benefit of her health. At least, that was the rumor."

"I know of a lot of wealthy families in this town, but I never heard of a Kate Gilbert," Farland said. "I think I'll make a little investigation."

"But why on earth should she be taking a hand in my affairs?" Prale wanted to know.

"Why should you be accused of murder? Why should men tell lies about you?" Farland asked. "Excuse me for a time; I'm going down to the hotel office to find out a few things."

Farland hurried away, and the police detective entered the suite again and made himself comfortable. Jim Farland went directly to the office of the hotel and looked at a city directory. He found no Kate Gilbert listed, except a seamstress who resided in Brooklyn. The telephone directory gave him no help.

But that was not conclusive, of course. A thousand Kate Gilberts might be living in New York, in apartments or at hotels, without having a private telephone.

"Have to get a line on that girl!" Farland told himself. "She's got something to do with this. I'll bet my reputation on it."

Jim Farland went to the smoking room and sat down in a corner. He tried to think it out, groped for a starting point. He considered all the persons connected with the case, one at a time.

Farland knew that Sidney Prale had told the truth. Why, then, had George Lerton told a falsehood about meeting Prale and talking to him, when the truth would have helped to establish an alibi? Why had the clothing merchant and the barber lied?

"I suppose I'll have to use stern methods," Farland told himself. "Old police stuff, I suppose. Well, I'm the man that can do it, take it from me!"

He went up to Prale's suite again.

"Can't find out anything about that woman," he reported. "And I want to get in touch with her. Keep your eyes peeled for her, Sid, and arrange for me to catch sight of her, if you can. Now you'd better take a little rest. You've been through an experience to-day. I'm going out to get busy, and I'm going to take Murk with me."

"What for?" Murk demanded.

"You're going to help me, old boy."

"Me work with a cop?" Murk exclaimed.

"To help Mr. Prale."

"Well, that's different," Murk said. "Wait until I get my hat."

\vspace{2\nbs}
\ChapterDeco[c1]{\decoglyph{e9665}}
\clearpage
\thispagestyle{empty}


\scenebreak
\scenebreak
{\centering\textsc{the end}\par}

\clearpage

\null

\centering\textsc{www.TalesofMurder.com}\par

\vspace*{10\nbs}

%\centering\InlineImage[0, 3em]{/home/darkstar/dox/working-files/LaTeX/atticus.jpg}

TALES OF MURDER PRESS, LLC

\null

\scshape{675 TOWN CENTER BLVD
BLDG 1A STE 200 PMB 530
GARLAND, TEXAS 75040}

\null

\textit{atticus@talesofmurder.com}
\vfill


\end{document}
