% !TeX TS-program = LuaLaTeX
% !TeX encoding = UTF-8
\documentclass{novel}
%%% METADATA (FILE DATA):
\SetTitle{The Corpse Steps Out}
\SetAuthor{Craig Rice}
\SetPDFX{X-1a:2001}
\SetTrimSize{4.25in}{6.875in}
\SetMediaSize{4.5in}{7.12in}
\SetMargins{0.5in}{0.5in}{0.5in}{0.7in}
\SetParentFont{Libertinus Serif}
\SetFontSize{9.5pt}
\SetHeadFootStyle{5}
\SetHeadJump{1.5}
\SetFootJump{1.5}
\SetLooseHead{50}
\SetEmblems{}{} % Default blanks.
\SetHeadFont[\parentfontfeatures,Letters=SmallCaps,Scale=0.92]{\parentfontname}
\SetPageNumberStyle{\thepage}
\SetVersoHeadText{\theAuthor}
\SetRectoHeadText{\theTitle}
%%% CHAPTERS:
\SetChapterStartStyle{footer} % Equivalent to empty, when style has no footer.
\SetChapterStartHeight{10}
\SetChapterFont[Numbers=Lining,Scale=1.6]{\parentfontname}
\SetSubchFont[Numbers=Lining,Scale=1.2]{\parentfontname}
\SetScenebreakIndent{false}
%%% BEGIN DOCUMENT:
\begin{document}
\frontmatter
\thispagestyle{empty}
% Half-Title Page.
\begin{parascale}[2]
\vspace*{3\nbs}
\centering\charscale[0.75]{The Corpse Steps Out }\par
\centering\charscale[0.75]{The Corpse Steps Out LINE 2}\par
\centering{The Corpse Steps Out LINE 3}\par
\end{parascale}
\clearpage
\thispagestyle{empty}
\null % Necessary for blank page.
% Alternatively, List of Books.
\clearpage
\thispagestyle{empty}
% Title Page.
\begin{parascale}[4]
\centering\charscale[0.75]{The Corpse Steps Out }\par
\centering\charscale[0.75]{The Corpse Steps Out LINE 2}\par
\centering{The Corpse Steps Out LINE 3}\par
\end{parascale}
\vspace*{2\nbs}

\begin{parascale}[1]
\centering\textit{SERIES}\par
\vspace*{3\nbs}
\charscale[2]{Craig Rice}\par
\end{parascale}
\vfill
\begin{parascale}[1]
% \centering\InlineImage[0, 3em]{/home/darkstar/dox/working-files/LaTeX/atticus.jpg}

A Tales of Murder Press, LLC\par
\textit{Noir} Novel\par
\end{parascale}
\clearpage
\thispagestyle{empty}
% Copyright Page.
\null\vfill
\allsmcp{First edition} Simon & Schuster, March 1940 by Tales of Murder Press, LLC\par
\null\null
\allsmcp{ISBN}\par
\null\null
\vfill
\begin{adjustwidth}{3em}{3em}
\textit{This novel is in the public domain.} Certain \mbox{elements} in this edition are Copyright © 2024 Tales of Murder Press, LLC
\end{adjustwidth}
\clearpage
\thispagestyle{empty}
\clearpage % because ToC must start recto
\thispagestyle{empty}
\begin{toc}[0.5]{0em}
{\centering\charscale[1.25]{Contents}\par}
\null

\tocitem*[1]{None}{1}
\tocitem*[2]{None}{2}
\tocitem*[3]{None}{3}
\tocitem*[4]{None}{4}
\tocitem*[5]{None}{5}
\tocitem*[6]{None}{6}
\tocitem*[7]{None}{7}
\tocitem*[8]{None}{8}
\tocitem*[9]{None}{9}
\tocitem*[10]{None}{10}
\tocitem*[11]{None}{11}
\tocitem*[12]{None}{12}
\tocitem*[13]{None}{13}
\tocitem*[14]{None}{14}
\tocitem*[15]{None}{15}
\tocitem*[16]{None}{16}
\tocitem*[17]{None}{17}
\tocitem*[18]{None}{18}
\tocitem*[19]{None}{19}
\tocitem*[20]{None}{20}
\tocitem*[21]{None}{21}
\tocitem*[22]{None}{22}
\tocitem*[23]{None}{23}
\tocitem*[24]{None}{24}
\tocitem*[25]{None}{25}
\tocitem*[26]{None}{26}
\tocitem*[27]{None}{27}
\tocitem*[28]{None}{28}
\tocitem*[29]{None}{29}
\tocitem*[30]{None}{30}
\tocitem*[31]{None}{31}
\tocitem*[32]{None}{32}
\tocitem*[33]{None}{33}
\tocitem*[34]{None}{34}


\end{toc}
\clearpage

\mainmatter
\cleartorecto
\thispagestyle{empty}


\begin{ChapterStart}
\vspace{3\nbs}
\ChapterSubtitle[l]{Chapter ch1}
\ChapterTitle[l]{Chapter ch1}
\end{ChapterStart}
\FirstLine{\noindent Everything in the big, shabby room was painfully familiar. Not one thing had been changed in the months since she had seen it last. There were the same faded tan curtains at the window, one still hanging a little askew; the same pictures; even the same discolored spot on the wall over the fireplace.}

She stood for one moment, listening. Nothing stirred. Yet for that moment she had found herself waiting for someone to speak.

It was a room she had never thought to see again. Certainly not on such an errand. Suddenly she shuddered, one hand grasping a sharp corner of the mantel for support, remembering the last time she had seen it, when she had walked out swearing it was the last time.

Involuntarily her eyes turned toward the floor of the kitchenette. The light from a tarnished bridge lamp reflected on the little pool of blood that seemed like a shadow reaching out toward the room. Once more she resisted an impulse to turn and flee.

Was someone watching her?

No, that was impossible. She had shut and locked the door. There was no one, could be no one, save herself, alive in the room.

Yet everywhere she turned, she could feel eyes following.

Suddenly she noticed that the tips of four pale fingers showed beneath the dingy green curtains of the kitchenette. For an instant, she clung to the mantel, fighting back the waves of weakness and nausea that threatened to engulf her. What if she should faint, here in this room, alone with that thing in the kitchenette? What if someone should come in and find her here?

For the barest breath of time, she decided in favor of flight.

But she knew there could be no escape from the things she still had to do. It was the voice coming from the radio set that reminded her. Suddenly she was aware that it was still going on. All this time the radio had been going, dance music, voices, crazy rhythms, singing, laughter.

Had it been turned on, she wondered, an hour before?

She detached her fingers slowly from the edge of the mantel and walked over to the window, telling herself that right now in ten thousand, a hundred thousand, a million rooms, loudspeakers were still turned on, families still gathered before their radio sets. Not so very long ago switches had clicked and listeners had settled back in their easy chairs to wait for her voice. Right now, out on the Pacific coast, more listeners were eyeing their clocks, making ready to tune in on the rebroadcast.

Now, between broadcasts, there was the thing she must do.

One long indrawn breath, her eyes closed, and then she walked slowly around the room, carefully avoiding the soiled green curtains of the kitchenette, reassuring herself with the touch of familiar objects, the look of familiar things.

Suddenly a voice, deep, warm, chocolatish, came from the loudspeaker.

“You’re nobody’s sweetheart now. ...”

She wheeled to stare at the object of wood and wire and, as she turned, a grotesque flicker of light momentarily transformed the fingertips below the kitchenette curtains into living, curling, and then uncurling things.

“It just don’t seem right, somehow, That you’re nobody’s sweetheart now....”

With one quick frenzied movement, she clicked the singing thing off in the middle of a word.

In the unsuspected silence, the harsh, indisputable ticking of a clock reminded her that she had very little time left. At the sound of it, her strength seemed to return. All at once she ceased to be the great radio star, the photographed and glamorous personality, the wife of a well-known socialite, the protected darling of the fan magazines. She was back in her childhood again, back in the days when every mouthful of food depended on resource and cunning, when each day’s living had to be fought for with desperation. She could still fight, she reminded herself, with the same cunning, the same desperate frenzy.

Resolutely she wrenched her eyes away from the kitchenette and began searching the room, hurriedly, frantically, but still with a sort of disordered efficiency. No one in the world—no one alive in the world—knew that room better. She searched the imitation spinet desk, with the long cigarette burn still showing on the veneer, remembering with a little shudder the night it had been made there. Nothing in the desk but newspaper clippings and unpaid bills. The chest of drawers in the closet was only a confusion of soiled shirts and socks. She hunted through the bookshelves, filled with inexpensive and unread editions of standard classics, and pulled out one book after another, shaking it, reaching behind the rows. She felt under the pillows of the double bed that disguised itself as a studio couch, extended experimental fingers under the mattress.

There was still the little hiding place behind the cheap Venetian mirror, where they had once left notes for each other. She lifted out the mirror, ran her fingers carefully along the ledge, while purplish dust accumulated on her fingertips. Nothing there. Nothing but one discarded hairpin, dust-covered and rusted. She held it a moment on the palm of her hand, staring at it, and recognizing it as her own. Had it been there all this time?

But the thing she had come to find, the thing she must find, the reason for her terrible errand, was nowhere in the shabby room.

Was she being watched?

She stood, breathless, listening. There was the faint dripping of water from the cold-water faucet in the kitchenette. (Hadn’t that faucet been fixed in all these months?) It sounded like the slow, remorseless, inexorable ticking of a clock.

There was so little time left!

Again she held herself back from headlong flight. Too much depended on her now. So much? Everything! Surely, she told herself, it was not so terrible a thing to do. Worse things had been done in this world, and bravely, too. Yes, even she herself had done them.

She was not only fighting for herself. There were others to fight for, she remembered them one by one, while slowly the courage she had lost came back to her.

There was no other way.

She went into the kitchenette, knelt on the floor, and carefully, methodically, began searching the dead man’s pockets.


\scenebreak
\scenebreak
{\centering\textsc{the end}\par}

\clearpage

\null

\centering\textsc{www.TalesofMurder.com}\par

\vspace*{10\nbs}

%\centering\InlineImage[0, 3em]{/home/darkstar/dox/working-files/LaTeX/atticus.jpg}

TALES OF MURDER PRESS, LLC

\null

\scshape{675 TOWN CENTER BLVD
BLDG 1A STE 200 PMB 530
GARLAND, TEXAS 75040}

\null

\textit{atticus@talesofmurder.com}
\vfill


\end{document}
